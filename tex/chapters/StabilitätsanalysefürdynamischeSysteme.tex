\chapter{Stabilitätsanalyse für dynamische Systeme}
\label{\detokenize{odestability/stabilitaetsanalyse:stabilitatsanalyse-fur-dynamische-systeme}}\label{\detokenize{odestability/stabilitaetsanalyse::doc}}
\par
In diesem Abschnitt beschäftigen wir uns mit der Stabilitätstheorie für kontinuierliche dynamische Systeme.
Hierbei interessieren wir uns für die Frage, wie sich \emph{kleine Störungen} von bestimmten Zuständen des Systems auf die Lösungen der zu Grunde liegenden gewöhnlichen Differentialgleichungen auswirken.
Der untersuchte Zustand kann beispielsweise ein periodischer Orbit oder eine Ruhelage des dynamischen Systems sein.
Letztere sind oftmals von besonderes Interesse, da man in vielen technischen und physikalischen Anwendungen daran interessiert ist das System in eine oder nahe einer Gleichgewichtslage zu bringen.

\par
Im Folgenden werden wir verschiedene Stabilitätsbegriffe für dynamische Systeme einführen und speziell Kriterien für die Stabilität von Ruhelagen diskutieren.


\section{Stabilitätsbegriffe}
\label{\detokenize{odestability/stabilitaetsbegriffe:stabilitatsbegriffe}}\label{\detokenize{odestability/stabilitaetsbegriffe::doc}}
\par
Im Folgenden wollen wir grundlegende Begriffe der Stabilitätsanalyse von Ruhelagen einführen und diskutieren.
Wie in \cref{ode/fluesse:s-fluesse}  definiert, nennen wir einen Punkt \(x\in U\) im Phasenraum \(U\) \textbf{Ruhelage}, falls für den zugehörigen Phasenfluss \(\Phi \colon I \times U \rightarrow U\) des dynamischen Systems gilt: \(\Phi(t,x) = x, \forall t \in I\), d.h., wenn für alle \(t \in I\) der Zustand \(x \in U\) ein \textbf{Fixpunkt des Flusses} ist.

\par
Für autonome Differentialgleichungssysteme mit
\begin{align*}
\dot{x}(t) = F(x)
\end{align*}
\par
ist \(x \in U\) auch eine Ruhelage, falls \(F(x) = 0\) gilt, d.h., falls \(x\) eine Nullstelle von \(F\) ist.
Das ist einfach zu verstehen, da die Zeitableitung auf der linken Seite für eine Ruhelage Null ist und somit die Funktion \(F\), die nur vom Ort abhängt, sich nicht ändern kann.

\par
Anschaulich versteht man unter der Stabilitätsanalyse von Ruhelagen die mathematische Untersuchung, ob benachbarte Lösungen von einer Ruhelage wegstreben oder nicht.
Dies ist insbesondere in technischen Anwendungen wichtig, da man dort häufig danach strebt ein dynamisches System in eine Gleichgewichtslage zu bringen.
Da dies nur bis zu einer gewissen Genauigkeit möglich ist, muss man also mit kleinen Störungen rechnen.

\par
Ist eine Ruhelage stabil, dann bleiben benachbarte Lösungen auch für zukünftige Zeitpunkte \(t \in I\) nahe der Ruhelage.
Ist sie jedoch nicht stabil, so muss das im Allgemeinen nicht gelten und die Lösungen können dann mit der Zeit von der Ruhelage divergieren.
Diese Anschauung wollen wir in der folgenden Definition mathematisch formalisieren.
Hierbei werden wir den Stabilitätsbegriff für allgemeine Lösungen einführen und später Ruhelagen als ein Spezialfall dieser Lösungen interpretieren.
\begin{definition}{(Stabilität von Lösungen)}{odestability/stabilitaetsbegriffe:def:Stabilitaet}



\par
Sei \(\Phi \colon I \times U \rightarrow U\) der Phasenfluss zu dem Vektorfeld \(F\in C^1(U;\R^n)\) auf \(U\), dass durch die rechte Seite des zugehörigen Differentialgleichungssystems gegeben ist.

\par
1. Eine Lösung \(t \in [0,\infty) \mapsto \Phi_t(x)\) heißt \textbf{(Lyapunov )stabil}, wenn zu jedem \(\epsilon > 0\) ein \(\delta>0\) existiert mit:
\begin{align*}
\|x-y\|<\delta \ \Rightarrow \ \sup_{t\geq0}\|\Phi_t(x)-\Phi_t(y)\|<\epsilon.
\end{align*}
\par
2. Eine Lösung \( t \in [0,\infty) \mapsto \Phi_t(x)\) heißt \textbf{asymptotisch stabil}, wenn ein \(\delta > 0\) existiert mit:
\begin{align*}
\|x-y\|<\delta \ \Rightarrow \ \lim_{t\to\infty}\|\Phi_t(x)-\Phi_t(y)\|=0.
\end{align*}
\par
3. Eine Lösung heißt \textbf{instabil}, wenn sie nicht (Lyapunov )stabil ist.
\end{definition}

\begin{emphBox}{Aleksandr Lyapunov}{}

\par
\href{https://de.wikipedia.org/wiki/Alexander\_Michailowitsch\_Ljapunow}{Alexander Michailowitsch Ljapunow} (Geboren 6. Juni 1857 in Jaroslawl; Gestorben 3. November 1918 in Odessa) war ein russischer Mathematiker und Physiker.
\end{emphBox}

\par
Es ist klar, dass der Begriff der asymptotischen Stabilität \emph{stärker} als der Begriff der Lyapunov Stabilität von Lösungen ist, da jede asymptotisch stabile Lösung auch schon Lyapunov stabil ist.
Die Umkehrung gilt jedoch im Allgemeinen nicht.
Dies wird durch das folgende Beispiel nochmal illustriert.
\begin{example}{(Stabilitätsanalyse für den harmonischer Oszillator)}{odestability/stabilitaetsbegriffe:example-1}



\par
Der Phasenfluss für den harmonischen Oszillator ist, wie wir in \cref{ode/fluesse:ex:oscillations} gesehen haben, gegeben durch
\begin{align*}
\Phi(t, (p,x)) = \begin{pmatrix}
p \cos(\omega t) - m x \sin(\omega t)\\
\frac{p}{\omega m}\sin(\omega t) + x\cos(\omega t)
\end{pmatrix}
\end{align*}
\par
Wir suchen nun einen Fixpunkt \((p_r,x_r) \in U\) des Flusses der unabhängig ist vom Zeitpunkt \(t\).
Man sieht leicht ein, dass eine \textbf{Ruhelage} sich bei \((p_r,x_r) = (0,0)^T \in U\) befindet, da \(\Phi(t,(0,0)) = (0,0)^T\) ist für alle \(t \in I\).
Die gefundene Ruhelage ist \textbf{Lyapunov stabil}, denn wie wir im Phasenporträt in \hyperref[\detokenize{ode/fluesse:fig-harmonic-oscillator}]{Abb.\@ \ref{\detokenize{ode/fluesse:fig-harmonic-oscillator}}} gesehen haben, ist jeder Orbit um die Ruhelage \((0,0)\) periodisch. Damit kann das dynamische System insgesamt nicht wegstreben von der Ruhelage.

\par
Mathematisch lässt sich diese Eigenschaft wie folgt zeigen.
Für ein beliebiges \(\epsilon > 0\) sei \((p,y) \in U\) ein Punkt im Phasenraum mit periodischen Orbit \(O(p,y)\) um die Ruhelage \((p_r,x_r) = (0,0)^T \in U\), so dass dessen maximaler Abstand zur Ruhelage kleiner als \(\epsilon\) ist, d.h.
\begin{align*}
\sup_{t \geq 0} ||\Phi_t(p_r,x_r) - \Phi_t(p,y)|| < \epsilon
\end{align*}
\par
Auf Grund der ersten Eigenschaft des Phasenflusses \(\Phi_0(p,y) = (p,y)\) gilt dann aber schon
\begin{align*}
||(p_r, x_r) - (p,y)|| = ||\Phi_0(p_r, x_r) - \Phi_0(p,y)|| < \epsilon.
\end{align*}
\par
Wählen wir nun \(\delta \coloneqq \epsilon\), so haben wir gezeigt, dass die Ruhelage \((p_r, x_r) = (0,0)^T\) Lyapunov stabil ist.
Sie ist jedoch auf Grund der Periodizität der Orbits um die Ruhelage \textbf{nicht asymptotisch stabil}, da für beliebige Punkte \((p,y) \in U\) mit \(||(p_r,x_r) - (p,y)|| < \delta\) für ein \(\delta > 0\) gilt
\begin{align*}
\lim_{t\to\infty}\|\Phi_t(p_r, x_r)-\Phi_t(p,y)\| \neq 0.
\end{align*}\end{example}

\par
Im allgemeinen Fall der gedämpften Schwingungsgleichung in \cref{ode/fluesse:ex:oscillations} hängt die Stabilität der Ruhelage im Ursprung intuitiverweise von der Reibungskonstanten ab, wie folgende Bemerkung festhält.
\begin{remark}{(Stabilität bei der gedämpften Schwingungsgleichung)}{odestability/stabilitaetsbegriffe:remark-2}



\par
Für den Fall der gedämpften Schwingungsgleichung in \eqref{equation:ode/fluesse:eq:schwingungsgleichung} lässt sich folgendes Stabilitätsverhalten der Ruhelage im Ursprung in Abhängigkeit der Reibungskonstanten \(r \in \R\) beobachten:
\begin{enumerate}

\item {} 
\par
Die Ruhelage ist \textbf{asymptotisch stabil} für den Fall mit positiver Reibung \(r>0\).

\item {} 
\par
Die Ruhelage ist \textbf{Lyapunov stabil} für den reibungsfreien Fall \(r=0\).

\item {} 
\par
Die Ruhelage ist \textbf{instabil} für den Fall einer negativen Reibung \(r < 0\), d.h. für einen externen Antrieb.

\end{enumerate}
\end{remark}


\section{Stabilität von Ruhelagen}
\label{\detokenize{odestability/ruhelagen:stabilitat-von-ruhelagen}}\label{\detokenize{odestability/ruhelagen::doc}}
\par
Zunächst wollen wir die Stabilität von dynamischen System im einfachen Fall von Ruhelagen für allgemeine \textbf{lineare} Differentialgleichungssysteme untersuchen.
Diese Familie von gewöhnlichen Differentialgleichungssystemen haben wir schon in Kapitel 8 in \cite{Ten21} kennen gelernt.

\par
Das folgende Theorem beschreibt die Existenz und Eindeutigkeit einer Ruhelage eines dynamischen System, das durch ein lineares Differentialgleichungssystem charakterisiert wird und gibt Bedingungen für die Stabilität der Ruhelage.
\begin{theorem}{}{odestability/ruhelagen:thm:stablin}



\par
Sei \(A\in \C^{n\times n}\) eine Matrix mit den Eigenwerten \(\lambda_1,\dots, \lambda_n\in \C\).
Dann beschreibt der zugehörige Phasenfluss \(\Phi\) zum homogenen linearen Differentialgleichungssystem
\begin{align*}
\dot{x}(t) = Ax(t)
\end{align*}
\par
eine Ruhelage in \(\mathbf{0} \in \C^n\).
Diese ist sogar eindeutig, falls \(\lambda_i\neq 0, i=1,\ldots,n\) gilt.

\par
Für
\begin{align*}
\gamma \coloneqq \max_{i=1,\dots,n} \mathcal{Re}(\lambda_i)
\end{align*}
\par
kann die Stabilität der Ruhelage wie folgt charakterisiert werden:
\begin{enumerate}

\item {} 
\par
Falls \(\gamma <0\) gilt, ist die Ruhelage \(\mathbf{0}\) \emph{asymptotisch stabil}

\item {} 
\par
Falls \(\gamma >0\) gilt, ist die Ruhelage \(\mathbf{0}\) \emph{instabil}.

\end{enumerate}
\end{theorem}

\begin{proof}
 Wir wissen, dass für einen beliebigen Startpunkt \(x_0 \in U\) im Phasenraum der Phasenfluss \(\Phi \colon I \times U \rightarrow U\) eine Lösung des Differentialgleichungssystems realisiert.
Für homogene, lineare Differentialgleichungssysteme haben wir bereits in \cref{ode/repetition:s-lineare-dglsysteme}  Lösungen mittels des \emph{Matrixexponentials} hergeleitet.

\par
Sei \(J = S^{-1}AS\) die Jordansche Normalform von \(A\) mit Transformationsmatrizen \(S^{-1},S \in \C^{n \times n}\), so erhalten wir die Abschätzung
\begin{align*}
\|\Phi_t(x_0)\| &= \|e^{tA}x_0\| = \|S^{-1}e^{tJ}Sx_0\| = \|S^{-1}e^{tD}e^{tN}Sx_0\| \\
&\leq \|S^{-1}\| \cdot \|e^{tD}\| \cdot \|e^{tN}\| \cdot \|S\| \cdot \|x_0\| \leq C_1 \cdot \|e^{tD}\| \cdot \|e^{t N}\|,
\end{align*}
\par
für eine Konstante \(C_1 > 0\), die unabhängig von \(t\) ist.
Hierbei haben wir ausgenutzt, dass sich die Jordannormalform \(J\) von \(A\) als Summe einer Diagonalmatrix \(D\) mit den Eigenwerten \(\lambda_i \in \C\), \(i=1,\ldots,n\) von \(A\) und einer nilpotenten Matrix \(N\) schreiben lässt als \(J = D + N\).
Diese Matrizen kommutieren, d.h., \(D \cdot N = N \cdot D\).

\par
Wir sehen nun ein, dass \(e^{tN}\) wegen der Nilpotenz von \(N\) eine endliche Reihe bildet der Form
\begin{align*}
e^{tN} = \sum_{k=0}^m \frac{(tN)^k}{k!} = \sum_{k=0}^m t^k\frac{N^k}{k!},
\end{align*}
\par
welches ein Polynom vom Grad \(m\) darstellt, wobei \(m \in \N\) der Nilpotenzindex der Matrix \(N\) ist.

\par
Sei nun \(\epsilon > 0\) beliebig klein gewählt.
Dann lässt sich die Norm des Polynoms mit einer genügend großen Konstanten \(C_2 > 0\), die von \(\epsilon\) jedoch nicht von \(t\) abhängt, durch eine gewöhnliche Exponentialfunktion abschätzen mit
\begin{align*}
 \|e^{tN}\| = \| \sum_{k=0}^m t^k\frac{N^k}{k!} \| \leq \sum_{k=0}^m t^k \frac{\|N^k\|}{k!} \leq C_2  e^{t \epsilon}.
\end{align*}
\par
Wählen wir nun \(\gamma \coloneqq \max_{i=1,\dots,n} \mathcal{Re}(\lambda_i)\), so folgt direkt, dass gilt
\begin{align*}
||e^{tD}|| \leq C_3 e^{t\gamma}.
\end{align*}
\par
Insgesamt erhalten wir also für die Norm des Phasenflusses
\begin{align}\label{equation:odestability/ruhelagen:eq:abschaetzungew}
\|\Phi_t(x_0)\| \leq C_1 \cdot \|e^{tN}\| \cdot \|e^{tD}\| \leq C_1 \cdot C_2 e^{t \epsilon} \cdot C_3 e^{t\gamma} = C e^{t \epsilon} e^{t\gamma}.
\end{align}
\par
Da \(\epsilon > 0\) beliebig klein ist, können wir \(|\epsilon| < |\gamma|\) wählen.
Damit hängt das Verhalten der Norm des Flusses nur noch vom Vorzeichen von \(\gamma\) ab.
Wir unterscheiden daher zwei Fälle:

\par
1. Wenn \(\gamma >0\) ist, so existiert zum Eigenwert \(\gamma\) von \(A\) ein zugehöriger Eigenvektor \(v\in U\), so dass die Eigenwertgleichung \(A v = \gamma v\) gilt.
Nach \cref{ode/repetition:lem:mpotew} ist dann \(e^{t\gamma}\) ein Eigenwert des Matrixexponentials \(e^{tA}\) mit zugehörigem Eigenvektor \(v\).
Insgesamt erhalten wir also
\begin{align*}
||\Phi_t(\alpha v)|| = ||e^{tA}\alpha v|| = ||\lambda e^{t\gamma} \alpha v|| \to \infty, \quad \text{ für } \ t \to \infty, \quad  \forall \alpha>0.\end{align*}
\par
Also enthält jede beliebig kleine Umgebung der Ruhelage \(0\) Punkte, für die die entsprechenden Lösungen divergieren.
In diesem Fall ist die Ruhelage also \textbf{instabil}.

\par
2. Falls \(\gamma <0\) gilt, so gilt auch \(\gamma + \epsilon <0\) und wir können abschätzen,
\begin{align*}
0\leq \|\Phi_t(x_0)-0\|\leq C e^{t (\gamma + \epsilon)} \to 0 \quad \text{ für } \ t \to \infty.
\end{align*}
\par
Dies liefert uns also \textbf{asymptotische Stabilität} der Ruhelage \(\mathbf{0}\).
\end{proof}

\par
Wir haben also gesehen, dass im Fall eines homogenen, linearen Differentialgleichungssystems die \(\mathbf{0}\) immer eine Ruhelage des zugehörigen dynamischen Systems darstellt, deren Stabilität einzig vom Vorzeichen des größten Eigenwerts abhängt.


\subsection{Linearisierung um Ruhelage}
\label{\detokenize{odestability/ruhelagen:linearisierung-um-ruhelage}}\label{\detokenize{odestability/ruhelagen:s-linearisierung-ruhelage}}
\par
In diesem Abschnitt wollen wir unsere Erkentnisse zur Stabilitätsanalysie vom Fall eines linearen Differentialgleichungssystems auf den allgemeinen Fall übertragen, da man es in den meisten Anwendungen leider nur selten mit linearen Differentialgleichungen zu tun hat.
Darüber hinaus ist es erstrebenswert Stabilitätsaussagen zu Differentialgleichungen zu machen, deren Lösungen man nicht explizit analytisch herleiten kann.
Daher betrachten wir im Folgenden das Anfangswertproblem eines \textbf{allgemeinen Differentialgleichungssystem erster Ordnung} auf dem Phasenraum \(U\in \R^n\), das nicht notwendigerweise linear sein muss und für ein Vektorfeld \(F\in C^1(U;\R^n)\) wie folgt formuliert ist
\begin{align}\label{equation:odestability/ruhelagen:eq:awpallg}
\dot{x}(t) &= F(x(t)), \quad \forall t \in I \subset \R^+_0\\
x(0) &= x_0.
\end{align}
\par
Wir nehmen an, dass \(x_F \in U\) eine Ruhelage des dynamischen Systems ist, so dass dementsprechend \(F(x_F) = 0\) gilt.
Durch einfache Translation der Koordinaten des Systems um \(x_F \in U\), können wir ohne Beschränkung der Allgemeinheit annehmen, dass die Ruhelage sich im Nullpunkt befindet.

\par
Im Folgenden definieren wir zwei wichtige Werkzeuge zur Untersuchung der Stabilität von Ruhelagen für allgemeine Differentialgleichungssysteme.
\begin{definition}{(Linearisierung und Abweichung)}{odestability/ruhelagen:def:linearisierung}



\par
Sei \(F\in C^1(U;\R^n)\) ein Vektorfeld auf dem Phasenraum \(U \subset \R^n\) und \(0\) eine Ruhelage des dynamischen Systems, dass durch das allgemeine Differentialgleichungssystem in \eqref{equation:odestability/ruhelagen:eq:awpallg} charakterisiert wird.
Sei nun \((DF)(x)\) die Jacobi Matrix der Funktion \(F\) im Punkt \(x \in U\) (vgl. Kapitel 6.2 in \cite{Ten21}).
Dann bezeichnen wir mit \(A := (DF)(0)\) die \textbf{Linearisierung} von \(F\) in der Ruhelage \(0 \in U\).
Außerdem bezeichnen wir die Funktion \(R \in C^1(U; \R^n)\) mit
\begin{align*}
R(x) \ \coloneqq \ F(x) - Ax
\end{align*}
\par
als die \textbf{Abweichung} (auch \textbf{Residuum} genannt) des Vektorfeldes \(F\) von seiner Linearisierung \(A\) in der Ruhelage.
\end{definition}

\par
Mit diesen Hilfswerkzeugen werden wir im Folgenden zeigen, dass die Lösung des Differentialgleichungssystem in führender Ordnung durch die Linearisierung \(A\) von \(F\) kontrolliert werden, solange wir uns nah genug zur Ruhelage befinden. Dies wird durch das folgende Lemma ausgedrückt.
\begin{lemma}{}{odestability/ruhelagen:lem:intexpglgn}



\par
Wir betrachten das Anfangswertproblem aus \eqref{equation:odestability/ruhelagen:eq:awpallg} auf dem Phasenraum \(U \subset \R^n\) für ein Vektorfeld \(F\in C^1(U;\R^n)\).
Außerdem sei \(A \coloneqq (DF)(0)\) die Linearisierung des Vektorfelds in der Ruhelage \(0\) des dynamischen Systems und \(R(x) \coloneqq F(x) - Ax\) die Abweichung von \(F\) von seiner Linearisierung \(A\) im Nullpunkt.

\par
Dann lassen sich Lösungen des Differentialgleichungssystems mittels der Linearisierung \(A\) und der Abweichung \(R\) explizit angeben als
\begin{align*}
x(t) = e^{At}x_0 + \int_0^t e^{A(t-s)} R(x(s))\, \mathrm{d}s, \quad \forall t \in I.
\end{align*}\end{lemma}

\begin{proof}
 Wir setzen zunächst die unbekannte Lösung \(x(t)\) des Anfangswertproblems \eqref{equation:odestability/ruhelagen:eq:awpallg} in der allgemeinen Form
\begin{align*}
x(t) = e^{At}c(t),\quad \text{mit }c(0) = x_0
\end{align*}
\par
an, und suchen eine Bestimmungsgleichung für die unbekannte Funktion \(c(t)\) mittels \textbf{Variation der Konstanten} (vgl. Kapitel 8.2 in \cite{Ten21}).

\par
Mittels der Rechenregeln für das Matrixexponentials in \cref{ode/repetition:rem:matrixexponentialregeln} können wir die Ableitung der Funktion \(x\) mittels Produktregel angeben als
\begin{align*}
\dot{x}(s) = A e^{As}c(s)+ e^{As}\dot{c}(s) = Ax(s) + e^{As}\dot{c}(s).
\end{align*}
\par
Aus der Definition des Residuums in \cref{odestability/ruhelagen:def:linearisierung} folgt aber auch
\begin{align*}
\dot{x}(s) = F(x(s)) = Ax(s) + R(x(s)).
\end{align*}
\par
Vergleichen wir die beiden Gleichungen, so sieht man ein, dass
\begin{align*}
e^{As}\dot{c}(s) = R(x(s))
\end{align*}
\par
gelten muss.
Äquivalent können wir auch folgern, dass \(\dot{c}(s) = e^{-As}R(x(s))\) gilt.

\par
Nach dem Hauptsatz der Differential  und Integralrechnung (vgl. Theorem 5.3 in \cite{Ten21}) gilt dann für die unbekannte Funktion \(c\) der folgende Zusammenhang
\begin{align*}
c(t) = c(0) + \int_0^t \dot{c}(s)\, \mathrm{d}s = x_0+ \int_0^t e^{-As}R(x(s)) \, \mathrm{d}s.
\end{align*}
\par
Setzen wir dies in die erste Gleichung unserer Ansatzfunktion ein und nutzen die Rechenregeln des Matrixexponnentials aus \cref{ode/repetition:rem:matrixexponentialregeln}  so erhalten wir schließlich die Aussage des Lemmas
\begin{align*}
x(t) = e^{At}x_0+ \int_0^t e^{A(t-s)}R(x(s)) \, \mathrm{d}s.
\end{align*}\end{proof}

\par
Auf den ersten Blick nützt uns die Identität in \cref{odestability/ruhelagen:lem:intexpglgn} nicht viel, denn auch auf der rechten Seite taucht \(x(s)\), also die unbekannte Lösung des Anfangswertproblems \eqref{equation:odestability/ruhelagen:eq:awpallg} auf.
Es stellt sich jedoch heraus, dass wir die \textbf{Gronwall Ungleichung} auf diese Integralgleichung anwenden können.
Diese wichtige Abschätzung in der Theorie von Differentialgleichungen ähnelt Münchhausens Methode, sich an den eigenen Haaren aus dem Sumpf zu ziehen.

\begin{emphBox}{Thomas Gronwall}{}

\par
\href{https://de.wikipedia.org/wiki/Thomas\_Hakon\_Gr\%C3\%B6nwall}{Thomas Hakon Gronwall} (Geboren 16. Januar 1877 in Dylta Bruk bei Axberg/Gemeinde Örebro; Gestorben 9. Mai 1932 in New York, NY) war ein schwedischer Mathematiker.
\end{emphBox}
\begin{lemma}{(Gronwall Ungleichung)}{odestability/ruhelagen:lemma:Gronwall}



\par
Für zwei stetige Funktionen \(f,g\in C([t_0,t_1]; \R^+)\) gelte für eine Konstante \(a \geq 0\) die Ungleichung
\begin{align*}
f(t) \leq a + \int_{t_0}^t f(s)g(s)\, \mathrm{d}s \quad \forall t\in [t_0,t_1].
\end{align*}
\par
Dann lässt sich der Wert der Funktion \(f\) durch die Funktion \(g\) wie folgt abschätzen
\begin{align*}
f(t) \leq a \exp{ \left(\int_{t_0}^t g(s)\, \mathrm{d}s \right)} \quad \forall t\in [t_0,t_1].
\end{align*}\end{lemma}

\begin{proof}
 Wir definieren zunächst eine Hilfsfunktion
\begin{align*}
h(t) \ \coloneqq \ a + \int_{t_0}^t f(s)g(s)\, \mathrm{d}s
\end{align*}
\par
und bemerken, dass \(0 \leq f(t) \leq h(t)\) nach Voraussetzung gilt für alle \(t \in [t_0, t_1]\).
Nun führen wir eine einfache Fallunterscheidung durch:

\par
1. Ist \(h(t)=0\), so folgt mit der Abschätzung \(f(t) \leq h(t)\) schon, dass \(f(t) = 0\) gelten muss, so dass die Behauptung des Lemmas trivialerweise erfüllt ist.

\par
2. Sei also im Folgenden \(h(t) > 0\).
Aus dem Haupsatz der Integral  und Differentialrechnung wissen wir, dass \(h'(t) = f(t)g(t)\) gilt.
Wegen \(f(t) \leq h(t)\) für alle \(t \in [t_0, t_1]\) folgt sofort, dass
\begin{align*}
f(t)g(t) \leq h(t)g(t) \quad \forall t \in [t_0,t_1].
\end{align*}
\par
Kombinieren wir diese Abschätzung mit der Identität der Ableitung \(h'(t)\), so erhalten wir durch Umstellen
\begin{align*}
\frac{h'(t)}{h(t)} \leq g(t) \quad \forall t \in [t_0, t_1].
\end{align*}
\par
Da wir \(h(t) > 0\) angenommen haben erhalten wir durch Integration beider Seiten die Abschätzung
\begin{align*}
\int_{t_0}^t \frac{h'(s)}{h(s)} \, \mathrm{d}s \leq \int_{t_0}^t g(s) \, \mathrm{d}s\end{align*}
\par
für alle \(t \in [t_0, t_1]\).
Für die linke Seite können wir das Integral explizit angeben als
\begin{align*}
\int_{t_0}^t \frac{h'(s)}{h(s)} \, \mathrm{d}s = \ln(h(t)) - \ln(h(t_0)) = \ln(h(t)) - \ln(a) = \ln\left(\frac{h(t)}{a}\right).
\end{align*}
\par
Es gilt also nun
\begin{align*}
\ln \left(\frac{h(t)}{a}\right) \leq \int_{t_0}^t g(s)\, \mathrm{d}s.
\end{align*}
\par
Durch Anwenden der Exponentialfunktion auf beiden Seiten und Ausnutzen der Voraussetzung \(f(t) \leq h(t)\) erhalten wir schließlich die Behauptung des Lemmas
\begin{align*}
 f(t) \leq h(t)\leq a \exp{\left( \int_{t_0}^t g(s)\, ds \right)} \quad \forall t \in [t_0,t_1].
\end{align*}\end{proof}

\par
Wir wollen folgende Bemerkungen zur Gronwall Ungleichung festhalten.
\begin{remark}{}{odestability/ruhelagen:remark-4}



\par
1. Die in \cref{odestability/ruhelagen:lemma:Gronwall} beschriebene Gronwall Ungleichung ist eigentlich ein Spezialfall für eine konstante Funktion \(a(t) \equiv a \geq 0\).
Die ursprünglich bewiesene Aussage gilt auch für allgemeinere Funktionen.

\par
2. Man kann sich die Abschätzung in der Gronwall Ungleichung leicht merken wenn man Gleichheit der beiden Seiten annimmt.
Die Integralgleichung
\begin{align*}
f(t) = a + \int_{t_0}^t f(s)g(s)\, \mathrm{d}s \quad t\in [t_0,t_1]
\end{align*}
\par
entspricht nämlich dem \textbf{linearen Anfangswertproblem}
\begin{align*}
\dot{f}(t) &= f(t)\cdot g(t) \quad \forall t \in [t_0, t_1], \\
f(t_0) &= a,
\end{align*}
\par
welches für alle \(t \in [t_0, t_1]\) die folgende explizite Lösung besitzt
\begin{align*}
f(t) = a \exp{\left( \int_{t_0}^t g(s)\, \mathrm{d}s \right)}.
\end{align*}\end{remark}

\par
Wir werden die Resultate der beiden Lemmata in den folgenden Abschnitten anwenden, um die Stabilität von Ruhelagen eines allgemeinen dynamischen Systems durch eine Linearisierung zu untersuchen.


\subsection{Asymptotische Stabilität von Ruhelagen}
\label{\detokenize{odestability/ruhelagen:asymptotische-stabilitat-von-ruhelagen}}
\par
Durch die explizite Darstellung von Lösungen allgemeiner Differentialgleichungssysteme basierend auf der Linearisierung und Abweichung des Vektorfeldes \(F \colon U \rightarrow \R^n\) in \cref{odestability/ruhelagen:lem:intexpglgn} und der Gronwall Ungleichung in \cref{odestability/ruhelagen:lemma:Gronwall} sind wir nun in der Lage die Stabilität einer Ruhelage eines dynamischen Systems zu analysieren.

\par
Wir formulieren direkt das Hauptresultat, dass uns ein hinreichendes Kriterium für \textbf{asymptotische Stabilität} der Ruhelage basierend auf den Eigenwerten der Linearisierung liefert.
\begin{theorem}{(Asymptotische Stabilität von Ruhelagen)}{odestability/ruhelagen:thm:stabasymallg}



\par
Sei \(F \in C^1(U; \R^n)\) ein Vektorfeld auf dem offenen Phasenraum \(U \subset \R^n\).
Eine Ruhelage \(x_F \in  U \subset \R^n\) des dynamischen Systems, das durch das allgemeine Differentialgleichungssystem
\begin{align*}
\dot{x}(t) = F(x(t)), \quad \forall t \in \R^+_0
\end{align*}
\par
charakterisiert wird, ist \textbf{asymptotisch stabil} wenn für die Eigenwerte \(\lambda_i \in \C, i=1,\ldots,n\) der Linearisierung \(A \, \coloneqq \, (Df)(x_F)\) gilt
\begin{align*}
\mathcal{Re}(\lambda_i)<0, \quad \text{für } i=1,\ldots,n.
\end{align*}\end{theorem}

\begin{proof}
 Wie bereits in \cref{odestability/ruhelagen:s-linearisierung-ruhelage}  diskutiert können wir durch Translation der Koordinaten des dynamischen Systems annehmen, dass ohne Beschränkung der Allgemeinheit \(x_F = 0 \in U\) gilt.
Da \(U\subseteq\R^n\) nach Vorraussetzung offen ist, können wir eine offene Kugel \(B_{{r^\ast}}(0) \coloneqq \{y \in U \colon ||y|| < {r^\ast}\}\) mit Radius \({r^\ast} > 0\) als Umgebung der Ruhelage \(0\) finden, so dass \(B_{r^\ast}(0) \subset U\) gilt.

\par
Wir nehmen im Folgenden an, dass der Realteil der Eigenwerte \(\lambda_i \in \C, i=1,\ldots,n\) der Linearisierung \(A \, \coloneqq \, Df(0)\) echt negativ ist, d.h., für ein geeignetes \(\Lambda > 0\) gilt die Abschätzung
\begin{align*}
\mathcal{Re}(\lambda_i)< -\Lambda, \quad \text{für } i=1,\ldots,n.\end{align*}
\par
Dann gibt es analog zum Beweis von \cref{odestability/ruhelagen:thm:stablin} eine Konstante \(c>0\), so dass gilt
\begin{align}\label{equation:odestability/ruhelagen:eq:normexp}
\|e^{At}\| \leq c\cdot e^{-\Lambda t}\quad \forall t\in \R^+_0.
\end{align}
\par
Hierbei haben wir ausgenutzt, dass wir die Konstante \(\epsilon > 0\) in \eqref{equation:odestability/ruhelagen:eq:abschaetzungew} so klein wählen können, dass \(\gamma + \epsilon < -\Lambda\) gilt.

\par
Wir können nun einen Radius \(r\in (0,{r^\ast})\) bestimmen, so dass die folgende Abschätzung gilt
\begin{align}\label{equation:odestability/ruhelagen:eq:residuum}
\|R(x)\| \leq \frac{\Lambda}{2c} \|x\|, \quad \forall \|x\| \leq r.
\end{align}
\par
Dies liegt an der totalen Differenzierbarkeit des Vektorfelds \(F\) in der Ruhelage (vgl. Kapitel 6.2 in \cite{Ten21}), denn dies bedeutet, dass das Residuum in der Nähe der Ruhelage schnell genug gegen Null konvergiert, so dass gilt
\begin{align*}
\lim_{x\to 0} \frac{\|R(x)\|}{\|x\|} = \lim_{x\to 0}\frac{\|F(x)- (DF)(0)\cdot x\|}{\|x\|} = 0.
\end{align*}
\par
Wir wollen im Folgenden zeigen, dass wenn der Anfangswert unserer unbekannten Lösung des Differentialgleichungssystems beschränkt ist durch
\begin{align*}
\|x(0)\| \leq \epsilon <\frac{r}{c},
\end{align*}
\par
dann soll schon für die Norm der Lösung für beliebiges \(t \geq 0\) gelten
\begin{align*}
\|x(t)\| \leq c\epsilon e^{-\frac{\Lambda t}{2}}.
\end{align*}
\par
Da \(c\epsilon e^{- \frac{\Lambda t}{2}} \leq c\epsilon < r <\tilde{r}\) gilt, liegt die Lösung somit noch in der offenen Kugel \(B_{{r^\ast}}(0) \subset U\) und konvergiert für \(t \rightarrow \infty\) gegen 0, was den Satz beweist.

\par
Nehmen wir also an, dass \(\|x(0)\| \leq \epsilon <\frac{r}{c}\) gelte.
Nun können wir nach \cref{odestability/ruhelagen:lem:intexpglgn} die unbekannte Lösung durch ihre Linearisierung darstellen als
\begin{align*}
x(t) = e^{At}x_0 + \int_0^t e^{A(t-s)} R(x(s))\, \mathrm{d}s.
\end{align*}
\par
Nehmen wir also die Norm der unbekannten Lösung in dieser Darstellung und nutzen die Abschätzungen \eqref{equation:odestability/ruhelagen:eq:normexp} und \eqref{equation:odestability/ruhelagen:eq:residuum}, so erhalten wir
\begin{align*}
\|x(t)\|\leq ce^{-\Lambda t}\|x_0\| + \int_0^tce^{-\Lambda (t-s)}\frac{\Lambda}{2c}\|x(s)\|\, \mathrm{d}s, \quad \forall \|x\| \leq r.
\end{align*}
\par
Multiplizieren wir beide Seiten der Ungleichung mit \(e^{\Lambda t}\) und definieren uns eine Hilfsfunktion \(f(t):=e^{\Lambda t}\|x(t)\|\), dann erhalten wir
\begin{align*}
f(t)\leq \underbrace{c\|x_0\|}_{=:a} + \int_0^t \underbrace{\frac{\Lambda}{2}}_{=:g(s)} f(s)\, \mathrm{d}s.
\end{align*}
\par
Für diese Form der Ungleichung bietet es sich an das \cref{odestability/ruhelagen:lemma:Gronwall} zur Gronwall Ungleichung anzuwenden, durch das wir schließlich folgendes Resultat bekommen
\begin{align*}
f(t) \leq c \|x_0\| \exp{\left( \frac{1}{2} \int_0^t \Lambda \, \mathrm{d}s \right) }
\leq c \epsilon e^{\frac{\Lambda}{2} t} \leq r e^{\frac{\Lambda}{2} t}.
\end{align*}
\par
Durch Multiplikation beider Seiten mit \(e^{-\Lambda t}\) führt dies zur finalen Abschätzung
\begin{align*}
 \|x(t)\|\leq re^{-\frac{\Lambda}{2}t}, \quad \forall t\in\R^+_0.
\end{align*}
\par
Wir sehen also ein, dass die unbekannte Lösung für alle nicht negativen Zeiten in der offenen Kugel \(B_r(0) \subset B_{{r^\ast}}(0) \subset U\) enthalten ist und offensichtlich gegen Null konvergiert.
Damit ist die Ruhelage \(0 \in U\) asymptotisch stabil.
\end{proof}

\par
Folgende Bemerkung geht speziell auf ein Detail des Beweises ein, das eine Aussage zum Konvergenzradius der Lösungen eines dynamisches Systems zulässt.
\begin{remark}{(Attraktionsbassin)}{odestability/ruhelagen:remark-6}



\par
Der Beweis von \cref{odestability/ruhelagen:thm:stabasymallg} liefert zusätzlich die Aussage, dass alle Punkte \(x\in U\) im Phasenraum mit
\begin{align*}
\|x\| < \frac{r}{c}
\end{align*}
\par
zu Orbits gehören, die gegen die Ruhelage \(0 \in U\) konvergieren.
Diesen attraktiven Einzugsbereich der Ruhelage nennt man auch das \textbf{Attraktionsbassin} der Ruhelage.
\end{remark}


\subsection{Lyapunov Stabilität von Ruhelagen}
\label{\detokenize{odestability/ruhelagen:lyapunov-stabilitat-von-ruhelagen}}
\par
Während ein hinreichendes Kriterium für das Vorliegen \emph{asymptotischer Stabilität} die strikte Ungleichung \(Re(\lambda_i)<0\) für die Eigenwerte \(\lambda_i\) der Jacobi Matrix war, ist die Situation bezüglich der Lyapunov Stabilität einer Ruhelage \textbf{komplizierter}.
Hierzu wollen wir ein Resultat für den Fall von linearen dynamischen Systemen im Folgenden formulieren.
\begin{theorem}{(Lyapunov Stabilität von Ruhelagen)}{odestability/ruhelagen:thm:stablyaplinear}



\par
Sei \(A\in \R^{n\times n}\) eine Matrix mit den Eigenwerten \(\lambda_1,\dots, \lambda_n\in \C\).
Besitzen die Eigenwerte \(\lambda_i \in \C, i=1,\ldots,n\) von \(A\) einen nicht positiven Realteil \(Re(\lambda_i) \leq 0\), und ist im Fall \(Re(\lambda_i)=0\) die geometrische Vielfachheit gleich der algebraischen Vielfachheit des Eigenwerts, dann ist \(0\in \R^n\) eine \textbf{Lyapunov stabile} Ruhelage des dynamischen Systems, dass durch das lineare Differentialgleichungssystem
\begin{align*}
\dot{x}(t) = Ax(t), \quad  \forall t \in I \subset \R^+_0
\end{align*}
\par
charakterisiert wird.
\end{theorem}

\begin{proof}
 Aus \cref{odestability/ruhelagen:thm:stablin} wissen wir bereits, dass im Fall eines linearen dynamischen Systems \(\vec{0} \in U\) eine Ruhelage im Phasenraum \(U \subset \R^n\) ist.
Seien \(\lambda_1, \ldots, \lambda_k \in \C\) für \(k \leq n\) die paarweise verschiedenen Eigenwerte der Matrix \(A\).
Wir betrachten wieder die Jordansche Normalform \(J = S^{-1}AS\) der Matrix \(A\) für Transformationsmatrizen \(S,S^{-1} \in \C^{n \times n}\) und
\begin{align*}
J=
\begin{pmatrix}
J_{r_1}(\lambda_1)& & & 0\\
 & J_{r_2}(\lambda_2) & & \\
 & & \ddots & \\
 0 & & & J_{r_k}(\lambda_k)
\end{pmatrix}.
\end{align*}
\par
Hierbei bezeichnen \(r_i \in \N, i=1,\ldots, k\) die algebraischen Vielfachheiten der zugehörigen Eigenwerte und jeder Jordanblock (vgl. Kapitel 2.7 in \cite{Ten21})) hat die Gestalt
\begin{align*}
 J_r(\lambda) \ \coloneqq \ \begin{pmatrix}
\lambda & 1 & & 0\\
 & \ddots & \ddots & \\
 & & \ddots & 1\\
 0 & & & \lambda
 \end{pmatrix} \in \C^{r\times r}
\end{align*}
\par
Mit den Rechenregeln für das Matrixexponential aus \cref{ode/repetition:rem:matrixexponentialregeln} folgt
\begin{align*}
e^{Jt} = \begin{pmatrix}
\exp{(J_{r_1}(\lambda_1)t)} & & 0\\
 & \ddots & \\
 0& & \exp{(J_{r_k}(\lambda_k)t)}
 \end{pmatrix}.
\end{align*}
\par
Betrachten wir nun die Norm der Lösungen des homogenen, linearen Differentialgleichungssystems für einen Startwert \(x_0 \in U\) mit
\begin{align*}
\| \Phi_t(x_0) \| = \|e^{At}x_0\| = \|S^{-1}e^{Jt}S x_0\| \leq \|S^{-1}\| \|e^{Jt}\| \|S\| \|x_0\|,
\end{align*}
\par
so sehen wir ein, dass die Ruhelage \(\vec{0} \in U\) \textbf{Lyapunov stabil} ist wenn für alle Jordanblöcke \(J_{r_i}(\lambda_i), i=1,\ldots,k\) von \(J\) der Ursprung \(0\in \C^{r_i}\) eine Lyapunov stabile Ruhelage des folgenden linearen Differentialgleichungssystems ist
\begin{align*}
 \dot{y}(t) = J_{r_i}(\lambda_i) y(t), \quad t \in I \subset \R^+_0.
\end{align*}
\par
Dies ist bereits gegeben falls für einen Eigenwert \(Re(\lambda_i)<0\) gilt, denn damit folgt aus \cref{odestability/ruhelagen:thm:stablin} sogar schon \textbf{asymptotische Stabilität}, welche Lyapunov Stabilität induziert.

\par
Betrachten wir also nun einen komplexen Eigenwert \(\lambda_i \in \C\) von \(A\) mit \(Re(\lambda_i)=0\) und für den die geometrische Vielfachheit nach Vorraussetzung gleich der algebraischen Vielfachheit ist.
In diesem Fall ist der ihm zugeordnete Jordanblock eine Diagonalmatrix auf deren Hauptdiagonale der Eigenwert \(\lambda_i \in \C\) steht, da alle Jordankästchen eindimensional sind.
In diesem Fall sehen wir, dass die Norm des Matrixexponentials beschränkt ist und wir dadurch \textbf{Lyapunov Stabilität} der Ruhelage gezeigt haben, da gilt
\begin{align*}
\|e^{J_{r_i}(\lambda_i)t)}\| = |e^{\lambda_i t}| = |e^0e^{\mathcal{Im}(\lambda_i) t}| = |\cos{(\mathcal{Im}(\lambda_i)t)} + i \sin{(\mathcal{Im}(\lambda_i)t)}| = 1.
\end{align*}
\par
Für diese Umformung haben wir die Definition der komplexen Exponentialfunktion genutzt, für die gilt:
\begin{align*}
e^z = e^{x+iy} = e^xe^iy = e^x(\cos(y) + i\sin(y)), \quad \text{für } z = x+iy \in \C.
\end{align*}\end{proof}

\par
Das folgende Beispiel illustriert, dass eine Ruhelage instabil werden kann, wenn die geometrische Vielfachheit nicht mit der algebraischen Vielfachheit übereinstimmt für einen Eigenwert \(\lambda =0\) der Koeffizientenmatrix \(A\).
\begin{example}{}{odestability/ruhelagen:example-8}



\par
Sei \(U \subset \R^2\) der Phasenraum und wir betrachten das homogene, lineare Differentialgleichungssystem
\begin{align*}
\dot{x}(t) = A x(t), \quad \forall t \in \R_0^+
\end{align*}
\par
für eine Koeffizientenmatrix
\begin{align*}
A = \begin{pmatrix} 0&1\\0&0\end{pmatrix}.
\end{align*}
\par
Wie man leicht nachrechnet besitzt diese Matrix den Eigenwert \(\lambda = 0\) mit algebraischer Vielfachheit \(2\) und geometrischer Vielfachheit \(1\) zum Eigenvektor \(v = (1,0)^T \in \R^2\).
Die Vielfachheiten des Eigenwert \textbf{stimmen} also \textbf{nicht überein}.

\par
Aus \cref{odestability/ruhelagen:thm:stablin} wissen wir, dass eine Ruhelage in \(\vec{0} \in \R^2\) existiert.
Man sieht jedoch leicht ein, dass sogar jeder Punkt \(x_0 = (y, 0) \in U\) eine Ruhelage des Systems darstellt, da diese Punkte ein Vielfaches des Eigenvektors zum Eigenwert \(\lambda = 0\) darstellen und somit im Kern der Matrix \(A\) liegen, d.h., für diese Punkte ist die rechte Seite des Differentialgleichungssystems \(\vec{0} \in \R^2\) und somit liegt eine Ruhelage vor.

\par
Wir wollen die Stabilität dieser Ruhelagen im Folgenden untersuchen.
Hierzu betrachten wir die Norm des Phasenflusses \(\Phi \colon I \times U \rightarrow U\), der für einen gegebenen Anfangswert \(x_0 = (y,z) \in U\) mit \(z \neq 0\) der die Lösung des Differentialgleichungssystems beschreibt mit
\begin{align*}
\| \Phi_t(x_0) \| &= \| e^{At}x_0 \| = \| \sum_{k=0}^\infty \frac{(At)^k}{k!} x_0\| = \| [\underbrace{(At)^0}_{=I_2} + (At)^1] x_0\| \\
&= \| \begin{pmatrix} 1 & t \\ 0 & 1\end{pmatrix}\begin{pmatrix} y \\ z \end{pmatrix} \| = \| \begin{pmatrix} y + tz \\ z\end{pmatrix} \| \overset{t\to \infty}{\longrightarrow} \infty.
\end{align*}
\par
Wir sehen also, dass für jeden Anfangswert \(x_0 = (y,z)\) mit \(z \neq 0\) die Lösung des Differentialgleichungssystems divergiert und somit ist jede Ruhelage des dynamischen Systems \textbf{instabil}.
\end{example}

\par
Leider kann man nicht wie im Fall der asymptotischen Stabilität vom linearen auf den nichtlinearen Fall schließen, wie das folgende Beispiel zeigt.
\begin{example}{}{odestability/ruhelagen:example-9}



\par
Wir betrachten eine gewöhnliche Differentialgleichung 1. Ordnung der Form
\begin{align*}
\dot{x}(t) = \alpha x(t) + \beta x^3(t), \forall t \in \R^+_0.
\end{align*}
\par
mit freien Parametern \(\alpha, \beta \in \R\).

\par
Wie man einsieht ist \(0\) eine Ruhelage des dynamischen Systems, das durch diese Differentialgleichung charakterisiert wird.
Wir betrachten die Linearisierung der Differentialgleichung in der Ruhelage mit \(A := (DF)(0) = \alpha\) und erhalten
\begin{align*}
\dot{x}(t) = A x(t) = \alpha x(t), \quad \forall t \in \R^+_0.
\end{align*}
\par
Folgende Fallunterscheidung zeigt nun das Stabilitätsverhalten der Ruhelage in Abhängigkeit der gewählten Parameter \(\alpha, \beta \in \R\):


\begin{center}
\centering
\begin{tabularx}{\linewidth}[{\linewidth}]{|c|c|c|}\hline

\par

& 
\par
linearisierte Gleichung
& 
\par
nicht lineare Gleichung
\\
\hline
\par
\(\alpha<0\)
&
\par
asymptotisch stabil
&
\par
asymptotisch stabil
\\
\hline
\par
\(\alpha>0\)
&
\par
instabil
&
\par
instabil
\\
\hline
\par
\(\alpha=0\)
&
\par
Lyapunov stabil
&
\par
asymptotisch stabil für \(\beta<0\)
\\
\hline
\par

&
\par

&
\par
stabil für \(\beta =0 \)
\\
\hline
\par

&
\par

&
\par
instabil \(\beta > 0\)
\\
\hline
\end{tabularx}
\end{center}

\par
Wie man sieht hängt die Stabilität im nichtlinearen Fall nicht nur vom Parameter \(\alpha\), sondern ebenfalls von \(\beta\) ab, was eine Stabilitätsanalyse deutlich komplizierter macht.
\end{example}


