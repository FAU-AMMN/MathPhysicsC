\chapter{Maß  und Integrationstheorie}
\label{\detokenize{masstheorie/intro_masstheorie:masz-und-integrationstheorie}}\label{\detokenize{masstheorie/intro_masstheorie::doc}}
\par
Im ersten Semester haben wir bereits das \emph{Riemann Integral} zur Berechnung des Flächeninhalts zwischen einer Funktion \(f \colon [a,b] \rightarrow \R\) und der x Achse eingeführt (siehe Kapitel 7 in \cite{Bur20}).
Die grundlegende Idee hierbei war es die x Achse in (unterschiedlich große) Intervalle zu unterteilen und mittels dieser Zerlegung Rechtecke zwischen dem Graphen der Funktion und der x Achse zu konstruieren.
Durch das Produkt der Seitenlängen dieser Rechtecke lässt sich nämlich deren Flächeninhalt leicht berechnen und die Summe all dieser Flächeninhalte approximiert den wahren Flächeninhalt zwischen dem Graphen und \(f\) und der x Achse.
Dieses Vorgehen ist zur Erinnerung nochmal in Abbildung \hyperref[\detokenize{masstheorie/intro_masstheorie:fig-riemann-integral}]{Fig.\@ \ref{\detokenize{masstheorie/intro_masstheorie:fig-riemann-integral}}} illustriert.

\begin{figure}[htbp]
\centering


\noindent\includegraphics[width=\textwidth]{../\string_build/html/\string_images/ober\string_untersummen.png}
\caption{Illustration zweier Approximationen des Riemann Integrals einer Funktion durch den Flächeninhalt von Rechtecken. Die grünen und lila Rechtecke visualisieren die Unter  bzw. Obersummen bezüglich des in rot dargestellten Graphen der Funktion. Quelle: \href{https://de.wikipedia.org/wiki/Riemannsches\_Integral}{Wikipedia; Riemannsches Integral}.}\label{\detokenize{masstheorie/intro_masstheorie:fig-riemann-integral}}\end{figure}

\par
Da wir bisher nur die Integration von eindimensionalen Funktionen auf kompakten Intervallen kennengelernt haben, wollen wir in diesem Kapitel den Begriff des Integals auf mehrdimensionale Funktionen \(f \colon \R^n \rightarrow \R\) erweitern.
Das Riemann Integral lässt sich problemlos auf mehrdimensionalen Funktionen verallgemeinern, indem man das Integral nicht durch den Flächeninhalt von Rechtecken approximiert, sondern hierfür das \emph{Volumen von entsprechenden Quadern} \(Q \subset \R^n\) berechnet.
Es lässt sich insbesondere zeigen, dass jede stetige mehrdimensionale Funktion auf einem (nicht ausgeartetem) Quader Riemann integrierbar ist.

\par
Es hat sich jedoch in der Entwicklung der Mathematik herausgestellt, dass der Begriff des Riemann Integrals zu starke Forderungen an die zu Grunde liegenden Funktionen stellt und damit wichtige und interessante Funktionsklassen nicht integrierbar waren.
Als Beispiel hierfür seien \href{https://de.wikipedia.org/wiki/Fraktal}{fraktale Mengen und Funktionen} genannt, welche auch zur Modellierung von Prozessen in der Natur genutzt werden.

\par
Aus diesem Grund hat sich ein eigenes Feld innerhalb der modernen Analysis gebildet   die sogenannte \textbf{Maßtheorie}.
Es widmet sich hauptsächlich der Untersuchung von Maßen zur Berechnung von Längen, Flächen und Volumina in unterschiedlichen mathematischen Strukturen und Räumen.
Es wird klar, dass Integration und die Berechnung von Volumina eng miteinander zusammen hängen, denn ist \(A\subset\R^n\) eine (messbare) Teilmenge, dann ist ihr Maß gleich dem Integral ihrer charakteristischen Funktion, d.h.,
\begin{align*}
\int_{\R^n} \mathbb{1}_A(x)\,\mathrm{d}x.
\end{align*}
\par
In einem gewissen Sinn sind Maße ein \emph{fundamentaleres Konzept} als das der Integration, da sich jede Integration auf die Berechnung von Maßen stützt.

\par
Eins der berühmtesten Beispiele zur Motivation der Maßtheorie ist im Folgenden erklärt.
\begin{example}{(Dirichlet Funktion)}{masstheorie/intro_masstheorie:ex:dirichletFunktion}



\par
Wir betrachten das kompakte Intervall \([0,1] \subset \R\) und definieren hierauf die sogenannte \textbf{Dirichlet Funktion} \(\mathbb{1}_\Q \colon [0,1] \rightarrow \{0,1\}\) mit
\begin{align*}
\mathbb{1}_\Q(x) := \begin{cases} 1, \ \text{ falls } x \in \Q, \\ 0, \ \text{ sonst }.\end{cases}
\end{align*}
\par
Diese Abbildung kann als \emph{charakteristische Funktion} der rationalen Zahlen \(\Q\) aufgefasst werden.
Man sieht leicht ein, dass diese Funktion \textbf{nicht Riemann integrierbar} ist, da alle Untersummen stets \(0\) und alle Obersummen stets \(1\) sind.
\end{example}

\par
Um eine Funktion wie die Dirichlet Funktion in \cref{masstheorie/intro_masstheorie:ex:dirichletFunktion} zu integrieren sieht man zunächst ein, dass die rationalen Zahlen \(\Q\) in den reellen Zahlen \(\R\) als abzählbare Menge eine sogenannte \emph{Nullmenge} im Sinne der Maßtheorie repräsentieren.
Wir werden im Laufe der Vorlesung die nötigen Werkzeuge der Maßtheorie einführen um diesen Umstand zu verstehen und einen verallgemeinerten Begriff des Integrals definieren, der diese Nullmenge berücksichtigt   das \textbf{Lebesgue Integral}.
Dieses Integral ist eine echte Verallgemeinerung des Riemann Integrals und wir werden einsehen, dass die Menge der Lebesgue integrierbaren Funktionen eine Obermenge der Menge der Riemann integrierbaren Funktionen ist.
Außerdem verhält sich das Lebesgue Integral bei Grenzwertbildungen einfacher als das Riemann Integral.


\section{Maßtheorie}
\label{\detokenize{masstheorie/masstheorie:masztheorie}}\label{\detokenize{masstheorie/masstheorie::doc}}
\par
Ein \href{https://de.wikipedia.org/wiki/Ma\%c3\%9f\_(Mathematik)}{\textbf{Maß}} \(\mu\) auf einer Menge \(M\) wie z.B. dem \(\R^n\)
ordnet geeigneten Teilmengen \(A\subseteq M\)
Zahlen
\begin{align*}
\mu(A)\in[0,\infty]:=[0,\infty)\cup\{\infty\}
\end{align*}
\par
zu, eben das Maß von \(A\).


\subsection{\protect\(\sigma\protect\) Algebren und Maße}
\label{\detokenize{masstheorie/masstheorie:sigma-algebren-und-masze}}
\par
Ein Mengensystem ist eine Menge von Mengen. Wir nennen die Potenzmengen \(2^M \equiv\mathcal{P}(M)\) von \(M\) die Menge aller möglichen Teilmengen von \(M\). In der Maßtheorie sind Mengensysteme \(\mathcal{A} \subseteq \mathcal{P}(M)\) zentral, nämlich die der \emph{meßbaren} Teilmengen von \(M\). Das sind die Mengen, denen ein Maß zugeordnet wird, also eine nicht negative Zahl oder Unendlich. Schauen wir uns also an, welche Eigenschaften ein Maß haben muss.

\par
Zunächst müssen wir festlegen, welche Teilmengen der Grundmenge \(M\) überhaupt messbar sein sollen. Wir wählen also wie gesagt ein Mengensystem \(\mathcal{A} \subseteq \mathcal{P}(M)\) in der Potenzmenge von \(M\) aus.
Am bequemsten wäre es, alles messen zu können, also \(\mathcal{A} = \mathcal{P}(M)\), aber das ist nicht immer möglich. Wir fordern, dass \(\mathcal{A}\) eine \(\sigma\) Algebra ist.
\begin{definition}{}{masstheorie/masstheorie:def:sigmaalgebra}


\begin{itemize}
\item {} 
\par
\(\mathcal{A} \subseteq \mathcal{P}(M)\) heißt \textbf{\(\sigma\) Algebra (von \(M\))}, wenn

\par
a) \(M\in \mathcal{A}\)

\par
b) \(A\in \mathcal{A}\) impliziert, dass \(A^c:=M\backslash A\in \mathcal{A}\)

\par
c) \(A_n\in \mathcal{A}\ (n\in\N)\) impliziert, dass \(\bigcup_{n\in \N} A_n\in \mathcal{A}\).

\item {} 
\par
Für eine \(\sigma\) Algebra \(\mathcal{A} \subseteq \mathcal{P}(M)\) von \(M\) heißt das Paar (\(M,\mathcal{A}\)) \textbf{Messraum}, und die Menge \(\mathcal{A}\) heißen \emph{messbar}.

\end{itemize}
\end{definition}

\par
Die kleinste \(\sigma\) Algebra von \(M\) ist damit \{\textbackslash{}emptyset, M\}, die Größte \(\mathcal{P}(M)\).

\par
Das Symbol \(\sigma\) soll an den Begriff der Summe erinnern, entsprechend der
dritten Forderung in Def. \textbackslash{}ref\{def:sigmaalgebra\}, also der Stabilität unter abzählbarer Vereinigung.
Sigma–Algebren sind offensichtlich auch unter dem Schnitt abzählbar vieler
Mengenstabil \(A_n\in \mathcal{A}\ (n\in\N)\) impliziert, dass \(\bigcap_{n\in \N} A_n = \left(\bigcup_{n\in\N} A_n^c\right)^c\in \mathcal{A}\).
\begin{definition}{}{masstheorie/masstheorie:definition-1}


\begin{itemize}
\item {} 
\par
Für einen Messraum (\(M, \mathcal{A}\)) heißt eine Abbildung \(\mu: \mathcal{A}\to [0, \infty]\) \textbf{Maß}, wenn

\par
a) \(\mu(\emptyset) = 0\)

\par
b) Für disjunkte (das heißt \(A_m\cap A_n = \emptyset\ (m\neq n)\)) \(A_n\in \mathcal{A}\ (n\in\N)\) gilt: \(\mu\left( \bigcup_{n\in\N}A_n \right) = \sum_{n\in\N}\mu (A_n)\) \textbackslash{}quad (abzählbare oder \(\sigma\) \textbf{Additivität}).

\item {} 
\par
Dann heißt das Tripel \((M, \mathcal{A}, \mu)\) \textbf{Maßraum}.

\item {} 
\par
Das Maß \(\mu\) heißt \textbf{endlich}, wenn \(\mu(M)<\infty\), und \textbf{Wahrscheinlichkeitsmaß}, wenn \(\mu(M)=1\).

\end{itemize}
\end{definition}
\begin{example}{}{masstheorie/masstheorie:example-2}



\par
Wichtige Maße sind z.B. die folgenden.
\begin{enumerate}

\item {} 
\par
Das \href{https://de.wikipedia.org/wiki/Z\%c3\%a4hlma\%c3\%9f\_(Ma\%c3\%9ftheorie)}{\textbf{Zählmaß}} \(m\) auf einer endlichen Menge \(M\), mit

\end{enumerate}
\begin{align*}
m(A):=|A|\qquad (A\subseteq M).
\end{align*}
\par
Hier sind insbesondere alle Teilmengen messbar.
\begin{enumerate}

\item {} 
\par
Das \textbf{Lebesgue–Maß} \(\lambda^n\) auf dem \(\R^n\), das wir bald kennen lernen, zeichnet sich dadurch aus, dass es einem verschobenen Körper das gleiche Volumen zuordnet wie dem unverschobenen, es also \emph{translationsinvariant} ist, und der Einheitswürfel \([0,1]^n\) Maß \(1\) besitzt.

\item {} 
\par
Das \href{https://de.wikipedia.org/wiki/Diracma\%c3\%9f}{\textbf{Dirac Maß}} \(\delta_x\) ist im Punkt \(x\) des \(\R^n\) konzentriert, und für \(A\subset\R^n\) ist.

\end{enumerate}
\begin{align*}
\delta_x(A)\equiv \int_A\delta_x := \begin{cases} 1, &  x \in A\\ 0, & \text{sonst.} \end{cases}

\end{align*}
\par
Dieses Maß ist also nicht translationsinvariant. Es wird beispielsweise in der Elektrodynamik benutzt, um eine bei \(x\) lokalisierte punktförmige Ladung zu beschreiben.
\begin{enumerate}

\item {} 
\par
Wir wollen z.B. auch die Länge der Spur einer \emph{Kurve} oder allgemeiner den Flächeninhalt einer \(d\)–dimensionalen Fläche im \(\R^n\) messen. Auch das dafür benutzte Maß \(\mu_d\) ist translations  und rotationsinvariant, es ordnet aber der \(d\)–dimensionalen Einheitsfläche \([0,1]^d\times\{0\} \subset\R^d\times\R^{n-d}=\R^n\) Maß 1 zu. Entsprechend hat aber für \(d<n\) der Einheitswürfel Maß \(\mu_d([0,1]^n)=\infty\).

\item {} 
\par
Man kann sogar sog. \href{https://de.wikipedia.org/wiki/Hausdorff-Ma\%c3\%9f}{\textbf{Hausdorff Maß}} \(\mu_d\) konstruieren, die Mengen beliebiger fraktaler Dimension \(d\in[0,n]\) messen. Genau genommen \emph{definiert} man die Dimension der Menge \(A\subset \R^n\) durch \(d(A):=\inf\{d'>0\mid \mu_{d'}(A)=0\}.\)

\item {} 
\par
Im Zusammenhang mit dem sog. Feynmanschen \href{https://de.wikipedia.org/wiki/Pfadintegral}{\textbf{Pfadintegral}} der Quantenmechanik wird auf dem unendlich dimensionalen Raum \(M\) der Wege zwischen zwei Punkten des Konfigurationsraumes \(\R^d\) ein \href{https://de.wikipedia.org/wiki/Wahrscheinlichkeitsma\%c3\%9f}{\textbf{Wahrscheinlichkeitsmaß}} (also ein Maß \(\mu\) auf \(M\) mit \(\mu (M)=1\)) definiert. Dabei erhalten Wege, die in der Nähe von Lösungskurven der DGL der Klassischen Mechanik sind, ein großes Gewicht.

\end{enumerate}
\end{example}


\subsection{Borel \protect\(\sigma\protect\) Algebren und  Maße}
\begin{definition}{(Borel \protect\(\sigma\protect\) Algebra)}{\detokenize{masstheorie/masstheorie:borel-sigma-algebren-und-masze}}\label{masstheorie/masstheorie:definition-3}



\par
S. 106 und 119 Schulz Baldes
\end{definition}
\begin{definition}{(Lokale Endlichkeit von Maßen)}{masstheorie/masstheorie:definition-4}



\par
Sei \(\sigma \colon \B(\Omega) \rightarrow [0, \infty]\) ein Maß auf einem topologischem Raum \(\Omega\).
Wir nennen das Maß \(\sigma\) \textbf{lokal endlich}, falls jeder Punkt \(x \in \Omega\) eine lokale Umgebung mit endlichem Maß besitzt.
\end{definition}

\par
Es ist klar, dass das Lebesgue Maß auf dem Raum \(\R^n\) lokal endlich ist.
\begin{definition}{(Borel Maß)}{masstheorie/masstheorie:definition-5}



\par
Ein lokal endliches Maß \(\sigma \colon \B(\Omega) \rightarrow [0, \infty]\) auf der Borelschen \(\sigma\) Algebra eines topologischen Raums \(\Omega\) heißt Borel Maß.
\end{definition}


\subsection{Riemann  und Lebesgue messbare Mengen}
\begin{definition}{(Mehrdimensionale Quader)}{\detokenize{masstheorie/masstheorie:riemann-und-lebesgue-messbare-mengen}}\label{masstheorie/masstheorie:definition-6}



\par
Seien \(a = (a_1,\ldots,a_n) \in \R^n\) und \(b = (b_1,\ldots,b_n) \in \R^n\) zwei Punkte im \textbackslash{}R\textasciicircum{}n.
Wir definieren zunächst folgende Anordnungsrelation mit
\begin{align*}
a < b \qquad \Leftrightarrow \qquad a_i < b_i \quad i=1,\ldots,n.
\end{align*}
\par
Analog können wir die Anordnungsrelationen \(a \leq b, a > b\) und \(a \geq b\) definieren und darüber im Folgenden \textbf{offene, halboffene} und \textbf{abgeschlossene Quader} im \(\R^n\) respektive beschreiben durch
\begin{align*}
(a,b) = \lbrace x \in \R^n : a < x < b \rbrace,\\
(a,b] = \lbrace x \in \R^n : a < x \leq b \rbrace,\\
[a,b] = \lbrace x \in \R^n : a \leq x \leq b \rbrace,.
\end{align*}\end{definition}
\begin{definition}{(Mehrdimensionale Treppenfunktion)}{masstheorie/masstheorie:definition-7}



\par
Wir nennen eine Funktion \(f \colon \R^n \rightarrow \C\) \textbf{Treppenfunktion}, falls es paarweise disjunkte Quader \(Q_1, \ldots, Q_k \subset \R^n\) gibt, so dass die folgenden Eigenschaften gelten:
\begin{enumerate}

\item {} 
\par
Die Funktion \(f\) ist konstant auf jedem der Quader, d.h.,

\end{enumerate}
\begin{align*}
f|_{Q_i} = c \in \C, \quad 1 \leq i \leq k,
\end{align*}\begin{enumerate}

\item {} 
\par
Die Funktion ist überall Null außerhalb der Quader, d.h.,

\end{enumerate}
\begin{align*}
f|_{\R^n \setminus (Q_1 \cup \ldots \cup Q_k)} = 0.
\end{align*}\end{definition}

\par
Illustration hier von Quadern und Treppenfunktionen!
\begin{definition}{(Ring)}{masstheorie/masstheorie:definition-8}



\par
Ein Mengensystem \(\mathcal{R} \subset \mathcal{P}(\Omega)\) heißt \textbf{Ring} auf einer Menge \(\Omega\), falls die folgenden Eigenschaften erfüllt sind:
\begin{enumerate}

\item {} 
\par
\(\emptyset \in \mathcal{R}\)

\item {} 
\par
\$A,B \textbackslash{}in \textbackslash{}mathcal\{R\} \textbackslash{}Rightarrow (A \textbackslash{}setminus B) \textbackslash{}in \textbackslash{}mathcal\{R\}

\item {} 
\par
\$A,B \textbackslash{}in \textbackslash{}mathcal\{R\} \textbackslash{}Rightarrow (A \textbackslash{}cup B) \textbackslash{}in \textbackslash{}mathcal\{R\}

\end{enumerate}
\end{definition}
\begin{lemma}{(Der von halboffenen Quadern erzeugte Ring)}{masstheorie/masstheorie:lemma-9}



\par
S. 79 Schulz Baldes
\end{lemma}
\begin{remark}{(Riemann messbare Mengen)}{masstheorie/masstheorie:remark-10}



\par
S. 78 Schulz Baldes
\end{remark}
\begin{example}{(Riemann Messbarkeit)}{masstheorie/masstheorie:example-11}



\par
Ein Beispiel messbar, eins nicht.
\end{example}

\par
Verglichen mit der Reichhaltigkeit der Potenzmenge \(\mathcal{P}(\Omega)\) sind nur relativ wenige Mengen Riemann messbar.
\begin{definition}{(Äußeres Lebesguesches Maß)}{masstheorie/masstheorie:definition-12}



\par
S. 78 u. 89 Schulz Baldes
\end{definition}
\begin{theorem}{(Eigenschaften des Lebesgue Maßes)}{masstheorie/masstheorie:theorem-13}



\par
S. 89 Schulz Baldes
\end{theorem}
\begin{definition}{}{masstheorie/masstheorie:definition-14}



\par
S. 92 Schulz Baldes
\end{definition}
\begin{example}{(Lebesgue messbare Mengen)}{masstheorie/masstheorie:example-15}



\par
S. 92 Schulz Baldes
\end{example}
\begin{definition}{}{masstheorie/masstheorie:definition-16}



\par
S. 98 Schulz Baldes
\end{definition}
\begin{lemma}{(Eigenschaften von Lebesgue Nullmengen)}{masstheorie/masstheorie:lemma-17}



\par
S. 98 Schulz Baldes
\end{lemma}
\begin{theorem}{(Lebesgue Messbarkeit von Teilmengen im \protect\(\R^n\protect\))}{masstheorie/masstheorie:theorem-18}



\par
Sowohl offene als auch abgeschlossene Teilmengen des \(\R^n\) sind Lebesgue messbar.
\end{theorem}

\begin{proof}
 S. 99 Schulz Baldes
\end{proof}


\section{Lebesgue Integral}
\label{\detokenize{masstheorie/lebesgue_integral:lebesgue-integral}}\label{\detokenize{masstheorie/lebesgue_integral::doc}}
\par
S.121ff. Schulz Baldes


\section{Integrationstechniken}
\label{\detokenize{masstheorie/integrationstechnik:integrationstechniken}}\label{\detokenize{masstheorie/integrationstechnik::doc}}

\subsection{Satz von Fubini}
\label{\detokenize{masstheorie/integrationstechnik:satz-von-fubini}}

\subsection{Satz von Cavalieri}
\label{\detokenize{masstheorie/integrationstechnik:satz-von-cavalieri}}

