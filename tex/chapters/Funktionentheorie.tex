\chapter{Funktionentheorie}
\label{\detokenize{complexanalysis/complexanalysis:funktionentheorie}}\label{\detokenize{complexanalysis/complexanalysis::doc}}
\par
Im letzten Kapitel der Vorlesung widmen wir uns der \emph{Funktionentheorie}.
Diese befasst sich hauptsächlich mit der Theorie differenzierbarer komplexer Funktionen.
Da viele Konzepte der reellen Analysis verwendet werden, wird dieses Gebiet auch häufig \textbf{komplexe Analysis} genannt.

\par
Die Grundlagen des Körpers der komplexen Zahlen wurden bereits in {[}{]} behandelt.


\section{Cauchy Riemann Gleichungen}
\label{\detokenize{complexanalysis/cauchyriemann:cauchy-riemann-gleichungen}}\label{\detokenize{complexanalysis/cauchyriemann::doc}}
\par
Der zentrale Begriff der Funktionentheorie ist der einer \emph{holomorphen Funktion}.
\begin{definition}{(Holomorphe Funktion)}{complexanalysis/cauchyriemann:def:holomorph}



\par
Sei \(D \subset \C\) eine offene Teilmenge und \(f \colon D \rightarrow \C\) eine stetige Funktion.

\par
Wir nennen die Funktion \(f\) \textbf{holomorph} auf der Teilmenge \(D\), wenn es für jeden Punkt \(z \in D\) eine komplexe Ableitung der Funktion \(f\) gibt mit
\begin{align*}
f'(z) = \lim_{h\rightarrow 0} \frac{f(z+h) - f(z)}{h}.
\end{align*}\end{definition}

\par
Der Differentialquotient in \cref{complexanalysis/cauchyriemann:def:holomorph} erinnert sehr an die Definition der Ableitung einer reellen Funktion.
Der grundlegende Unterschied ist hier jedoch, dass \(h \in \C\) komplex ist.
Interpretiert man den Körper der komplexen Zahlen als zweidimensionalen reellen Vektorraum, so muss für die beiden Richtungsableitungen \(x := \operatorname{Re}(z)\) und \(y := \operatorname{Im}(z)\) gelten
\begin{align*}
\partial_x f(z) = \lim_{\epsilon \rightarrow 0} \frac{f(z+\epsilon) - f(z)}{\epsilon} = \lim_{\epsilon \rightarrow 0} \frac{f(z+i\epsilon) - f(z)}{i\epsilon} = -i \partial_y f(z), \quad \epsilon \in \R.
\end{align*}
\par
Dieser Zusammenhang ist charakteristisch für holomorphe Funktionen und wird im folgenden Satz präzisiert.
\begin{theorem}{(Cauchy Riemann Gleichungen)}{complexanalysis/cauchyriemann:theorem-1}



\par
Sei \(D \subset \C\) eine offene Teilmenge und \(f \colon D \rightarrow \C\) eine stetige Funktion für die gilt
\begin{align*}
f(z) = u(z) + i v(z), \qquad u,v \colon D \rightarrow \R.
\end{align*}
\par
Dann ist die Funktion \(f\) genau dann holomorph auf der Teilmenge \(D\), wenn die folgenden \textbf{Cauchy Riemann Gleichungen} auf ganz \(D\) gelten:
\begin{align*}
\partial_x u = \partial_y v, \qquad \partial_y u = -\partial_x v.
\end{align*}\end{theorem}

\begin{proof}
 Schulz Baldes S.313
\end{proof}
\begin{example}{(Holomorphe Funktionen)}{complexanalysis/cauchyriemann:example-2}



\par
Ableitung eines komplexen Monoms  > Beispiel 10.4 auf S.314 in Schulz Baldes

\par
\(f(z) := \overline{z}\) ist nicht holomorph  > Beispiel 10.7 auf S.315 in Schulz Baldes
\end{example}

\par
Eine besondere Klasse von Funktionen sind \emph{analytische Funktonen}, die sich lokal mit Hilfe von Reihen darstellen lassen.
\begin{definition}{(Analytische Funktion)}{complexanalysis/cauchyriemann:definition-3}



\par
Sei \(D \subset \C\) eine Teilmenge und \(f \colon D \rightarrow \C\) eine Funktion.

\par
Wir nennen die Funktion \(f\) \textbf{analytisch} in einem Punkt \(z_0 \in D\) genau dann, wenn ein \(\epsilon > 0\) existiert, so dass sich jeder Funktionswert \(f(z) \in \C\) in einer entsprechenden lokalen Umgebung durch eine absolut konvergente Reihe darstellen lässt mit
\begin{align*}
f(z) = \sum_{n \geq 0} a_n (z-z_0)^n, \qquad \forall |z - z_0| < \epsilon,
\end{align*}
\par
wobei \((a_n)_{n_\in\N}\) eine Folge in \(\C\) ist.

\par
Wir nennen die Funktion \(f\) analytisch auf der Teilmenge \(D\), wenn sie analytisch ist für alle Punkte \(z_0 \in D\).
\end{definition}

\par
Der folgende Satz beschreibt den Zusammenhang zwischen analytischen und holomorphen Funktionen.
\begin{theorem}{}{complexanalysis/cauchyriemann:thm:analytischHolomorph}



\par
Jede analytische Funktion \(f\) auf einer Teilmenge \(D \subset \C\) ist auch holomorph auf \(D\).
\end{theorem}

\begin{proof}
 Vertauschung des Limes \(h \rightarrow 0\) und der Summe.
\end{proof}

\par
Wie wir später im Hauptsatz der Funktionentheorie sehen werden gilt auch die Umkehrung.


\section{Kurvenintegrale}
\label{\detokenize{complexanalysis/kurvenintegrale:kurvenintegrale}}\label{\detokenize{complexanalysis/kurvenintegrale::doc}}

\subsection{Wege und Kurven}
\label{\detokenize{complexanalysis/kurvenintegrale:wege-und-kurven}}
\par
Wir beginnen diesen Abschnitt mit der grundlegenden Definition von Wegen und Kurven.
\begin{definition}{(Weg und Kurve)}{complexanalysis/kurvenintegrale:definition-0}



\par
Sei \((X,\tau)\) ein topologischer Raum und \(I = [a,b]\) ein reelles Intervall für \(a,b \in \R\).
Wir nennen eine stetige Funktion \(f \colon I \rightarrow X\) einen \textbf{Weg} in \(X\).
Wir bezeichnen einen Weg \(f\) als \textbf{glatt}, wenn er stetig differenzierbar ist und seine Ableitung für jeden Punkt \(x \in I\) ungleich Null ist.
Außerdem bezeichnen wir einen Weg \(f\) als \textbf{geschlossen}, wenn gilt \(f(a) = f(b)\).
Schließlich nennen wir einen Weg \(f\) \textbf{konstant}, wenn er für alle \(t \in [a,b]\) auf den gleichen Punkt in \(\C\) abbildet.

\par
Die Bildmenge \(f(I) \subset X\) nennen wir \textbf{Kurve} in \(X\).
\end{definition}

\begin{emphBox}{}{}
\par
Normalerweise nennen wir eine Abbildung \emph{glatt}, wenn sie unendlich oft differenzierbar ist.
Dies wird im Kontext von Wegen nicht gefordert und es genügt eine einfache stetige Differenzierbarkeit.
\end{emphBox}

\par
Wir wollen im Folgenden annehmen, dass \(f\) eine holomorphe Funktion auf einer offenen Teilmenge \(D \subset \C\) ist.
Wir betrachten \(f(z) \mathrm{d}z\) als die zugehörige \(1\) Form und außerdem sei \(\gamma \colon I \rightarrow D\) für \(I := [0,1]\) ein glatter Weg.
Wegen der Jacobischen Transformationsformel in \cref{masstheorie/integrationstechnik:thm:jacobitransformation} gilt dann für das Kurvenintegral von \(f\) bezüglich \(\gamma\)
\begin{align}\label{equation:complexanalysis/kurvenintegrale:eq:kurvenintegral}
\int_{\gamma} f(z) \, \mathrm{d}z = \int_I \gamma^*(f(z) \, \mathrm{d}z) = \int_0^1 f(\gamma(t)) \gamma'(t) \, \mathrm{d}t.
\end{align}
\par
Häufig ist die Annahme eines global glatten Weges \(\gamma\) eine zu starke Forderung.
Es genügt auch zu fordern, dass der Weg \(\gamma\) stückweise glatt ist und aus endlich vielen Teilstücken besteht.
In diesem Fall lässt sich das Integral \cref{complexanalysis/kurvenintegrale:equation-eq-kurvenintegral} als Summe der Integrale über die Teilstücke schreiben.

\par
\textbf{ToDo: Abbildung von stückweise glatten Wegen / Kurven}

\par
Wir definieren als Nächstes eine charakteristische Größe von geschlossenen Wegen in \(\C\), den sogenannten Index.
\begin{definition}{(Index)}{complexanalysis/kurvenintegrale:definition-1}



\par
Sei \(\gamma \colon [0,1] \rightarrow \C\) ein geschlossener Weg in \(\C\) und \(w \in \C \setminus \operatorname{Bild}(\gamma)\) ein beliebiger Punkt außerhalb der zugehörigen Kurve von \(\gamma\).
Wir bezeichnen als \textbf{Index} von \(w\) bezüglich des Wegs \(\gamma\) folgende charakteristische Größe
\begin{align*}
\operatorname{Ind}_\gamma(w) \ := \ \oint_\gamma \frac{1}{z - w} \frac{\mathrm{d}z}{2\pi i} \ = \ \int_0^1 \frac{\gamma'(t)}{\gamma(t - w)} \frac{\mathrm{d}z}{2\pi i} \in \mathbb{Z}.
\end{align*}
\par
Häufig wird der Index auch \textbf{Windungszahl} von \(\gamma\) um \(w\) genannt.
Sie ist eine \emph{topologische Invariante}, die anschaulich beschreibt, wie häufig sich die zugehörige Kurve um den Punkt \(w\) windet.
\end{definition}

\par
\textbf{ToDo: Abbildung mit Beispiel von \href{https://de.wikipedia.org/wiki/Umlaufzahl\_(Mathematik)}{Wikipedia}}
\begin{lemma}{}{complexanalysis/kurvenintegrale:lemma-2}



\par
Sei \(\gamma \colon [0,1] \rightarrow \C\) ein geschlossener Weg in \(\C\).
Dann ist die Abbildung, die jedem Punkt \(w \in \C \setminus \operatorname{Bild}(\gamma)\) außerhalb der zugehörigen Kurve seinen Index \(\operatorname{Ind_\gamma}(w)\) konstant auf jeder Zusammenhangskomponente bezüglich der Kurve von \(\gamma\).
\end{lemma}

\begin{proof}
 Schulz Baldes S.318f.
\end{proof}


\subsection{Homotopie}
\begin{definition}{(Homotopie)}{\detokenize{complexanalysis/kurvenintegrale:homotopie}}\label{complexanalysis/kurvenintegrale:definition-3}



\par
Sei \(I := [0,1]\) ein reelles Intervall und \(D \subset \C\) eine Teilmenge.
Wir nennen zwei Wege \(\gamma, \Gamma \colon I \rightarrow D\) \textbf{homotop} in der Teilmenge \(D\) genau dann, wenn eine stetige Abbildung \(H \colon I \times I \rightarrow D\) existiert, so dass
\begin{align*}
H(t,0) = \gamma(t), \qquad H(t,1) = \Gamma(t).
\end{align*}
\par
In diesem Fall nennen wir die Abbildung \(H\) eine \textbf{Homotopie} zwischen den Wegen \(\gamma\) und \(\Gamma\).
\end{definition}

\par
Man kann zeigen, dass der Begriff der Homotopie zwischen Wegen in einer Teilmenge \(D \subset \C\) eine \emph{Äquivalenzrelation} auf Wegen in \(D\) induziert.
Die zugehörigen Äquivalenzklassen werden auch \emph{Homotopieklassen} genannt.

\par
Für geschlossene Wege impliziert Homotopie eine besondere Eigenschaft bezüglich des Kurvenintegrals, wie folgendes Lemma festhält.
\begin{lemma}{}{complexanalysis/kurvenintegrale:lemma-4}



\par
Sei \(D \subset \C\) eine Teilmenge und seien \(\gamma\) und \(\Gamma\) geschlossene, homotope Wege in \(D\).
Sei außerdem \(f \colon D \rightarrow \C\) eine holomorphe Funktion.

\par
Dann gilt
\begin{align*}
\oint_\gamma f(z) \, \mathrm{d}z = \oint_\Gamma f(z) \, \mathrm{d}z.
\end{align*}\end{lemma}

\begin{proof}
 Schulz Baldes S.321
\end{proof}

\par
Eine besondere Klasse von Wegen sind solche, die nullhomotop sind.
\begin{definition}{(Nullhomotoper Weg)}{complexanalysis/kurvenintegrale:definition-5}



\par
Wir nennen einen Weg \(\gamma\) \textbf{nullhomotop} in einer Teilmenge \(D \subset \C\) genau dann, wenn \(\gamma\) homotop in \(D\) zu einem konstanten Weg ist.
\end{definition}

\par
Wir realisieren also, dass sich nullhomotope Wege in einer Teilmenge \(D \subset \C\) zu einem Punkt \(w \in D\) zusammenziehen lassen.
Darüber hinaus zeigt der folgende Satz, dass das Kurvenintegral einer holomorphe Funktionen auf einem nullhomotopen Weg verschwindet.
\begin{theorem}{(Satz von Cauchy)}{complexanalysis/kurvenintegrale:theorem-6}



\par
Sei \(\gamma\) ein nullhomotoper Weg in einer Teilmenge \(D \subset \C\), der sich zu einem Punkt \(w \in D\) zusammenziehen lässt.
Sei darüber hinaus \(f \colon D \rightarrow \C\) eine stetige Funktion, welche zudem holomorph auf der Menge \(D \setminus \{w\}\) sei.

\par
Dann gilt für das Kurvenintegral
\begin{align*}
\oint_\gamma f(z) \, \mathrm{d}z = 0.
\end{align*}\end{theorem}

\begin{proof}
 Schulz Baldes S.322
\end{proof}


\subsection{Cauchyscher Integralsatz}
\label{\detokenize{complexanalysis/kurvenintegrale:cauchyscher-integralsatz}}
\par
Wir wollen nun einen der zentralen Aussagen der Funktionentheorie formulieren, die Cauchysche Integralformel.
Diese besagt, dass sich die Werte einer holomorphen Funktion im Inneren eines bestimmten Gebietes bereits durch die Werte auf dem Gebietsrand bestimmen lassen.
\begin{theorem}{(Cauchyscher Integralsatz)}{complexanalysis/kurvenintegrale:theorem-7}



\par
Sei \(D \subset \C\) ein Sterngebiet, d.h., \(D\) ist eine offene Menge in der mindestens einen Punkt \(z_0 \in \C\) gibt, so dass die Verbindungsstrecke jedes beliebigen Punktes \(z \in D\) zu \(z_0\) vollständig in \(D\) liegt.
Sei außerdem \(\gamma\) ein geschlossener Weg in \(D\).

\par
Dann lässt sich der Funktionswert von \(f\) in jedem Punkt \(z \in D \setminus \operatorname{Bild}(\gamma)\) darstellen durch das Kurvenintegral
\begin{align*}
f(z) \ = \ \frac{1}{\operatorname{Ind}_\gamma(z)} \oint_\gamma \frac{f(\zeta)}{\zeta - z} \frac{\mathrm{d}\zeta}{2\pi i}.
\end{align*}\end{theorem}

\begin{proof}
 Schulz Baldes S.323
\end{proof}

\par
Wir werden den Cauchyschen Integralsatz später noch in Form des sogenannten \emph{Residuensatzes} stark verallgemeinern.
Für den Moment erlaubt uns dessen Aussage jedoch zu zeigen, dass jede holomorphe Funktion bereits analytisch ist, was die Umkehrung zu \cref{complexanalysis/cauchyriemann:thm:analytischHolomorph} ist.
\begin{theorem}{(Holomorphe Funktionen sind analytisch)}{complexanalysis/kurvenintegrale:theorem-8}



\par
Sei \(\epsilon > 0\) und \(z_0 \in D\) ein Punkt in einer offenen Teilmenge \(D \subset \C\).
Sei außerdem \(f \colon B_\epsilon(z_0) \rightarrow \C\) eine holomorphe Funktion.

\par
Dann ist die Funktion \(f\) für jeden Punkt \(z \in B_\epsilon(z_0)\) durch eine konvergente Potenzreihe darstellbar (also analytisch) als
\begin{align*}
f(z) = \sum_{n=1}^\infty a_n (z-z_0)^n,
\end{align*}
\par
deren Koeffizienten \((a_n)_{n\in\N}\) für alle \(\epsilon' < \epsilon\) gegeben sind durch
\begin{align*}
a_n \ = \ \oint_{\partial B_{\epsilon'}(z_0)} \frac{f(\zeta)}{(\zeta - z_0)^{n+1}} \frac{\mathrm{d}\zeta}{2\pi i}.
\end{align*}
\par
Insbesondere ist \(f\) unendlich oft komplex differenzierbar und für alle \(n \in \N\) gilt für die \(n\) te Ableitung von \(f\)
\begin{align*}
f^{(n)}(z_0) \ = \ n! \oint_{\partial B_{\epsilon'}(z_0)} \frac{f(\zeta)}{(\zeta - z_0)^{n+1}} \frac{\mathrm{d}\zeta}{2\pi i} = n! a_n.
\end{align*}\end{theorem}

\begin{proof}
 Schulz Baldes S.325f.
\end{proof}

\par
Das folgende Korollar erlaubt eine direkte Abschätzung der \(n\) ten Ableitung einer holomorphen Funktion.
\label{complexanalysis/kurvenintegrale:corollary-9}
\begin{emphBox}{}{}{Corollary 7.1 (Cauchy Abschätzungen)}



\par
Sei \(\epsilon > 0\) und \(z_0 \in D\) ein Punkt in einer offenen Teilmenge \(D \subset \C\).
Außerdem sei \(f \colon B_\epsilon(z_0) \rightarrow \C\) eine holomorphe Funktion.
Dann gilt für alle \(0 < \epsilon' < \epsilon\) die folgende genannte \textbf{Cauchy Abschätzung}
\begin{align*}
|f^{(n)}(z_0)| \ \leq \ \frac{n!}{\epsilon'^n} \max_{|z - z_0|=\epsilon'} |f(z)|.
\end{align*}\end{emphBox}


\section{Laurententwicklung und Residuensatz}
\label{\detokenize{complexanalysis/residuensatz:laurententwicklung-und-residuensatz}}\label{\detokenize{complexanalysis/residuensatz::doc}}

\subsection{Singularitäten homomorpher Funktionen}
\label{\detokenize{complexanalysis/residuensatz:singularitaten-homomorpher-funktionen}}
\par
In diesem Abschnitt beschäftigen wir uns mit speziell ausgezeichneten Punkten, den sogenannten Singularitäten.
\begin{definition}{(Singularitäten)}{complexanalysis/residuensatz:definition-0}



\par
Sei \(D \subset \C\) eine offene Teilmenge und \(z_0 \in D\) ein Punkt.

\par
1. Wenn \(f \colon D \setminus \{z_0\} \rightarrow \C\) eine holomorphe Funktion ist.
Dann nennen wir den Punkt \(z_0\) eine \textbf{isolierte Singularität} von \(f\).

\par
2. Wir nennen den Punkt \(z_0\) eine \textbf{hebbare Singularität}, wenn \(z_0\) eine isolierte Singularität einer holomorphen Funktion \(f \colon D \setminus \{z_0\} \rightarrow \C\) ist und es eine holomorphe Funktion \(g \colon D \rightarrow \C\) gibt, so dass \(g(z) = f(z)\) gilt für alle \(z \in D \setminus \{z_0\}\).

\par
3. Wir nennen den Punkt \(z_0\) einen \textbf{Pol}, wenn für alle Folgen \(z_n \rightarrow z_0\) gilt
\begin{align*}
\lim_{n\rightarrow \infty} |f(z_n)| = \infty.
\end{align*}
\par
4. Wir nennen den Punkt \(z_0\) eine \textbf{wesentliche Singularität}, wenn \(z_0\) weder hebbar noch Pol ist.
\end{definition}

\par
Der Satz von Casorati Weierstraß erlaubt es wesentliche Singularitäten zu charakterisieren.
\begin{remark}{(Casorati Weierstraß)}{complexanalysis/residuensatz:remark-1}



\par
Sei \(D \subset \C\) eine offene Teilmenge und \(z_0 \in D\) ein Punkt.

\par
Der Punkt \(z_0\) ist genau dann eine wesentliche Singularität einer holomorphen Funktion \(f \colon D \setminus \{z_0\} \rightarrow \C\), wenn für alle \(\epsilon > 0\) die Menge der Funktionswerte \(f(B_\epsilon(z_0)) \setminus \{z_0\})\) dicht in \(\C\) liegt.
\end{remark}

\par
\textbf{ToDo: Hier Beispiel zu Singularitäten? Schulz Baldes S.329}
\begin{theorem}{(Riemannscher Hebbarkeitssatz)}{complexanalysis/residuensatz:theorem-2}



\par
Sei \(D \subset \C\) eine offene Teilmenge und \(z_0 \in D\) ein Punkt.
Sei außerdem \(f \colon D \setminus \{z_0\} \rightarrow \C\) eine holomorphe Funktion.
Falls eine Umgebung \(U \subset D\) von \(z_0\) gibt, so dass \(f\) auf \(U \setminus \{z_0\}\) beschränkt ist, so kann man einen Funktionswert \(f(z_0)\) in \(z_0\) so wählen, dass die Funktion \(f\) auf der gesamten Teilmenge \(D\) holomorph ist, d.h., der Punkt \(z_0\) ist eine hebbare Singularität.
\end{theorem}

\begin{proof}
 Schulz Baldes S.327
\end{proof}

\par
Das folgende Lemma charakterisiert Pole einer holomorphen Funktion.
\begin{lemma}{}{complexanalysis/residuensatz:lem:pole}



\par
Sei \(D \subset \C\) eine offene Teilmenge und \(z_0 \in D\) eine isolierte Singularität einer holomorphen Funktion \(f \colon D \setminus \{z_0\} \rightarrow \C\).

\par
Dann sind folgende Aussagen äquivalent:

\par
1. Der Punkt \(z_0\) ist ein Pol der Funktion \(f\).

\par
2. Es existiert ein \(m \in \N\), so dass die Funktion \((z - z_0)^m f(z)\) beschränkt in einer lokalen Umgebung von \(z_0\) ist, jedoch die Funktion \((z - z_0)^{m-1} f(z)\) unbeschränkt ist.

\par
Die Ordnung der Funktion \(f\) im Pol \(z_0\) ist dann definiert als
\begin{align*}
\operatorname{Ord}_{z_0}(f) := -m.
\end{align*}\end{lemma}

\begin{proof}
 Schulz Baldes S.330
\end{proof}

\par
Dieser Begriff von Ordnung setzt den Begriff der Ordnung von Polynomen für holomorphe Funktionen fort.
Häufig spricht man jedoch nur von der Ordnung \(m > 0\) eines Pols.


\subsection{Laurent Reihe}
\label{\detokenize{complexanalysis/residuensatz:laurent-reihe}}
\par
Die Beobachtung aus \cref{complexanalysis/residuensatz:lem:pole} motiviert die folgende Definition der Laurent Reihe, die nach \{prf:ref\}`` immer an einem Pol von Ordnung \(m\) existiert.
\label{complexanalysis/residuensatz:definition-4}
\begin{emphBox}{}{}{ (Laurent Reihe)}



\par
Sei \(D \subset \C\) eine offene Teilmenge und \(z_0 \in D\) Pol von Ordnung \(m\) einer holomorphen Funktion \(f \colon D \setminus \{z_0\} \rightarrow \C\).

\par
Dann definieren wir die \textbf{Laurent Reihe} von \(f\) um den Pol \(z_0\) durch
\begin{align*}
f(z) := \sum_{n=-m}^\infty a_n (z-z_0)^n.
\end{align*}
\par
Als \textbf{Hauptteil} der Laurent Reihe bezeichnen wir den Term
\begin{align*}
\sum_{n=-m} a_n (z-z_0)^n
\end{align*}
\par
und das \textbf{Residuum} von \(f\) bei \(z_0\) als
\begin{align*}
\operatorname{Res}_{z_0}(f) = a_{-1}.
\end{align*}\begin{definition}{}{complexanalysis/residuensatz:definition-5}



\par
Sei \(D \subset \C\) eine offene Teilmenge.
Wir nennen eine Funktion \(f \colon D \rightarrow \C\) \textbf{meromorph} auf \(D\) genau dann, wenn eine lokalendliche Menge \(P\) existiert, so dass die Funktion \(f\) holomorph auf \(D \setminus P\) mit Polen in \(P\) ist.
\end{definition}
\end{emphBox}
\begin{example}{(Meromorphe Funktionen)}{complexanalysis/residuensatz:example-6}



\par
Rationale Funktionen oder konkretes Beispiel

\par
Schulz Baldes S.332
\end{example}


\subsection{Cauchyscher Residuensatz}
\label{\detokenize{complexanalysis/residuensatz:cauchyscher-residuensatz}}
\par
Das folgende Lemma erlaubt die explizite Berechnung des Residuums.
\begin{lemma}{(Berechnung des Residuums)}{complexanalysis/residuensatz:lemma-7}



\par
Sei \(D \subset \C\) eine offene Teilmenge und \(z_0 \in D\) Pol einer holomorphen Funktion \(f \colon D \setminus \{z_0\} \rightarrow \C\).

\par
Für genügend kleine \(\epsilon > 0\) lässt sich das Residuum von \(f\) bei \(z_0\) angeben als
\begin{align*}
\operatorname{Res}_{z_0}(f) = \oint_{\partial B_\epsilon(z_0)} f(z) \frac{\mathrm{d}z}{2\pi i}.
\end{align*}
\par
Falls der Pol von Ordnung \(-m\) ist, lässt sich das Residuum von \(f\) bei \(z_0\) sogar angeben als
\begin{align*}
\operatorname{Res}_{z_0}(f) = \partial_z^{m-1}\left( (z-z_0)^m \frac{f(z)}{(m-1)!}\right)|_{z=z_0}.
\end{align*}\end{lemma}

\begin{proof}
 Schulz Baldes S.333f.
\end{proof}
\begin{example}{(Berechnung des Residuums)}{complexanalysis/residuensatz:example-8}



\par
Rationale Funktion bei Schulz Baldes S.335
\end{example}

\par
Der folgende Residuensatz von Cauchy stellt eine zentrale Aussage der Funktionentheorie vor.
\begin{theorem}{(Cauchyscher Residuensatz)}{complexanalysis/residuensatz:theorem-9}



\par
Sei \(D \subset \C\) eine offene Teilmenge und \(f \colon D \rightarrow \C\) eine meromorphe Funktion mit endlicher Menge \(P \subset D\) von Polstellen.
Sei außerdem \(\gamma\) ein geschlossener und zusammenziehbarer Weg in \(D\) mit \(\operatorname{Bild}(\gamma) \cap P = \emptyset\).

\par
Dann gilt der folgende Zusammenhang
\begin{align*}
\int_\gamma f(z) \frac{\mathrm{d}z}{2\pi i} = \sum_{z_0 \in P} \operatorname{Ind}_\gamma(z_0) \operatorname{Res}_{z_0}(f).
\end{align*}\end{theorem}

\begin{proof}
 Schulz Baldes S.337
\end{proof}
\begin{remark}{}{complexanalysis/residuensatz:remark-10}



\par
Für holomorphe Funktionen \(f\) entspricht der Residuensatz gerade dem Cauchyschen Integralsatz.
Wenn \(D\) als Sterngebiet angenommen wird ist die Zusammenziehbarkeit des Wegs \(\gamma\) immer erfüllt.
\end{remark}
\begin{example}{}{complexanalysis/residuensatz:example-11}



\par
Viele konkrete Beispiele in Schulz Baldes S.338 344
\end{example}


