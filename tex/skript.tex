%% Generated by Sphinx.
\def\sphinxdocclass{jupyterBook}
\documentclass[letterpaper,10pt,english]{jupyterBook}
\ifdefined\pdfpxdimen
   \let\sphinxpxdimen\pdfpxdimen\else\newdimen\sphinxpxdimen
\fi \sphinxpxdimen=.75bp\relax
%% turn off hyperref patch of \index as sphinx.xdy xindy module takes care of
%% suitable \hyperpage mark-up, working around hyperref-xindy incompatibility
\PassOptionsToPackage{hyperindex=false}{hyperref}
%% memoir class requires extra handling
\makeatletter\@ifclassloaded{memoir}
{\ifdefined\memhyperindexfalse\memhyperindexfalse\fi}{}\makeatother

\PassOptionsToPackage{warn}{textcomp}

\catcode`^^^^00a0\active\protected\def^^^^00a0{\leavevmode\nobreak\ }
\usepackage{cmap}
\usepackage{fontspec}
\defaultfontfeatures[\rmfamily,\sffamily,\ttfamily]{}
\usepackage{amsmath,amssymb,amstext}
\usepackage{polyglossia}
\setmainlanguage{english}



\setmainfont{FreeSerif}[
  Extension      = .otf,
  UprightFont    = *,
  ItalicFont     = *Italic,
  BoldFont       = *Bold,
  BoldItalicFont = *BoldItalic
]
\setsansfont{FreeSans}[
  Extension      = .otf,
  UprightFont    = *,
  ItalicFont     = *Oblique,
  BoldFont       = *Bold,
  BoldItalicFont = *BoldOblique,
]
\setmonofont{FreeMono}[
  Extension      = .otf,
  UprightFont    = *,
  ItalicFont     = *Oblique,
  BoldFont       = *Bold,
  BoldItalicFont = *BoldOblique,
]


\usepackage[Bjarne]{fncychap}
\usepackage[,numfigreset=1,mathnumfig]{sphinx}

\fvset{fontsize=\small}
\usepackage{geometry}


% Include hyperref last.
\usepackage{hyperref}
% Fix anchor placement for figures with captions.
\usepackage{hypcap}% it must be loaded after hyperref.
% Set up styles of URL: it should be placed after hyperref.
\urlstyle{same}


\usepackage{sphinxmessages}



        % Start of preamble defined in sphinx-jupyterbook-latex %
         \usepackage[Latin,Greek]{ucharclasses}
        \usepackage{unicode-math}
        % fixing title of the toc
        \addto\captionsenglish{\renewcommand{\contentsname}{Contents}}
        \hypersetup{
            pdfencoding=auto,
            psdextra
        }
        % End of preamble defined in sphinx-jupyterbook-latex %
        

\title{Mathematik für Physikstudierende C}
<<<<<<< HEAD
\date{Nov 12, 2021}
=======
\date{Nov 13, 2021}
>>>>>>> main
\release{}
\author{J.\@{} Laubmann, T.\@{} Roith, D.\@{} Tenbrinck}
\newcommand{\sphinxlogo}{\vbox{}}
\renewcommand{\releasename}{}
\makeindex
\begin{document}

\pagestyle{empty}
\sphinxmaketitle
\pagestyle{plain}
\sphinxtableofcontents
\pagestyle{normal}
\phantomsection\label{\detokenize{intro::doc}}


\noindent\sphinxincludegraphics{{intro_1_0}.png}

\sphinxAtStartPar
Das vorliegende Skript begleitet die \sphinxstylestrong{Vorlesung Mathematik für Physikstudierende C} und ist im Wintersemester 21/22 an der FAU Erlangen\sphinxhyphen{}Nürnberg entstanden. Es soll den Studierenden zusätzlich zur virtuellen Vorlesung als Nachschlagewerk dienen und ist ausführlicher und genauer gehalten als die Vorlesungsnotizen.

\begin{DUlineblock}{0em}
\item[] \sphinxstylestrong{\Large Referenz}
\end{DUlineblock}

\sphinxAtStartPar
Das Skript orientiert sich teilweise an dem Vorlesungsskript “Mathematik für Physikstudierende 3” {[}\hyperlink{cite.references:id7}{Kna20}{]} von Prof.Dr.Andreas Knauf (FAU) aus dem Sommersemester 2020 und den Folien zu “Mathematik für Physiker 3” von Prof.Dr.Hermann Schulz\sphinxhyphen{}Baldes (FAU) {[}\hyperlink{cite.references:id10}{SB18}{]}. Weiterhin wird in der Vorlesung oft auch auf das Buch “Mathematische Physik: Klassiche Mechanik” {[}\hyperlink{cite.references:id8}{Kna17}{]} von Prof.Knauf verwiesen, was wir Ihnen als zusätzliches Nachschlagewerk empfehlen können.


\chapter{Gewöhnliche Differentialgleichungen für dynamische Systeme}
\label{\detokenize{ode/ode:gewohnliche-differentialgleichungen-fur-dynamische-systeme}}\label{\detokenize{ode/ode::doc}}
\sphinxAtStartPar
In diesem ersten Kapitel der Vorlesung wollen wir weiterführende Konzepte zum Thema gewöhnlicher Differentialgleichungen einführen.
Insbesondere wollen wir uns mit gewöhnlichen Differentialgleichungen für dynamische Systeme beschäftigen.
Hierfür wiederholen wir zunächst die wichtigsten Aussagen und Begriffe, die Sie in Kaptiel 8 {[}\hyperlink{cite.references:id12}{Ten21}{]} kennengelernt haben.
Anschließend definieren wir zwei grundlegende mathematische Werkzeuge um dynamische Systeme zu charakterisieren, nämlich Flüsse und Phasenportraits.
Zum Schluss wollen wir diese zur Untersuchung und Lösung von Hamiltonschen Differentialgleichungen nutzen, welche eine insbesondere in der klassischen Mechanik innerhalb der Physik eine wichtige Rolle spielen.


\section{Einführung in dynamische Systeme}
\label{\detokenize{ode/dynamicSystems:einfuhrung-in-dynamische-systeme}}\label{\detokenize{ode/dynamicSystems::doc}}
\sphinxAtStartPar
Dynamische Systeme spielen eine zentrale Rolle bei der Beschreibung zeitabhängiger Prozesse in vielen verschiedenen Anwendungsgebieten, wie zum Beispiel der Biologie oder der Physik.
Durch diese Art von mathematischen Modellen ist es beispielsweise möglich das Ausschwingen eines Pendels zu beschreiben oder den Bestand zweier unterschiedlicher Populationen über die Zeit in einer Räuber\sphinxhyphen{}Beute Beziehung zu untersuchen.

\sphinxAtStartPar
Maßgeblich für dynamische Systeme ist die Beobachtung, dass die beschriebenen Prozesse nicht von der Wahl des Anfangszeitpunktes abhängig sind, sondern lediglich von dem gewählten Anfangszustand.
Wir werden diese Eigenschaft später in Sektion {\hyperref[\detokenize{ode/fluesse:s-fluesse}]{\sphinxcrossref{\DUrole{std,std-ref}{Phasenflüsse und Phasenportraits}}}} noch genauer mathematisch charakterisieren.

\sphinxAtStartPar
Je nach Anwendungsgebiet können dynamische Systeme entweder \sphinxstylestrong{diskret} oder \sphinxstylestrong{kontinuierlich} in der Zeitentwicklung sein.
Wir wollen im Folgenden zwei Beispiele zur Illustration des Unterschieds in der Zeitmodellierung diskutieren.


\subsection{Diskrete dynamische Systeme}
\label{\detokenize{ode/dynamicSystems:diskrete-dynamische-systeme}}
\sphinxAtStartPar
Zur Veranschaulichung von diskreten dynamischen System wollen wir uns im Folgenden mit einem Beispiel aus der Biologie beschäftigen.
\label{ode/dynamicSystems:ex:bacteria}
\begin{sphinxadmonition}{note}{Example 1.1 (Wachstum von Bakterien)}



\sphinxAtStartPar
In diesem Beispiel wollen wir annehmen, dass wir das \sphinxstylestrong{exponentielle Wachstum} von Bakterien durch Zellteilung als diskretes dynamisches System zu festen, äquidistanten Zeitpunkten \(t_0, t_1, \ldots \in I\) in einem offenen Zeitintervall \(I\subset\R^+_0\) untersuchen wollen.
Wir modellieren die (ungefähre) Anzahl der Bakterien zu einem Zeitpunkt \(t \in I\) als Funktion \(F \colon I \rightarrow \R_0^+\).
Da die Zeitpunkte äquidistant gewählt sind können wir eine einheitliche Wachstumsrate \(\alpha \in \R^+\) mit \(\alpha > 1\) annehmen, so dass für alle \(n \in \N\) gilt:
\begin{equation*}
\begin{split}F(t_{n+1}) = \alpha \cdot F(t_n).\end{split}
\end{equation*}
\sphinxAtStartPar
Wir erkennen, dass der Prozess des Bakterienwachstums nicht von der konkreten Wahl des Startzeitpunkts \(t_0 \in I\) abhängt, sondern nur von anfänglichen Anzahl der Bakterien \(F_0 \coloneqq F(t_0)\). \hyperref[\detokenize{ode/dynamicSystems:fig-bacteria}]{Fig.\@ \ref{\detokenize{ode/dynamicSystems:fig-bacteria}}} zeigt, dass eine unterschiedliche Wahl des Anfangszeitpunkt bei gleicher Wahl der Anfangspopulation keinen Effekt auf die zeitliche Dynamik hat.

\sphinxAtStartPar
Dies können wir wie folgt mathematisch verifizieren. Seien \(t_m, t_n \in I\) mit \(n,m \in \N\) zwei unterschiedliche Anfangszeitpunkte für die die gleiche Anfangspopulation \(F_0 \in \N\) von Bakterien angenommen wird, d.h.,
\begin{equation*}
\begin{split}F(t_m) = F_0 = F(t_n).\end{split}
\end{equation*}
\sphinxAtStartPar
Betrachten wir nun für die beiden unterschiedlichen Anfangszeitpunkte das Bakterienwachstum nach \(k \in \N\) äquidistanten Zeitschritten, so ergibt sich:
\begin{equation*}
\begin{split}F(t_{m+k}) = \alpha \cdot F(t_{m+k-1}) = \ldots = \alpha^k \cdot F(t_{m}) = \alpha^k \cdot F_0 = \alpha^k \cdot F(t_n) = F(t_{n+k}).\end{split}
\end{equation*}
\sphinxAtStartPar
Wir erkennen also, dass unabhängig vom gewählten Anfangszeitpunkt die Bakterienpopulation nach \(k \in \N\) Zeitschritten gleich ist.
\end{sphinxadmonition}

\begin{figure}[htbp]
\centering
\capstart

\noindent\sphinxincludegraphics{{C:/Tim/Uni/Lectures/MathPhysicsC/_build/jupyter_execute/dynamicSystems_3_0}.png}
\caption{Visualisierung für Beispiel {\hyperref[\detokenize{ode/dynamicSystems:ex:bacteria}]{\sphinxcrossref{Example 1.1}}}. Wir erkennen, dass die Dynamik der Koloniegröße nicht von der Startzeit abhängt, sondern nur vom Anfangswert. Zu beachten gilt, es ist ein diskretes System, die angezeichneten kontinuierlichen Linien dienen lediglich zur Veranschaulichung der Dynamik.}\label{\detokenize{ode/dynamicSystems:fig-bacteria}}\end{figure}

\sphinxAtStartPar
Diskrete dynamische Systeme tauchen auch in anderen spannenden Anwendungen auf, wie beispielsweise in der \sphinxhref{https://de.wikipedia.org/wiki/Bifurkation\_(Mathematik)\#Bifurkationsdiagramm}{Chaostheorie} und in der \sphinxhref{https://de.wikipedia.org/wiki/Markow-Kette}{Stochastik}.


\subsection{Kontinuierliche dynamische Systeme}
\label{\detokenize{ode/dynamicSystems:kontinuierliche-dynamische-systeme}}
\sphinxAtStartPar
Im Unterschied zu diskreten dynamischen Systemen wird die Zeit bei kontinuierlichen dynamischen Systemen nicht an abzählbar vielen Punkten modelliert, sondern als Kontinuum.
Im Folgenden beschreiben wir das physikalische Experiment des freien Falls als Spezialfall eines kontinuierlichen dynamischen Systems.
\label{ode/dynamicSystems:ex:freefall}
\begin{sphinxadmonition}{note}{Example 1.2 (Freier Fall)}



\sphinxAtStartPar
In diesem Beispiel betrachten wir ein physikalisches Modell für den freien Fall eines Steins mit Masse \(m \in \R^+\), den wir in einer Hand halten, bis wir ihn zu einem definierten Anfangszeitpunkt \(t_0 \in I\) mit \(I \subset \R^+_0\) fallen lassen.

\sphinxAtStartPar
Die aktuelle Entfernung des Steins zum Boden zu einem Zeitpunkt \(t \in I\), d.h. seine gegenwärtige Höhe, ist gegeben durch eine monoton\sphinxhyphen{}fallende Funktion \(F \colon I \rightarrow \R^+_0\).
Unsere Hand befindet sich zum Anfangszeitpunkt \(t_0\) in einer Höhe von \(F_0 > 0\).
Für jeden beliebigen Zeitpunkt \(t > t_0\) lässt sich die aktuelle Höhe des fallenden Steins mit Hilfe des Newtonschen Gravitationsgesetzes wie folgt angeben:
\begin{equation*}
\begin{split}F(t) = \max(0, F_0 - \frac{1}{2}gt^2),\end{split}
\end{equation*}
\sphinxAtStartPar
wobei \(g \approx 9,81 \frac{m}{s^2}\) die Erdbeschleunigungskonstante bezeichnet.

\sphinxAtStartPar
Aus \hyperref[\detokenize{ode/dynamicSystems:fig-free-fall}]{Fig.\@ \ref{\detokenize{ode/dynamicSystems:fig-free-fall}}} wird klar, dass auch hier die Dynamik des freien Falls nicht von der Wahl des Anfangszeitpunkts \(t_0 \in I\) abhängt.
Anschaulich gesprochen, würde der Stein genauso fallen, wenn wir ihn noch einige Sekunden länger festhalten würden.
\end{sphinxadmonition}

\begin{figure}[htbp]
\centering
\capstart

\noindent\sphinxincludegraphics{{C:/Tim/Uni/Lectures/MathPhysicsC/_build/jupyter_execute/dynamicSystems_6_0}.png}
\caption{Visualisierung für Beispiel {\hyperref[\detokenize{ode/dynamicSystems:ex:freefall}]{\sphinxcrossref{Example 1.2}}}. Wir erkennen, dass die Dynamik der Fallhöhe nicht von der Startzeit abhängt, sondern nur von der Starthöhe.}\label{\detokenize{ode/dynamicSystems:fig-free-fall}}\end{figure}

\sphinxAtStartPar
Häufig kommen zur Beschreibung von kontinuierlichen dynamischen Systemen sogenannte \sphinxstylestrong{autonome gewöhnliche Differentialgleichungen} zum Einsatz, wie die in Beispiel {\hyperref[\detokenize{ode/dynamicSystems:ex:freefall}]{\sphinxcrossref{Example 1.2}}} implizit genutzten Bewegungsgleichungen.
Wir werden diese Art von Differentialgleichungen in Kapitel {\hyperref[\detokenize{ode/fluesse:s-fluesse}]{\sphinxcrossref{\DUrole{std,std-ref}{Phasenflüsse und Phasenportraits}}}} mathematisch genauer betrachten.


\section{Wiederholung: Gewöhnliche Differentialgleichungen}
\label{\detokenize{ode/repetition:wiederholung-gewohnliche-differentialgleichungen}}\label{\detokenize{ode/repetition::doc}}
\sphinxAtStartPar
In diesem Abschnitt werden wir kurz die wichtigsten Definitionen und Ergebnisse zu gewöhnlichen Differentialgleichungen aus Kapitel 8 in {[}\hyperlink{cite.references:id12}{Ten21}{]} wiederholen und um neue Begriffe erweitern, mit denen wir die Theorie dynamischer Systeme mathematisch untersuchen können.


\subsection{Gewöhnliche Differentialgleichungen}
\label{\detokenize{ode/repetition:gewohnliche-differentialgleichungen}}
\sphinxAtStartPar
Wir erinnern uns zunächst an die Definition eines gewöhnlichen Differentialgleichungssystems \(m\)\sphinxhyphen{}ter Ordnung als Grundlage für unsere weiteren Betrachtungen.
\label{ode/repetition:def:DGL}
\begin{sphinxadmonition}{note}{Definition 1.1 (Gewöhnliches Differentialgleichungssystem)}



\sphinxAtStartPar
Seien \(n,m \in \N\).
Wir betrachten im Folgenden eine offene Teilmenge \(U\subset (\R^n)^{m+1}\) und ein offenes Intervall \(I\subset\R\).
Es sei außerdem \(F:I\times U\rightarrow\R^n\) eine stetige Funktion, dann nennen wir
\begin{equation}\label{equation:ode/repetition:eq:DGL}
\begin{split}F(t,x(t),x'(t),\ldots,x^{(m)}(t)) = 0\end{split}
\end{equation}
\sphinxAtStartPar
ein \sphinxstylestrong{gewöhnliches Differentialgleichungssystem (DGL)} \(m\)\sphinxhyphen{}ter Ordnung von \(n\) Gleichungen.
Gilt \(n=1\), das heißt die Funktion \(F\) ist skalarwertig, so sprechen wir von einer \sphinxstylestrong{gewöhnlichen Differentialgleichung}.

\sphinxAtStartPar
Eine Funktion \(\phi\in C^m(I;\R^n)\) heißt \sphinxstylestrong{Lösung des Differentialgleichungssystems}, falls gilt,
\begin{equation*}
\begin{split}F(t, \phi(t), \phi'(t), \ldots, \phi^{(m)}(t)) = 0 \quad \forall t\in I.\end{split}
\end{equation*}
\sphinxAtStartPar
Wenn wir die DGL nach der höchsten auftauchenden Ableitung auflösen können, so dass sie die folgende Form hat
\begin{equation*}
\begin{split}x^{(m)}(t) = F(t,x(t),x'(t),\ldots,x^{(m-1)}(t)),\end{split}
\end{equation*}
\sphinxAtStartPar
so nennen wir die DGL \sphinxstylestrong{explizit}, ansonsten wird sie \sphinxstylestrong{implizit} genannt.
\end{sphinxadmonition}

\sphinxAtStartPar
Folgende Bemerkung beschreibt eine alternative Notation von gewöhnlichen Differentialgleichungen 1. und 2. Ordnung, die häufig in der Literatur im Kontext dynamischer Systeme auftaucht.
\label{ode/repetition:remark-1}
\begin{sphinxadmonition}{note}{Remark 1.1 (Zeitableitungen bei gewöhnlichen Differentialgleichungen)}



\sphinxAtStartPar
Viele physikalische Phänomene können durch zeitabhängige gewöhnliche Differentialgleichungen 1. und 2. Ordnung beschrieben werden.
In diesen Fällen verwendet man häufig die Variable \(t \in \R^+_0\) als unabhängige Variable anstatt einer Variable \(x \in \R\).
Auch ändert sich häufig die Notation der Zeitableitungen der gesuchten Funktion \(x\), so dass folgende Korrespondenz für die ersten beiden Ableitungen entsteht:
\begin{enumerate}
\sphinxsetlistlabels{\arabic}{enumi}{enumii}{}{.}%
\item {} 
\sphinxAtStartPar
\(x'(t) \ \ \hat{=} \ \ \dot{x}(t)\),

\item {} 
\sphinxAtStartPar
\(x''(t) \ \ \hat{=} \ \ \ddot{x}(t)\).

\end{enumerate}

\sphinxAtStartPar
Damit lässt sich das gewöhnliche Differentialgleichungssystem aus \eqref{equation:ode/repetition:eq:DGL} schreiben als
<<<<<<< HEAD
\begin{equation}\label{equation:ode/repetition:eq:DGL_time}
\begin{split}F(z, y(t), \dot{y}(t), \ldots, y{(m)}(t)) = 0 \quad \forall t\in I.\end{split}
=======
\begin{equation}\label{equation:ode/repetition:eq:DGLtime}
\begin{split}F(t, x(t), \dot{x}(t), \ldots, x{(m)}(t)) = 0 \quad \forall t\in I.\end{split}
>>>>>>> main
\end{equation}\end{sphinxadmonition}


\subsection{Autonome Differentialgleichungen}
\label{\detokenize{ode/repetition:autonome-differentialgleichungen}}
\sphinxAtStartPar
Im Fall von dynamischen Systemen erhält der Definitionsbereich der Funktion \(F\) einer gewöhnlichen Differentialgleichung einen besonderen Namen, wie die folgende Bemerkung erklärt.
\label{ode/repetition:remark-2}
\begin{sphinxadmonition}{note}{Remark 1.2 ((Erweiterter) Phasenraum)}



\sphinxAtStartPar
Wird eine gewöhnliche Differentialgleichung als mathematisches Modell für ein kontinuierliches dynamisches System genutzt, so wird die offene Menge \(U\subset (\R^n)^{m+1}\) auch als \sphinxstylestrong{Phasenraum} bezeichnet.
Der Definitionsbereich \(I\times U\) der stetigen Funktion \(F\) wird auch als \sphinxstylestrong{erweiterter Phasenraum} bezeichnet.

\sphinxAtStartPar
Der Phasenraum beschreibt die Menge aller möglichen Zustände des dynamischen Systems.
Jeder Punkt des Phasenraums wird hierbei eindeutig einem Zustand des Systems zugeordnet.

\sphinxAtStartPar
In Kapitel {\hyperref[\detokenize{ode/fluesse:s-fluesse}]{\sphinxcrossref{\DUrole{std,std-ref}{Phasenflüsse und Phasenportraits}}}} werden wir spezielle Diagramme basierend auf dem Begriff des erweiterten Phasenraum betrachten (auch Phasenportraits genannt), um Lösungen von dynamischen Systemen mathematisch zu charakterisieren.
\end{sphinxadmonition}

\sphinxAtStartPar
Im Fall von \sphinxstylestrong{kontinuierlichen dynamischen Systemen} spielt eine Familie von gewöhnlichen Differentialgleichungen eine wichtige Rolle, die wir im Folgenden definieren wollen.
<<<<<<< HEAD
Diese zeichnen sich dadurch aus, dass die Funktion \(F\) in \eqref{equation:ode/repetition:eq:DGL_time} nicht explizit von der Zeit abhängt.
=======
Diese zeichnen sich dadurch aus, dass die Funktion \(F\) in \eqref{equation:ode/repetition:eq:DGLtime} nicht explizit von der Zeit abhängt.
>>>>>>> main
\label{ode/repetition:definition-3}
\begin{sphinxadmonition}{note}{Definition 1.2 (Autonome DGL)}



\sphinxAtStartPar
Hängt die Funktion \(F\) in {\hyperref[\detokenize{ode/repetition:def:DGL}]{\sphinxcrossref{Definition 1.1}}} nicht explizit von der Zeit ab, d.h., wir haben \(F:U\rightarrow\R^n\) dann heißt die Gleichung
\begin{equation}\label{equation:ode/repetition:eq:autonomeDGL}
\begin{split}F(x(t), x'(t), \ldots, x^{(m)}(t)) = 0 \quad \forall t\in I\end{split}
\end{equation}
\sphinxAtStartPar
\sphinxstylestrong{autonome DGL}.
\end{sphinxadmonition}

\sphinxAtStartPar
Im folgenden Beispiel wollen wir unterschiedliche gewöhnliche Differentialgleichungen darauf prüfen, ob sie autonom sind.
\label{ode/repetition:example-4}
\begin{sphinxadmonition}{note}{Example 1.3 (Autonome Differentialgleichungen)}



\sphinxAtStartPar
Wir betrachten drei verschiedene gewöhnliche Differentialgleichungen und untersuchen diese auf ihre Zeitabhängigkeit.
Der Einfachheit\sphinxhyphen{}halber konzentrieren wir uns hierbei auf gewöhnliche Differentialgleichungen 1. Ordnung.
Sei hierzu  im Folgenden \(I \subset \R\) ein offenes Intervall.

\sphinxAtStartPar
1. Die gewöhnliche Differentialgleichung
\begin{equation*}
\begin{split}2x'(t) = x(t)\cdot t \quad \forall t \in I\end{split}
\end{equation*}
\sphinxAtStartPar
ist \sphinxstylestrong{nicht autonom}, da die rechte Seite der Gleichung durch die Funktion
\begin{equation*}
\begin{split}F(t,x(t)) = x(t) \cdot x\end{split}
\end{equation*}
\sphinxAtStartPar
beschrieben wird und diese Funktion explizit vom Funktionsargument \(t \in I\) abhängt.



\sphinxAtStartPar
2. Die gewöhnliche Differentialgleichung
\begin{equation*}
\begin{split}2t\cdot \dot{x}(t) = x(t)\cdot t \quad \forall t \in I\end{split}
\end{equation*}
\sphinxAtStartPar
ist hingegen \sphinxstylestrong{autonom}, da die Gleichung in folgende explizite Form überführt werden kann
\begin{equation*}
\begin{split}\dot{x}(t) = \frac{1}{2} x(t) \quad \forall t \in I\end{split}
\end{equation*}
\sphinxAtStartPar
und somit die rechte Seite der Gleichung durch die Funktion
\begin{equation*}
\begin{split}F(t,x(t)) = \frac{1}{2}x(t)\end{split}
\end{equation*}
\sphinxAtStartPar
beschrieben wird, welche nicht explizit vom Funktionsargument \(t \in I\) abhängt.



\sphinxAtStartPar
3. Im Fall der gewöhnlichen Differentialgleichung
\begin{equation*}
\begin{split}2x'(t) = x(t)\cdot \sin(g(t)) \quad \forall t \in I\end{split}
\end{equation*}
\sphinxAtStartPar
können wir für beliebige Funktionen \(g \colon I \rightarrow \R\) \sphinxstylestrong{nicht entscheiden}, ob sie autonom ist wenn keine konkrete Form der Funktion \(g\) gegeben ist.
\end{sphinxadmonition}


\subsection{Anfangswertprobleme}
\label{\detokenize{ode/repetition:anfangswertprobleme}}
\sphinxAtStartPar
Um gewöhnliche Differentialgleichungen zu lösen, betrachtet man in der Regel sogenannte Anfangswertprobleme.
Hierbei wählt man einen ausgezeichneten Zeitpunkt \(t_0\in I\) aus dem Zeitintervall \(I\), an welchem man die Lösung explizit durch einen Anfangswert \(x_0\in U\) vorgibt.
Dieses Vorgehen wird in der folgenden Definition nochmal kurz wiederholt.
\label{ode/repetition:def:anfangswertproblem}
\begin{sphinxadmonition}{note}{Definition 1.3}



\sphinxAtStartPar
Sei ein gewöhnliches Differentialgleichungssystem 1. Ordnung wie in {\hyperref[\detokenize{ode/repetition:def:DGL}]{\sphinxcrossref{Definition 1.1}}} gegeben, wobei \(I \times U \subset \R_0^+ \times \R^n\) den erweiterten Phasenraum des Systems bezeichnet.
Sei außerdem \(t_0 \in I\) ein Anfangszeitpunkt und \(x_0 \in U\) der zugehörige Anfangszustand.

\sphinxAtStartPar
Dann nennen wir das Gleichungssystem
\begin{equation}\label{equation:ode/repetition:eq:AWP}
\begin{split}\dot{x}(t) &= F(t, x(t))\quad\forall t\in I, \\
x(t_0) &= x_0\end{split}
\end{equation}
\sphinxAtStartPar
\sphinxstylestrong{Anfangswertproblem} des gewöhnlichen Differentialgleichungssystems.
Sofern nicht explizit angegeben werden wir im Folgenden annehmen, dass ohne Beschränkung der Allgemeinheit \(t_0=0\) gilt.
\end{sphinxadmonition}

\sphinxAtStartPar
Die explizite Wahl des Anfangszeitpunkts und \sphinxhyphen{}zustands erlaubt es erst eine gewöhnliche Differentialgleichung eindeutig zu lösen.
Ohne diese zusätzlichen Informationen könnte man lediglich Funktionenscharen als Lösungsmenge angeben.
Dies wird durch das folgende Beispiel nochmal dargestellt.
\label{ode/repetition:example-6}
\begin{sphinxadmonition}{note}{Example 1.4}



\sphinxAtStartPar
Wir betrachten eine sehr einfache gewöhnliche Differentialgleichung erster Ordnung, die sich explizit in folgender Form schreiben lässt:
\begin{equation*}
\begin{split}x'(t) = x(t) \quad \forall t \in \R.\end{split}
\end{equation*}
\sphinxAtStartPar
Man sieht leicht ein, dass Lösungen dieser Differentialgleichung Funktionen \(x \colon \R \rightarrow \R\) von der Form
\begin{equation*}
\begin{split}x(t) = c\cdot e^t\end{split}
\end{equation*}
\sphinxAtStartPar
für eine beliebige Konstante \(c \in \R\) sein müssen.
Um diese Funktionenschar weiter einzuschränken und eine eindeutige Lösung zu erhalten, müssen wir noch Anfangswertbedindungen hinzunehmen.
Hierzu reicht es eine ausgewiesene Stelle \(t_0 \in \R\) und einen Funktionswert \(x_0 = x(t_0)\) festzulegen.

\sphinxAtStartPar
Wählen wir beispielsweise \(t_0 = 0\) und \(x_0 = x(0) = 2\), so erhalten wir als eindeutige Lösung der gewöhnlichen Differentialgleichung die Funktion
\begin{equation*}
\begin{split}x(t) = 2\cdot e^t.\end{split}
\end{equation*}
\sphinxAtStartPar
Wir sehen also, dass durch das Festlegen eines Anfangswert die unbekannte Konstante \(c \in \R\) als \(c=2\) eindeutig bestimmt wurde.
\end{sphinxadmonition}


\subsection{Existenz und Eindeutigkeit einer Lösung}
\label{\detokenize{ode/repetition:existenz-und-eindeutigkeit-einer-losung}}
\sphinxAtStartPar
Nicht jede gewöhnliche Differentialgleichung ist im Allgemeinen lösbar oder besitzt eindeutige Lösungen, wie das folgende Beispiel belegt.
\label{ode/repetition:example-7}
\begin{sphinxadmonition}{note}{Example 1.5}



\sphinxAtStartPar
Wir wollen im folgenden zwei Beispiele von autonomen, gewöhnlichen Differentialgleichungen erster Ordnung diskutieren, für die entweder die Existenz oder die Eindeutigkeit von Lösungen nicht gegeben ist.

\sphinxAtStartPar
1. Die gewöhnliche Differentialgleichung
\begin{equation*}
\begin{split}e^{x'(t)} \equiv 0 \quad \forall t \in \R\end{split}
\end{equation*}
\sphinxAtStartPar
besitzt keine Lösung, da die Exponentialfunktion strikt positiv ist und es somit keine Funktion \(y \colon \R \rightarrow \R\) gibt, so dass die obige Gleichung erfüllt werden kann.

\sphinxAtStartPar
2. Die gewöhnliche Differentialgleichung
\begin{equation*}
\begin{split}x'(t)(1-x'(t)) \equiv 0 \quad \forall t \in \R\end{split}
\end{equation*}
\sphinxAtStartPar
besitzt auf Grund ihrer Symmetrieeigenschaften zwei unterschiedliche Funktionenscharen als Lösung, nämlich
\begin{equation*}
\begin{split}x_1(t) = c \quad \text{ und } \quad x_2(t) = t + c \quad \forall t \in \R,\end{split}
\end{equation*}
\sphinxAtStartPar
wobei \(c \in \R\) eine beliebige Konstante darstellt.
\end{sphinxadmonition}

\sphinxAtStartPar
Die wichtigste Eigenschaft für die Existenz und Eindeutigkeit von Lösungen gewöhnlicher Differentialgleichungen ist die \sphinxstylestrong{(lokale) Lipschitzstetigkeit} der rechten Seite \(F \colon I \times U \rightarrow U\).
Diese wollen wir der Vollständigkeit halber im Folgenden definieren.

\begin{sphinxShadowBox}
\sphinxstylesidebartitle{Rudolf Lipschitz}

\sphinxAtStartPar
\sphinxhref{https://de.wikipedia.org/wiki/Rudolf\_Lipschitz}{Rudolf Otto Sigismund Lipschitz} (Geboren 14. Mai 1832 in Königsberg i. Pr.; Gestorben 7. Oktober 1903 in Bonn) war ein deutscher Mathematiker und Hochschullehrer. Er betreute die Doktorarbeit von \sphinxhref{https://en.wikipedia.org/wiki/Felix\_Klein}{Felix Klein}, weswegen der österreichische Mathematiker \sphinxhref{https://www.math.fau.de/angewandte-mathematik-1/mitarbeiter/prof-dr-martin-burger/}{Martin Burger} in direkter Linie im akademischen Stammbaum von Lipschitz abstammt, siehe \sphinxhref{https://genealogy.math.ndsu.nodak.edu/index.php}{Mathematics Genealogy Project}.
\end{sphinxShadowBox}
\label{ode/repetition:definition-8}
\begin{sphinxadmonition}{note}{Definition 1.4 ((Lokale) Lipschitzstetigkeit)}



\sphinxAtStartPar
Sei \(F \colon G \to \R^n\) eine Funktion mit dem erweiterten Phasenraum \(G \, \coloneqq \, I \times U \subset \R\times\R^n\).
Man sagt, dass \(F\) in \(G\) einer \sphinxstylestrong{globalen Lipschitz\sphinxhyphen{}Bedingung} genügt (bezüglich der Variablen \(x \in U\)) mit der Lipschitz\sphinxhyphen{}Konstanten \(L\geq0\), wenn gilt
\begin{equation*}
\begin{split}\Vert F(t,x) - F(t,\widetilde{x}) \Vert \leq L \Vert x-\widetilde{x}\Vert\quad\text{ für alle }(t,x), (t,\widetilde{x})\in G\,.\end{split}
\end{equation*}
\sphinxAtStartPar
Man sagt, \(F\) genüge in \(G\) einer \sphinxstylestrong{lokalen Lipschitz\sphinxhyphen{}Bedingung}, falls jeder Punkt \((t,x)\in G\) im erweiterten Phasenraum eine Umgebung \(V\) besitzt, sodass \(F\) in \(G\cap V\) einer Lipschitzbedingung mit einer gewissen (von \(V\) abhängigen) Konstanten \(L\in\R_0^+\) genügt.
\end{sphinxadmonition}

\sphinxAtStartPar
Für die \sphinxstylestrong{(lokale) Existenz von Lösungen} haben wir in Kapitel 8.4 {[}\hyperlink{cite.references:id12}{Ten21}{]} den Satz von Picard\sphinxhyphen{}Lindelöf formuliert, den wir im Folgenden wiederholen werden.
\label{ode/repetition:thm:piclindlokal}
\begin{sphinxadmonition}{note}{Theorem 1.1 (Lokaler Existenzsatz nach Picard–Lindelöf)}



\sphinxAtStartPar
Sei \(F\colon G\to\R^n\) eine stetige Funktion mit erweitertem Phasenraum \(G \coloneqq I \times U \subset \R\times\R^n\), die lokal Lipschitz\sphinxhyphen{}stetig auf \(G\) bezüglich der \(x\)\sphinxhyphen{}Variablen ist.
Dann existiert zu jedem Anfangswert \((t_0,x_0) \in G\) ein \(\varepsilon>0\), sowie genau eine Lösung
\begin{equation*}
\begin{split}\phi \colon \left[t_0-\varepsilon, t_0+\varepsilon\right] \to \R^n\end{split}
\end{equation*}
\sphinxAtStartPar
der gewöhnlichen Differentialgleichung
\begin{equation*}
\begin{split}\dot{x}(t) \ = \ F(t,x(t))\end{split}
\end{equation*}
\sphinxAtStartPar
unter der Anfangsbedingung \(\phi(t_0)=x_0\).
\end{sphinxadmonition}

\begin{sphinxShadowBox}
\sphinxstylesidebartitle{Ernst Lindelöf}

\sphinxAtStartPar
\sphinxhref{https://en.wikipedia.org/wiki/Ernst\_Leonard\_Lindel\%C3\%B6f}{Ernst Leonard Lindelöf} (Geboren 7. März 1870 in Helsingfors (Helsinki), Großfürstentum Finnland; Gestorben 4. Juni 1946 in Helsinki) war ein finnischer Mathematiker.
\end{sphinxShadowBox}

\begin{sphinxShadowBox}
\sphinxstylesidebartitle{Émile Picard}

\sphinxAtStartPar
\sphinxhref{https://de.wikipedia.org/wiki/\%C3\%89mile\_Picard}{Charles Émile Picard} (Geboren 24. Juli 1856 in Paris; Gestorben 11. Dezember 1941 ebenda) war ein französischer Mathematiker.
\end{sphinxShadowBox}

\begin{sphinxadmonition}{note}
\sphinxAtStartPar
Proof. Siehe Kapitel 12, Satz 4 Kapitel 8.4 {[}\hyperlink{cite.references:id4}{For17}{]}
\end{sphinxadmonition}

\sphinxAtStartPar
Bisher haben wir nur die Existenz und Eindeutigkeit von Lösungen gewöhnlicher Differentialgleichungen in lokalen Intervallen betrachtet.
Unter den strengeren Voraussetzungen einer rechten Seite \(F\) der gewöhnlichen Differentialgleichung, die einer globalen Lipschitzbedingung genügt, lässt sich jedoch eine \sphinxstylestrong{globale Existenzaussage} formulieren, die besonders für konkrete Anwendungen sehr praktisch ist.
\label{ode/repetition:satz:picardlindeloef}
\begin{sphinxadmonition}{note}{Theorem 1.2 (Globaler Existenzsatz nach Picard\sphinxhyphen{}Lindelöf)}



\sphinxAtStartPar
Sei \(F\colon G\to\R^n\) eine stetige Funktion mit erweitertem Phasenraum \(G \, \coloneqq \, I \times U \subset \R\times\R^n\), die eine globale Lipschitzbedingung auf \(G\) bezüglich der \(x\)\sphinxhyphen{}Variablen erfüllt.
Dann existiert zu jedem Anfangswert \((t_0,x_0) \in G\) eine globale Lösung
\begin{equation*}
\begin{split}\phi \colon I \to \R^n\end{split}
\end{equation*}
\sphinxAtStartPar
der gewöhnlichen Differentialgleichung
\begin{equation*}
\begin{split}\dot{x}(t) \ = \ F(t,x(t))\end{split}
\end{equation*}
\sphinxAtStartPar
unter der Anfangsbedingung \(\phi(t_0)=x_0\).
Es existieren außerdem keine weiteren (lokalen) Lösungen.
\end{sphinxadmonition}

\begin{sphinxadmonition}{note}
\sphinxAtStartPar
Proof. Siehe Kapitel 2.3 {[}\hyperlink{cite.references:id5}{Kna13}{]}
\end{sphinxadmonition}
\label{ode/repetition:cor:eindeutigkeitlinear}
\begin{sphinxadmonition}{note}{Corollary 1.1}



\sphinxAtStartPar
Das Anfangswertproblem jedes \sphinxstylestrong{linearen} gewöhnlichen Differentialgleichungssystems 1. Ordnung hat eine eindeutige globale Lösung.
\end{sphinxadmonition}

\begin{sphinxadmonition}{note}
\sphinxAtStartPar
Proof. Siehe Theorem 2.25, Kapitel 2.3 {[}\hyperlink{cite.references:id5}{Kna13}{]}
\end{sphinxadmonition}


\subsection{Lösungen von linearen Differentialgleichungssystemen}
\label{\detokenize{ode/repetition:losungen-von-linearen-differentialgleichungssystemen}}\label{\detokenize{ode/repetition:s-lineare-dglsysteme}}
\sphinxAtStartPar
Analog zu Kapitel 8 in {[}\hyperlink{cite.references:id12}{Ten21}{]} wollen wir uns mit Lösungen für \sphinxstylestrong{homogene lineare Differentialgleichungen} beschäftigen, jedoch dieses Mal nicht im skalaren Fall \(n=1\), sondern für ein Anfangswertproblem von der Form
\begin{equation}\label{equation:ode/repetition:eq:linhomdglsystem}
\begin{split}\dot{x}(t) &= A x(t), \quad \forall t \in I \subset \R^+_0, \\
x(t_0) &= x_0 \in U \subset \R^n.\end{split}
\end{equation}
\sphinxAtStartPar
Wir bemerken hierbei, dass im Gegensatz zum skalaren Fall hier die Koeffizientenmatrix \(A \in \C^{n\times n}\) nicht von der Zeit abhängt, wir also ein autonomes Differentialgleichungssystem betrachten.

\sphinxAtStartPar
Bevor wir Lösungen von \eqref{equation:ode/repetition:eq:linhomdglsystem} angeben, wollen wir ein hilfreiches Funktionalkalkül einführen, dass die Notation im Fall von Differentialgleichungssystemen erleichtert.
\label{ode/repetition:definition-12}
\begin{sphinxadmonition}{note}{Definition 1.5 (Matrixexponential)}



\sphinxAtStartPar
Sei \(n \in \N\) und \(A \in \C^{n \times n}\) eine beliebige quadratische Matrix.
Das \sphinxstylestrong{Matrixexponential} \(e^A\) von \(A\), ist diejenige \(n\times n\)\sphinxhyphen{}Matrix, welche durch die folgende Potenzreihe definiert ist:
\begin{equation*}
\begin{split}e^A \equiv \exp(A) \ \coloneqq \ \sum_{k=0}^\infty \frac{A^k}{k!} = I_n + A + \frac{A^2}{2} + \frac{A^3}{6} + \ldots.\end{split}
\end{equation*}
\sphinxAtStartPar
Analog zur gewöhnlichen Exponentialfunktion konvergiert die Reihe für alle \(A \in \C^{n \times n}\), woraus die Wohldefiniertheit der Definition folgt.
Für den Spezialfall \(n=1\) entspricht das Matrixexponential der gewöhnlichen Exponentialfunktion.
\end{sphinxadmonition}
\label{ode/repetition:rem:matrixexponentialregeln}
\begin{sphinxadmonition}{note}{Remark 1.3 (Rechenregeln für das Matrixexponential)}



\sphinxAtStartPar
Für das Matrixexponential gelten die gleichen Rechenregeln wie für die gewöhnliche Exponentialfunktion, wie zum Beispiel:
\begin{itemize}
\item {} 
\sphinxAtStartPar
\(e^{tA}e^{sA} = e^{(t+s)A}, \quad\) für \(s,t \in \R\)

\item {} 
\sphinxAtStartPar
\(\frac{d}{dt} e^{tA} = Ae^{tA}, \quad\) für \(t \in \R\)

\item {} 
\sphinxAtStartPar
\( e^{D} = \operatorname{diag}(e^{a_1}, \ldots, e^{a_n})\) ist Diagonalmatrix für eine Diagonalmatrix \(D = \operatorname{diag}(a_1, \ldots, a_n)\).

\end{itemize}
\end{sphinxadmonition}

\sphinxAtStartPar
Folgendes Lemma stellt einen interessanten Zusammenhang des Matrixexponentials zur Spektraltheorie her.
\label{ode/repetition:lem:mpotew}
\begin{sphinxadmonition}{note}{Lemma 1.1 (Eigenwerte des Matrixexponentials)}



\sphinxAtStartPar
Sei \(A \in \C^{n\times n}\) eine beliebige quadratische Matrix und sei \(\lambda \in \C\) ein Eigenwert von \(A\) zum
Eigenvektor \(v \in \C^n\).
Dann ist der Vektor \(v\) auch Eigenvektor des Matrixexponentials \(e^A\) zum zugehörigen Eigenwert \(e^\lambda\).
\end{sphinxadmonition}

\begin{sphinxadmonition}{note}
\sphinxAtStartPar
Proof. In der Hausaufgabe zu zeigen.
\end{sphinxadmonition}

\sphinxAtStartPar
Mit Hilfe des Matrixexponentials lässt sich die Lösung des homogenen linearen Differentialgleichungssystems \eqref{equation:ode/repetition:eq:linhomdglsystem} kompakt angeben, wie uns folgendes Lemma zeigt.
\label{ode/repetition:lemma-15}
\begin{sphinxadmonition}{note}{Lemma 1.2}



\sphinxAtStartPar
Sei \(n\in \N\), \(I \subset \R^+_0\) und \(A \in \C^{n\times n}\) eine beliebige quadratische Matrix.
Das Anfangswertproblem \eqref{equation:ode/repetition:eq:linhomdglsystem} hat die eindeutige Lösung
\begin{equation*}
\begin{split}x(t) = e^{A(t-t_0)}x_0, \quad \forall t \in I.\end{split}
\end{equation*}\end{sphinxadmonition}

\begin{sphinxadmonition}{note}
\sphinxAtStartPar
Proof. Wir zeigen zunächst, dass die Lösung \(x(t)\) die Anfangswertbedingung erfüllt:
\begin{equation*}
\begin{split}x(t_0) = e^{A(t_0-t_0)}x_0 = e^0x_0 = I_n x_0.\end{split}
\end{equation*}
\sphinxAtStartPar
Um zu zeigen, dass \(x(t)\) das lineare homogene Differentialgleichungssystem \eqref{equation:ode/repetition:eq:linhomdglsystem} löst, berechnen wir die entsprechende Zeitableitung als
\begin{equation*}
\begin{split}\dot{x}(t) = \frac{d}{dt}(e^{A(t-t_0)}x_0) = A \cdot e^{A(t-t_0)}x_0 = A x(t), \quad \forall t \in I.\end{split}
\end{equation*}
\sphinxAtStartPar
Vergleichen wir die linke und rechte Seite dieser Gleichung so erkennen wir, dass \(x(t)\) in der Tat eine Lösung des Differentialgleichungssystems ist.

\sphinxAtStartPar
Nach {\hyperref[\detokenize{ode/repetition:cor:eindeutigkeitlinear}]{\sphinxcrossref{Corollary 1.1}}} ist die Lösung eindeutig, da es sich um ein lineares Differentialgleichungssystem 1. Ordnung handelt.
\end{sphinxadmonition}

\sphinxAtStartPar
Im Allgemeinen kann man bei linearen Differentialgleichungssystemen nicht davon ausgehen, dass diese in der einfachsten Form wie in \eqref{equation:ode/repetition:eq:linhomdglsystem} vorliegen.
Außerdem ist die konkrete Berechnung des Matrixexponentials zur Bestimmung einer Lösungsfunktion \(x(t)\) in der Regel ungeeignet.
Hierzu wollen wir die abschließende Bemerkung machen.
\label{ode/repetition:remark-16}
\begin{sphinxadmonition}{note}{Remark 1.4}



\sphinxAtStartPar
1. Zur Berechnung einer konkreten Lösung \(x(t)\) des linearen homogenen Differentialgleichungssystems \eqref{equation:ode/repetition:eq:linhomdglsystem} bietet es sich an, die \sphinxstylestrong{Jordansche Normalform} \(J = SAS^{-1}\) von \(A\) aus Kapitel 2.7 in {[}\hyperlink{cite.references:id12}{Ten21}{]} auszunutzen, da für diese das Matrixexponential wie folgt berechnet werden kann:
\begin{equation*}
\begin{split}e^{tA} &=  \sum_{k=0}^\infty \frac{(t A)^k}{k!} = \sum_{k=0}^\infty \frac{(tS^{-1}JS)^k}{k!} 
\\&= 
S^{-1} \sum_{k=0}^\infty \frac{(tJ)^k}{k!} S = S^{-1} e^{tJ}S 
\\&= S^{-1} e^{t(D+N)}S = S^{-1} e^{tD} e^{tN} S\end{split}
\end{equation*}
\sphinxAtStartPar
für eine Transformationsmatrix \(S \in \C^{n \times n}\), eine Diagonalmatrix \(D \in \C^{n \times n}\) mit den Eigenwerten von \(A\) und einer nilpotenten Matrix \(N \in \C^{n \times n}\), für die die Reihendarstellung des zugehörigen Matrixexponentials nach endlich vielen Summanden (entsprechend dem Nilpotenzindex von \(N\)) abbricht.

\sphinxAtStartPar
2. Ist das vorliegende lineare Differentialgleichungssystem \sphinxstylestrong{inhomogen}, das heißt für eine stetige Störfunktion \(b \colon I \rightarrow \R^n\) von der Form
\begin{equation}\label{equation:ode/repetition:eq:lininhomdglsystem}
\begin{split}\dot{x}(t) &= A x(t) + b(t), \quad \forall t \in I \subset \R^+_0, \\
x(t_0) &= x_0 \in U \subset \R^n,\end{split}
\end{equation}
\sphinxAtStartPar
so lässt sich über die Variation der Konstanten aus Kapitel 8.2 in {[}\hyperlink{cite.references:id12}{Ten21}{]} eine eindeutige Lösung des Anfangswertproblems \eqref{equation:ode/repetition:eq:lininhomdglsystem} angeben als
\begin{equation*}
\begin{split}x(t) = e^{tA}x_0 + \int_0^t e^{(t-s)A}b(s) \, \mathrm{d}s.\end{split}
\end{equation*}
\sphinxAtStartPar
3. Im Falle eines homogenen, linearen Differentialgleichungssystems, das \sphinxstylestrong{nicht autonom} ist, das heißt die Koeffizientenmatrix \(A = A(t)\) ist zeitabhängig, können wir nicht mehr die Spektraltheorie zur konkreten Berechnung von Lösungen nutzen.
Formal lassen sich dennoch Lösungen als sogenanntes \sphinxstylestrong{zeitgeordnetes Produkt} angeben, was jedoch den Rahmen dieser Vorlesung sprengen würde.
\end{sphinxadmonition}


\section{Phasenflüsse und Phasenportraits}
\label{\detokenize{ode/fluesse:phasenflusse-und-phasenportraits}}\label{\detokenize{ode/fluesse:s-fluesse}}\label{\detokenize{ode/fluesse::doc}}
\sphinxAtStartPar
In diesem Abschnitt führen wir die grundlegende mathematischen Konzepte zur Analyse von kontinuierlichen dynamischen Systemen ein. Insbesondere diskutieren wir Flüsse als Lösungen von autonomen gewöhnlichen Differentialgleichungen und definieren sogenannte Phasenportraits, die es uns erlauben dynamische Systeme geometrisch zu interpretieren.


\subsection{Phasenflüsse}
\label{\detokenize{ode/fluesse:phasenflusse}}
\sphinxAtStartPar
Wir beginnen zunächst damit eine Klasse von Funktionen einzuführen, welche die Beschreibung zeitabhängiger Systeme vereinfacht.
Die folgende Definition ist zunächst sehr allgemein für beliebige dynamische Systreme gehalten und wird später im Kontext von konkreten Anwendungsbeispielen spezieller diskutiert.
\label{ode/fluesse:def:Fluss}
\begin{sphinxadmonition}{note}{Definition 1.6 (Fluss und dynamisches System)}



\sphinxAtStartPar
Sei \(U \subset \R^n\) eine offene Teilmenge und \(I=\R^+_0\), dann heißt eine Abbildung \(\Phi:I\times U\rightarrow U\) \sphinxstylestrong{(Phasen\sphinxhyphen{})Fluss}, falls gilt,
\begin{enumerate}
\sphinxsetlistlabels{\arabic}{enumi}{enumii}{}{.}%
\item {} 
\sphinxAtStartPar
\(\Phi(0, x) = x\) für alle \(x\in U\),

\item {} 
\sphinxAtStartPar
\(\Phi(t, \Phi(s,x)) = \Phi(s + t, \Phi(0, x)) = \Phi(s + t, x)\) für alle \(x\in U\) und alle \(s,t\in I\).

\end{enumerate}

\sphinxAtStartPar
Das Tripel \((I, U, \Phi)\) heißt \sphinxstylestrong{dynamisches System}.

\sphinxAtStartPar
Zur Vereinfachung der Notation schreibt man häufig auch das erste Argument des Flusses als Index wie folgt
\begin{equation*}
\begin{split}\Phi_t(x) \coloneqq \Phi(t, x).\end{split}
\end{equation*}\end{sphinxadmonition}

\sphinxAtStartPar
Für die Analyse von dynamischen Systemen beschreibt der Fluss die Bewegung im Phasenraum in Abhängigkeit zur Zeit.
Im Folgenden wollen wir speziell die \sphinxstylestrong{Lösungen einer autonomen DGL}
\begin{equation*}
\begin{split}\dot{x} = F(x).\end{split}
\end{equation*}
\sphinxAtStartPar
für \(F\in C^1(U;\R^n)\) als Fluss interpretieren.
Hierbei soll das zweite Argument des Flusses jeweils den Anfangswert \(x_0\in U\) angeben und \(\Phi(x_0) = \Phi(\cdot, x_0)\) dann eine Lösung der DGL sein, d.h.,
\begin{equation*}
\begin{split}\frac{\d}{\d t} \Phi(x_0) = F(\Phi(x_0))\end{split}
\end{equation*}
\sphinxAtStartPar
So werden durch den Phasenfluss die Lösungen des dynamischen Systems in Abhängigkeit vom Anfangszustand angegeben.
Im folgenden Beispiel betrachten wir den \sphinxstylestrong{Fluss eines Vektorfeldes}, das die rechte Seite eines gewöhnlichen Differentialgleichungssystems beschreibt.
\label{ode/fluesse:example-1}
\begin{sphinxadmonition}{note}{Example 1.6}



\sphinxAtStartPar
Sei \(I\subset \R_0^+\) ein offenes Zeitintervall.
Wir interessieren uns für Lösungen des autonomen gewöhnlichen Differentialgleichungssystems
\begin{equation*}
\begin{split}\dot{\vec{x}}(t) = F(\vec{x}) \quad \forall t\in I,\end{split}
\end{equation*}
\sphinxAtStartPar
dessen rechte Seite durch das Vektorfeld \(F \colon \R^2 \rightarrow \R^2\) mit \(F(x,y) \, \coloneqq \, (y, -x)\) gegeben ist.
Abbildung \textbackslash{}xxx illustriert das Vektorfeld in \(\R^2\).

\sphinxAtStartPar
Wir wollen den Fluss des Vektorfeldes \(F\) angeben, der die Bewegung entlang der Lösungskurven der durch das Vektorfeld gegebenen gewöhnlichen Differentialgleichung beschreibt.
Dieser ist gegeben durch
\begin{equation*}
\begin{split}\Phi(t,(x,y)) = (\cos(t)x + \sin(t)y, -\sin(t)x + \cos(t)y).\end{split}
\end{equation*}
\sphinxAtStartPar
Das die Funktion \(\Phi \colon I \times \R^2 \rightarrow \R^2\) ein Fluss ist, lässt sich leicht verifizieren durch Nachrechnen der beiden Eigenschaften eines Flusses aus Definition \textbackslash{}ref.

\sphinxAtStartPar
1. Es gilt \(\Phi(0, (x,y)) = (x,y)\) für beliebige Paare \((x,y) \in \R^2\), da
\begin{equation*}
\begin{split}\Phi(0, (x,y)) = (1\cdot x + 0\cdot y, - 0 \cdot x + 1 \cdot y) = (x,y).\end{split}
\end{equation*}
\sphinxAtStartPar
2. Es gilt \(\Phi(t, \Phi(s,(x,y)) = \Phi(s + t, (x,y))\) für beliebige Paare \((x,y) \in \R^2\) und Zeitpunkte \(s,t \in I\), da wegen der Additionstheoreme von Sinus und Cosinus gilt
\begin{equation*}
\begin{split}\Phi(t, \Phi(s,(x,y))) &= \Phi(t, (\cos(s)x + \sin(s)y, -\sin(s)x + \cos(s)y)) \\
&= [\cos(t)(\cos(s)x + \sin(s)y) + \sin(t)(-\sin(s)x + \cos(s)y), \\
& \ \ -\sin(t)(\cos(s)x + \sin(s)y) + \cos(t)(-\sin(s)x + \cos(s)y)]\\
&= \ [ (\cos(t)\cos(s) - \sin(t)\sin(s))x + (\cos(t)\sin(s) + \sin(t)\cos(s))y, \\
& \quad (-\sin(t)\cos(s) - \cos(t)\sin(s))x + (\cos(t)\cos(s) - \sin(t)\sin(s))y ] \\
&= (\cos(s+t)x + \sin(s+t)y, -\sin(s+t)x + \cos(s+t)y).\end{split}
\end{equation*}
\sphinxAtStartPar
Nun verfizieren wir noch, dass der Fluss tatsächlich Lösungen des gewöhnlichen Differentialgleichungssystems realisiert.
Es gilt
\begin{equation*}
\begin{split}\dot{\Phi}(t, (x,y)) &= \frac{d}{dt}(\cos(t)x + \sin(t)y, -\sin(t)x + \cos(t)y) 
\\&=
(-\sin(t)x + \cos(t)y, -\cos(t)x - \sin(t)y) 
\\&= 
F(\Phi(t,(x,y)), \quad \forall t \in I, (x,y) \in U.\end{split}
\end{equation*}
\sphinxAtStartPar
Offensichtlich ist der Fluss \(\Phi \colon I \times \R^2 \rightarrow \R^2\) Lösung des gewöhnlichen Differentialgleichungssystems.
\end{sphinxadmonition}


\subsection{Lokale Flüsse}
\label{\detokenize{ode/fluesse:lokale-flusse}}
\sphinxAtStartPar
Nach dem Satz von Picard\sphinxhyphen{}Lindelöf {\hyperref[\detokenize{ode/repetition:thm:piclindlokal}]{\sphinxcrossref{Theorem 1.1}}} wissen wir, dass für jeden Anfangswert \(x_0\in U\) ein \(\epsilon(x_0)>0\) existiert, so dass es lokal eine eindeutige Lösung \(\phi: [-\epsilon(x_0), \epsilon(x_0)]\) gibt, falls die rechte Seite \(F\) lokal Lipschitzstetig bezüglich der \(y\)\sphinxhyphen{}Variablen ist.
In diesem Fall müssen wir das Zeitintervall \(I(x_0)=[-\epsilon(x_0), \epsilon(x_0)]\) wählen und können also nicht wie in {\hyperref[\detokenize{ode/fluesse:def:Fluss}]{\sphinxcrossref{Definition 1.6}}} auf ganz \(I = \R^+_0\) als Zeitintervall arbeiten.
Stattdessen können wir nur Tupel der Form \((t, x_0) \in I(x_0) \times \{x_0\}\) betrachten, wobei \(x_0\in U\) fixiert ist und \(t\) aus dem lokalen Existenzintervall \(I(x_0)\) gewählt werden kann.

\sphinxAtStartPar
Diese Einschränkung führt uns auf den Begriff des \sphinxstylestrong{lokalen Phasenflusses}.
\label{ode/fluesse:def:LokFluss}
\begin{sphinxadmonition}{note}{Definition 1.7 (Lokaler Fluss)}



\sphinxAtStartPar
Sei \(U \subset \R^n\) eine offene Teilmenge und der erweiterte Phasenraum \(G\subset \R^+_0\times U\) sei gegeben als
\begin{equation*}
\begin{split}G = \bigcup_{x_0\in U} I(x_0) \times \{x_0\},\end{split}
\end{equation*}
\sphinxAtStartPar
wobei \(0\in I(x_0)\) für jedes \(x_0\in U\) gelte.

\sphinxAtStartPar
Dann heißt eine Abbildung \(\Phi: G\rightarrow U\) \sphinxstylestrong{lokaler (Phasen\sphinxhyphen{})Fluss}, falls
\begin{enumerate}
\sphinxsetlistlabels{\arabic}{enumi}{enumii}{}{.}%
\item {} 
\sphinxAtStartPar
\(\Phi(0,x) = x\) für alle \(x\in U\),

\item {} 
\sphinxAtStartPar
\(\Phi(t, \Phi(s, x)) = \Phi(s+t, x)\) für alle \(x\in U\) und alle \(s,t\) mit \(s, s+t\in I(x)\) und \(t\in I(\Phi(s,x))\).

\end{enumerate}
\end{sphinxadmonition}

\sphinxAtStartPar
Im nächsten Lemma werden wir sehen, dass die Lösung eines autonomen gewöhnlichen Differentialgleichungssystems tatsächlich als solch ein lokaler Fluss interpretiert werden kann.
In diesem Fall spircht man auch vom \sphinxstylestrong{Fluss einer Differentialgleichung}.
\label{ode/fluesse:lemma-3}
\begin{sphinxadmonition}{note}{Lemma 1.3}



\sphinxAtStartPar
Sei \(U\subset\R^n\) eine offene Teilmenge und es sei \(F \colon U \rightarrow \R^n\) eine lokal Lipschitzstetige Abbildung.
Dann existieren Intervalle \(I(x_0)\), so dass es für den erweiterten Phasenraum
\begin{equation*}
\begin{split}G = \bigcup_{x_0\in U} I(x_0)\times\{x_0\}\end{split}
\end{equation*}
\sphinxAtStartPar
eine Funktion \(\Phi \colon G\rightarrow \R^n\) gibt, mit folgenden Eigenschaften
\begin{enumerate}
\sphinxsetlistlabels{\arabic}{enumi}{enumii}{}{.}%
\item {} 
\sphinxAtStartPar
\(\frac{\d}{\d t} \Phi(t, x_0) = F(\Phi(t, x_0))\) für alle \((t,x_0)\in G\),

\item {} 
\sphinxAtStartPar
\(\Phi\) ist ein lokaler Fluss auf \(G\).

\end{enumerate}
\end{sphinxadmonition}

\begin{sphinxadmonition}{note}
\sphinxAtStartPar
Proof. Da die rechte Seite \(F\) des autonomen gewöhnlichen Differentialgleichungssystems nach Voraussetzung lokal Lipschitzstetig ist, existiert nach dem Satz von Picard\sphinxhyphen{}Lindelöf {\hyperref[\detokenize{ode/repetition:thm:piclindlokal}]{\sphinxcrossref{Theorem 1.1}}} für jedes \(x_0\in U\) ein \(\epsilon(x_0)>0\), so dass eine Lösung des Differentialgleichungssystems \(\Phi_{x_0} \colon [-\epsilon(x_0),\epsilon(x_0)] \rightarrow U\) mit dem Anfangswert \(x_0\) existiert, d.h.,
\begin{equation*}
\begin{split}\dot{\Phi}_{x_0}(t) &= F(\Phi_{x_0}(t)) \quad \forall t \in [-\epsilon(x_0),\epsilon(x_0)],\\
\Phi_{x_0}(0) &= x_0.\end{split}
\end{equation*}
\sphinxAtStartPar
Daher können wir den erweiterten Phasenraum als
\begin{equation*}
\begin{split}G = \bigcup_{x_0\in U} [-\epsilon(x_0),\epsilon(x_0)] \times\{x_0\}\end{split}
\end{equation*}
\sphinxAtStartPar
wählen und die Abbildung \(\Phi\) als Einschränkung auf die Funktionen \(\Phi_{x_0}\) so definieren, dass
\begin{equation*}
\begin{split}\frac{\d}{\d t} \Phi(t, x_0) &= \dot{\Phi}_{x_0}(t) = F(\Phi_{x_0}(t)) = F(\Phi(t, x_0))\\
\Phi(0, x_0) &= \Phi_{x_0}(0) = x_0\end{split}
\end{equation*}
\sphinxAtStartPar
für alle \((t, x_0)\in G\).
Damit haben wir sowohl die erste Aussage des Lemmas als auch die erste Flusseigenschaft aus {\hyperref[\detokenize{ode/fluesse:def:Fluss}]{\sphinxcrossref{Definition 1.6}}} gezeigt.

\sphinxAtStartPar
Die zweite Flusseigenschaft ist eine direkte Folgerung aus der Eindeutigkeit der Lösung des gewöhnlichen Differentialgleichungssystems.
Wir führen den Beweis trotzdem im Folgenden explizit aus.
Es sei \(x_0\in U, s\in [-\epsilon(x_0), \epsilon(x_0)]\) und zusätzlich \(t\) so gewählt, dass \(s+t \in [-\epsilon(x_0), \epsilon(x_0)]\) und \(t\in [-\epsilon(\Phi(s,x_0)), \epsilon(\Phi(s,x_0))]\).
Per Definition löst die Funktion
\begin{equation*}
\begin{split}\phi_1(\tau) \ \coloneqq \ \Phi(s + \tau, x_0)\end{split}
\end{equation*}
\sphinxAtStartPar
sowie auch die Funktion
\begin{equation*}
\begin{split}\phi_2(\tau) \ \coloneqq \ \Phi(\tau, \Phi(s,x_0))\end{split}
\end{equation*}
\sphinxAtStartPar
das gewöhnliche Differentialgleichungssystem auf dem Intervall \([t, \epsilon(x_0)]\), da \(\Phi\) eine Lösung ist.
Weiterhin wissen wir auf Grund der ersten Flusseigenschaft, dass
\begin{equation*}
\begin{split}\phi_1(0) = \Phi(s, x_0) = \Phi(0, \Phi(s, x_0)) = \phi_2(0).\end{split}
\end{equation*}
\sphinxAtStartPar
Somit stimmen also beide Funktionen an einem Punkt überein und sind somit schon auf dem gesamten Intervall \([t, \epsilon(x_0)]\) gleich, was eine direkte Folgerung aus dem Eindeutigkeitssatz 8.20 aus {[}\hyperlink{cite.references:id12}{Ten21}{]} ist.
Wir haben also insgesamt
\begin{equation*}
\begin{split}\Phi(s + \tau, x_0) = \phi_1(\tau) = \phi_2(\tau) = \Phi(\tau, \Phi(s,x_0))\end{split}
\end{equation*}
\sphinxAtStartPar
für jedes \(\tau\in [t, \epsilon(x_0)]\), was die zweite Flusseigenschaft aus {\hyperref[\detokenize{ode/fluesse:def:Fluss}]{\sphinxcrossref{Definition 1.6}}} zeigt.
\end{sphinxadmonition}


\subsection{Phasenporträts}
\label{\detokenize{ode/fluesse:phasenportrats}}
\sphinxAtStartPar
Die teilweise abstrakten Konzepte und Eigenschaften von Phasenflüssen aus den vorangegangenen Abschnitten werden wir im Folgenden mit einfachen geometrischen Anschauungen illustrieren.
Dafür benötigen wir zunächst die folgenden Definitionen.
\label{ode/fluesse:definition-4}
\begin{sphinxadmonition}{note}{Definition 1.8 (Phasenporträt)}



\sphinxAtStartPar
Es sei \(\Phi:G\rightarrow U\) ein Phasenfluss eines gewöhnlichen Differentialgleichungssystems für den erweiterten Phasenraum \(G = I \times U\subset \R^+_0\times \R^n\).
Dann können wir folgende Begriffe für den Fluss einführen:
\begin{itemize}
\item {} 
\sphinxAtStartPar
Für jedes \(x_0\in U\) heißt die Funktion \(t\mapsto \Phi(t, x_0)\) \sphinxstylestrong{Bahnkurve} durch \(x_0\).

\item {} 
\sphinxAtStartPar
Die Menge \(\mathcal{O}(x_0) := \{\Phi(t, x_0): (t, x_0)\in G\}\) heißt \sphinxstylestrong{Orbit} oder \sphinxstylestrong{Trajektorie} durch \(x_0\).

\item {} 
\sphinxAtStartPar
Ein Punkt \(x_0 \in U\) heißt \sphinxstylestrong{Ruhelage}, falls \(\mathcal{O}(x_0) = \{x_0\}\).

\item {} 
\sphinxAtStartPar
Ein Anfangswert \(x_0\in U\) heißt \sphinxstylestrong{periodisch} mit Periode \(T>0\), falls \(\Phi(T, x_0) = x_0\).

\end{itemize}

\sphinxAtStartPar
Wir nennen die Zerlegung des erweiterten Phasenraums \(G\) in Orbits ein \sphinxstylestrong{Phasenporträt} des dynamischen Systems \((I,U, \Phi)\).
\end{sphinxadmonition}

\sphinxAtStartPar
Phasenporträts erlauben es uns das charakteristische Verhalten kontinuierlicher dynamischer Systeme zu visualisieren und graphisch zu analysieren.
An ihnen lassen sich beispielsweise die Existenz und Stabilität von Fixpunkten und periodischen Orbits direkt erkennen.
Da ein Phasenporträt den gesamten Phasenraum zerlegt werden typischerweise nur einige charakteristische Orbits gezeichnet um die Übersichtlichkeit zu gewährleisten.
Aus dem gleichen Grund beschränkt man sich in der Regel außerdem auf ein\sphinxhyphen{} und zweidimensionale Phasenräume.

\sphinxAtStartPar
Ein klassisches Beispiel aus der Mechanik ist besonders gut geeignet, um die eben eingeführten Konzepte näher zu diskutieren \sphinxhyphen{} die gedämpfte Schwingungsgleichung.
\label{ode/fluesse:ex:oscillations}
\begin{sphinxadmonition}{note}{Example 1.7 (Gedämpfte Schwingungsgleichung)}



\sphinxAtStartPar
Die \sphinxstylestrong{gedämpfte Schwingungsgleichung} ist gegeben durch
\begin{equation}\label{equation:ode/fluesse:eq:schwingungsgleichung}
\begin{split}m\ddot{x}(t) + r\dot{x}(t) + kx(t)=0\end{split}
\end{equation}
\sphinxAtStartPar
und beschreibt beispielsweise die horizontale (eindimensionale) Auslenkung eines Federpendels, das durch Reibungsverluste Schwingungsenergie über die Zeit verliert.

\sphinxAtStartPar
Hierbei bezeichnet
\begin{itemize}
\item {} 
\sphinxAtStartPar
\(x(t)\) die horizontale Auslenkung des Federpendels zum Zeitpunkt \(t\),

\item {} 
\sphinxAtStartPar
\(m\) die Masse des Objekts,

\item {} 
\sphinxAtStartPar
\(r\) die Dämpfungskonstante,

\item {} 
\sphinxAtStartPar
\(k\) die Federkonstante.

\end{itemize}

\sphinxAtStartPar
Durch Einführung der Variablen \(p(t):= m\dot{x}(t)\) als Impuls erhalten wir das folgende gewöhnliche Differentialgleichungssystem
\begin{equation*}
\begin{split}\dot{p}(t) &= - \frac{r}{m}p(t) -kx(t), \\
\dot{x}(t) &= \frac{1}{m}p(t).\end{split}
\end{equation*}
\sphinxAtStartPar
Dies lässt sich in kompakter Form schreiben als:
\begin{equation*}
\begin{split}\begin{pmatrix} \dot{p} \\ \dot{x} \end{pmatrix}(t) = \begin{pmatrix} -\frac{r}{m} & -k \\ \frac{1}{m} & 0\end{pmatrix} \begin{pmatrix}p \\ x\end{pmatrix}(t)\end{split}
\end{equation*}
\sphinxAtStartPar
Betrachten wir speziell den ungedämpften Fall für \(r=0\), d.h. ohne Reibungsverluste, so geht die Gleichung in die \sphinxstylestrong{Bewegungsgleichung für einen harmonischen Oszillator} über.
In diesem Fall erhalten wir zum Anfangswert \((p,x) \in U \subset \R^2 \) die Lösung
\begin{equation*}
\begin{split}\Phi(t, (p,x)) = 
\begin{pmatrix}
p \cos(\omega t) - m x \sin(\omega t) \\
\frac{p}{\omega m}\sin(\omega t) + x\cos(\omega t)
\end{pmatrix},\end{split}
\end{equation*}
\sphinxAtStartPar
wobei \(\omega=\sqrt{\frac{k}{m}}\) die Eigenfrequenz des Systems ist.
\end{sphinxadmonition}

\sphinxAtStartPar
Wir wollen im Folgenden die Phasenporträts der gedämpften Schwingungsgleichung und des harmonischen Oszillators illustrieren.

\sphinxAtStartPar
In beiden Abbildungen wird die horizontale Auslenkung \(x(t)\) des Federpendels auf der x\sphinxhyphen{}Achse und der Impuls \(p(t) = m\dot{x}(t)\) auf der y\sphinxhyphen{}Achse aufgetragen.

\sphinxAtStartPar
Das in \hyperref[\detokenize{ode/fluesse:fig-harmonic-oscillator}]{Fig.\@ \ref{\detokenize{ode/fluesse:fig-harmonic-oscillator}}} dargestellte Phasenportrait illustriert anschaulich, dass im Fall des harmonischen Oszillators ohne Dämpfung die Orbits elliptisch sind und somit jeder Startwert \(x_0 \in U\) periodisch ist.

\begin{figure}[htbp]
\centering
\capstart

\noindent\sphinxincludegraphics{{C:/Tim/Uni/Lectures/MathPhysicsC/_build/jupyter_execute/fluesse_3_0}.png}
\caption{Visualisierung des Phasenporträts und einiger Orbits für den Phasenfluss des harmonischen Oszillators aus {\hyperref[\detokenize{ode/fluesse:ex:oscillations}]{\sphinxcrossref{Example 1.7}}}. Das Phasenporträt zeigt das charakteristische Verhalten von Lösungen der gedämpften Schwingungsgleichung für reibungsfreie Prozesse, d.h., für eine Dämpfungskonstante \(r = 0\).}\label{\detokenize{ode/fluesse:fig-harmonic-oscillator}}\end{figure}

\sphinxAtStartPar
Betrachten wir nun für eine positive Dämpfungskonstante \(r > 0\) den Fall der allgemeinen gedämpften Schwingungsgleichung, so sieht man am dargestellen Phasenportrait in \hyperref[\detokenize{ode/fluesse:fig-damped-oscillator}]{Fig.\@ \ref{\detokenize{ode/fluesse:fig-damped-oscillator}}}, dass die Trajektorien in den Ursprung konvergieren, der als Orbit in Ruhelage einen Fixpunkt des dynamischen Systems darstellt.
Dies macht auch physikalisch Sinn, da jedes Federpendel auf Grund der Reibung nach endlicher Zeit zum Stillstand kommt.

\begin{figure}[htbp]
\centering
\capstart

\noindent\sphinxincludegraphics{{C:/Tim/Uni/Lectures/MathPhysicsC/_build/jupyter_execute/fluesse_6_0}.png}
\caption{Visualisierung des Phasenporträts und einiger Orbits für den Phasenfluss der gedämpften Schwingungsgleichung aus {\hyperref[\detokenize{ode/fluesse:ex:oscillations}]{\sphinxcrossref{Example 1.7}}} für eine relativ groß gewählte Dämpfungskonstante \(r > 0\).}\label{\detokenize{ode/fluesse:fig-damped-oscillator}}\end{figure}


\section{Hamiltonsche Differentialgleichungen}
\label{\detokenize{ode/hamilton:hamiltonsche-differentialgleichungen}}\label{\detokenize{ode/hamilton::doc}}
\sphinxAtStartPar
Ein wichtiges Prinzip für viele physikalischen Anwendungen und dynamische Systeme sind \sphinxstyleemphasis{Erhaltungssätze} und die dazugehörigen \sphinxstyleemphasis{Erhaltungsgrößen}.
Aus der klassichen Mechanik kennen wir beispielsweise die \sphinxstyleemphasis{Energieerhaltung} oder die \sphinxstyleemphasis{Impulserhaltung}.
In {\hyperref[\detokenize{ode/fluesse:s-fluesse}]{\sphinxcrossref{\DUrole{std,std-ref}{Phasenflüsse und Phasenportraits}}}} haben wir Bewegungsgleichungen als System von gewöhnlichen Differentialgleichungen hergeleitet und gelöst, deshalb wollen wir nun die nötige Theorie entwickeln, die es uns erlaubt Erhaltungsgrößen direkt aus der Formulierung des Differentialgleichungssystems abzulesen.

\sphinxAtStartPar
Hamiltonsche Differentialgleichungen haben in der Physik eine besondere Rolle, insbesondere in der klassischen Mechanik bei Abwesenheit von Reibung.
Typischerweise tauchen diese bei der Untersuchung von Bewegungen im Phasenraum auf, d.h., bei der Betrachtung von Paaren aus Orts\sphinxhyphen{} und Impulswerten.
Ihre Lösungen liefern uns Trajektorien im Phasenraum für die die Gesamtenergie des Systems erhalten bleibt.
Dies macht sie für uns besonders interessant.

\sphinxAtStartPar
Bevor wir die hamiltonschen Differentialgleichungen und ihre Eigenschaften näher diskutieren führen wir zunächst ein wann wir ein Vektorfeld auf dem Phasenraum Hamiltonsch nennen und was eine Hamilton\sphinxhyphen{}Funktion dieses Vektorfelds ist.
\label{ode/hamilton:def:hamiltonsch}
\begin{sphinxadmonition}{note}{Definition 1.9 (Hamilton\sphinxhyphen{}Funktion)}



\sphinxAtStartPar
Sei \(n \in N\) die \sphinxstylestrong{Anzahl der Freiheitsgrade} des betrachteten dynamischen Systems und sei \(U\subset \R^n \times \R^n\) der zugehörige Phasenraum.
Wir nennen ein Vektorfeld \(X \colon U \rightarrow \R^{2n}\) mit \(X \in C^1(P;\R^{2n})\) \sphinxstylestrong{Hamiltonsch}, falls eine reellwertige Funktion \(H \colon U \rightarrow \R\) sowie eine Matrix \(J \, \coloneqq \, \begin{pmatrix}0 & -\mathbf{1}\\ \mathbf{1} & 0 \end{pmatrix} \in \R^{2n \times 2n}\) existiert, so dass sich das Vektorfeld darstellen lässt als
\begin{equation}\label{equation:ode/hamilton:eq:hamilton_Gleichung}
\begin{split}X(p,q) = J \, \nabla H (p,q) \quad \forall (p,q) \in U.\end{split}
\end{equation}
\sphinxAtStartPar
In diesem Fall nennen wir die Funktion \(H\) eine \sphinxstylestrong{Hamilton\sphinxhyphen{}Funktion} des Vektorfelds \(X\).
\end{sphinxadmonition}

\sphinxAtStartPar
Folgende Bemerkungen zur Hamilton\sphinxhyphen{}Funktion wollen wir festhalten.
\label{ode/hamilton:remark-1}
\begin{sphinxadmonition}{note}{Remark 1.5}


\begin{enumerate}
\sphinxsetlistlabels{\arabic}{enumi}{enumii}{}{.}%
\item {} 
\sphinxAtStartPar
Die Hamilton\sphinxhyphen{}Funktion lässt sich auch als Legendre\sphinxhyphen{}Transformation der Lagrange\sphinxhyphen{}Funktion des Systems herleiten, was weitere interessante Zusammenhänge in der Physik erklärt.
In dieser Vorlesung verzichten wir auf diesen Zugang zur Hamilton\sphinxhyphen{}Funktion und verweisen die interessierten Leser*innen auf Kapitel 2 {[}\hyperlink{cite.references:id9}{Nol11}{]}.

\item {} 
\sphinxAtStartPar
Im Folgenden werden wir annehmen, dass die Hamilton\sphinxhyphen{}Funktion \(H\) nicht explizit von der Zeitvariable \(t \in I\) abhängt, was jedoch im Allgemeinen sein kann.

\end{enumerate}
\end{sphinxadmonition}

\sphinxAtStartPar
Basierend auf der Hamilton\sphinxhyphen{}Funktion aus {\hyperref[\detokenize{ode/hamilton:def:hamiltonsch}]{\sphinxcrossref{Definition 1.9}}} können wir nun die Hamiltonschen Differentialgleichungen definieren.
\label{ode/hamilton:definition-2}
\begin{sphinxadmonition}{note}{Definition 1.10 (Hamiltonsche Differentialgleichung)}



\sphinxAtStartPar
Sei \(x(t) = (p(t),q(t)) \in U\) eine Bahnkurve des Phasenraums \(U \subset \R^{2n}\).
Wird das hamiltonsche Vektorfeld auf der linken Seite von \eqref{equation:ode/hamilton:eq:hamilton_Gleichung} als
\begin{equation*}
\begin{split}X = \dot{x}(t) = \begin{pmatrix} \dot{p} \\ \dot{q} \end{pmatrix} (t)\end{split}
\end{equation*}
\sphinxAtStartPar
gewählt, so lässt sich die Gleichung für \(J \, \coloneqq \, \begin{pmatrix}0 & -\mathbf{1}\\ \mathbf{1} & 0 \end{pmatrix} \in \R^{2n \times 2n}\) schreiben als
\begin{equation}\label{equation:ode/hamilton:eq:hamilton_DGL}
\begin{split}\dot{x}(t) = J \nabla H(x(t)).\end{split}
\end{equation}
\sphinxAtStartPar
In dieser Form wird die entstehende Differentialgleichung in \eqref{equation:ode/hamilton:eq:hamilton_DGL} \sphinxstylestrong{Hamiltonsche Differentialgleichung} genannt.

\sphinxAtStartPar
Äquivalent lässt sich dieses System von gewöhnlichen Differentialgleichungen auch explizit für die \(2n\) unbekannten Orts\sphinxhyphen{} und Impulsfunktionen \(q_i, p_i\) für \(1 \leq i \leq n\) schreiben als
\begin{equation*}
\begin{split}\dot{q_i}(t) = \frac{\partial H}{\partial p_i}(t), \quad \dot{p_i}(t) = -\frac{\partial H}{\partial q_i}(t), \quad i=1,\ldots,n.\end{split}
\end{equation*}\end{sphinxadmonition}

\sphinxAtStartPar
Für den einfachen Fall einer zeitunabhängigen Hamilton\sphinxhyphen{}Funktion \(H\) lässt sich beobachten, dass die Lösungskurven der Hamiltonschen Differentialgleichungen sich nicht schneiden und durch jeden Punkt des Phasenraums eine Lösungskurve verläuft.

\sphinxAtStartPar
Die Hamilton\sphinxhyphen{}Funktion \(H\) als Funktion des Phasenraumes kann als die Energie eines Systems von Teilchen aufgefasst werden.
Wir wollen uns die Rolle der Hamilton\sphinxhyphen{}Funktion \(H\) an Hand eines physikalischen Beispiels klar machen.
\label{ode/hamilton:example-3}
\begin{sphinxadmonition}{note}{Example 1.8 (Newtonsche Kraftgleichung)}



\sphinxAtStartPar
Im folgenden Beispiel wollen wir die Bewegung eines Teilchens mit Masse \(m>0\) in einem Kraftfeld \(F \colon \R^3 \rightarrow \R^3\)  untersuchen, welches nur vom Ort \(q \in \R^3\) abhängt.
Nach dem 2. Newtonschen Gesetz erhalten wir die Bewegungsgleichung
\begin{equation}\label{equation:ode/hamilton:eq:newton}
\begin{split}m\ddot{q}(t) = F(q(t)).\end{split}
\end{equation}
\sphinxAtStartPar
Die gewöhnliche Differentialgleichung 2. Ordnung in \eqref{equation:ode/hamilton:eq:newton} lässt sich durch die Definition des Impulses des Teilchens \(p(t) \, \coloneqq \, m \dot{q(t)}\) in ein gewöhnliches Differentialgleichungssystem 1. Ordnung überführen:
\begin{equation*}
\begin{split}\dot{p}(t) = F(q(t)), \quad \dot{q}(t) = \frac{1}{m}p(t).\end{split}
\end{equation*}
\sphinxAtStartPar
Wir nehmen zur Vereinfachung nun an, dass das gegebene Kraftfeld \(F\) \sphinxstyleemphasis{konservativ} sei, d.h., wir können annehmen, dass \(F = - \nabla V\) gilt für ein Potential \(V \colon \R^3 \rightarrow \R\) (z.B. die Erdanziehungskraft).
Dann können wir das physikalische Modell als kontinuierliches dynamisches System interpretieren mit dem erweiterten Phasenraum \(I \times U \subset \R^+_0 \times \R^6\).
Betrachten wir nun einen Punkt \(x = \begin{pmatrix} p \\ q\end{pmatrix} \in U\) im Phasenraum, so lässt sich das autonome gewöhnliche Differentialgleichungssystem kompakt schreiben als
\begin{equation}\label{equation:ode/hamilton:eq:newton_DGL}
\begin{split}\dot{x}(t) = \begin{pmatrix} \dot{p} \\ \dot{q} \end{pmatrix}(t) = \begin{pmatrix} -\nabla V(q) \\ \frac{p}{m} \end{pmatrix}(t)\end{split}
\end{equation}
\sphinxAtStartPar
Wählen wir nun die \sphinxstylestrong{Hamilton\sphinxhyphen{}Funktion} aus {\hyperref[\detokenize{ode/hamilton:def:hamiltonsch}]{\sphinxcrossref{Definition 1.9}}}
\begin{equation*}
\begin{split}H(p,q) \, \coloneqq \, \frac{||p||^2}{2m} + V(q),\end{split}
\end{equation*}
\sphinxAtStartPar
so erkennen wir, dass diese sich aus \sphinxstyleemphasis{kinetischer} und \sphinxstyleemphasis{potentieller Energie} zusammensetzt.
Durch diese Hamilton\sphinxhyphen{}Funktion \(H\) lässt sich \eqref{equation:ode/hamilton:eq:newton} als \sphinxstylestrong{Hamiltonsche Differentialgleichung} schreiben mit
\begin{equation*}
\begin{split}\dot{x}(t) = \begin{pmatrix}\dot{p} \\ \dot{q} \end{pmatrix}(t) = \begin{pmatrix} -\nabla V(q) \\ \frac{p}{m} \end{pmatrix}(t) = \begin{pmatrix}0 & -\mathbf{1}\\ \mathbf{1} & 0 \end{pmatrix} \begin{pmatrix} \frac{p}{m} \\ \nabla V(q) \end{pmatrix}(t) = J \nabla H(p(t),q(t)).\end{split}
\end{equation*}\end{sphinxadmonition}

\sphinxAtStartPar
Ergänzend wollen wir noch folgendes Beispiel einer Hamilton\sphinxhyphen{}Funktion nennen.
\label{ode/hamilton:example-4}
\begin{sphinxadmonition}{note}{Example 1.9}



\sphinxAtStartPar
Im Fall des eindimensionalen harmonischen Oszillators mit Masse \(m > 0\) aus {\hyperref[\detokenize{ode/fluesse:ex:oscillations}]{\sphinxcrossref{Example 1.7}}} lässt sich ebenfalls eine Hamilton\sphinxhyphen{}Funktion des dynamischen Systems angeben.
Sei \((x,p) \in U\) als Punkt des Phasenraums \(U \subset \R^2\) der Ort und Impuls eines Pendels.
Dann lässt sich die zugehörige Hamilton\sphinxhyphen{}Funktion \(H \colon U \rightarrow \R\) angeben als:
\begin{equation*}
\begin{split}H(x,p) = \frac{p^2}{2m} + \frac{m}{2} \omega^2 x^2.\end{split}
\end{equation*}
\sphinxAtStartPar
Hierbei bezeichnet \(\omega = \sqrt{\frac{k}{m}}\) die Eigenfrequenz des Systems und \(k > 0\) die Federkonstante.
\end{sphinxadmonition}

\sphinxAtStartPar
Bisher haben wir noch nicht den Grund diskutiert, warum die Hamilton\sphinxhyphen{}Funktion eine besondere Rolle im Kontext dynamischer Systeme spielt.
Das wollen wir nun im folgenden Satz nachholen.
\label{ode/hamilton:thm:hamconst}
\begin{sphinxadmonition}{note}{Theorem 1.3}



\sphinxAtStartPar
Sei \(n\in \N, U \subseteq \mathbb{R}^{2n}\) ein (offener) Phasenraum und \(J= \begin{pmatrix} 0 & - \mathbf{1} \\ \mathbf{1} & 0 \end{pmatrix} \in \mathbb{R}^{2n \times 2n}\).
Ist die Hamilton\sphinxhyphen{}Funktion \(H \in C^2(U; \mathbb{R})\), dann ist sie entlang der Lösungskurven der Hamiltonschen Differentialgleichung
\begin{equation*}
\dot x = J \nabla H(x)
\end{equation*}
\sphinxAtStartPar
konstant.
\end{sphinxadmonition}

\begin{sphinxadmonition}{note}
\sphinxAtStartPar
Proof. In der Hausaufgabe zu zeigen.
\end{sphinxadmonition}

\sphinxAtStartPar
{\hyperref[\detokenize{ode/hamilton:thm:hamconst}]{\sphinxcrossref{Theorem 1.3}}} sagt uns also, dass die Orbits des kontinuierlichen Systems innerhalb der Niveaumengen der Hamilton\sphinxhyphen{}Funktion verlaufen.
Dies erlaubt es uns dynamische Systeme auf diese häufig auch \sphinxstyleemphasis{Energieschalen} genannten Niveaumengen \(H^{-1}(E)\) für \(E \in \R\) zu restringieren.
Diese Energieschalen bilden Untermannigfaltigkeiten des Phasenraums \(U\).

\sphinxAtStartPar
Für den einfachen Fall eines Freiheitsgrades, d.h., für \(n = 1\), lassen sich für eine gegebene Hamilton\sphinxhyphen{}Funktion \(H\) die Orbits des dynamischen Systems bestimmen.
Für einen Punkt \(x \in U\) im Phasenraum \(U \subset \R^2\) unterscheiden wir zwei Fälle:
\begin{enumerate}
\sphinxsetlistlabels{\arabic}{enumi}{enumii}{}{.}%
\item {} 
\sphinxAtStartPar
Ist \(\nabla H(x) = 0\), so ist der Orbit wegem \eqref{equation:ode/hamilton:eq:hamilton_DGL} von der Form \(O(x) = {x}\).

\item {} 
\sphinxAtStartPar
Ist \(\nabla H(x) \neq 0\), so ist der Orbit \(O(x)\) gegeben durch die zusammenhängende Menge

\end{enumerate}
\begin{equation*}
\begin{split}O(x) = \{y \in U | H(y) = H(x), \nabla H(y) \neq 0\}\end{split}
\end{equation*}
\sphinxAtStartPar
Die Orientierung des Orbits erhält man durch die Richtung, die orthogonal zum Gradienten \(\nabla H\) steht, d.h., durch Drehung des Gradienten im Uhrzeigersinn um \(\frac{\pi}{2}\).
Die Matrix \(J\) entspricht eben einer solchen Drehung.
\label{ode/hamilton:remark-6}
\begin{sphinxadmonition}{note}{Remark 1.6}



\sphinxAtStartPar
Eine Formulierung der Bewegungsgleichungen eines dynamischen Systems als Hamiltonsche Differentialgleichungen hat den Vorteil, dass sie unter den sogenannten \sphinxstyleemphasis{kanonischen Transformationen} in manchen Fällen in eine einfachere, lösbare Form gebracht werden können.
\end{sphinxadmonition}


\section{Aufgaben}
\label{\detokenize{ode/ex:aufgaben}}\label{\detokenize{ode/ex::doc}}
\begin{sphinxadmonition}{note}{Aufgabe: DGL höherer Ordnung}

\sphinxAtStartPar
Gegeben sei folgende gewöhnliche Differentialgleichung 4.\textasciitilde{}Ordnung:
\begin{equation*}
\begin{split}x^{(4)}(t) = 7 x^{(3)}(t) - \dot x(t) + 5 x(t) + t^2\end{split}
\end{equation*}
\sphinxAtStartPar
Überführen Sie diese in ein System gewöhnlicher Differentialgleichungen 1.Ordnung.
\end{sphinxadmonition}

\begin{sphinxadmonition}{note}{Aufgabe: Autonome gewöhnliche Differentialgleichungen}

\sphinxAtStartPar
Entscheiden und begründen Sie mathematisch, ob die folgenden gewöhnlichen Differentialgleichungen \sphinxstylestrong{autonom} sind.

\sphinxAtStartPar
\sphinxstylestrong{a)} Differentialgleichung für harmonischen Oszillator:
\begin{equation*}
\begin{split}\ddot x(t) + \lambda x(t) = 0\end{split}
\end{equation*}
\sphinxAtStartPar
für eine Konstante \(\lambda \in \mathbb{R}\).

\sphinxAtStartPar
\sphinxstylestrong{b)} Newtonsche Kraftgleichung:
\begin{equation*}
\begin{split}m \ddot x(t) = F(t, x(t))\end{split}
\end{equation*}
\sphinxAtStartPar
für eine Konstante \(m > 0\), eine Kraft \(F: \mathbb{R} \times \mathbb{R}^3 \rightarrow \mathbb{R}^3\), welche von der Position im Raum \(x: \mathbb{R} \rightarrow \mathbb{R}^3\) und der Zeit \(t \in \mathbb{R}\) abhängt.

\sphinxAtStartPar
\sphinxstylestrong{c)} Newtonsche Kraftgleichung:
\begin{equation*}
\begin{split}m \ddot x(t) = F(t, x(t))\end{split}
\end{equation*}
\sphinxAtStartPar
für eine Konstante \(m > 0\), eine Kraft \(F: \mathbb{R}^3 \rightarrow \mathbb{R}^3\), welche im Gegensatz zur Situation in b) lediglich von der Position im Raum \(x: \mathbb{R} \rightarrow \mathbb{R}^3\) abhängt.

\sphinxAtStartPar
\sphinxstylestrong{d)} Mathieusche Differentialgleichung:
\begin{equation*}
\begin{split}\ddot x(t) + [\lambda + \gamma \cos(t)] ~ x(t) = 0\end{split}
\end{equation*}
\sphinxAtStartPar
für Konstanten \(\lambda, \gamma \in \mathbb{R}\).
\end{sphinxadmonition}

\begin{sphinxadmonition}{note}{Flüsse}

\sphinxAtStartPar
Für \(I = \mathbb{R}^0_+\) und \(U = \mathbb{R}^2\) betrachten wir die Abbildung \(\phi: I \times U \rightarrow U\) mit
\begin{equation*}
\begin{split}\phi(t, x) = 
\begin{pmatrix} \frac{x_2}{2} ~ \sin(\omega t) + x_1 ~ \cos(\omega t) \\ x_2 ~ \cos(\omega t) - 2 x_1 ~ \sin(\omega t) \end{pmatrix}, \end{split}
\end{equation*}
\sphinxAtStartPar
wobei \(x = \begin{pmatrix} x_1 \\ x_2 \end{pmatrix}\) gilt.
Zeigen Sie, dass diese Abbildung die mathematischen Eigenschaften eines Flusses erfüllt.
\end{sphinxadmonition}

\begin{sphinxadmonition}{note}{Phasenporträt gedämpfter Oszillator}

\sphinxAtStartPar
Wir betrachten die Bewegungsgleichung für den harmonischen Oszillator
\begin{equation*}
\begin{split}m ~ \ddot x(t) + r ~ \dot x(t) + k ~ x(t) = 0\end{split}
\end{equation*}
\sphinxAtStartPar
mit Masse \(m = 1  ~ kg\), Dämpfungskonstante \(r = 0.5 ~ \frac{kg}{s}\) und Federkonstante \(k = 1.5 ~ \frac{kg}{s^2}\).

\sphinxAtStartPar
Wie in Beispiel 1.3 im Skript führen wir den Impuls \(p(t) = m ~ \dot x(t)\) ein und erhalten das Differentialgleichungssystem erster Ordnung
\begin{equation*}
\begin{split}\dot x(t) &= \frac{1}{m} ~ p(t)\\
\dot p(t) &= -k ~ x(t) - \frac{r}{m} ~ p(t).\end{split}
\end{equation*}
\sphinxAtStartPar
Zeichnen Sie händisch ein Phasenporträt für dieses System in den Unbekannten \(x\) und \(p\), indem Sie für die folgenden Punkte \((x,p)\) die durch das Differentialsystem gegebene Steigung berechnen und einzeichnen:
\begin{equation*}
<<<<<<< HEAD
\begin{split}(-1, 0) \quad  (1, 0) \quad (0, -1) \quad  (0, 1) \quad (-0.75, -0.75) \quad  (-0.75, 0.75) \quad (0.75, -0.75) \quad (0.75, 0.75)\end{split}
=======
\begin{split}&(-1, 0) \quad  (1, 0) \quad (0, -1) \quad  (0, 1)\\
&(-0.75, -0.75) \quad  (-0.75, 0.75) \quad (0.75, -0.75) \quad (0.75, 0.75)\end{split}
>>>>>>> main
\end{equation*}\end{sphinxadmonition}

\begin{sphinxadmonition}{note}{Aufgabe: Eigenschaften Hamilton\sphinxhyphen{}Funktion}

\sphinxAtStartPar
Beweisen Sie die folgende Aussage:

\sphinxAtStartPar
Sei \(P \subseteq \mathbb{R}^{2m}\) ein (offener) Phasenraum und \(\mathbb{J} = \begin{pmatrix} 0 & - 𝟙 \\ 𝟙 & 0 \end{pmatrix} \in Mat(2m, \mathbb{R})\). Ist die Hamilton\sphinxhyphen{}Funktion \(H \in C^2(P, \mathbb{R})\), dann ist sie entlang der Lösungskurven der Hamiltonschen Differentialgleichung \(\dot x = \mathbb{J} \nabla H(x)\) konstant.
\end{sphinxadmonition}

<<<<<<< HEAD
=======
\begin{sphinxadmonition}{note}{Aufgabe: Hamilton\sphinxhyphen{}Funktion}

\sphinxAtStartPar
Zeigen Sie mathematisch, dass die Hamilton\sphinxhyphen{}Funktion eines eindimensionalen harmonischen Oszillators gegeben ist durch:
\begin{equation*}
\begin{split}H(x,p) = \frac{p^2}{2m} + \frac{m}{2} w^2 x^2,\end{split}
\end{equation*}
\sphinxAtStartPar
wobei \(w = \sqrt{\frac{k}{m}}\) gilt.
\end{sphinxadmonition}

>>>>>>> main

\chapter{Stabilitätsanalyse für dynamische Systeme}
\label{\detokenize{odestability/stabilitaetsanalyse:stabilitatsanalyse-fur-dynamische-systeme}}\label{\detokenize{odestability/stabilitaetsanalyse::doc}}
\sphinxAtStartPar
In diesem Abschnitt beschäftigen wir uns mit der Stabilitätstheorie für kontinuierliche dynamische Systeme.
Hierbei interessieren wir uns für die Frage, wie sich \sphinxstyleemphasis{kleine Störungen} von bestimmten Zuständen des Systems auf die Lösungen der zu Grunde liegenden gewöhnlichen Differentialgleichungen auswirken.
Der untersuchte Zustand kann beispielsweise ein periodischer Orbit oder eine Ruhelage des dynamischen Systems sein.
Letztere sind oftmals von besonderes Interesse, da man in vielen technischen und physikalischen Anwendungen daran interessiert ist das System in eine oder nahe einer Gleichgewichtslage zu bringen.

\sphinxAtStartPar
Im Folgenden werden wir verschiedene Stabilitätsbegriffe für dynamische Systeme einführen und speziell Kriterien für die Stabilität von Ruhelagen diskutieren.


\section{Stabilitätsbegriffe}
\label{\detokenize{odestability/stabilitaetsbegriffe:stabilitatsbegriffe}}\label{\detokenize{odestability/stabilitaetsbegriffe::doc}}
\sphinxAtStartPar
Im Folgenden wollen wir grundlegende Begriffe der Stabilitätsanalyse von Ruhelagen einführen und diskutieren.
Wie in {\hyperref[\detokenize{ode/fluesse:s-fluesse}]{\sphinxcrossref{\DUrole{std,std-ref}{Phasenflüsse und Phasenportraits}}}} definiert, nennen wir einen Punkt \(x\in U\) im Phasenraum \(U\) \sphinxstylestrong{Ruhelage}, falls für den zugehörigen Phasenfluss \(\Phi \colon I \times U \rightarrow U\) des dynamischen Systems gilt: \(\Phi(t,x) = x, \forall t \in I\), d.h., wenn für alle \(t \in I\) der Zustand \(x \in U\) ein \sphinxstylestrong{Fixpunkt des Flusses} ist.

\sphinxAtStartPar
Für autonome Differentialgleichungssysteme mit
\begin{equation*}
\begin{split}\dot{x}(t) = F(x)\end{split}
\end{equation*}
\sphinxAtStartPar
ist \(x \in U\) auch eine Ruhelage, falls \(F(x) = 0\) gilt, d.h., falls \(x\) eine Nullstelle von \(F\) ist.
Das ist einfach zu verstehen, da die Zeitableitung auf der linken Seite für eine Ruhelage Null ist und somit die Funktion \(F\), die nur vom Ort abhängt, sich nicht ändern kann.

\sphinxAtStartPar
Anschaulich versteht man unter der Stabilitätsanalyse von Ruhelagen die mathematische Untersuchung, ob benachbarte Lösungen von einer Ruhelage wegstreben oder nicht.
Dies ist insbesondere in technischen Anwendungen wichtig, da man dort häufig danach strebt ein dynamisches System in eine Gleichgewichtslage zu bringen.
Da dies nur bis zu einer gewissen Genauigkeit möglich ist, muss man also mit kleinen Störungen rechnen.

\sphinxAtStartPar
Ist eine Ruhelage stabil, dann bleiben benachbarte Lösungen auch für zukünftige Zeitpunkte \(t \in I\) nahe der Ruhelage.
Ist sie jedoch nicht stabil, so muss das im Allgemeinen nicht gelten und die Lösungen können dann mit der Zeit von der Ruhelage divergieren.
Diese Anschauung wollen wir in der folgenden Definition mathematisch formalisieren.
Hierbei werden wir den Stabilitätsbegriff für allgemeine Lösungen einführen und später Ruhelagen als ein Spezialfall dieser Lösungen interpretieren.
\label{odestability/stabilitaetsbegriffe:def:Stabilitaet}
\begin{sphinxadmonition}{note}{Definition 2.1 (Stabilität von Lösungen)}



\sphinxAtStartPar
Sei \(\Phi \colon I \times U \rightarrow U\) der Phasenfluss zu dem Vektorfeld \(F\in C^1(U;\R^n)\) auf \(U\), dass durch die rechte Seite des zugehörigen Differentialgleichungssystems gegeben ist.

\sphinxAtStartPar
1. Eine Lösung \(t \in [0,\infty) \mapsto \Phi_t(x)\) heißt \sphinxstylestrong{(Lyapunov\sphinxhyphen{})stabil}, wenn zu jedem \(\epsilon > 0\) ein \(\delta>0\) existiert mit:
\begin{equation*}
\begin{split}\|x-y\|<\delta \ \Rightarrow \ \sup_{t\geq0}\|\Phi_t(x)-\Phi_t(y)\|<\epsilon.\end{split}
\end{equation*}
\sphinxAtStartPar
2. Eine Lösung \( t \in [0,\infty) \mapsto \Phi_t(x)\) heißt \sphinxstylestrong{asymptotisch stabil}, wenn ein \(\delta > 0\) existiert mit:
\begin{equation*}
\begin{split}\|x-y\|<\delta \ \Rightarrow \ \lim_{t\to\infty}\|\Phi_t(x)-\Phi_t(y)\|=0.\end{split}
\end{equation*}
\sphinxAtStartPar
3. Eine Lösung heißt \sphinxstylestrong{instabil}, wenn sie nicht (Lyapunov\sphinxhyphen{})stabil ist.
\end{sphinxadmonition}

\begin{sphinxShadowBox}
\sphinxstylesidebartitle{Aleksandr Lyapunov}

\sphinxAtStartPar
\sphinxhref{https://de.wikipedia.org/wiki/Alexander\_Michailowitsch\_Ljapunow}{Alexander Michailowitsch Ljapunow} (Geboren 6. Juni 1857 in Jaroslawl; Gestorben 3. November 1918 in Odessa) war ein russischer Mathematiker und Physiker.
\end{sphinxShadowBox}

\sphinxAtStartPar
Es ist klar, dass der Begriff der asymptotischen Stabilität \sphinxstyleemphasis{stärker} als der Begriff der Lyapunov\sphinxhyphen{}Stabilität von Lösungen ist, da jede asymptotisch stabile Lösung auch schon Lyapunov\sphinxhyphen{}stabil ist.
Die Umkehrung gilt jedoch im Allgemeinen nicht.
Dies wird durch das folgende Beispiel nochmal illustriert.
\label{odestability/stabilitaetsbegriffe:example-1}
\begin{sphinxadmonition}{note}{Example 2.1 (Stabilitätsanalyse für den harmonischer Oszillator)}



\sphinxAtStartPar
Der Phasenfluss für den harmonischen Oszillator ist, wie wir in {\hyperref[\detokenize{ode/fluesse:ex:oscillations}]{\sphinxcrossref{Example 1.7}}} gesehen haben, gegeben durch
\begin{equation*}
\begin{split}\Phi(t, (p,x)) = \begin{pmatrix}
p \cos(\omega t) - m x \sin(\omega t)\\
\frac{p}{\omega m}\sin(\omega t) + x\cos(\omega t)
\end{pmatrix}\end{split}
\end{equation*}
\sphinxAtStartPar
Wir suchen nun einen Fixpunkt \((p_r,x_r) \in U\) des Flusses der unabhängig ist vom Zeitpunkt \(t\).
Man sieht leicht ein, dass eine \sphinxstylestrong{Ruhelage} sich bei \((p_r,x_r) = (0,0)^T \in U\) befindet, da \(\Phi(t,(0,0)) = (0,0)^T\) ist für alle \(t \in I\).
Die gefundene Ruhelage ist \sphinxstylestrong{Lyapunov\sphinxhyphen{}stabil}, denn wie wir im Phasenporträt in \hyperref[\detokenize{ode/fluesse:fig-harmonic-oscillator}]{Fig.\@ \ref{\detokenize{ode/fluesse:fig-harmonic-oscillator}}} gesehen haben, ist jeder Orbit um die Ruhelage \((0,0)\) periodisch. Damit kann das dynamische System insgesamt nicht wegstreben von der Ruhelage.

\sphinxAtStartPar
Mathematisch lässt sich diese Eigenschaft wie folgt zeigen.
Für ein beliebiges \(\epsilon > 0\) sei \((p,y) \in U\) ein Punkt im Phasenraum mit periodischen Orbit \(O(p,y)\) um die Ruhelage \((p_r,x_r) = (0,0)^T \in U\), so dass dessen maximaler Abstand zur Ruhelage kleiner als \(\epsilon\) ist, d.h.
\begin{equation*}
\begin{split}\sup_{t \geq 0} ||\Phi_t(p_r,x_r) - \Phi_t(p,y)|| < \epsilon\end{split}
\end{equation*}
\sphinxAtStartPar
Auf Grund der ersten Eigenschaft des Phasenflusses \(\Phi_0(p,y) = (p,y)\) gilt dann aber schon
\begin{equation*}
\begin{split}||(p_r, x_r) - (p,y)|| = ||\Phi_0(p_r, x_r) - \Phi_0(p,y)|| < \epsilon.\end{split}
\end{equation*}
\sphinxAtStartPar
Wählen wir nun \(\delta \coloneqq \epsilon\), so haben wir gezeigt, dass die Ruhelage \((p_r, x_r) = (0,0)^T\) Lyapunov\sphinxhyphen{}stabil ist.
Sie ist jedoch auf Grund der Periodizität der Orbits um die Ruhelage \sphinxstylestrong{nicht asymptotisch stabil}, da für beliebige Punkte \((p,y) \in U\) mit \(||(p_r,x_r) - (p,y)|| < \delta\) für ein \(\delta > 0\) gilt
\begin{equation*}
\begin{split}\lim_{t\to\infty}\|\Phi_t(p_r, x_r)-\Phi_t(p,y)\| \neq 0.\end{split}
\end{equation*}\end{sphinxadmonition}

\sphinxAtStartPar
Im allgemeinen Fall der gedämpften Schwingungsgleichung in {\hyperref[\detokenize{ode/fluesse:ex:oscillations}]{\sphinxcrossref{Example 1.7}}} hängt die Stabilität der Ruhelage im Ursprung intuitiverweise von der Reibungskonstanten ab, wie folgende Bemerkung festhält.
\label{odestability/stabilitaetsbegriffe:remark-2}
\begin{sphinxadmonition}{note}{Remark 2.1 (Stabilität bei der gedämpften Schwingungsgleichung)}



\sphinxAtStartPar
Für den Fall der gedämpften Schwingungsgleichung in \eqref{equation:ode/fluesse:eq:schwingungsgleichung} lässt sich folgendes Stabilitätsverhalten der Ruhelage im Ursprung in Abhängigkeit der Reibungskonstanten \(r \in \R\) beobachten:
\begin{enumerate}
\sphinxsetlistlabels{\arabic}{enumi}{enumii}{}{.}%
\item {} 
\sphinxAtStartPar
Die Ruhelage ist \sphinxstylestrong{asymptotisch stabil} für den Fall mit positiver Reibung \(r>0\).

\item {} 
\sphinxAtStartPar
Die Ruhelage ist \sphinxstylestrong{Lyapunov\sphinxhyphen{}stabil} für den reibungsfreien Fall \(r=0\).

\item {} 
\sphinxAtStartPar
Die Ruhelage ist \sphinxstylestrong{instabil} für den Fall einer negativen Reibung \(r < 0\), d.h. für einen externen Antrieb.

\end{enumerate}
\end{sphinxadmonition}


\section{Stabilität von Ruhelagen}
\label{\detokenize{odestability/ruhelagen:stabilitat-von-ruhelagen}}\label{\detokenize{odestability/ruhelagen::doc}}
\sphinxAtStartPar
Zunächst wollen wir die Stabilität von dynamischen System im einfachen Fall von Ruhelagen für allgemeine \sphinxstylestrong{lineare} Differentialgleichungssysteme untersuchen.
Diese Familie von gewöhnlichen Differentialgleichungssystemen haben wir schon in Kapitel 8 in {[}\hyperlink{cite.references:id12}{Ten21}{]} kennen gelernt.

\sphinxAtStartPar
Das folgende Theorem beschreibt die Existenz und Eindeutigkeit einer Ruhelage eines dynamischen System, das durch ein lineares Differentialgleichungssystem charakterisiert wird und gibt Bedingungen für die Stabilität der Ruhelage.
<<<<<<< HEAD
\label{ode_stability/ruhelagen:theorem:stabilität_linear}
=======
\label{odestability/ruhelagen:thm:stablin}
>>>>>>> main
\begin{sphinxadmonition}{note}{Theorem 2.1}



\sphinxAtStartPar
Sei \(A\in \C^{n\times n}\) eine Matrix mit den Eigenwerten \(\lambda_1,\dots, \lambda_n\in \C\).
Dann beschreibt der zugehörige Phasenfluss \(\Phi\) zum homogenen linearen Differentialgleichungssystem
\begin{equation*}
\begin{split}\dot{x}(t) = Ax(t)\end{split}
\end{equation*}
\sphinxAtStartPar
eine Ruhelage in \(\mathbf{0} \in \C^n\).
Diese ist sogar eindeutig, falls \(\lambda_i\neq 0, i=1,\ldots,n\) gilt.

\sphinxAtStartPar
Für
\begin{equation*}
\begin{split}\gamma \coloneqq \max_{i=1,\dots,n} \mathcal{Re}(\lambda_i)\end{split}
\end{equation*}
\sphinxAtStartPar
kann die Stabilität der Ruhelage wie folgt charakterisiert werden:
\begin{enumerate}
\sphinxsetlistlabels{\arabic}{enumi}{enumii}{}{.}%
\item {} 
\sphinxAtStartPar
Falls \(\gamma <0\) gilt, ist die Ruhelage \(\mathbf{0}\) \sphinxstyleemphasis{asymptotisch stabil}

\item {} 
\sphinxAtStartPar
Falls \(\gamma >0\) gilt, ist die Ruhelage \(\mathbf{0}\) \sphinxstyleemphasis{instabil}.

\end{enumerate}
\end{sphinxadmonition}

\begin{sphinxadmonition}{note}
\sphinxAtStartPar
Proof. Wir wissen, dass für einen beliebigen Startpunkt \(x_0 \in U\) im Phasenraum der Phasenfluss \(\Phi \colon I \times U \rightarrow U\) eine Lösung des Differentialgleichungssystems realisiert.
Für homogene, lineare Differentialgleichungssysteme haben wir bereits in {\hyperref[\detokenize{ode/repetition:s-lineare-dglsysteme}]{\sphinxcrossref{\DUrole{std,std-ref}{Lösungen von linearen Differentialgleichungssystemen}}}} Lösungen mittels des \sphinxstyleemphasis{Matrixexponentials} hergeleitet.

\sphinxAtStartPar
Sei \(J = S^{-1}AS\) die Jordansche Normalform von \(A\) mit Transformationsmatrizen \(S^{-1},S \in \C^{n \times n}\), so erhalten wir die Abschätzung
\begin{equation*}
\begin{split}\|\Phi_t(x_0)\| &= \|e^{tA}x_0\| = \|S^{-1}e^{tJ}Sx_0\| = \|S^{-1}e^{tD}e^{tN}Sx_0\| \\
&\leq \|S^{-1}\| \cdot \|e^{tD}\| \cdot \|e^{tN}\| \cdot \|S\| \cdot \|x_0\| \leq C_1 \cdot \|e^{tD}\| \cdot \|e^{t N}\|,\end{split}
\end{equation*}
\sphinxAtStartPar
für eine Konstante \(C_1 > 0\), die unabhängig von \(t\) ist.
Hierbei haben wir ausgenutzt, dass sich die Jordannormalform \(J\) von \(A\) als Summe einer Diagonalmatrix \(D\) mit den Eigenwerten \(\lambda_i \in \C\), \(i=1,\ldots,n\) von \(A\) und einer nilpotenten Matrix \(N\) schreiben lässt als \(J = D + N\).
Diese Matrizen kommutieren, d.h., \(D \cdot N = N \cdot D\).

\sphinxAtStartPar
Wir sehen nun ein, dass \(e^{tN}\) wegen der Nilpotenz von \(N\) eine endliche Reihe bildet der Form
\begin{equation*}
\begin{split}e^{tN} = \sum_{k=0}^m \frac{(tN)^k}{k!} = \sum_{k=0}^m t^k\frac{N^k}{k!},\end{split}
\end{equation*}
\sphinxAtStartPar
welches ein Polynom vom Grad \(m\) darstellt, wobei \(m \in \N\) der Nilpotenzindex der Matrix \(N\) ist.

\sphinxAtStartPar
Sei nun \(\epsilon > 0\) beliebig klein gewählt.
Dann lässt sich die Norm des Polynoms mit einer genügend großen Konstanten \(C_2 > 0\), die von \(\epsilon\) jedoch nicht von \(t\) abhängt, durch eine gewöhnliche Exponentialfunktion abschätzen mit
\begin{equation*}
\begin{split} \|e^{tN}\| = \| \sum_{k=0}^m t^k\frac{N^k}{k!} \| \leq \sum_{k=0}^m t^k \frac{\|N^k\|}{k!} \leq C_2  e^{t \epsilon}.\end{split}
\end{equation*}
\sphinxAtStartPar
Wählen wir nun \(\gamma \coloneqq \max_{i=1,\dots,n} \mathcal{Re}(\lambda_i)\), so folgt direkt, dass gilt
\begin{equation*}
\begin{split}||e^{tD}|| \leq C_3 e^{t\gamma}.\end{split}
\end{equation*}
\sphinxAtStartPar
Insgesamt erhalten wir also für die Norm des Phasenflusses
<<<<<<< HEAD
\begin{equation}\label{equation:ode_stability/ruhelagen:eq:abschaetzung_ew}
=======
\begin{equation}\label{equation:odestability/ruhelagen:eq:abschaetzungew}
>>>>>>> main
\begin{split}\|\Phi_t(x_0)\| \leq C_1 \cdot \|e^{tN}\| \cdot \|e^{tD}\| \leq C_1 \cdot C_2 e^{t \epsilon} \cdot C_3 e^{t\gamma} = C e^{t \epsilon} e^{t\gamma}.\end{split}
\end{equation}
\sphinxAtStartPar
Da \(\epsilon > 0\) beliebig klein ist, können wir \(|\epsilon| < |\gamma|\) wählen.
Damit hängt das Verhalten der Norm des Flusses nur noch vom Vorzeichen von \(\gamma\) ab.
Wir unterscheiden daher zwei Fälle:

\sphinxAtStartPar
1. Wenn \(\gamma >0\) ist, so existiert zum Eigenwert \(\gamma\) von \(A\) ein zugehöriger Eigenvektor \(v\in U\), so dass die Eigenwertgleichung \(A v = \gamma v\) gilt.
Nach {\hyperref[\detokenize{ode/repetition:lem:mpotew}]{\sphinxcrossref{Lemma 1.1}}} ist dann \(e^{t\gamma}\) ein Eigenwert des Matrixexponentials \(e^{tA}\) mit zugehörigem Eigenvektor \(v\).
Insgesamt erhalten wir also
\begin{equation*}
\begin{split}||\Phi_t(\alpha v)|| = ||e^{tA}\alpha v|| = ||\lambda e^{t\gamma} \alpha v|| \to \infty, \quad \text{ für } \ t \to \infty, \quad  \forall \alpha>0. \end{split}
\end{equation*}
\sphinxAtStartPar
Also enthält jede beliebig kleine Umgebung der Ruhelage \(0\) Punkte, für die die entsprechenden Lösungen divergieren.
In diesem Fall ist die Ruhelage also \sphinxstylestrong{instabil}.

\sphinxAtStartPar
2. Falls \(\gamma <0\) gilt, so gilt auch \(\gamma + \epsilon <0\) und wir können abschätzen,
\begin{equation*}
\begin{split}0\leq \|\Phi_t(x_0)-0\|\leq C e^{t (\gamma + \epsilon)} \to 0 \quad \text{ für } \ t \to \infty.\end{split}
\end{equation*}
\sphinxAtStartPar
Dies liefert uns also \sphinxstylestrong{asymptotische Stabilität} der Ruhelage \(\mathbf{0}\).
\end{sphinxadmonition}

\sphinxAtStartPar
Wir haben also gesehen, dass im Fall eines homogenen, linearen Differentialgleichungssystems die \(\mathbf{0}\) immer eine Ruhelage des zugehörigen dynamischen Systems darstellt, deren Stabilität einzig vom Vorzeichen des größten Eigenwerts abhängt.


\subsection{Linearisierung um Ruhelage}
\label{\detokenize{odestability/ruhelagen:linearisierung-um-ruhelage}}\label{\detokenize{odestability/ruhelagen:s-linearisierung-ruhelage}}
\sphinxAtStartPar
:label: s:linearisierung\_ruhelage

\sphinxAtStartPar
In diesem Abschnitt wollen wir unsere Erkentnisse zur Stabilitätsanalysie vom Fall eines linearen Differentialgleichungssystems auf den allgemeinen Fall übertragen, da man es in den meisten Anwendungen leider nur selten mit linearen Differentialgleichungen zu tun hat.
Darüber hinaus ist es erstrebenswert Stabilitätsaussagen zu Differentialgleichungen zu machen, deren Lösungen man nicht explizit analytisch herleiten kann.
Daher betrachten wir im Folgenden das Anfangswertproblem eines \sphinxstylestrong{allgemeinen Differentialgleichungssystem erster Ordnung} auf dem Phasenraum \(U\in \R^n\), das nicht notwendigerweise linear sein muss und für ein Vektorfeld \(F\in C^1(U;\R^n)\) wie folgt formuliert ist
\begin{equation}\label{equation:odestability/ruhelagen:eq:awpallg}
\begin{split}\dot{x}(t) &= F(x(t)), \quad \forall t \in I \subset \R^+_0\\
x(0) &= x_0.\end{split}
\end{equation}
\sphinxAtStartPar
Wir nehmen an, dass \(x_F \in U\) eine Ruhelage des dynamischen Systems ist, so dass dementsprechend \(F(x_F) = 0\) gilt.
Durch einfache Translation der Koordinaten des Systems um \(x_F \in U\), können wir ohne Beschränkung der Allgemeinheit annehmen, dass die Ruhelage sich im Nullpunkt befindet.

\sphinxAtStartPar
Im Folgenden definieren wir zwei wichtige Werkzeuge zur Untersuchung der Stabilität von Ruhelagen für allgemeine Differentialgleichungssysteme.
\label{odestability/ruhelagen:def:linearisierung}
\begin{sphinxadmonition}{note}{Definition 2.2 (Linearisierung und Abweichung)}



\sphinxAtStartPar
Sei \(F\in C^1(U;\R^n)\) ein Vektorfeld auf dem Phasenraum \(U \subset \R^n\) und \(0\) eine Ruhelage des dynamischen Systems, dass durch das allgemeine Differentialgleichungssystem in \eqref{equation:odestability/ruhelagen:eq:awpallg} charakterisiert wird.
Sei nun \((DF)(x)\) die Jacobi\sphinxhyphen{}Matrix der Funktion \(F\) im Punkt \(x \in U\) (vgl. Kapitel 6.2 in {[}\hyperlink{cite.references:id12}{Ten21}{]}).
Dann bezeichnen wir mit \(A := (DF)(0)\) die \sphinxstylestrong{Linearisierung} von \(F\) in der Ruhelage \(0 \in U\).
Außerdem bezeichnen wir die Funktion \(R \in C^1(U; \R^n)\) mit
\begin{equation*}
\begin{split}R(x) \ \coloneqq \ F(x) - Ax\end{split}
\end{equation*}
\sphinxAtStartPar
als die \sphinxstylestrong{Abweichung} (auch \sphinxstylestrong{Residuum} genannt) des Vektorfeldes \(F\) von seiner Linearisierung \(A\) in der Ruhelage.
\end{sphinxadmonition}

\sphinxAtStartPar
Mit diesen Hilfswerkzeugen werden wir im Folgenden zeigen, dass die Lösung des Differentialgleichungssystem in führender Ordnung durch die Linearisierung \(A\) von \(F\) kontrolliert werden, solange wir uns nah genug zur Ruhelage befinden. Dies wird durch das folgende Lemma ausgedrückt.
\label{odestability/ruhelagen:lem:intexpglgn}
\begin{sphinxadmonition}{note}{Lemma 2.1}



\sphinxAtStartPar
Wir betrachten das Anfangswertproblem aus \eqref{equation:odestability/ruhelagen:eq:awpallg} auf dem Phasenraum \(U \subset \R^n\) für ein Vektorfeld \(F\in C^1(U;\R^n)\).
Außerdem sei \(A \coloneqq (DF)(0)\) die Linearisierung des Vektorfelds in der Ruhelage \(0\) des dynamischen Systems und \(R(x) \coloneqq F(x) - Ax\) die Abweichung von \(F\) von seiner Linearisierung \(A\) im Nullpunkt.

\sphinxAtStartPar
Dann lassen sich Lösungen des Differentialgleichungssystems mittels der Linearisierung \(A\) und der Abweichung \(R\) explizit angeben als
\begin{equation*}
\begin{split}x(t) = e^{At}x_0 + \int_0^t e^{A(t-s)} R(x(s))\, \mathrm{d}s, \quad \forall t \in I.\end{split}
\end{equation*}\end{sphinxadmonition}

\begin{sphinxadmonition}{note}
\sphinxAtStartPar
Proof. Wir setzen zunächst die unbekannte Lösung \(x(t)\) des Anfangswertproblems \eqref{equation:odestability/ruhelagen:eq:awpallg} in der allgemeinen Form
\begin{equation*}
\begin{split}x(t) = e^{At}c(t),\quad \text{mit }c(0) = x_0\end{split}
\end{equation*}
\sphinxAtStartPar
an, und suchen eine Bestimmungsgleichung für die unbekannte Funktion \(c(t)\) mittels \sphinxstylestrong{Variation der Konstanten} (vgl. Kapitel 8.2 in {[}\hyperlink{cite.references:id12}{Ten21}{]}).

\sphinxAtStartPar
Mittels der Rechenregeln für das Matrixexponentials in {\hyperref[\detokenize{ode/repetition:rem:matrixexponentialregeln}]{\sphinxcrossref{Remark 1.3}}} können wir die Ableitung der Funktion \(x\) mittels Produktregel angeben als
\begin{equation*}
\begin{split}\dot{x}(s) = A e^{As}c(s)+ e^{As}\dot{c}(s) = Ax(s) + e^{As}\dot{c}(s).\end{split}
\end{equation*}
\sphinxAtStartPar
Aus der Definition des Residuums in {\hyperref[\detokenize{odestability/ruhelagen:def:linearisierung}]{\sphinxcrossref{Definition 2.2}}} folgt aber auch
\begin{equation*}
\begin{split}\dot{x}(s) = F(x(s)) = Ax(s) + R(x(s)).\end{split}
\end{equation*}
\sphinxAtStartPar
Vergleichen wir die beiden Gleichungen, so sieht man ein, dass
\begin{equation*}
\begin{split}e^{As}\dot{c}(s) = R(x(s))\end{split}
\end{equation*}
\sphinxAtStartPar
gelten muss.
Äquivalent können wir auch folgern, dass \(\dot{c}(s) = e^{-As}R(x(s))\) gilt.

\sphinxAtStartPar
Nach dem Hauptsatz der Differential\sphinxhyphen{} und Integralrechnung (vgl. Theorem 5.3 in {[}\hyperlink{cite.references:id12}{Ten21}{]}) gilt dann für die unbekannte Funktion \(c\) der folgende Zusammenhang
\begin{equation*}
\begin{split}c(t) = c(0) + \int_0^t \dot{c}(s)\, \mathrm{d}s = x_0+ \int_0^t e^{-As}R(x(s)) \, \mathrm{d}s.\end{split}
\end{equation*}
\sphinxAtStartPar
Setzen wir dies in die erste Gleichung unserer Ansatzfunktion ein und nutzen die Rechenregeln des Matrixexponnentials aus {\hyperref[\detokenize{ode/repetition:rem:matrixexponentialregeln}]{\sphinxcrossref{Remark 1.3}}}, so erhalten wir schließlich die Aussage des Lemmas
\begin{equation*}
\begin{split}x(t) = e^{At}x_0+ \int_0^t e^{A(t-s)}R(x(s)) \, \mathrm{d}s.\end{split}
\end{equation*}\end{sphinxadmonition}

\sphinxAtStartPar
Auf den ersten Blick nützt uns die Identität in {\hyperref[\detokenize{odestability/ruhelagen:lem:intexpglgn}]{\sphinxcrossref{Lemma 2.1}}} nicht viel, denn auch auf der rechten Seite taucht \(x(s)\), also die unbekannte Lösung des Anfangswertproblems \eqref{equation:odestability/ruhelagen:eq:awpallg} auf.
Es stellt sich jedoch heraus, dass wir die \sphinxstylestrong{Gronwall\sphinxhyphen{}Ungleichung} auf diese Integralgleichung anwenden können.
Diese wichtige Abschätzung in der Theorie von Differentialgleichungen ähnelt Münchhausens Methode, sich an den eigenen Haaren aus dem Sumpf zu ziehen.

\begin{sphinxShadowBox}
\sphinxstylesidebartitle{Thomas Gronwall}

\sphinxAtStartPar
\sphinxhref{https://de.wikipedia.org/wiki/Thomas\_Hakon\_Gr\%C3\%B6nwall}{Thomas Hakon Gronwall} (Geboren 16. Januar 1877 in Dylta Bruk bei Axberg/Gemeinde Örebro; Gestorben 9. Mai 1932 in New York, NY) war ein schwedischer Mathematiker.
\end{sphinxShadowBox}
\label{odestability/ruhelagen:lemma:Gronwall}
\begin{sphinxadmonition}{note}{Lemma 2.2 (Gronwall\sphinxhyphen{}Ungleichung)}



\sphinxAtStartPar
Für zwei stetige Funktionen \(f,g\in C([t_0,t_1]; \R^+)\) gelte für eine Konstante \(a \geq 0\) die Ungleichung
\begin{equation*}
\begin{split}f(t) \leq a + \int_{t_0}^t f(s)g(s)\, \mathrm{d}s \quad \forall t\in [t_0,t_1].\end{split}
\end{equation*}
\sphinxAtStartPar
Dann lässt sich der Wert der Funktion \(f\) durch die Funktion \(g\) wie folgt abschätzen
\begin{equation*}
\begin{split}f(t) \leq a \exp{ \left(\int_{t_0}^t g(s)\, \mathrm{d}s \right)} \quad \forall t\in [t_0,t_1].\end{split}
\end{equation*}\end{sphinxadmonition}

\begin{sphinxadmonition}{note}
\sphinxAtStartPar
Proof. Wir definieren zunächst eine Hilfsfunktion
\begin{equation*}
\begin{split}h(t) \ \coloneqq \ a + \int_{t_0}^t f(s)g(s)\, \mathrm{d}s\end{split}
\end{equation*}
\sphinxAtStartPar
und bemerken, dass \(0 \leq f(t) \leq h(t)\) nach Voraussetzung gilt für alle \(t \in [t_0, t_1]\).
Nun führen wir eine einfache Fallunterscheidung durch:

\sphinxAtStartPar
1. Ist \(h(t)=0\), so folgt mit der Abschätzung \(f(t) \leq h(t)\) schon, dass \(f(t) = 0\) gelten muss, so dass die Behauptung des Lemmas trivialerweise erfüllt ist.

\sphinxAtStartPar
2. Sei also im Folgenden \(h(t) > 0\).
Aus dem Haupsatz der Integral\sphinxhyphen{} und Differentialrechnung wissen wir, dass \(h'(t) = f(t)g(t)\) gilt.
Wegen \(f(t) \leq h(t)\) für alle \(t \in [t_0, t_1]\) folgt sofort, dass
\begin{equation*}
\begin{split}f(t)g(t) \leq h(t)g(t) \quad \forall t \in [t_0,t_1].\end{split}
\end{equation*}
\sphinxAtStartPar
Kombinieren wir diese Abschätzung mit der Identität der Ableitung \(h'(t)\), so erhalten wir durch Umstellen
\begin{equation*}
\begin{split}\frac{h'(t)}{h(t)} \leq g(t) \quad \forall t \in [t_0, t_1].\end{split}
\end{equation*}
\sphinxAtStartPar
Da wir \(h(t) > 0\) angenommen haben erhalten wir durch Integration beider Seiten die Abschätzung
\begin{equation*}
\begin{split}\int_{t_0}^t \frac{h'(s)}{h(s)} \, \mathrm{d}s \leq \int_{t_0}^t g(s) \, \mathrm{d}s \end{split}
\end{equation*}
\sphinxAtStartPar
für alle \(t \in [t_0, t_1]\).
Für die linke Seite können wir das Integral explizit angeben als
\begin{equation*}
\begin{split}\int_{t_0}^t \frac{h'(s)}{h(s)} \, \mathrm{d}s = \ln(h(t)) - \ln(h(t_0)) = \ln(h(t)) - \ln(a) = \ln\left(\frac{h(t)}{a}\right).\end{split}
\end{equation*}
\sphinxAtStartPar
Es gilt also nun
\begin{equation*}
\begin{split}\ln \left(\frac{h(t)}{a}\right) \leq \int_{t_0}^t g(s)\, \mathrm{d}s.\end{split}
\end{equation*}
\sphinxAtStartPar
Durch Anwenden der Exponentialfunktion auf beiden Seiten und Ausnutzen der Voraussetzung \(f(t) \leq h(t)\) erhalten wir schließlich die Behauptung des Lemmas
\begin{equation*}
\begin{split} f(t) \leq h(t)\leq a \exp{\left( \int_{t_0}^t g(s)\, ds \right)} \quad \forall t \in [t_0,t_1].\end{split}
\end{equation*}\end{sphinxadmonition}

\sphinxAtStartPar
Wir wollen folgende Bemerkungen zur Gronwall\sphinxhyphen{}Ungleichung festhalten.
\label{odestability/ruhelagen:remark-4}
\begin{sphinxadmonition}{note}{Remark 2.2}



\sphinxAtStartPar
1. Die in {\hyperref[\detokenize{odestability/ruhelagen:lemma:Gronwall}]{\sphinxcrossref{Lemma 2.2}}} beschriebene Gronwall\sphinxhyphen{}Ungleichung ist eigentlich ein Spezialfall für eine konstante Funktion \(a(t) \equiv a \geq 0\).
Die ursprünglich bewiesene Aussage gilt auch für allgemeinere Funktionen.

\sphinxAtStartPar
2. Man kann sich die Abschätzung in der Gronwall\sphinxhyphen{}Ungleichung leicht merken wenn man Gleichheit der beiden Seiten annimmt.
Die Integralgleichung
\begin{equation*}
\begin{split}f(t) = a + \int_{t_0}^t f(s)g(s)\, \mathrm{d}s \quad t\in [t_0,t_1]\end{split}
\end{equation*}
\sphinxAtStartPar
entspricht nämlich dem \sphinxstylestrong{linearen Anfangswertproblem}
\begin{equation*}
\begin{split}\dot{f}(t) &= f(t)\cdot g(t) \quad \forall t \in [t_0, t_1], \\
f(t_0) &= a,\end{split}
\end{equation*}
\sphinxAtStartPar
welches für alle \(t \in [t_0, t_1]\) die folgende explizite Lösung besitzt
\begin{equation*}
\begin{split}f(t) = a \exp{\left( \int_{t_0}^t g(s)\, \mathrm{d}s \right)}.\end{split}
\end{equation*}\end{sphinxadmonition}

\sphinxAtStartPar
Wir werden die Resultate der beiden Lemmata in den folgenden Abschnitten anwenden, um die Stabilität von Ruhelagen eines allgemeinen dynamischen Systems durch eine Linearisierung zu untersuchen.


\subsection{Asymptotische Stabilität von Ruhelagen}
<<<<<<< HEAD
\label{\detokenize{ode_stability/ruhelagen:asymptotische-stabilitat-von-ruhelagen}}
\sphinxAtStartPar
Durch die explizite Darstellung von Lösungen allgemeiner Differentialgleichungssysteme basierend auf der Linearisierung und Abweichung des Vektorfeldes \(F \colon U \rightarrow \R^n\) in \sphinxcode{\sphinxupquote{lemma:int\_exp\_glgn}} und der Gronwall\sphinxhyphen{}Ungleichung in {\hyperref[\detokenize{ode_stability/ruhelagen:lemma:Gronwall}]{\sphinxcrossref{Lemma 2.2}}} sind wir nun in der Lage die Stabilität einer Ruhelage eines dynamischen Systems zu analysieren.

\sphinxAtStartPar
Wir formulieren direkt das Hauptresultat, dass uns ein hinreichendes Kriterium für \sphinxstylestrong{asymptotische Stabilität} der Ruhelage basierend auf den Eigenwerten der Linearisierung liefert.
\label{ode_stability/ruhelagen:theorem:stabilitaet_asymptotisch_allg}
=======
\label{\detokenize{odestability/ruhelagen:asymptotische-stabilitat-von-ruhelagen}}
\sphinxAtStartPar
Durch die explizite Darstellung von Lösungen allgemeiner Differentialgleichungssysteme basierend auf der Linearisierung und Abweichung des Vektorfeldes \(F \colon U \rightarrow \R^n\) in {\hyperref[\detokenize{odestability/ruhelagen:lem:intexpglgn}]{\sphinxcrossref{Lemma 2.1}}} und der Gronwall\sphinxhyphen{}Ungleichung in {\hyperref[\detokenize{odestability/ruhelagen:lemma:Gronwall}]{\sphinxcrossref{Lemma 2.2}}} sind wir nun in der Lage die Stabilität einer Ruhelage eines dynamischen Systems zu analysieren.

\sphinxAtStartPar
Wir formulieren direkt das Hauptresultat, dass uns ein hinreichendes Kriterium für \sphinxstylestrong{asymptotische Stabilität} der Ruhelage basierend auf den Eigenwerten der Linearisierung liefert.
\label{odestability/ruhelagen:thm:stabasymallg}
>>>>>>> main
\begin{sphinxadmonition}{note}{Theorem 2.2 (Asymptotische Stabilität von Ruhelagen)}



\sphinxAtStartPar
Sei \(F \in C^1(U; \R^n)\) ein Vektorfeld auf dem offenen Phasenraum \(U \subset \R^n\).
Eine Ruhelage \(x_F \in  U \subset \R^n\) des dynamischen Systems, das durch das allgemeine Differentialgleichungssystem
\begin{equation*}
\begin{split}\dot{x}(t) = F(x(t)), \quad \forall t \in \R^+_0\end{split}
\end{equation*}
\sphinxAtStartPar
charakterisiert wird, ist \sphinxstylestrong{asymptotisch stabil} wenn für die Eigenwerte \(\lambda_i \in \C, i=1,\ldots,n\) der Linearisierung \(A \, \coloneqq \, (Df)(x_F)\) gilt
\begin{equation*}
\begin{split}\mathcal{Re}(\lambda_i)<0, \quad \text{für } i=1,\ldots,n.\end{split}
\end{equation*}\end{sphinxadmonition}

\begin{sphinxadmonition}{note}
\sphinxAtStartPar
<<<<<<< HEAD
Proof. Wie bereits in \sphinxcode{\sphinxupquote{s:linearisierung\_ruhelage}} diskutiert können wir durch Translation der Koordinaten des dynamischen Systems annehmen, dass ohne Beschränkung der Allgemeinheit \(x_F = 0 \in U\) gilt.
Da \(U\subseteq\R^n\) nach Vorraussetzung offen ist, können wir eine offene Kugel \(B_\vec{r}(0) \coloneqq \{y \in U \colon ||y|| < \vec{r}\}\) mit Radius \(\vec{r} > 0\) als Umgebung der Ruhelage \(0\) finden, so dass \(B_\vec{r}(0) \subset U\) gilt.
=======
Proof. Wie bereits in {\hyperref[\detokenize{odestability/ruhelagen:s-linearisierung-ruhelage}]{\sphinxcrossref{\DUrole{std,std-ref}{Linearisierung um Ruhelage}}}} diskutiert können wir durch Translation der Koordinaten des dynamischen Systems annehmen, dass ohne Beschränkung der Allgemeinheit \(x_F = 0 \in U\) gilt.
Da \(U\subseteq\R^n\) nach Vorraussetzung offen ist, können wir eine offene Kugel \(B_{{r^\ast}}(0) \coloneqq \{y \in U \colon ||y|| < {r^\ast}\}\) mit Radius \({r^\ast} > 0\) als Umgebung der Ruhelage \(0\) finden, so dass \(B_{r^\ast}(0) \subset U\) gilt.
>>>>>>> main

\sphinxAtStartPar
Wir nehmen im Folgenden an, dass der Realteil der Eigenwerte \(\lambda_i \in \C, i=1,\ldots,n\) der Linearisierung \(A \, \coloneqq \, Df(0)\) echt negativ ist, d.h., für ein geeignetes \(\Lambda > 0\) gilt die Abschätzung
\begin{equation*}
\begin{split}\mathcal{Re}(\lambda_i)< -\Lambda, \quad \text{für } i=1,\ldots,n. \end{split}
\end{equation*}
\sphinxAtStartPar
<<<<<<< HEAD
Dann gibt es analog zum Beweis von {\hyperref[\detokenize{ode_stability/ruhelagen:theorem:stabilit_xe4t_linear}]{\sphinxcrossref{Theorem 2.1}}} eine Konstante \(c>0\), so dass gilt
\begin{equation}\label{equation:ode_stability/ruhelagen:eq:abschaetzung_norm_exponential}
\begin{split}\|e^{At}\| \leq c\cdot e^{-\Lambda t}\quad \forall t\in \R^+_0.\end{split}
\end{equation}
\sphinxAtStartPar
Hierbei haben wir ausgenutzt, dass wir die Konstante \(\epsilon > 0\) in \DUrole{xref,std,std-ref}{eq:abschaetzung\_ew} so klein wählen können, dass \(\gamma + \epsilon < -\Lambda\) gilt.

\sphinxAtStartPar
Wir können nun einen Radius \(r\in (0,\vec{r})\) bestimmen, so dass die folgende Abschätzung gilt
\begin{equation}\label{equation:ode_stability/ruhelagen:eq:abschaetzung_residuum}
=======
Dann gibt es analog zum Beweis von {\hyperref[\detokenize{odestability/ruhelagen:thm:stablin}]{\sphinxcrossref{Theorem 2.1}}} eine Konstante \(c>0\), so dass gilt
\begin{equation}\label{equation:odestability/ruhelagen:eq:normexp}
\begin{split}\|e^{At}\| \leq c\cdot e^{-\Lambda t}\quad \forall t\in \R^+_0.\end{split}
\end{equation}
\sphinxAtStartPar
Hierbei haben wir ausgenutzt, dass wir die Konstante \(\epsilon > 0\) in \eqref{equation:odestability/ruhelagen:eq:abschaetzungew} so klein wählen können, dass \(\gamma + \epsilon < -\Lambda\) gilt.

\sphinxAtStartPar
Wir können nun einen Radius \(r\in (0,{r^\ast})\) bestimmen, so dass die folgende Abschätzung gilt
\begin{equation}\label{equation:odestability/ruhelagen:eq:residuum}
>>>>>>> main
\begin{split}\|R(x)\| \leq \frac{\Lambda}{2c} \|x\|, \quad \forall \|x\| \leq r.\end{split}
\end{equation}
\sphinxAtStartPar
Dies liegt an der totalen Differenzierbarkeit des Vektorfelds \(F\) in der Ruhelage (vgl. Kapitel 6.2 in {[}\hyperlink{cite.references:id12}{Ten21}{]}), denn dies bedeutet, dass das Residuum in der Nähe der Ruhelage schnell genug gegen Null konvergiert, so dass gilt
\begin{equation*}
\begin{split}\lim_{x\to 0} \frac{\|R(x)\|}{\|x\|} = \lim_{x\to 0}\frac{\|F(x)- (DF)(0)\cdot x\|}{\|x\|} = 0.\end{split}
\end{equation*}
\sphinxAtStartPar
Wir wollen im Folgenden zeigen, dass wenn der Anfangswert unserer unbekannten Lösung des Differentialgleichungssystems beschränkt ist durch
\begin{equation*}
\begin{split}\|x(0)\| \leq \epsilon <\frac{r}{c},\end{split}
\end{equation*}
\sphinxAtStartPar
dann soll schon für die Norm der Lösung für beliebiges \(t \geq 0\) gelten
\begin{equation*}
\begin{split}\|x(t)\| \leq c\epsilon e^{-\frac{\Lambda t}{2}}.\end{split}
\end{equation*}
\sphinxAtStartPar
<<<<<<< HEAD
Da \(c\epsilon e^{- \frac{\Lambda t}{2}} \leq c\epsilon < r <\tilde{r}\) gilt, liegt die Lösung somit noch in der offenen Kugel \(B_{\vec{r}}(0) \subset U\) und konvergiert für \(t \rightarrow \infty\) gegen 0, was den Satz beweist.

\sphinxAtStartPar
Nehmen wir also an, dass \(\|x(0)\| \leq \epsilon <\frac{r}{c}\) gelte.
Nun können wir nach \sphinxcode{\sphinxupquote{lemma:intexpglgn}} die unbekannte Lösung durch ihre Linearisierung darstellen als
=======
Da \(c\epsilon e^{- \frac{\Lambda t}{2}} \leq c\epsilon < r <\tilde{r}\) gilt, liegt die Lösung somit noch in der offenen Kugel \(B_{{r^\ast}}(0) \subset U\) und konvergiert für \(t \rightarrow \infty\) gegen 0, was den Satz beweist.

\sphinxAtStartPar
Nehmen wir also an, dass \(\|x(0)\| \leq \epsilon <\frac{r}{c}\) gelte.
Nun können wir nach {\hyperref[\detokenize{odestability/ruhelagen:lem:intexpglgn}]{\sphinxcrossref{Lemma 2.1}}} die unbekannte Lösung durch ihre Linearisierung darstellen als
>>>>>>> main
\begin{equation*}
\begin{split}x(t) = e^{At}x_0 + \int_0^t e^{A(t-s)} R(x(s))\, \mathrm{d}s.\end{split}
\end{equation*}
\sphinxAtStartPar
<<<<<<< HEAD
Nehmen wir also die Norm der unbekannten Lösung in dieser Darstellung und nutzen die Abschätzungen {\hyperref[\detokenize{ode_stability/ruhelagen:equation-eq-abschaetzung-norm-exponential}]{\sphinxcrossref{(2.3)}}} und {\hyperref[\detokenize{ode_stability/ruhelagen:equation-eq-abschaetzung-residuum}]{\sphinxcrossref{(2.4)}}}, so erhalten wir
=======
Nehmen wir also die Norm der unbekannten Lösung in dieser Darstellung und nutzen die Abschätzungen \eqref{equation:odestability/ruhelagen:eq:normexp} und \eqref{equation:odestability/ruhelagen:eq:residuum}, so erhalten wir
>>>>>>> main
\begin{equation*}
\begin{split}\|x(t)\|\leq ce^{-\Lambda t}\|x_0\| + \int_0^tce^{-\Lambda (t-s)}\frac{\Lambda}{2c}\|x(s)\|\, \mathrm{d}s, \quad \forall \|x\| \leq r.\end{split}
\end{equation*}
\sphinxAtStartPar
Multiplizieren wir beide Seiten der Ungleichung mit \(e^{\Lambda t}\) und definieren uns eine Hilfsfunktion \(f(t):=e^{\Lambda t}\|x(t)\|\), dann erhalten wir
\begin{equation*}
\begin{split}f(t)\leq \underbrace{c\|x_0\|}_{=:a} + \int_0^t \underbrace{\frac{\Lambda}{2}}_{=:g(s)} f(s)\, \mathrm{d}s.\end{split}
\end{equation*}
\sphinxAtStartPar
<<<<<<< HEAD
Für diese Form der Ungleichung bietet es sich an das {\hyperref[\detokenize{ode_stability/ruhelagen:lemma:Gronwall}]{\sphinxcrossref{Lemma 2.2}}} zur Gronwall\sphinxhyphen{}Ungleichung anzuwenden, durch das wir schließlich folgendes Resultat bekommen
=======
Für diese Form der Ungleichung bietet es sich an das {\hyperref[\detokenize{odestability/ruhelagen:lemma:Gronwall}]{\sphinxcrossref{Lemma 2.2}}} zur Gronwall\sphinxhyphen{}Ungleichung anzuwenden, durch das wir schließlich folgendes Resultat bekommen
>>>>>>> main
\begin{equation*}
\begin{split}f(t) \leq c \|x_0\| \exp{\left( \frac{1}{2} \int_0^t \Lambda \, \mathrm{d}s \right) }
\leq c \epsilon e^{\frac{\Lambda}{2} t} \leq r e^{\frac{\Lambda}{2} t}.\end{split}
\end{equation*}
\sphinxAtStartPar
<<<<<<< HEAD
Durch Multiplikation beider Seiten mit \$e\textasciicircum{}\{\sphinxhyphen{}\textbackslash{}Lambda t\} führt dies zur finalen Abschätzung
=======
Durch Multiplikation beider Seiten mit \(e^{-\Lambda t}\) führt dies zur finalen Abschätzung
>>>>>>> main
\begin{equation*}
\begin{split} \|x(t)\|\leq re^{-\frac{\Lambda}{2}t}, \quad \forall t\in\R^+_0.\end{split}
\end{equation*}
\sphinxAtStartPar
<<<<<<< HEAD
Wir sehen also ein, dass die unbekannte Lösung für alle nicht\sphinxhyphen{}negativen Zeiten in der offenen Kugel \(B_r(0) \subset B_{\vec{r}}(0) \subset U\) enthalten ist und offensichtlich gegen Null konvergiert.
=======
Wir sehen also ein, dass die unbekannte Lösung für alle nicht\sphinxhyphen{}negativen Zeiten in der offenen Kugel \(B_r(0) \subset B_{{r^\ast}}(0) \subset U\) enthalten ist und offensichtlich gegen Null konvergiert.
>>>>>>> main
Damit ist die Ruhelage \(0 \in U\) asymptotisch stabil.
\end{sphinxadmonition}

\sphinxAtStartPar
Folgende Bemerkung geht speziell auf ein Detail des Beweises ein, das eine Aussage zum Konvergenzradius der Lösungen eines dynamisches Systems zulässt.
<<<<<<< HEAD
\label{ode_stability/ruhelagen:remark-6}
=======
\label{odestability/ruhelagen:remark-6}
>>>>>>> main
\begin{sphinxadmonition}{note}{Remark 2.3 (Attraktionsbassin)}



\sphinxAtStartPar
<<<<<<< HEAD
Der Beweis von {\hyperref[\detokenize{ode_stability/ruhelagen:theorem:stabilitaet_asymptotisch_allg}]{\sphinxcrossref{Theorem 2.2}}} liefert zusätzlich die Aussage, dass alle Punkte \(x\in U\) im Phasenraum mit
=======
Der Beweis von {\hyperref[\detokenize{odestability/ruhelagen:thm:stabasymallg}]{\sphinxcrossref{Theorem 2.2}}} liefert zusätzlich die Aussage, dass alle Punkte \(x\in U\) im Phasenraum mit
>>>>>>> main
\begin{equation*}
\begin{split}\|x\| < \frac{r}{c}\end{split}
\end{equation*}
\sphinxAtStartPar
zu Orbits gehören, die gegen die Ruhelage \(0 \in U\) konvergieren.
Diesen attraktiven Einzugsbereich der Ruhelage nennt man auch das \sphinxstylestrong{Attraktionsbassin} der Ruhelage.
\end{sphinxadmonition}


\subsection{Lyapunov\sphinxhyphen{}Stabilität von Ruhelagen}
\label{\detokenize{odestability/ruhelagen:lyapunov-stabilitat-von-ruhelagen}}
\sphinxAtStartPar
<<<<<<< HEAD
Während ein hinreichendes Kriterium für das Vorliegen \sphinxstyleemphasis{asymptotischer Stabilität} die strikte Ungleichung \(Re(\lambda_i)<0\) für die Eigenwerte \(\lambda_i\) der Jacobi\sphinxhyphen{}Matrix war, ist die Situation bezüglich der Liapunov\sphinxhyphen{}Stabilität einer Ruhelage \sphinxstylestrong{komplizierter}.
Hierzu wollen wir ein Resultat für den Fall von linearen dynamischen Systemen im Folgenden formulieren.
\label{ode_stability/ruhelagen:theorem:stabilitaet_lyapunov_linear}
=======
Während ein hinreichendes Kriterium für das Vorliegen \sphinxstyleemphasis{asymptotischer Stabilität} die strikte Ungleichung \(Re(\lambda_i)<0\) für die Eigenwerte \(\lambda_i\) der Jacobi\sphinxhyphen{}Matrix war, ist die Situation bezüglich der Lyapunov\sphinxhyphen{}Stabilität einer Ruhelage \sphinxstylestrong{komplizierter}.
Hierzu wollen wir ein Resultat für den Fall von linearen dynamischen Systemen im Folgenden formulieren.
\label{odestability/ruhelagen:thm:stablyaplinear}
>>>>>>> main
\begin{sphinxadmonition}{note}{Theorem 2.3 (Lyapunov\sphinxhyphen{}Stabilität von Ruhelagen)}



\sphinxAtStartPar
Sei \(A\in \R^{n\times n}\) eine Matrix mit den Eigenwerten \(\lambda_1,\dots, \lambda_n\in \C\).
<<<<<<< HEAD
Besitzen die Eigenwerte \(\lambda_i \in \C, i=1,\ldots,n\) von \(A\) einen nicht\sphinxhyphen{}positiven Realteil \(Re(\lambda_i) \leq 0\), und ist im Fall \(Re(\lambda_i)=0\) die geometrische Vielfachheit gleich der algebraischen Vielfachheit des Eigenwerts, dann ist \(0\in \R^n\) eine \sphinxstylestrong{Liapunov\sphinxhyphen{}stabile} Ruhelage des dynamischen Systems, dass durch das lineare Differentialgleichungssystem
=======
Besitzen die Eigenwerte \(\lambda_i \in \C, i=1,\ldots,n\) von \(A\) einen nicht\sphinxhyphen{}positiven Realteil \(Re(\lambda_i) \leq 0\), und ist im Fall \(Re(\lambda_i)=0\) die geometrische Vielfachheit gleich der algebraischen Vielfachheit des Eigenwerts, dann ist \(0\in \R^n\) eine \sphinxstylestrong{Lyapunov\sphinxhyphen{}stabile} Ruhelage des dynamischen Systems, dass durch das lineare Differentialgleichungssystem
>>>>>>> main
\begin{equation*}
\begin{split}\dot{x}(t) = Ax(t), \quad  \forall t \in I \subset \R^+_0\end{split}
\end{equation*}
\sphinxAtStartPar
charakterisiert wird.
\end{sphinxadmonition}

\begin{sphinxadmonition}{note}
\sphinxAtStartPar
<<<<<<< HEAD
Proof. Aus {\hyperref[\detokenize{ode_stability/ruhelagen:theorem:stabilit_xe4t_linear}]{\sphinxcrossref{Theorem 2.1}}} wissen wir bereits, dass im Fall eines linearen dynamischen Systems \(\vec{0} \in U\) eine Ruhelage im Phasenraum \(U \subset \R^n\) ist.
=======
Proof. Aus {\hyperref[\detokenize{odestability/ruhelagen:thm:stablin}]{\sphinxcrossref{Theorem 2.1}}} wissen wir bereits, dass im Fall eines linearen dynamischen Systems \(\vec{0} \in U\) eine Ruhelage im Phasenraum \(U \subset \R^n\) ist.
>>>>>>> main
Seien \(\lambda_1, \ldots, \lambda_k \in \C\) für \(k \leq n\) die paarweise verschiedenen Eigenwerte der Matrix \(A\).
Wir betrachten wieder die Jordansche Normalform \(J = S^{-1}AS\) der Matrix \(A\) für Transformationsmatrizen \(S,S^{-1} \in \C^{n \times n}\) und
\begin{equation*}
\begin{split}J=
\begin{pmatrix}
J_{r_1}(\lambda_1)& & & 0\\
 & J_{r_2}(\lambda_2) & & \\
 & & \ddots & \\
 0 & & & J_{r_k}(\lambda_k)
\end{pmatrix}.\end{split}
\end{equation*}
\sphinxAtStartPar
Hierbei bezeichnen \(r_i \in \N, i=1,\ldots, k\) die algebraischen Vielfachheiten der zugehörigen Eigenwerte und jeder Jordanblock (vgl. Kapitel 2.7 in {[}\hyperlink{cite.references:id12}{Ten21}{]})) hat die Gestalt
\begin{equation*}
\begin{split} J_r(\lambda) \ \coloneqq \ \begin{pmatrix}
\lambda & 1 & & 0\\
 & \ddots & \ddots & \\
 & & \ddots & 1\\
 0 & & & \lambda
 \end{pmatrix} \in \C^{r\times r}\end{split}
\end{equation*}
\sphinxAtStartPar
<<<<<<< HEAD
Mit den Rechenregeln für das Matrixexponential aus {\hyperref[\detokenize{ode/repetition:rem:matrixexponential_regeln}]{\sphinxcrossref{Remark 1.3}}} folgt
=======
Mit den Rechenregeln für das Matrixexponential aus {\hyperref[\detokenize{ode/repetition:rem:matrixexponentialregeln}]{\sphinxcrossref{Remark 1.3}}} folgt
>>>>>>> main
\begin{equation*}
\begin{split}e^{Jt} = \begin{pmatrix}
\exp{(J_{r_1}(\lambda_1)t)} & & 0\\
 & \ddots & \\
 0& & \exp{(J_{r_k}(\lambda_k)t)}
 \end{pmatrix}.\end{split}
\end{equation*}
\sphinxAtStartPar
Betrachten wir nun die Norm der Lösungen des homogenen, linearen Differentialgleichungssystems für einen Startwert \(x_0 \in U\) mit
\begin{equation*}
\begin{split}\| \Phi_t(x_0) \| = \|e^{At}x_0\| = \|S^{-1}e^{Jt}S x_0\| \leq \|S^{-1}\| \|e^{Jt}\| \|S\| \|x_0\|,\end{split}
\end{equation*}
\sphinxAtStartPar
<<<<<<< HEAD
so sehen wir ein, dass die Ruhelage \(\vec{0} \in U\) \sphinxstylestrong{Lyapunov\sphinxhyphen{}stabil} ist wenn für alle Jordanblöcke \(J_{r_i}(\lambda_i), i=1,\ldots,k\) von \(J\) der Ursprung \(0\in \C^{r_i}\) eine Liapunov\sphinxhyphen{}stabile Ruhelage des folgenden linearen Differentialgleichungssystems ist
=======
so sehen wir ein, dass die Ruhelage \(\vec{0} \in U\) \sphinxstylestrong{Lyapunov\sphinxhyphen{}stabil} ist wenn für alle Jordanblöcke \(J_{r_i}(\lambda_i), i=1,\ldots,k\) von \(J\) der Ursprung \(0\in \C^{r_i}\) eine Lyapunov\sphinxhyphen{}stabile Ruhelage des folgenden linearen Differentialgleichungssystems ist
>>>>>>> main
\begin{equation*}
\begin{split} \dot{y}(t) = J_{r_i}(\lambda_i) y(t), \quad t \in I \subset \R^+_0.\end{split}
\end{equation*}
\sphinxAtStartPar
<<<<<<< HEAD
Dies ist bereits gegeben falls für einen Eigenwert \(Re(\lambda_i)<0\) gilt, denn damit folgt aus {\hyperref[\detokenize{ode_stability/ruhelagen:theorem:stabilit_xe4t_linear}]{\sphinxcrossref{Theorem 2.1}}} sogar schon \sphinxstylestrong{asymptotische Stabilität}, welche Lyapunov\sphinxhyphen{}Stabilität induziert.
=======
Dies ist bereits gegeben falls für einen Eigenwert \(Re(\lambda_i)<0\) gilt, denn damit folgt aus {\hyperref[\detokenize{odestability/ruhelagen:thm:stablin}]{\sphinxcrossref{Theorem 2.1}}} sogar schon \sphinxstylestrong{asymptotische Stabilität}, welche Lyapunov\sphinxhyphen{}Stabilität induziert.
>>>>>>> main

\sphinxAtStartPar
Betrachten wir also nun einen komplexen Eigenwert \(\lambda_i \in \C\) von \(A\) mit \(Re(\lambda_i)=0\) und für den die geometrische Vielfachheit nach Vorraussetzung gleich der algebraischen Vielfachheit ist.
In diesem Fall ist der ihm zugeordnete Jordanblock eine Diagonalmatrix auf deren Hauptdiagonale der Eigenwert \(\lambda_i \in \C\) steht, da alle Jordankästchen eindimensional sind.
In diesem Fall sehen wir, dass die Norm des Matrixexponentials beschränkt ist und wir dadurch \sphinxstylestrong{Lyapunov\sphinxhyphen{}Stabilität} der Ruhelage gezeigt haben, da gilt
<<<<<<< HEAD
\begin{equation*}
\begin{split}\|e^{J_{r_i}(\lambda_i)t)}\| = |e^{\lambda_i t}| = |e^0e^{\mathcal{Im}(\lambda_i) t}| = |\cos{(\mathcal{Im}(\lambda_i)t)} + i \sin{(\mathcal{Im}(\lambda_i)t)}| = 1.\end{split}
\end{equation*}
\sphinxAtStartPar
Für diese Umformung haben wir die Definition der komplexen Exponentialfunktion genutzt, für die gilt:
\begin{equation*}
=======
\begin{equation*}
\begin{split}\|e^{J_{r_i}(\lambda_i)t)}\| = |e^{\lambda_i t}| = |e^0e^{\mathcal{Im}(\lambda_i) t}| = |\cos{(\mathcal{Im}(\lambda_i)t)} + i \sin{(\mathcal{Im}(\lambda_i)t)}| = 1.\end{split}
\end{equation*}
\sphinxAtStartPar
Für diese Umformung haben wir die Definition der komplexen Exponentialfunktion genutzt, für die gilt:
\begin{equation*}
>>>>>>> main
\begin{split}e^z = e^{x+iy} = e^xe^iy = e^x(\cos(y) + i\sin(y)), \quad \text{für } z = x+iy \in \C.\end{split}
\end{equation*}\end{sphinxadmonition}

\sphinxAtStartPar
Das folgende Beispiel illustriert, dass eine Ruhelage instabil werden kann, wenn die geometrische Vielfachheit nicht mit der algebraischen Vielfachheit übereinstimmt für einen Eigenwert \(\lambda =0\) der Koeffizientenmatrix \(A\).
<<<<<<< HEAD
\label{ode_stability/ruhelagen:example-8}
=======
\label{odestability/ruhelagen:example-8}
>>>>>>> main
\begin{sphinxadmonition}{note}{Example 2.2}



\sphinxAtStartPar
Sei \(U \subset \R^2\) der Phasenraum und wir betrachten das homogene, lineare Differentialgleichungssystem
\begin{equation*}
\begin{split}\dot{x}(t) = A x(t), \quad \forall t \in \R_0^+\end{split}
\end{equation*}
\sphinxAtStartPar
für eine Koeffizientenmatrix
\begin{equation*}
\begin{split}A = \begin{pmatrix} 0&1\\0&0\end{pmatrix}.\end{split}
\end{equation*}
\sphinxAtStartPar
Wie man leicht nachrechnet besitzt diese Matrix den Eigenwert \(\lambda = 0\) mit algebraischer Vielfachheit \(2\) und geometrischer Vielfachheit \(1\) zum Eigenvektor \(v = (1,0)^T \in \R^2\).
Die Vielfachheiten des Eigenwert \sphinxstylestrong{stimmen} also \sphinxstylestrong{nicht überein}.
<<<<<<< HEAD

\sphinxAtStartPar
Aus {\hyperref[\detokenize{ode_stability/ruhelagen:theorem:stabilit_xe4t_linear}]{\sphinxcrossref{Theorem 2.1}}} wissen wir, dass eine Ruhelage in \(\vec{0} \in \R^2\) existiert.
Man sieht jedoch leicht ein, dass sogar jeder Punkt \(x_0 = (y, 0) \in U\) eine Ruhelage des Systems darstellt, da diese Punkte ein Vielfaches des Eigenvektors zum Eigenwert \(\lambda = 0\) darstellen und somit im Kern der Matrix \(A\) liegen, d.h., für diese Punkte ist die rechte Seite des Differentialgleichungssystems \(\vec{0} \in \R^2\) und somit liegt eine Ruhelage vor.

\sphinxAtStartPar
Wir wollen die Stabilität dieser Ruhelagen im Folgenden untersuchen.
Hierzu betrachten wir die Norm des Phasenflusses \(\Phi \colon I \times U \rightarrow U\), der für einen gegebenen Anfangswert \(x_0 = (y,z) \in U\) mit \(z \neq 0\) der die Lösung des Differentialgleichungssystems beschreibt mit
\begin{equation*}
\begin{split}\| \Phi_t(x_0) \| &= \| e^{At}x_0 \| = \| \sum_{k=0}^\infty \frac{(At)^k}{k!} x_0\| = \| [\underbrace{(At)^0}_{=I_2} + (At)^1] x_0\| \\
&= \| \begin{pmatrix} 1 & t \\ 0 & 1\end{pmatrix}\begin{pmatrix} y \\ z \end{pmatrix} \| = \| \begin{pmatrix} y + tz \\ z\end{pmatrix} \| \overset{t\to \infty}{\longrightarrow} \infty.\end{split}
\end{equation*}
\sphinxAtStartPar
Wir sehen also, dass für jeden Anfangswert \(x_0 = (y,z)\) mit \(z \neq 0\) die Lösung des Differentialgleichungssystems divergiert und somit ist jede Ruhelage des dynamischen Systems \sphinxstylestrong{instabil}.
\end{sphinxadmonition}

\sphinxAtStartPar
Leider kann man nicht wie im Fall der asymptotischen Stabilität vom linearen auf den nichtlinearen Fall schließen, wie das folgende Beispiel zeigt.
\label{ode_stability/ruhelagen:example-9}
=======

\sphinxAtStartPar
Aus {\hyperref[\detokenize{odestability/ruhelagen:thm:stablin}]{\sphinxcrossref{Theorem 2.1}}} wissen wir, dass eine Ruhelage in \(\vec{0} \in \R^2\) existiert.
Man sieht jedoch leicht ein, dass sogar jeder Punkt \(x_0 = (y, 0) \in U\) eine Ruhelage des Systems darstellt, da diese Punkte ein Vielfaches des Eigenvektors zum Eigenwert \(\lambda = 0\) darstellen und somit im Kern der Matrix \(A\) liegen, d.h., für diese Punkte ist die rechte Seite des Differentialgleichungssystems \(\vec{0} \in \R^2\) und somit liegt eine Ruhelage vor.

\sphinxAtStartPar
Wir wollen die Stabilität dieser Ruhelagen im Folgenden untersuchen.
Hierzu betrachten wir die Norm des Phasenflusses \(\Phi \colon I \times U \rightarrow U\), der für einen gegebenen Anfangswert \(x_0 = (y,z) \in U\) mit \(z \neq 0\) der die Lösung des Differentialgleichungssystems beschreibt mit
\begin{equation*}
\begin{split}\| \Phi_t(x_0) \| &= \| e^{At}x_0 \| = \| \sum_{k=0}^\infty \frac{(At)^k}{k!} x_0\| = \| [\underbrace{(At)^0}_{=I_2} + (At)^1] x_0\| \\
&= \| \begin{pmatrix} 1 & t \\ 0 & 1\end{pmatrix}\begin{pmatrix} y \\ z \end{pmatrix} \| = \| \begin{pmatrix} y + tz \\ z\end{pmatrix} \| \overset{t\to \infty}{\longrightarrow} \infty.\end{split}
\end{equation*}
\sphinxAtStartPar
Wir sehen also, dass für jeden Anfangswert \(x_0 = (y,z)\) mit \(z \neq 0\) die Lösung des Differentialgleichungssystems divergiert und somit ist jede Ruhelage des dynamischen Systems \sphinxstylestrong{instabil}.
\end{sphinxadmonition}

\sphinxAtStartPar
Leider kann man nicht wie im Fall der asymptotischen Stabilität vom linearen auf den nichtlinearen Fall schließen, wie das folgende Beispiel zeigt.
\label{odestability/ruhelagen:example-9}
>>>>>>> main
\begin{sphinxadmonition}{note}{Example 2.3}



\sphinxAtStartPar
Wir betrachten eine gewöhnliche Differentialgleichung 1. Ordnung der Form
\begin{equation*}
\begin{split}\dot{x}(t) = \alpha x(t) + \beta x^3(t), \forall t \in \R^+_0.\end{split}
\end{equation*}
\sphinxAtStartPar
mit freien Parametern \(\alpha, \beta \in \R\).

\sphinxAtStartPar
Wie man einsieht ist \(0\) eine Ruhelage des dynamischen Systems, das durch diese Differentialgleichung charakterisiert wird.
Wir betrachten die Linearisierung der Differentialgleichung in der Ruhelage mit \(A := (DF)(0) = \alpha\) und erhalten
\begin{equation*}
\begin{split}\dot{x}(t) = A x(t) = \alpha x(t), \quad \forall t \in \R^+_0.\end{split}
\end{equation*}
\sphinxAtStartPar
Folgende Fallunterscheidung zeigt nun das Stabilitätsverhalten der Ruhelage in Abhängigkeit der gewählten Parameter \(\alpha, \beta \in \R\):


\begin{savenotes}\sphinxattablestart
\centering
\begin{tabulary}{\linewidth}[t]{|T|T|T|}
\hline

\sphinxAtStartPar

&\sphinxstyletheadfamily 
\sphinxAtStartPar
linearisierte Gleichung
&\sphinxstyletheadfamily 
\sphinxAtStartPar
nicht lineare Gleichung
\\
\hline
\sphinxAtStartPar
\(\alpha<0\)
&
\sphinxAtStartPar
asymptotisch stabil
&
\sphinxAtStartPar
asymptotisch stabil
\\
\hline
\sphinxAtStartPar
\(\alpha>0\)
&
\sphinxAtStartPar
instabil
&
\sphinxAtStartPar
instabil
\\
\hline
\sphinxAtStartPar
\(\alpha=0\)
&
\sphinxAtStartPar
Lyapunov\sphinxhyphen{}stabil
&
\sphinxAtStartPar
asymptotisch stabil für \(\beta<0\)
\\
\hline
\sphinxAtStartPar

&
\sphinxAtStartPar

&
\sphinxAtStartPar
stabil für \(\beta =0 \)
\\
\hline
\sphinxAtStartPar

&
\sphinxAtStartPar

&
\sphinxAtStartPar
instabil \(\beta > 0\)
\\
\hline
\end{tabulary}
\par
\sphinxattableend\end{savenotes}
<<<<<<< HEAD

\sphinxAtStartPar
Wie man sieht hängt die Stabilität im nichtlinearen Fall nicht nur vom Parameter \(\alpha\), sondern ebenfalls von \(\beta\) ab, was eine Stabilitätsanalyse deutlich komplizierter macht.
\end{sphinxadmonition}


\chapter{Vektoranalysis}
\label{\detokenize{vektoranalysis/vektoranalysis:vektoranalysis}}\label{\detokenize{vektoranalysis/vektoranalysis::doc}}
\sphinxAtStartPar
In diesem Kapitel führen wir wichtige Konzepte der \sphinxstyleemphasis{Vektoranalysis} ein. Insbesondere schaffen wir die Grundlagen für eine spezielle Art der Mehrdimensionalen Integration, das Integrieren über sog. \sphinxstyleemphasis{Untermannigfaltigkeiten} des \(\R^n\).
Um diese Integration durchzuführen, entwickeln wir das Kalkül der \sphinxstyleemphasis{Differentialformen} auf Mannigfaltigkeiten.

\sphinxAtStartPar
Dieses Kalkül lässt auch den \sphinxstyleemphasis{geometrischen Gehalt} physikalischer
Theorien wie Elektrodynamik oder Allgemeine Relativitätstheorie klar
hervortreten. So lassen sich beispielsweise die sog. Maxwellschen Gleichungen der
Elektrodynamik in Differentialformenkalkül schreiben.

\sphinxAtStartPar
Als zusätzliche Literatur und Referenz für diese Thematiken empfehlen wir das Buch von \sphinxstyleemphasis{Agricola} und \sphinxstyleemphasis{Friedrich}, {[}\hyperlink{cite.references:id13}{AF13}{]}.


\section{Multilinearformen}
\label{\detokenize{vektoranalysis/multilinear:multilinearformen}}\label{\detokenize{vektoranalysis/multilinear::doc}}
\sphinxAtStartPar
In diesem Abschnitt wollen wir Multilinearformen kennenlernen. Für Vektorräume \(\V, W\) über einem Körper \(\K\) haben Sie bereits den Begriff der Linearform, also einer linearen Abbildung \(\varphi:\V\rightarrow W\) kennengelernt. Die Idee der Multilinearform ist anstatt einem, gleich \(k\)\sphinxhyphen{}viele Vektorräume \(V_1,\ldots,V_k\) über \(\K\) zu betrachten und das Konzept der Lineratität auf Abbildung \(\varphi:\V_1\times\ldots\V_k\rightarrow W\) zu übertragen. Zur Vereinfachung werden wir im folgenden nur den Körper \(\K=\R\) betrachten, in den meisten Fällen lassen sich die Konzepte aber direkt auf allgemeine Körper übertragen.

\sphinxAtStartPar
Wir beginnen zunächst mit einer Wiederolung und betrachten die schon bekannten Linearformen. Insbesondere soll der nächste Abschnitt als Wiederholung zum vorherigen Semester die verschiedenen Begriffe des Dualraums abgrenzen.


\subsection{Dualräume}
\label{\detokenize{vektoranalysis/multilinear:dualraume}}
\sphinxAtStartPar
Für einen reellen Vektorraum \(\V\) wollen wir lineare Abbildung \(\varphi:V\to\R\) betrachten.
\label{vektoranalysis/multilinear:definition-0}
\begin{sphinxadmonition}{note}{Definition 3.1 (Algebraischer Dualraum)}

=======
>>>>>>> main


\sphinxAtStartPar
Es sei \(\V\) ein \(\R\)\sphinxhyphen{}Vektorraum, die Menge
\begin{equation*}
\begin{split}\V^\ast := \{\varphi:\V\rightarrow\R: \varphi\text{ ist linear}\}\end{split}
\end{equation*}
\sphinxAtStartPar
<<<<<<< HEAD
heißt \sphinxstylestrong{algebraischer Dualraum}.
=======
Wie man sieht hängt die Stabilität im nichtlinearen Fall nicht nur vom Parameter \(\alpha\), sondern ebenfalls von \(\beta\) ab, was eine Stabilitätsanalyse deutlich komplizierter macht.
>>>>>>> main
\end{sphinxadmonition}

\sphinxAtStartPar
Aus {[}\hyperlink{cite.references:id12}{Ten21}{]} ist bereits der Begriff des \sphinxstyleemphasis{topologischen Dualraums} bekannt, welcher allerdings eine etwas restriktivere Definition hat. Sie fordert noch zusätzlich die Stetigkeit der linearen Abbildungen.

\begin{sphinxadmonition}{danger}{Danger:}
\sphinxAtStartPar
Der algebraische Dualraum ist im allgemeinen nicht gleich dem topologischen. Der Hauptzweck dieses Abschnitts ist es diese Tatsache klar zu machen und die Unterschiede der beiden Definitionen zu verstehen.
\end{sphinxadmonition}
\label{vektoranalysis/multilinear:definition-1}
\begin{sphinxadmonition}{note}{Definition 3.2 (Topologischer Dualraum)}



\sphinxAtStartPar
Es sei \(\V\) ein normierter \(\R\)\sphinxhyphen{}Vektorraum für einen Körper \(\R\), dann heißt die Menge
\begin{equation*}
\begin{split}\V^\prime := \{\varphi:\V\rightarrow\R: \varphi\text{ ist linear und stetig}\}\end{split}
\end{equation*}
\sphinxAtStartPar
\sphinxstylestrong{topologischer Dualraum}.
\end{sphinxadmonition}
\label{vektoranalysis/multilinear:remark-2}
\begin{sphinxadmonition}{note}{Remark 3.1}



\sphinxAtStartPar
Damit die obige Definition sinnvoll ist, ist es in der Tat nicht notwendig, dass \(X\) ein normierter Raum ist. Es reicht anzunehmen, dass \(\V\) ein topologischer Vektorraum ist.
\end{sphinxadmonition}

\sphinxAtStartPar
In unserem Kontext spielt allerdings der Begriff des algebraischen Dualraums eine wichtige Rolle, welcher im Folgenden eingeführt wird. Man erkennt sofort, dass stets \(\V^\prime\subset \V^\ast\) gilt. Weiterhin stimmen die beiden Räume im endlich\sphinxhyphen{}dimensionalen Fall überein.
\label{vektoranalysis/multilinear:lemma-3}
\begin{sphinxadmonition}{note}{Lemma 3.1}



\sphinxAtStartPar
Für \(n\in\N\) sei \(\V\) ein \(n\)\sphinxhyphen{}dimensionaler \(\R\)\sphinxhyphen{}Vektorraum, dann gilt
\begin{equation*}
\begin{split}V^\prime = V^\ast.\end{split}
\end{equation*}\end{sphinxadmonition}
\label{vektoranalysis/multilinear:remark-4}
\begin{sphinxadmonition}{note}{Remark 3.2}



\sphinxAtStartPar
Die Norm auf \(\V\) in der obigen Aussage ist durch das Standardskalarprodukt induziert.
\end{sphinxadmonition}

\begin{sphinxadmonition}{note}
\sphinxAtStartPar
Proof. Siehe Übung.
\end{sphinxadmonition}


\subsection{k\sphinxhyphen{}Multilinearformen}
\label{\detokenize{vektoranalysis/multilinear:k-multilinearformen}}
\sphinxAtStartPar
Wir verallgeminern nun den Begriff der Linearität in der folgenden Definition.
\label{vektoranalysis/multilinear:def:multilinear}
\begin{sphinxadmonition}{note}{Definition 3.3}



\sphinxAtStartPar
Für \(i=1,\ldots,k\) sei \(\V_i\), sowie \(W\) ein reeller Vektorraum. Eine Abbildung
\begin{equation*}
\begin{split}\varphi:\V_1\times\ldots\times \V_k\ \to W\end{split}
\end{equation*}
\sphinxAtStartPar
heißt k\sphinxhyphen{}\sphinxstylestrong{(multi)linear}, wenn für beliebige \(z_i\in\V_i\) und jede Komponente \(i\in\{1,\ldots,k\}\) die Abbildung
\begin{equation*}
\begin{split}V_i &\to W\\
x&\mapsto \varphi_i(x):= \varphi(z_1,\ldots, z_{i-1}, x, z_{i+1},\ldots,z_k)\end{split}
\end{equation*}
\sphinxAtStartPar
linear ist. Die Menge aller \(k\)\sphinxhyphen{}linearen Abbildungen wird mit \(L^k(\V_1\times\ldots\times \V_k, W)\) bezeichnet.
\end{sphinxadmonition}
\label{vektoranalysis/multilinear:remark-6}
\begin{sphinxadmonition}{note}{Remark 3.3}

<<<<<<< HEAD


\sphinxAtStartPar
Ausgeschrieben bedeutet die Bedingung in der obigen Definition, dass für alle \(z_i\in V_i\), \(\lambda\in\R\),
und insbesondere für jedes \(i\in\{1,\ldots,k\}\), \(x,y\in \V_i\)  gilt,
\begin{equation*}
\begin{split}\varphi(z_1,\ldots,z_{i-1},\lambda x, z_{i+1},\ldots,z_k) = \lambda
\varphi(z_1,\ldots,z_{i-1}, x, z_{j+1}, \ldots,z_k)\end{split}
\end{equation*}
\sphinxAtStartPar
und
\begin{equation*}
\begin{split}&\varphi(z_1,\ldots,z_{i-1},x+y,z_{j+1},\ldots,z_k)\\
= 
&\varphi(z_1,\ldots,x,\ldots,z_k) + \varphi(z_1,\ldots,y,\ldots,z_k).\end{split}
\end{equation*}\end{sphinxadmonition}

\sphinxAtStartPar
Falls alle Vektorräume übereinstimmen, d.h., \(\V_i = \V\) für alle \(i=1,\ldots,k\), so schreibt man auch \(L^k(\V\times\ldots\times \V,W) = L^k(\V,W)\).

\sphinxAtStartPar
Viele multilineare Abbildungen sind schon aus der Linearen Algebra vertraut. Im folgenden Beispiel wiederholen wir einige bekannte Beispiele unter dem Aspekt der Multilinearität.
\label{vektoranalysis/multilinear:ex:multi}
\begin{sphinxadmonition}{note}{Example 3.1}



\sphinxAtStartPar
Wir betrachten Beispiele für verschiedene \(k\in\N\).

\sphinxAtStartPar
\sphinxstylestrong{\(k=1\)}:

\sphinxAtStartPar
In diesem Fall haben wir bereits gesehen, dass \(L^1(\V,\R) = \V^\ast\).

\sphinxAtStartPar
\sphinxstylestrong{\(k=2\)}:

\sphinxAtStartPar
Es sei \(\V=\R^n\) mit kanonischem innerem Produkt \(\langle\cdot,\cdot\rangle\). Für \(A\in\R^{n,n}\) ist
\begin{equation*}
\begin{split}\varphi:\V\times \V\to\R\\ 
\varphi(x, y) :=\langle x,A y \rangle\end{split}
\end{equation*}
\sphinxAtStartPar
eine \sphinxstylestrong{Bilinearform}. Sie heißt \sphinxstyleemphasis{symmetrisch},
falls
\begin{equation*}
\begin{split}\varphi(x, y) = \varphi( y, x)\qquad (x, y\in \V)\end{split}
\end{equation*}
\sphinxAtStartPar
und \sphinxstyleemphasis{antisymmetrisch} falls
\begin{equation*}
\begin{split}\varphi(x, y) = -\varphi( y, x)\qquad (x, y\in \V).\end{split}
\end{equation*}
\sphinxAtStartPar
\sphinxstylestrong{\(k=n\)}:

\sphinxAtStartPar
Es sei \(\V=\R^n\). Die \(n\)\sphinxhyphen{}lineare Abbildung
\begin{equation*}
\begin{split}\varphi(z_1,\ldots,z_n) := \det((z_1,\ldots,z_n))\end{split}
\end{equation*}
\sphinxAtStartPar
heißt \sphinxstylestrong{Determinantenform}. Wir beachten, dass hierbei jedes \(z_i\in\R^n\) ein Vektor und \((z_1,\ldots,z_n)\) eine Matrix ist.
Die Form gibt das orientierte Volumen des von \(z_1,\ldots,z_n\) aufgespannten Parallelotops an.
\end{sphinxadmonition}


\subsection{Der Vektorraum der Multilinearformen}
\label{\detokenize{vektoranalysis/multilinear:der-vektorraum-der-multilinearformen}}
\sphinxAtStartPar
Die Definition einer \(k\)\sphinxhyphen{}linearen Abbildung ermöglicht es uns sehr direkt eine Vektorraumstruktur zu definieren.
\label{vektoranalysis/multilinear:lemma-8}
\begin{sphinxadmonition}{note}{Lemma 3.2}



\sphinxAtStartPar
Es seien \(\V_1,\ldots,\V_k\) sowie \(W\) reelle Vektorräume, dann ist die Menge \(L^k(\V_1\times\ldots\V_k,W)\) ein Vektorraum über \(\R\) bezüglich der Addition
\begin{equation*}
\begin{split}(\varphi_1+\varphi_2)(z_1,\ldots,z_k) := \varphi_1(z_1,\ldots,z_k) +
\varphi_2(z_1,\ldots,z_k),\quad \varphi_1,\varphi_2\in L^k(\V,\R)\end{split}
\end{equation*}
\sphinxAtStartPar
und der Skalarmultiplikation
\begin{equation*}
\begin{split}(\lambda\varphi)(z_1,\ldots,z_k) := \lambda\big(\varphi(z_1,\ldots,z_k)\big),\quad\varphi\in L^k(\V,\R), \lambda\in\R.\end{split}
\end{equation*}\end{sphinxadmonition}

\begin{sphinxadmonition}{note}
\sphinxAtStartPar
Proof. Siehe Übung.
\end{sphinxadmonition}

\sphinxAtStartPar
Als wichtigen Spezialfall erhalten wir für \(k=1\) den Dualraum \(L^1(\V,\R)\). Für diesen Vektorraum können wir eine spezielle Basis charakterisieren, die sogenannte \sphinxstylestrong{duale Basis}.
\label{vektoranalysis/multilinear:lemma-9}
\begin{sphinxadmonition}{note}{Lemma 3.3 (Duale Basis)}



\sphinxAtStartPar
Es sei \(\V\) ein \(n\)\sphinxhyphen{}dimensionaler \(\R\)\sphinxhyphen{}Vektorraum mit einer Basis \(B = (b_1,\ldots,b_n)\), die Abbildungen
\(\eta_i:\V\rightarrow\R\) für \(i=1,\ldots,n\),
\begin{equation*}
\begin{split}\eta_i(z) = \eta_i\left(\sum_{i=1}^n \alpha_i b_i\right) := \alpha_i\end{split}
\end{equation*}
\sphinxAtStartPar
bilden einen Basis von \(\V^\ast\), die sogenannte \sphinxstylestrong{duale Basis} zu \(B\).
\end{sphinxadmonition}
\label{vektoranalysis/multilinear:remark-10}
\begin{sphinxadmonition}{note}{Remark 3.4}



\sphinxAtStartPar
Insbesondere zeigt diese Aussage, dass \(\dim(\V) = \dim(\V^\ast)\).
\end{sphinxadmonition}

\begin{sphinxadmonition}{note}
\sphinxAtStartPar
Proof. Wir zeigen zunächst, dass \(\eta_i\in\V^\ast\). Dazu sei \(x,y\in\V\), dann existieren skalare
\(\alpha_i^x,\alpha_i^y\in\R\) für \(i=1,\ldots,n\), s.d.,
\begin{equation*}
\begin{split}x &= \sum_{i=1}^n \alpha_i^x b_i\\
y &= \sum_{i=1}^n \alpha_i^y b_i.\end{split}
\end{equation*}
\sphinxAtStartPar
Somit haben wir
\begin{equation*}
\begin{split}\eta_i(x+y) &= \eta_i\left(\sum_{i=1}^n \alpha_i^x b_i + \alpha_i^y b_i\right) 
\\&=
\eta_i\left(\sum_{i=1}^n (\alpha_i^x + \alpha_i^y) b_i\right) 
\\&=
\alpha_i^x + \alpha_i^y
\\&=
\eta_i\left(\sum_{i=1}^n \alpha_i^x b_i\right)  + \eta_i\left(\sum_{i=1}^n \alpha_i^y b_i\right)
\\&=
\eta_i(x) + \eta_i(y).\end{split}
\end{equation*}
\sphinxAtStartPar
Weiterhin gilt für \(\lambda\in\R\)
\begin{equation*}
\begin{split}\eta_i(\lambda x) &= \eta_i\left(\lambda \sum_{i=1}^n \alpha_i^x b_i\right) 
\\&=
\eta_i\left(\sum_{i=1}^n (\lambda \alpha_i^x) b_i\right) 
\\&=
\lambda \alpha_i^x
\\&=
\lambda \eta_i(x)\end{split}
\end{equation*}
\sphinxAtStartPar
und damit ist \(\eta_i\) linear.
Sei nun \(\phi\in\V^\ast\), dann gilt
\begin{equation*}
\begin{split}\phi(x) = \phi\left(\sum_{i=1}^n \alpha_i^x b_i\right) = \sum_{i=1}^n \alpha_i^x \phi(b_i) = 
\sum_{i=1}^n \eta_i(x) \phi(b_i),\end{split}
\end{equation*}
\sphinxAtStartPar
insbesondere gilt also \(\phi = \sum_{i=1}^n \eta_i \phi(b_i\)).

\sphinxAtStartPar
Somit bilden die Abbildungen \(\eta_i\) ein Erzeugenden System, da jedes \(\phi\) als Linearkombination dargestellt werden kann.
Weiterhin seien \(a_i\in\R\) gegeben, s.d., \(0 = \sum_{i=1}^n a_i \eta_i\), damit folgt für jedes \(j=1,\ldots,n\),
\begin{equation*}
\begin{split}0 = \left(\sum_{i=1}^n a_i \eta_i\right)(b_j) = \sum_{i=1}^n a_i \eta_i(b_j) = a_j\end{split}
\end{equation*}
\sphinxAtStartPar
und damit ist die Aussage bewiesen.
\end{sphinxadmonition}
\label{vektoranalysis/multilinear:remark-11}
\begin{sphinxadmonition}{note}{Remark 3.5}



\sphinxAtStartPar
Die Aussage lässt sich auf den Fall eines unendlich\sphinxhyphen{}dimensionalen Vektorraums übertragen. Hierfür erinnern wir daran, dass für einen Vektorraum \(V\) stets eine Basis \(B^V = \{b_i^v:i\in I\}\subset V\) existiert wobei \(I\) eine (nicht notwendigerweise endliche) Indexmenge ist. Insbesondere bemerken wir, dass wir hier von einer \sphinxstylestrong{Hamelbasis} sprechen, d.h., für jedes Element \(v\in V\) gibt es eindeutig bestimmte Koeffizienten \(\alpha_i, i\in I\) s.d.
\begin{equation*}
\begin{split}v = \sum_{i\in I} \alpha_i b_i.\end{split}
\end{equation*}
\sphinxAtStartPar
Der wichtige Punkt ist aber, dass nur \sphinxstylestrong{endlich viele} \(\alpha_i\) ungleich null sind und die Summation somit keine eigentlich unendliche Reihe beschreibt sondern nur eine endliche Summe. Diese Konzept ist insbesondere verschieden vom Begriff der \sphinxhref{https://de.wikipedia.org/wiki/Schauderbasis}{Schauderbasis}
\end{sphinxadmonition}

\begin{sphinxShadowBox}
\sphinxstylesidebartitle{Georg Hamel}

\sphinxAtStartPar
\sphinxhref{https://de.wikipedia.org/wiki/Georg\_Hamel}{Georg Karl Wilhelm Hamel} (Geboren 12. September 1877 in Düren; Gestorben 4. Oktober 1954 in Landshut) war ein deutscher Mathematiker.
\end{sphinxShadowBox}

\begin{sphinxShadowBox}
\sphinxstylesidebartitle{Juliusz Schauder}

\sphinxAtStartPar
\sphinxhref{https://de.wikipedia.org/wiki/Juliusz\_Schauder}{Juliusz Paweł Schauder} (Geboren 21. September 1899 in Lemberg; Gestorben September 1943) war ein polnischer Mathematiker.
\end{sphinxShadowBox}

\sphinxAtStartPar
Wir halten weiterhin fest, dass sich der doppelt duale Raum im endlich\sphinxhyphen{}dimensionalen Fall leicht charakterisieren lässt.
\label{vektoranalysis/multilinear:lem:doubledual}
\begin{sphinxadmonition}{note}{Lemma 3.4}



\sphinxAtStartPar
Es sei \(\V\) ein \(n\)\sphinxhyphen{}dimensionaler reeller Vektorraum, dann gilt, dass die Abbildung
\begin{equation*}
\begin{split}\Psi:\V\rightarrow \V^{\ast\ast}
\Psi(x) := (\varphi\mapsto \varphi(x))\end{split}
\end{equation*}
\sphinxAtStartPar
ein Isomorphismus ist.
\end{sphinxadmonition}


\subsection{Äußere Formen}
\label{\detokenize{vektoranalysis/multilinear:auszere-formen}}
\sphinxAtStartPar
In {\hyperref[\detokenize{vektoranalysis/multilinear:ex:multi}]{\sphinxcrossref{Example 3.1}}} haben wir für \(k=2\) bereits den Begriff der Antisymmetrie kennengelernt. Dieser Fall lässt sich auf beliebige \(k\in\N\) verallgemeinern, was zur Definition der äußeren Form führt.
\label{vektoranalysis/multilinear:aeussere_Form}
\begin{sphinxadmonition}{note}{Definition 3.4 (Äußere Form)}



\sphinxAtStartPar
Es sei \(\V\) ein \(n\)–dimensionaler \(\R\)\sphinxhyphen{}Vektorraum und \(k\in\N\). Dann heißt \(\varphi\in L^k(E,\R)\)
\sphinxstylestrong{äußere} \(k\)\sphinxstylestrong{\sphinxhyphen{}Form},wenn sie \sphinxstylestrong{antisymmetrisch} ist, d.h., für alle \(1\leq i<l\leq k\) und
\(z\in \V^k\) gilt
\begin{equation*}
\begin{split}\varphi(z_1,\ldots,z_i,\ldots,z_l,\ldots,z_k) =-\varphi(z_1,\ldots,z_l,\ldots,z_i,\ldots,z_k).\end{split}
\end{equation*}
\sphinxAtStartPar
Der Unterraum der äußeren \(k\)\sphinxhyphen{}Formen wird mit \(\Lambda^k(\V)\subset L^k(\V,\R)\) bezeichnet.
\end{sphinxadmonition}
\label{vektoranalysis/multilinear:example-14}
\begin{sphinxadmonition}{note}{Example 3.2}



\sphinxAtStartPar
\(k=1\):

\sphinxAtStartPar
Hier fallen alle bisherigen Definitionen zusammen, d.h.,
\begin{equation*}
\begin{split}\Lambda^1(\V) = L^1(\V,\R)= \V^\ast\end{split}
\end{equation*}
\sphinxAtStartPar
\(k=2\):

\sphinxAtStartPar
Für \(A\in\R^{2,2}\) definiert die Abbildung \((x,y)\mapsto\langle x,A y\rangle\) eine äußere \(2\)\sphinxhyphen{}Form auf \(\R^n\) genau dann, wenn die Matrix \(A\) antisymmetrisch ist. D.h., falls \(A^T=-A\) gilt.

\sphinxAtStartPar
\(k=n\):

\sphinxAtStartPar
Die Determinantenform ist bis auf ihre Vielfachen die einzige äußere \(n\)\sphinxhyphen{}Form auf dem \(\R^n\).
\end{sphinxadmonition}

\sphinxAtStartPar
Wir beweisen zwei kleine Hilfsaussagen zu äußeren Formen
\label{vektoranalysis/multilinear:lemma-15}
\begin{sphinxadmonition}{note}{Lemma 3.5}



\sphinxAtStartPar
Es sein \(\V\) ein reeller Vektorraum und \(\varphi\in\Lambda^k(V)\).
\begin{itemize}
\item {} 
\sphinxAtStartPar
Für jede Permutation \(\pi:\{1,\ldots,k\}\rightarrow\{1,\ldots,k\}\) gilt

\end{itemize}
\begin{equation*}
\begin{split}\varphi(z_{\pi(1)},\ldots,z_{\pi(k)}) = \sign(\pi) \varphi(z_1,\ldots,z_k).\end{split}
\end{equation*}\begin{itemize}
\item {} 
\sphinxAtStartPar
Sind \(z_1,\ldots, z_k\) linear abhängig, so gilt \(\varphi(z_1,\ldots,z_k) = 0\).

\end{itemize}
\end{sphinxadmonition}

\begin{sphinxadmonition}{note}
\sphinxAtStartPar
Proof. Die erste Behauptung folgt direkt aus der Tatsache, dass sich jede Permutation als Verkettung endlich vieler Transpositionen schreiben lässt.
Für die zweite Behauptung sehen wir zunächst, dass
\begin{equation*}
\begin{split}\varphi(z_1,\ldots,x,\ldots, x,\ldots,z_k) = -\varphi(z_1,\ldots,x,\ldots,x,\ldots,z_k)\end{split}
\end{equation*}
\sphinxAtStartPar
und somit \(\varphi(z_1,\ldots,x,\ldots,x,\ldots,z_k)=0\). Sind die \(z_i\) nun linear abhängig, so existieren Skalare \(\alpha_i\), s.d.,
ein \(j\) existiert mit \(\alpha_j\neq 0\) und
\begin{equation*}
\begin{split}\sum_{i=1}^n \alpha_i z_i = 0 \Leftrightarrow z_j = 
\frac{1}{\alpha_j} \sum_{i\neq j} \alpha_i z_i.\end{split}
\end{equation*}
\sphinxAtStartPar
Somit folgt
\begin{equation*}
\begin{split}\varphi(z_1,\ldots,z_j,\ldots,z_k) = 
\sum_{i\neq j} \alpha_i \varphi(z_1,\ldots,z_{j-1},z_i,z_{j+1},\ldots,z_k) = 0.\end{split}
\end{equation*}\end{sphinxadmonition}

\sphinxAtStartPar
Wir werden nun eine Methode kennelernen die es uns erlaubt eine äußere \(k\)\sphinxhyphen{}Form als sogenanntes \sphinxstylestrong{äußeres Produkt} von \(k\) vielen Linearformen zu erhalten.
\label{vektoranalysis/multilinear:definition-16}
\begin{sphinxadmonition}{note}{Definition 3.5 (Äußeres Produkt)}



\sphinxAtStartPar
Für einen Vektorraum \(\V\) ist das \sphinxstylestrong{äußere Produkt} von \(\omega_1,\ldots,\omega_k\in\Lambda^1(\V)\)
durch
\begin{equation*}
\begin{split}\omega_1\wedge\ldots\wedge\omega_k:V^k&\to\R\\
(z_1,\ldots,z_k)&\mapsto 
\det
\begin{pmatrix}
\omega_1(z_1)&\ldots&\omega_k(z_1)\\ 
\vdots&&\vdots\\
\omega_1(z_k)&\ldots&\omega_k(z_k)
\end{pmatrix}\end{split}
\end{equation*}
\sphinxAtStartPar
definiert.
\end{sphinxadmonition}
\label{vektoranalysis/multilinear:lemma-17}
\begin{sphinxadmonition}{note}{Lemma 3.6}



\sphinxAtStartPar
Für einen Vektorraum \(\V\) ist das äußere Produkt von \(\omega_1,\ldots,\omega_k\in\Lambda^1(\V)\) eine \(k\)\sphinxhyphen{}Linearform.
\end{sphinxadmonition}

\begin{sphinxadmonition}{note}
\sphinxAtStartPar
Proof. Siehe Übung.
\end{sphinxadmonition}

\sphinxAtStartPar
Insbesondere gilt damit für die Dualbasis \(\eta_1,\ldots,\eta_n\) von \(\V^*\), dass
\begin{equation*}
\begin{split}\eta_{i_1}\wedge\ldots\wedge\eta_{i_k}\in\Lambda^k(\V)\end{split}
\end{equation*}
\sphinxAtStartPar
für beliebige Indexkombinationen \(i_1,\ldots,i_k \in \{1,\ldots,n \}\). Wegen der Eigenschaften der Determinante gilt
\begin{equation*}
\begin{split}\eta_{i_1}\wedge\ldots\wedge\eta_{i_k} = \sign(\pi)\, \eta_{\pi(i_1)}\wedge\ldots\wedge\eta_{\pi(i_k)} \end{split}
\end{equation*}
\sphinxAtStartPar
wobei \(\pi:\{i_1,\ldots,i_k\}\rightarrow\{i_1,\ldots,i_k\}\) eine Permutation ist, s.d.,
\begin{equation*}
\begin{split}\pi(i_1) <= \ldots <= \pi(i_j) <= \ldots <= \pi(i_k).\end{split}
\end{equation*}
\sphinxAtStartPar
Desweiteren gilt auch
\begin{equation*}
\begin{split}\eta_{i_1}\wedge\ldots\wedge\eta_{i_k} &\neq 0\\
&\Leftrightarrow \\
i_{j}\neq i_l&\text{ für } j\neq l.\end{split}
\end{equation*}
\sphinxAtStartPar
Wir können nun jede \(k\)\sphinxhyphen{}Form \(\omega\in\Lambda^k(E)\) eindeutig als Linearkombination
\begin{equation*}
\begin{split}\omega = \sum_{1\leq i_1<\ldots<i_k\leq n}\omega_{i_1\ldots i_k}
\alpha_{i_1}\wedge\ldots\wedge\alpha_{i_k}\end{split}
\end{equation*}
\sphinxAtStartPar
mit Koeffizienten
\begin{equation*}
\begin{split}\omega_{i_1\ldots i_k} := \omega(e_{i_1},\ldots,e_{i_k})\in\R\end{split}
\end{equation*}
\sphinxAtStartPar
darstellen. Da
die Indexmengen \(\{i_1,\ldots ,i_k\}\) die \(k\)–elementigen Teilmengen von\(\{1,\ldots,n\}\) durchlaufen, gilt
für \(\dim(E)=n\)
\begin{equation*}
\begin{split}\dim\left(\Lambda^k(E)\right) = {n\choose k}.\end{split}
\end{equation*}
\sphinxAtStartPar
Das  \sphinxstyleemphasis{äußere Produkt}
Produkt der \(k\)–Form \(\omega\) mit einer \(l\)–Form
\begin{equation*}
\begin{split}\psi := \sum_{1\leq j_1<\ldots<j_{l}\leq n}\psi_{j_1\ldots j_l}\,
\alpha_{j_1}\wedge\ldots\wedge\alpha_{j_l}\end{split}
\end{equation*}
\sphinxAtStartPar
wird nun distributiv als \(\omega\wedge\psi\in\Lambda^{k+l}(E)\),
\begin{equation*}
\begin{split}\omega\wedge\psi := \sum_{1\leq i_1<\ldots<i_k\leq n} \sum_{1\leq
j_1<\ldots<j_l\leq n} \omega_{i_1\ldots i_k} \psi_{j_1\ldots j_l}
\alpha_{i_1}\wedge\ldots\wedge\alpha_{i_k}\wedge\alpha_{j_1}\wedge
\ldots\wedge\alpha_{j_l}\end{split}
\end{equation*}
\sphinxAtStartPar
definiert. All diejenigen Summanden, bei denen ein Indexpaar \(i_r=j_s\)vorkommt, sind gleich Null, denn \(\alpha_l\wedge\alpha_l = -\alpha_l\wedge\alpha_l=0\).
\begin{itemize}
\item {} 
\sphinxAtStartPar
Das äußere Produkt ist \sphinxstyleemphasis{assoziativ}, d.h. für beliebige äußere Formen auf \(E\) gilt

\end{itemize}
\begin{equation*}
\begin{split}(\omega\wedge\psi)\wedge\rho = \omega\wedge(\psi\wedge\rho).\end{split}
\end{equation*}\begin{itemize}
\item {} 
\sphinxAtStartPar
Weiter gilt für eine \(k\)–Form \(\omega\) und eine \(l\)–Form \(\psi\)

\end{itemize}
\begin{equation*}
\begin{split}\omega\wedge\psi = (-1)^{k\cdot l}\psi\wedge\omega,\end{split}
\end{equation*}
\sphinxAtStartPar
denn wir müssen \(k\!\cdot\! l\)–mal Eins–Formen kommutieren, um von der einen
zur anderen Gestalt zu gelangen.
\label{vektoranalysis/multilinear:symplektische Form auf dem $\R^{2n}$}
\begin{sphinxadmonition}{note}{Example 3.3 (Wikipedia \sphinxstyleemphasis{Symplektische Form})}


\begin{equation*}
\begin{split}\omega := \sum_{i=1}^n\alpha_i\wedge\alpha_{i+n}\in\Lambda^2(\R^{2n}).\end{split}
\end{equation*}
\sphinxAtStartPar
Für \(n=2\) ergibt sich
\begin{equation*}
\begin{split}\omega = \alpha_1\wedge\alpha_3+\alpha_2\wedge\alpha_4,\end{split}
\end{equation*}
\sphinxAtStartPar
also
\begin{equation*}
\begin{split}\omega\wedge\omega &=& (\alpha_1\wedge\alpha_3+\alpha_2\wedge\alpha_4)
\wedge(\alpha_1\wedge\alpha_3+\alpha_2\wedge\alpha_4)\\
&=& \underbrace{\alpha_1\wedge\alpha_3\wedge\alpha_1\wedge\alpha_3}_0 +
\alpha_2\wedge\alpha_4\wedge\alpha_1\wedge\alpha_3\\
&& + \alpha_1\wedge\alpha_3\wedge\alpha_2\wedge\alpha_4 + \underbrace
{\alpha_2\wedge\alpha_4\wedge\alpha_2\wedge\alpha_4}_0\\
&=& (-1)^3\alpha_1\wedge\alpha_2\wedge\alpha_3\wedge\alpha_4 +
(-1)^1\alpha_1\wedge\alpha_2\wedge\alpha_3\wedge\alpha_4\\
&=& -2\alpha_1\wedge\alpha_2\wedge\alpha_3\wedge\alpha_4.\end{split}
\end{equation*}
\sphinxAtStartPar
Die symplektische Form \(\omega\) hat eine Schlüsselrolle in der Klassischen Mechanik. Dort bezeichnet man die Koordinaten \(x_1,\ldots,
x_n\) als Impulskoordinaten, die Koordinaten \(x_{n+1},\ldots, x_{2n}\) als
Ortskoordinaten.
\end{sphinxadmonition}
\label{vektoranalysis/multilinear:example-19}
\begin{sphinxadmonition}{note}{Example 3.4 (Vektoren und äußere Formen)}



\sphinxAtStartPar
Wir ordnen nun Vektoren
\(v = \begin{pmatrix} v_1\\ \vdots\\ v_n \end{pmatrix} =\sum_{k=1}^nv_ke_k\in\R^n\) verschiedene äußere Formen zu.
\begin{itemize}
\item {} 
\sphinxAtStartPar
Das kanonische innere Produkt im \(\R^n\) vermittelt einen Isomorphismus

\end{itemize}
\begin{equation*}
\begin{split}v\mapsto v^*,\ v^*(u) :=\, < v,u > \qquad(u\in\R^n)\end{split}
\end{equation*}
\sphinxAtStartPar
des \(\R^n\) und seines Dualraumes. Die Eins–Form \(v^*\) besitzt dabei die Gestalt
\begin{equation*}
\begin{split}v^* = \sum_{i=1}^nv_i\alpha_i
\in\Lambda^1(\R^n).\end{split}
\end{equation*}\begin{itemize}
\item {} 
\sphinxAtStartPar
\(v\in\R^n\) wird auch eine \((n-1)\)–Form \(\omega_v\in\Lambda^{n-1}(\R^n)\),

\end{itemize}
\begin{equation*}
\begin{split}\omega_v(u_2,\ldots,u_n) := \det(v,u_2,\ldots,u_n) \qquad (u_2,\ldots,u_n\in\R^n)\end{split}
\end{equation*}
\sphinxAtStartPar
zugeordnet. Speziell im \(\R^3\) finden wir die \(2\)–Form
\begin{equation*}
\begin{split}\omega_v = v_1\alpha_2\wedge\alpha_3+v_2\alpha_3\wedge\alpha_1+v_3
\alpha_1\wedge\alpha_2.\end{split}
\end{equation*}\begin{itemize}
\item {} 
\sphinxAtStartPar
Wir betrachten jetzt speziell den (physikalisch wichtigen) \(\R^3\).
Das äußere Produkt zweier solcher \(1\)–Formen ergibtauf dem \(\R^3\) die \(2\)–Form

\end{itemize}
\begin{equation*}
\begin{split}v^*\wedge u^* &=& (v_1\alpha_1+v_2\alpha_2+v_3\alpha_3)
\wedge(u_1\alpha_1+u_2\alpha_2+u_3\alpha_3)\\
&=& (v_1u_2-v_2u_1)\alpha_1\wedge\alpha_2+(v_2u_3-v_3u_2)\alpha_2\wedge
\alpha_3\\
&& + (v_3u_1-v_1u_3)\alpha_3\wedge\alpha_1\\
&=& \omega_{v\times u}.\end{split}
\end{equation*}
\sphinxAtStartPar
Wir haben auf diese Weise das \sphinxhref{https://de.wikipedia.org/wiki/Kreuzprodukt}{\sphinxstyleemphasis{Kreuzprodukt}}
\begin{equation*}
\begin{split}v\times u=\begin{pmatrix} v_2u_3-v_3u_2\\v_3u_1-v_1u_3 \\ v_1u_2-v_2u_1 \end{pmatrix} \in\R^3\end{split}
\end{equation*}
\sphinxAtStartPar
zweier Vektoren \(v,u\in\R^3\) gewonnen.
\end{sphinxadmonition}
\label{vektoranalysis/multilinear:theorem-20}
\begin{sphinxadmonition}{note}{Theorem 3.1}



\sphinxAtStartPar
Die Vektoren \(w_1,\ldots,w_k\in E^*\) sind genau dann linear abhängig, wenn
\begin{equation*}
\begin{split}w_1\wedge\ldots\wedge w_k=0.\end{split}
\end{equation*}\end{sphinxadmonition}

\begin{sphinxadmonition}{note}
\sphinxAtStartPar
Proof. * Wenn sie linear abhängig sind, können wir einen Index \(i\in\{1,\ldots, k\}\) finden, für den \(w_i\) eine Linearkombination \(w_i=\sum_{\stackrel{l=1}{l\neq i}}^k c_l w_l\) ist. Damit gilt aber
\begin{equation*}
\begin{split}w_1\wedge\ldots\wedge w_k = \sum_{\stackrel{l=1}{l\neq i}}^kc_l\, w_1 \wedge \ldots \wedge w_{i-1}\wedge w_l\wedge w_{i+1 \wedge\ldots\wedge w_k = 0,\end{split}
\end{equation*}
\sphinxAtStartPar
denn in jedem Summanden kommt \(w_l\) doppelt vor.
\begin{itemize}
\item {} 
\sphinxAtStartPar
Andernfalls können wir die Vektoren \(w_1,\ldots,w_k\) zu einer Basis

\end{itemize}
\begin{equation*}
\begin{split}w_1,\ldots,w_n \text{ mit } n:=\dim(E^*)\end{split}
\end{equation*}
\sphinxAtStartPar
ergänzen, sodass \(w_1\wedge\ldots\wedge w_n\neq0\) ist.
Dann ist aber auch \(w_1\wedge\ldots\wedge w_k\neq0\).
\end{sphinxadmonition}
\label{vektoranalysis/multilinear:Grassmannalgebra "uber $E$.}
\begin{sphinxadmonition}{note}{Definition 3.6 (Grassmannalgebra)}



\sphinxAtStartPar
Für einen endlich\sphinxhyphen{}dimensionalen \(\R\)\sphinxhyphen{}Vektorraum \(E\) heißt der reelle Vektorraum
\begin{equation*}
\begin{split}\Lambda^*(E) := \bigoplus_{k=0}^{\dim(E)}\Lambda^k(E)\end{split}
\end{equation*}
\sphinxAtStartPar
(mit \(\Lambda^0(E):=\R\)) mit der durch das Dachprodukt
gegebenen Multiplikation die \sphinxstylestrong{äußere} oder
\sphinxhref{https://de.wikipedia.org/wiki/Gra\%C3\%9Fmann-Algebra}{\sphinxstylestrong{Grassmann\sphinxhyphen{}Algebra}}
\end{sphinxadmonition}
\label{vektoranalysis/multilinear:remark-22}
\begin{sphinxadmonition}{note}{Remark 3.6}


\begin{itemize}
\item {} 
\sphinxAtStartPar
\(\dim(\Lambda^*(E)) = 2^{\dim(E)}\), denn \(\sum_{k=0}^n{n\choose k} = 2^n\).

\item {} 
\sphinxAtStartPar
Für beliebige \(k,l\in\N_0\) ist für alle \(\omega\in\Lambda^k(E)\) und
\(\varphi\in\Lambda^l(E)\):\textbackslash{} \(\omega\wedge\varphi\in\Lambda^{k+l}(E)\), aber für
\(m>\dim(E)\) ist \(\dim(\Lambda^m(E))=0\).

\end{itemize}
\end{sphinxadmonition}
\label{vektoranalysis/multilinear:pull-back von $\omega$ mit $f$.}
\begin{sphinxadmonition}{note}{Definition 3.7}



\sphinxAtStartPar
Für eine lineare Abbildung \(f:E\to F\) endlichdimensionaler \(\R\)–Vektorräume und \(\omega\in\Lambda^k(F)\) heißt die durch
\begin{equation*}
\begin{split}f^*(\omega)( v_1,\ldots, v_k) := \omega \big(f( v_1),\ldots,f( v_k)\big)
\qquad (v_1,\ldots,v_k\in E)\end{split}
\end{equation*}
\sphinxAtStartPar
definierte \(k\)–Form \(f^*(\omega)\) die \sphinxstylestrong{Zurückziehung} (engl.
\sphinxstylestrong{pull–back}).
\end{sphinxadmonition}

\sphinxAtStartPar
Es gilt offensichtlich \(f^*(\omega)\in\Lambda^k(E)\), denn \(f^*(\omega)\)
ist \(k\)–linear und antisymmetrisch.
\label{vektoranalysis/multilinear:theorem-24}
\begin{sphinxadmonition}{note}{Theorem 3.2}


\begin{itemize}
\item {} 
\sphinxAtStartPar
Die Abbildung \(f^*:\Lambda^*(F)\to\Lambda^*(E)\) ist linear.

\item {} 
\sphinxAtStartPar
Für \(g\in L(F,G)\) ist \((g\circ f)^*=f^*\circ g^*\).

\item {} 
\sphinxAtStartPar
Für die identische Abbildung \(Id_E:E\to E\) ist \(Id_E^* = Id_{\Lambda^*(E)}\).

\item {} 
\sphinxAtStartPar
Für eine invertierbare Abbildung \(f\in {\rm GL}(E,F)\) ist \((f^*)^{-1}=(f^{-1})^*\).

\item {} 
\sphinxAtStartPar
\(f^*(\alpha\wedge\beta) = f^*(\alpha)\wedge f^*(\beta)\).

\end{itemize}
\end{sphinxadmonition}

\begin{sphinxadmonition}{note}
\sphinxAtStartPar
Proof. Für alle Vektoren \(v_1,\ldots,v_k\in E\) gilt
\begin{itemize}
\item {} 
\sphinxAtStartPar
Mit \(\alpha, \beta\in\Lambda^k(F)\) und \(c_1,c_2\in\R\) ist

\end{itemize}
\begin{equation*}
\begin{split}f^*(c_1\alpha+c_2\beta)(v_1,\ldots,v_k)
 &=& (c_1\alpha+c_2\beta) \big(f(v_1),\ldots,f(v_k)\big)\\
&=& c_1\alpha\big(f(v_1),\ldots,f(v_k)\big) + c_2\beta\big(f(v_1),\ldots,f(v_k)\big)\\
&=& c_1f^*\alpha(v_1,\ldots,v_k)+c_2f^*\beta(v_1,\ldots,v_k).\end{split}
\end{equation*}\begin{itemize}
\item {} 
\sphinxAtStartPar
\((g\circ f)^*\alpha( v_1,\ldots, v_k) = \alpha\big(g\circ f( v_1),\ldots, g\circ f( v_k)\big)= g^*\alpha\big(f( v_1),\ldots,f( v_k)\big)\\=  f^*\circ g^*\alpha( v_1,\ldots, v_k)\)

\item {} 
\sphinxAtStartPar
\(Id_E^*(\alpha)(v_1,\ldots,v_k) = \alpha\big(Id_E(v_1),\ldots,Id_E(v_k)\big)
= \alpha(v_1,\ldots,v_k)\).

\item {} 
\sphinxAtStartPar
Folgt aus 2. und 3.: \((f^{-1})^*f^* = (f\circ f^{-1})^* = Id_F^* =
Id_{\Lambda^*(F)}\).

\item {} 
\sphinxAtStartPar
Hausaufgabe.

\end{itemize}
\end{sphinxadmonition}


\section{Tensoren und Tensorprodukte}
\label{\detokenize{vektoranalysis/tensor:tensoren-und-tensorprodukte}}\label{\detokenize{vektoranalysis/tensor::doc}}
\sphinxAtStartPar
In diesem Kapitel widmen wir uns einem wichtigen aber komplizierten Thema der Vektoranalysis, nämlich Tensoren und Tensorprodukten.
Der Begriff hat sehr viele verschiedene Anschauungsmöglichkeiten (siehe \sphinxhref{https://de.wikipedia.org/wiki/Tensorprodukt}{Wikipedia}) weshalb es nicht leicht ist eine Einführung zu geben die gleichzeitig allgemein, aber auch verständlich ist. Da Tensoren aber eine wichtige Rolle in der Physik spielen werden wir uns hier damit beschäftigen.


\subsection{Motivation}
\label{\detokenize{vektoranalysis/tensor:motivation}}
\sphinxAtStartPar
Wir betrachten zwei Beispiele aus der Physik, welche auf Tensoren zurückgreifen.
\label{vektoranalysis/tensor:remark-0}
\begin{sphinxadmonition}{note}{Remark 3.7}



\sphinxAtStartPar
Der Begriff Tensor wurde von Hamilton in der Mitte des 19. Jahrhunderts eingeführt. Er leitete die Bezeichnung vom latinischen \sphinxstyleemphasis{tendere} (spannen) ab, da die ursprüngliche Anwendung derartiger Objekte in der Elastizitätstheorie Anwendung fand.
\end{sphinxadmonition}


\subsubsection{Der Cauchy Spannungstensor}
\label{\detokenize{vektoranalysis/tensor:der-cauchy-spannungstensor}}
\begin{sphinxShadowBox}
\sphinxstylesidebartitle{Augustin Cauchy}

\sphinxAtStartPar
\sphinxhref{https://de.wikipedia.org/wiki/Augustin-Louis\_Cauchy}{Augustin\sphinxhyphen{}Louis Cauchy} (Geboren 21. August 1789 in Paris; Gestorben 23. Mai 1857 in Sceaux) war ein französischer Mathematiker.
\end{sphinxShadowBox}

\sphinxAtStartPar
Mechanische Spannung beschreibt die innere Beanspruchung und Kräfte in einem Volumen \(V\subset\R^3\) die aufgrund einer äußeren Belastungen auftreten. Die grundlegende Idee ist das \sphinxstylestrong{Euler\sphinxhyphen{}Cauchy Stress Prinzip}, welches beschreibt, dass auf jede Schnittfläche \(A\subset\R^2\) welche ein Volumen in zwei Teile trennt, von diesen zwei Komponenten eine Spannung auf \(A\) ausgewirkt wird, welche durch den \sphinxstylestrong{Spannungsvektor} \(\mathbf{T}^n\) beschrieben wird. Der Spannungsvektor ist hierbei von der Dimension “Kraft pro Fläche”.

\begin{figure}[htbp]
\centering
\capstart

\noindent\sphinxincludegraphics[height=250\sphinxpxdimen]{{stress_vector}.png}
\caption{Visualisierung für Normal\sphinxhyphen{} und Scherspannung an einer Schnittfläche. Quelle: \sphinxhref{https://en.wikipedia.org/wiki/Cauchy\_stress\_tensor}{Wikipedia; Cauchy Stress Tensor}.}\label{\detokenize{vektoranalysis/tensor:fig-stress}}\end{figure}

\sphinxAtStartPar
Wie in \hyperref[\detokenize{vektoranalysis/tensor:fig-stress}]{Fig.\@ \ref{\detokenize{vektoranalysis/tensor:fig-stress}}} visualisiert teilt sich die Spannung in zwei Komponenten auf:

\sphinxAtStartPar
\sphinxstylestrong{Normalspannung:}

\sphinxAtStartPar
Dieser Teil des Spannungsvektor zeigt in Richtung der normalen \(\mathbf{n}\) welche orthogonal auf der Schnittfläche stehen.

\sphinxAtStartPar
\sphinxstylestrong{Scherspannung:}

\sphinxAtStartPar
Dieser Teil des Spannungstensors ist parallel zur Schnittfläche.

\sphinxAtStartPar
Man erkennt nun, dass die Spannung in \(V\) nicht durch einen einzigen Vektor ausgedrückt werden kann. Einerseits hängt sie vom betrachteten Punkt \(P\in V\) ab und zudem von der Orientierung der Schnittfläche. Allerdings hat Cauchy gezeigt, dass ein Tensorfeld \(\mathbf{\sigma}(x)\) existiert, s.d.,
\begin{equation*}
\begin{split}T^{\mathbf{n}}(x) = \mathbf{n}\cdot \mathbf{\sigma}(x),\end{split}
\end{equation*}
\sphinxAtStartPar
d.h. in jedem Punkt \(x\in V\) ist der Stressvektor linear im Normalenvektor \(\mathbf{n}\).

\begin{figure}[htbp]
\centering
\capstart

\noindent\sphinxincludegraphics[height=250\sphinxpxdimen]{{stress_tensor_comp}.png}
\caption{Quelle: \sphinxhref{https://de.wikipedia.org/wiki/Spannungstensor}{Wikipedia; Spannungstensor}.}\label{\detokenize{vektoranalysis/tensor:fig-stress-comp}}\end{figure}

\sphinxAtStartPar
Hierfür betrachtet man einen freigeschnittenen Würfel wie in \hyperref[\detokenize{vektoranalysis/tensor:fig-stress-comp}]{Fig.\@ \ref{\detokenize{vektoranalysis/tensor:fig-stress-comp}}} und definiert für die drei verschiedenen Flächen (orthogonal zu den Einheitsvektoren) den Stresstensor
\begin{equation*}
\begin{split}\mathbf{T}^{e_i}:= \sum_{j=1}^3 \sigma_{ij} e_j.\end{split}
\end{equation*}
\sphinxAtStartPar
So haben wir z.B. für \(\mathbf{T}^{e_1}\) die Normalspannung gegeben durch \(\sigma_{11} e_1\) und die zwei Scherspannungskomponenten \(\sigma_{12} e_2, \sigma_{13} e_3\). Insgesamt erhält man neun Komponenten \(\sigma_{ij}\) welche über die Definition
\begin{equation*}
\begin{split}\mathbf{\sigma} := \sum_{i=1}^3 e_i \otimes \mathbf{T}^{e_i} = \sum_{i=1}^3\sum_{j=1}^3 \sigma_{ij} (e_i\otimes e_j)\end{split}
\end{equation*}
\sphinxAtStartPar
den Cauchy Stresstensor \(\mathbf{\sigma}\) ergebene. Hierbei bezeichnet \(\otimes\) das \sphinxstylestrong{dyadische Produkt} zweier Vektoren. Für \(x\in\R^n,y\in\R^m\) definieren wir
\begin{equation*}
\begin{split}x \otimes y := 
\begin{pmatrix}
x_1y_1 &\ldots &x_1 y_m\\
\vdots &\ddots & \vdots\\
x_n y_1&\ldots& x_n y_m
\end{pmatrix}.\end{split}
\end{equation*}
\sphinxAtStartPar
Wir werden später sehen, dass man die Idee \(\sigma\) über das dyadische Produkt zu definieren abstrahieren kann, was auf den allgemeinen Tensorbegriff führt.


\subsubsection{Quantenverschränkung}
\label{\detokenize{vektoranalysis/tensor:quantenverschrankung}}

\subsection{Das Tensorprodukt}
\label{\detokenize{vektoranalysis/tensor:das-tensorprodukt}}
\sphinxAtStartPar
Wir wollen nun das Tensorprodukt von Vektorräumen abstrakt einführen und es später konkret realisieren.
\label{vektoranalysis/tensor:definition-1}
\begin{sphinxadmonition}{note}{Definition 3.8 (Tensorprodukt)}



\sphinxAtStartPar
Es seien \(V,W\) zwei reelle Vektorräume. Ein reeler Vektorraum \(X\) heißt \sphinxstylestrong{Tensorproduktraum} falls eine bilineare Abbildung \(\otimes:V\times W\rightarrow X\) existiert, s.d., die folgende \sphinxstylestrong{universelle Eigenschaft} gilt:

\sphinxAtStartPar
Für jede Bilinearform \(\phi\in L^2(V\times W, Y)\) in einen beliebigen reellen Vektorraum \(Y\), existiert eine eindeutige lineare Abbildung
\(p \in L^1(X, Y)\), s.d. gilt
\begin{equation*}
\begin{split}\phi(v,w) = p((v\otimes w)) = p(\otimes(v,w))\quad\forall (v,w)\in V\times W.\end{split}
\end{equation*}
\sphinxAtStartPar
In diesem Fall schreibt man auch \(X = V\otimes X\), \(\otimes\) heißt Tensorprodukt und zusätzlich ist die Schreibweise \(\otimes(v,w)=:v\otimes w\) üblich.
\end{sphinxadmonition}

\sphinxAtStartPar
\sphinxstylestrong{Was bedeutet das?}

\sphinxAtStartPar
Diese Definition erscheint auf den ersten Blick abstrakt und unverständlich. Was ist jetzt also ein Tensorprodukt?

\sphinxAtStartPar
\sphinxstylestrong{Das Tensorprodukt ist universell:}

\sphinxAtStartPar
Wir haben benutzten in der Definition oben das kartesische Produkt \(\times\) welches eindeutig definiert ist. Im Gegensatz dazu gibt es nicht \sphinxstyleemphasis{ein} Tensorprodukt \(\otimes\) oder \sphinxstyleemphasis{einen} Tensorproduktraum \(V\otimes W\). Wir haben die Freiheit \(\otimes\) zu wählen und wann immer die universelle Eigenschaft erfüllt ist, heißt dann \(V\otimes W\) Tensorproduktraum. Derartige Konzepte nennt man in der Algebra \sphinxstyleemphasis{universell}.

\sphinxAtStartPar
\sphinxstylestrong{Was bedeutet die universelle Eigenschaft?}

\sphinxAtStartPar
Wie wir weiter unten noch genauer beschreiben werden, stellt die universelle Eigenschaft eine wichtige Beziehung zwischen dem Raum der bilinearen Abbildungen auf \(V\times W\) und dem Dualraum von \(V\otimes W\) her. Sofern wir das Tensorprodukt gegeben haben erhalten wir alle Bilinearformen schon über einfache Linearformen auf \(V\otimes W\).


\subsection{Existenz und Konstruktion}
\label{\detokenize{vektoranalysis/tensor:existenz-und-konstruktion}}
\sphinxAtStartPar
Wir können ein Tensorprodukt konkret konstruieren indem wir uns auf die Basis der Vektorräume \(V\) und \(W\) zurückziehen. Diese Tatsache formulieren wir in der folgenden Aussage.
\label{vektoranalysis/tensor:theorem-2}
\begin{sphinxadmonition}{note}{Theorem 3.3}



\sphinxAtStartPar
Für zwei reelle Vektorräume \(V, W\) existiert stets mindestens ein Tensorprodukt \(\otimes\in L^2(V\times W, V\otimes W)\).
\end{sphinxadmonition}

\begin{sphinxadmonition}{note}
\sphinxAtStartPar
Proof. Der folgende Beweis ist ein sogenannter konstruktiver Beweis, d.h., wir zeigen die Existenz eines Objekts indem wir es explizit angeben. Es gibt auch nicht\sphinxhyphen{}konstruktive Existenzbeweise.

\sphinxAtStartPar
Es sei \(B^V = \{b_i^V: i\in I^V\}\) eine Basis von \(V\) und analog \(B^W = \{b_i^W: i\in I^W\}\)  eine Basis von \(W\) für Indexmengen \(I^V, I^W\). Wir betrachten das kartesische Produkt
\begin{equation*}
\begin{split}J := I^V \times I^W = \{(i,j): i\in I^V, j\in I^W\}.\end{split}
\end{equation*}
\sphinxAtStartPar
Es sei nun \(X\) ein Vektorraum dessen Basis sich durch \(J\) indizieren lässt, d.h., es existiert eine Menge
\begin{equation*}
\begin{split}B^X = \{b_{ij}^X: (i,j)\in J\}\end{split}
\end{equation*}
\sphinxAtStartPar
s.d. \(B^X\) eine Basis von \(X\) ist. Ein solcher Vektorraum existiert, da z.B. das kartesische Produkt \(V\times W\) diese Eigenschaft erfüllt.

\sphinxAtStartPar
Wir definieren nun eine bilineare Abbildung \(\otimes: V\times W\to X\) über
\begin{equation*}
\begin{split}b_i^V \otimes b_j^W := b_{ij}^X\quad\forall (i,j)\in J.\end{split}
\end{equation*}
\sphinxAtStartPar
Beachte, \(\otimes\) ist durch die Definition auf \(J\) eindeutig festgelegt, da für beliebige \((v,w)\in V\times W\) endlich viele Faktoren
\(\alpha_{i_1},\ldots,\alpha_{i_m}\) und \(\beta_{j_1},\ldots, \beta_{j_n}\) existieren s.d.
\begin{equation*}
\begin{split}\otimes(v,w) 
&= 
\otimes\big(\sum_{k=1}^n \alpha_{i_k} b_{i_k}^V, \sum_{l=1}^m \beta_{j_l} b_{j_l}^W\big) 
\\&= 
\sum_{k=1}^n \sum_{l=1}^m \otimes\left(b_{i_k}^V, b_{j_l}^W\right)
\\&=
\sum_{k=1}^n \sum_{l=1}^m b_{i_kj_l}^X.\end{split}
\end{equation*}
\sphinxAtStartPar
Wir müssen nun die universelle Eigenschaft zeigen, sei dazu \(\phi\in L^2(V\times W, Y)\) eine Bilinearform auf einen reellen Vektorraum \(Y\), dann können wir eine Linearform auf \(p:X\to Y\) definieren durch (analog reicht es die Definition auf den Basiselementen anzugeben)
\begin{equation*}
\begin{split}p(b_{ij}^X) := \phi(b_i^V, b_j^W).\end{split}
\end{equation*}
\sphinxAtStartPar
Dann gilt nämlich, unter Ausnutzung der Linearität von \(p\) und obiger Rechnung, dass
\begin{equation*}
\begin{split}p(\otimes(v,w)) 
&=
\sum_{k=1}^n \sum_{l=1}^m p(b_{i_kj_l}^X)
\\&=
\sum_{k=1}^n \sum_{l=1}^m \phi\left(b_{i_k}^V, b_{j_l}^W\right)
\\&
\phi\big(\sum_{k=1}^n  b_{i_k}^V,\sum_{l=1}^m b_{j_l}^W\big)
\\&
\phi(v,w)\end{split}
\end{equation*}
\sphinxAtStartPar
und somit gilt die universelle Eigenschaft. Insbesondere, da \(p\) durch die obige Definition eindeutig festgelegt ist.
\end{sphinxadmonition}

\sphinxAtStartPar
Als Korollar erhalten wir somit, dass eine Basis des Tensorproduktraums durch das kartesische Produkt der ursprünglichen Basen konstruiert werden kann. Hieran sieht man qualitativ den Unterschied zwischen \(V\otimes W\) und \(V\otimes W\).
\label{vektoranalysis/tensor:corollary-3}
\begin{sphinxadmonition}{note}{Corollary 3.1}



\sphinxAtStartPar
Für zwei reelle Vektorräume \(V,W\) mit Basen \(B^V = \{b_i^V: i\in I^V\}, B^W = \{b_i^W: i\in I^W\}\) und ein Tensorprodukt \(\otimes:V\times W\to V\otimes W\) ist
\begin{equation*}
\begin{split}\{b_i^V\otimes b_j^W: i\in I^V, j\in I^W\}\end{split}
\end{equation*}
\sphinxAtStartPar
eine Basis von \(V\otimes W\).
\end{sphinxadmonition}

\sphinxAtStartPar
Wir wissen nun, dass mindestens ein Tensorprodukt existiert, es stellt sich also die Frage inwiefern sich verschiedene derartige Abbildungen auf den gleichen Vektorräumen \(V,W\) unterschieden. Seien dazu \(\otimes_1, \otimes_2\) je zwei Tensorprodukte auf \(V\times W\). Wegen der universellen Eigenschaft gibt es lineare Abbildungen \(p_1: V\otimes_1 W\to W\otimes_2 V\) und \(p_2: V\otimes_2 W\to W\otimes_1 V\), s.d.,
\begin{equation*}
\begin{split}\otimes_2 &= p_1 \circ \otimes_1\\
\otimes_1 &= p_2 \circ \otimes_2.\end{split}
\end{equation*}
\sphinxAtStartPar
und somit
\begin{equation*}
\begin{split}\otimes_2 &= p_1\circ p_2 \circ \otimes_2\\
\otimes_1 &= p_2\circ p_1 \circ \otimes_1.\end{split}
\end{equation*}
\sphinxAtStartPar
Da wir aber die Basis von \(V\otimes_2 W\) über Elemente \(\otimes_2(b_i^V, b_j^W)\) charakterisieren können, und aus der ersten Gleichung folgt, dass
\begin{equation*}
\begin{split}p_1\circ p_2(\otimes(b_i^V,b_j^W)) = \otimes(b_i^V, b_j^W)\end{split}
\end{equation*}
\sphinxAtStartPar
wissen wir dass \(p_1\circ p_2 = \mathrm{Id}\). Das folgt da \(p_1\circ p_2\) als lineare Abbildung schon ganz auf den Basiselementen festgelegt ist.
Analog folgt \(p_2\circ p_1 = \mathrm{Id}\) und somit sind \(p_1, p_2\) isomorph zueinander. D.h. wir haben insgesamt gezeigt, dass verschiedene Tensorprodukte stets isomorph zueinander sind.
\label{vektoranalysis/tensor:lemma-4}
\begin{sphinxadmonition}{note}{Lemma 3.7}



\sphinxAtStartPar
Es seien \(V,W\) zwei reelle Vektorräume und \(\otimes_1,\otimes_2\) zwei Tensorprodukte. Dann existiert genau ein Isomorphismus \(p:V\otimes_1 W\to V\otimes_2 W\), s.d.
\begin{equation*}
\begin{split}\otimes_2 = p\circ \otimes_1.\end{split}
\end{equation*}\end{sphinxadmonition}


\subsection{Tensoren als Linearformen}
\label{\detokenize{vektoranalysis/tensor:tensoren-als-linearformen}}
\sphinxAtStartPar
Als Einleitung in das Thema wollen wir Tensoren zunächst als Linearformen auf \(\V_1\times\ldots\times\V_k\)
betrachten wobei für \(i=1,\ldots,k\) \(\V_i\) reelle endlich dimensionale Vektorräume sind.
Man schreibt in diesem Fall auch
\begin{equation*}
\begin{split}\V_1\otimes\ldots\otimes\V_k = L(\V_1\times\ldots\V_k,\R)\end{split}
\end{equation*}
\sphinxAtStartPar
wobei \(\otimes\) das Tensorprodukt bezeichnet.

\sphinxAtStartPar
Der wichtige Spezialfall ist hier allerdings nun nicht \(\V^k\) sondern ein kartesisches Produkt der Form
\begin{equation*}
\begin{split}(V^\ast)^r\times V^s.\end{split}
\end{equation*}\label{vektoranalysis/tensor:definition-5}
\begin{sphinxadmonition}{note}{Definition 3.9}



\sphinxAtStartPar
Es sei \(\V\) ein reeller endlich\sphinxhyphen{}dimensionaler Vektorraum, dann nennt man
\begin{equation*}
\begin{split}T^r_s(V) := L((V^\ast)^r\times V^s, \R)\end{split}
\end{equation*}
\sphinxAtStartPar
Menge der \(r\)\sphinxhyphen{}fach \sphinxstylestrong{kontravarianten} und \(s\)\sphinxhyphen{}fach \sphinxstylestrong{kovarianten} Tensoren, oder alternativ Tensoren der Stufe \((r,s)\).
\end{sphinxadmonition}

\sphinxAtStartPar
Wir wollen diese abstrakte Definition nun mit einfachen Beispielen veranschaulichen zunächst für \(r+s=1\).
\label{vektoranalysis/tensor:example-6}
\begin{sphinxadmonition}{note}{Example 3.5}



\sphinxAtStartPar
Tensoren der Stufe \((1,0)\) können mit Elementen des Vektorraums selbst identifiziert werden, denn
\begin{equation*}
\begin{split}T^1_0(V) = L((V^\ast), \R) = \V^{\ast\ast}\cong \V\end{split}
\end{equation*}
\sphinxAtStartPar
mit der Identifikation aus {\hyperref[\detokenize{vektoranalysis/multilinear:lem:doubledual}]{\sphinxcrossref{Lemma 3.4}}}. Weiterhin sind Tensoren der Stufe \((0,1)\) Elemente des
Dualraums, also einfach Linearformen auf \(\V\), sogenannte \sphinxstyleemphasis{Kovektoren}.
\end{sphinxadmonition}

\sphinxAtStartPar
Als weiteren Spezialfall erhalten wir Multilinearformen.
\label{vektoranalysis/tensor:example-7}
\begin{sphinxadmonition}{note}{Example 3.6}



\sphinxAtStartPar
Tensoren der Stufe \((0,k)\) sind \(k\)\sphinxhyphen{}Linearformen, da \(T^0_k(V) = L^k(V)\).
\end{sphinxadmonition}
\label{vektoranalysis/tensor:example-8}
\begin{sphinxadmonition}{note}{Example 3.7}



\sphinxAtStartPar
Aus einer linearen Abbildung \(A:\V\to\V\) erhält man direkt einen Tensor der Stufe \((1,1)\) über die Abbildung
\begin{equation*}
\begin{split}\varphi, v \mapsto \varphi(Av).\end{split}
\end{equation*}\end{sphinxadmonition}


\section{Differentialformen}
\label{\detokenize{vektoranalysis/diffformen:differentialformen}}\label{\detokenize{vektoranalysis/diffformen::doc}}
\sphinxAtStartPar
In diesem Kapitel werden wir nun \sphinxhref{https://de.wikipedia.org/wiki/Differentialform}{Differentialformen} einführen. Die entscheidende Neuerung im Vergleich zum vorhergehenden Kapitel, ist
dass wir zusätzlich zur Vektorraumstruktur nun ein Konzept von Räumlichkeit einführen, speziell betrachten wir eine offene Menge \(U\subset\R^n\). Ein weiterer wichtiger Aspekt, ist dass wir im Folgenden mit glatten Funktion arbeiten wollen, d.h., mit dem Raum \(C^\infty(U,\R^n)\).

\sphinxAtStartPar
Eine Differentialform \(\omega\) auf \(U\subseteq\R^n\) ist eine von Ort zu Ort variierende äußere Form, deren Variation wir als glatt voraussetzen.

\sphinxAtStartPar
Wir schreiben eine allgemeine \sphinxstyleemphasis{\(k\)–Form} \(\omega\) in der \sphinxstyleemphasis{Grundform}
\begin{equation*}
\begin{split}\omega = \sum_{1\leq i_1<\ldots<i_k\leq n}\omega_{i_1\ldots i_k}
dx_{i_1}\wedge\ldots\wedge dx_{i_k}\in\Omega^k(U),\end{split}
\end{equation*}
\sphinxAtStartPar
wobei
\begin{itemize}
\item {} 
\sphinxAtStartPar
die \(\omega_{i_1\ldots i_k}\in \Omega^0(U):=C^\infty(U,\R)\), also glatte reelle Funktionen auf \(U\) sind,

\item {} 
\sphinxAtStartPar
und die \(dx_i\) den Koordinatenfunktionen \(x_i:\R^n\to\R\) zugeordnete \(1\)–Differentialformen sind (\(dx_i\in\Omega^1(\R^n)\)).

\item {} 
\sphinxAtStartPar
Den Raum der \(k\)–Differentialformen schreiben wir ab jetzt zur Unterscheidung vom Raum der äußeren \(k\)–Formen mit dem Symbol \(\Omega\) statt \(\Lambda\).

\end{itemize}

\sphinxAtStartPar
Die \(dx_i\) sind durch ihre Wirkung auf ein Vektorfeld \(v:U\to
\R^n\) definiert, und \(dx_i(v)( y) := v_i( y)\).
\(1\)–Differentialformen machen also aus Vektorfeldern Funktionen, und für \(k\) Vektorfelder \(v^{(l)}:U\to\R^n\) ist für das \(\omega\) aus der Grundform
\begin{equation*}
\begin{split}\omega\left(v^{(1)},\ldots,v^{(k)}\right) := \sum_{1\leq i_1<\ldots<i_k\leq n}
\omega_{i_1\ldots i_k}\cdot\det\begin{pmatrix} dx_{i_1}(v^{(1)})&\ldots& dx_{i_k}(v^{(1)})\\
\vdots&&\vdots\\
dx_{i_1}(v^{(k)})&\ldots& dx_{i_k}(v^{(k)}) \end{pmatrix}\end{split}
\end{equation*}
\sphinxAtStartPar
definiert. Das Ergebnis ist also eine reelle Funktion auf \(U\).\textbackslash{}
Die Rechenregeln übertragen sich von den äußeren Formen auf die Differentialformen.

\sphinxAtStartPar
Auf dem \(\R\)–Vektorraum
\begin{equation*}
\begin{split}\Omega^*(U) := \bigoplus_{k=0}^n\Omega^k(U)\end{split}
\end{equation*}
\sphinxAtStartPar
der Differentialformen betrachten wir jetzt
den \sphinxstyleemphasis{Differentialoperator} \(d\), der durch
\begin{itemize}
\item {} 
\sphinxAtStartPar
\(df := \sum_{i=1}^n\frac{\partial f}{\partial x_i}dx_i\) für Funktionen
\(f\in C^\infty(U,\R) = \Omega^0(U)\)

\item {} 
\sphinxAtStartPar
und \(d\omega := \sum_{1\leq i_1<\ldots<i_k\leq n}d\omega_{i_1\ldots i_k}
\wedge dx_{i_1}\wedge\ldots\wedge dx_{i_k}\) für \(k\)–Formen \textbackslash{}linebreak
\(\omega = \sum_{1\leq i_1<\ldots<i_k\leq n}\omega_{i_1\ldots i_k}
dx_1\wedge\ldots\wedge dx_{i_k}\)

\end{itemize}

\sphinxAtStartPar
definiert ist. \(d\) verwandelt eine \(k\)–Form also in eine \((k+1)\)–Form.
\label{vektoranalysis/diffformen:aeussere Ableitung}
\begin{sphinxadmonition}{note}{Definition 3.10}



\sphinxAtStartPar
Die lineare Abbildung \(d:\Omega^*(U)\to\Omega^*(U)\) heißt \sphinxhref{https://de.wikipedia.org/wiki/\%C3\%84u\%C3\%9Fere\_Ableitung}{\sphinxstylestrong{äußere Ableitung}}.
\end{sphinxadmonition}
\label{vektoranalysis/diffformen:ex:10.14}
\begin{sphinxadmonition}{note}{Example 3.8 (Äußere Ableitung)}


\begin{enumerate}
\sphinxsetlistlabels{\arabic}{enumi}{enumii}{}{.}%
\item {} 
\sphinxAtStartPar
Für \(\omega\in\Omega^0(\R^3)\) ist \(d\omega = \frac{\partial\omega}{\partial x_1}dx_1+
\frac{\partial\omega}{\partial x_2}dx_2+\frac{\partial\omega}{\partial x_3}dx_3\).

\item {} 
\sphinxAtStartPar
Für \(\omega = \omega_1dx_1+\omega_2dx_2+\omega_3dx_3\in\Omega^1(\R^3)\) ist

\end{enumerate}
\begin{equation*}
\begin{split}d\omega &=& (d\omega_1)\wedge dx_1+(d\omega_2)\wedge dx_2+(d\omega_3)\wedge
dx_3\\
&=& \left(\frac{\partial\omega_2}{\partial x_1}-\frac{\partial\omega_1}{\partial x_2}\right)
dx_1\wedge dx_2+ \left(\frac{\partial\omega_3}{\partial x_2}-\frac{\partial\omega_2}{\partial x_3}\right)
dx_2\wedge dx_3\\
&& + \left(\frac{\partial\omega_1}{\partial x_3}-\frac{\partial\omega_3}{\partial x_1}\right)
dx_3\wedge dx_1\end{split}
\end{equation*}\begin{enumerate}
\sphinxsetlistlabels{\arabic}{enumi}{enumii}{}{.}%
\item {} 
\sphinxAtStartPar
Für \(\omega = \omega_{12}dx_1\wedge dx_2+\omega_{23}dx_2\wedge dx_3
+\omega_{31}dx_3\wedge dx_1 \in\Omega^2(\R^3)\) ist

\end{enumerate}
\begin{equation*}
\begin{split}d\omega = \left(\frac{\partial\omega_{12}}{\partial x_3} + \frac{\partial\omega_{23}}{\partial x_1}
+ \frac{\partial\omega_{31}}{\partial x_2}\right)dx_1\wedge dx_2\wedge dx_3.\end{split}
\end{equation*}\begin{enumerate}
\sphinxsetlistlabels{\arabic}{enumi}{enumii}{}{.}%
\item {} 
\sphinxAtStartPar
Für \(\omega\in\Omega^3(\R^3)\) ist \(d\omega=0\).

\end{enumerate}
\end{sphinxadmonition}
\label{vektoranalysis/diffformen:Antiderivation}
\begin{sphinxadmonition}{note}{Theorem 3.4}



\sphinxAtStartPar
\(d\) ist eine \sphinxhref{https://de.wikipedia.org/wiki/Derivation\_(Mathematik)\#Antiderivationen}{\sphinxstylestrong{Antiderivation}}, d.h. für \(\alpha\in\Omega^k(U)\) und \(\beta\in\Omega^l(U)\) ist
\begin{equation*}
\begin{split}d(\alpha\wedge\beta) = (d\alpha)\wedge\beta+(-1)^k\alpha\wedge d\beta.\end{split}
\end{equation*}\end{sphinxadmonition}

\begin{sphinxadmonition}{note}
\sphinxAtStartPar
Proof. Wegen der Linearität von \(d\) genügt es, diese Gleichung für Monome
\begin{equation*}
\begin{split}\alpha := f\underbrace{dx_{i_1}\wedge\ldots\wedge dx_{i_k}}_{\tilde
{\alpha}},\ \beta := g\underbrace{dx_{j_1}\wedge\ldots\wedge dx_{j_l}}_
{\tilde{\beta}},\ f,g\in C^\infty(U,\R)\end{split}
\end{equation*}
\sphinxAtStartPar
zu beweisen.
Es gilt
\begin{equation*}
\begin{split}d(\alpha\wedge\beta) &=& d(f\cdot g)\tilde{\alpha}\wedge
\tilde{\beta} = \big((df)g+f(dg)\big)\,\tilde{\alpha}\wedge\tilde{\beta}\\
&=& (df)\tilde{\alpha}\wedge g\tilde{\beta}+ (-1)^kf\tilde{\alpha}\end{split}
\end{equation*}\end{sphinxadmonition}
\label{vektoranalysis/diffformen:thm:dd}
\begin{sphinxadmonition}{note}{Theorem 3.5}



\sphinxAtStartPar
Auf \(\Omega^*(U)\) gilt
\end{sphinxadmonition}

\begin{sphinxadmonition}{note}
\sphinxAtStartPar
Proof. 1. Für \(f\in\Omega^0(U)\) ist
\begin{equation*}
\begin{split}ddf &=& d\left(\sum_{i=1}^n\frac{\partial f}
{\partial x_i}dx_i\right) = \sum_{i=1}^n\sum_{l=1}^n\frac{\partial^2f}{\partial x_l\partial x_i}
dx_l\wedge dx_i\\
& =& \sum_{1\leq r< s\leq n}\left(\frac{\partial^2 f}{\partial x_r
\partial x_s} - \frac{\partial^2f}{\partial x_s\partial x_r}\right)dx_r\wedge dx_s = 0,\end{split}
\end{equation*}
\sphinxAtStartPar
da wir wegen der Glattheit von \(f\) die partiellen Ableitungen vertauschen
können.
\begin{enumerate}
\sphinxsetlistlabels{\arabic}{enumi}{enumii}{}{.}%
\item {} 
\sphinxAtStartPar
Für \(\omega = \sum\omega_{i_1\ldots i_k}dx_{i_1}\wedge\ldots\wedge dx_{i_k}
\in\Omega^k(U)\) ist\textbackslash{}

\end{enumerate}
\begin{equation*}
\begin{split}dd\omega = \sum(\underbrace{dd\omega_{i_1\ldots i_k}}_0)
\wedge dx_{i_1}\wedge\ldots\wedge dx_{i_k} = 0,\end{split}
\end{equation*}
\sphinxAtStartPar
denn gemäß Satz {\hyperref[\detokenize{vektoranalysis/diffformen:Antiderivation}]{\sphinxcrossref{Theorem 3.4}}} wird die äußere Ableitung auf die
1\sphinxhyphen{}Formen \(d\omega_{i_1\ldots i_k}\) und \(dx_{i_l}\) angewandt, und nach Teil 1.
ist das Ergebnis Null.
\end{sphinxadmonition}
\label{vektoranalysis/diffformen:geschlossen:exakt}
\begin{sphinxadmonition}{note}{Definition 3.11}



\sphinxAtStartPar
Eine Differentialform \(\vv\in\Omega^*(U)\) heißt
\begin{itemize}
\item {} 
\sphinxAtStartPar
\sphinxstylestrong{geschlossen}, wenn \(d\vv=0\), *\sphinxstylestrong{exakt}, wenn \(\vv=d\psi\) für ein \(\psi\in\Omega^*(U)\) gilt.

\end{itemize}

\sphinxAtStartPar
Nach Satz {\hyperref[\detokenize{vektoranalysis/diffformen:thm:dd}]{\sphinxcrossref{Theorem 3.5}}} sind exakte Differentialformen geschlossen.\textbackslash{} Für \(k\)–Formen auf konvexen offenen Teimengen \(U\subseteq \R^n\) gilt für \(k\ge 1\)auch die Umkehrung (sog.
\sphinxhref{https://de.wikipedia.org/wiki/Poincar\%c3\%a9-Lemma}{\sphinxstylestrong{Poincaré\sphinxhyphen{}Lemma}} ),  siehe Kapitel \sphinxcode{\sphinxupquote{sect:Poinca}}).
\end{sphinxadmonition}


\chapter{Bibliography}
\label{\detokenize{references:bibliography}}\label{\detokenize{references::doc}}
\sphinxAtStartPar


\begin{sphinxthebibliography}{For17}
\bibitem[AF13]{references:id13}
\sphinxAtStartPar
Ilka Agricola and Thomas Friedrich. \sphinxstyleemphasis{Globale Analysis \sphinxhyphen{} Differentialformen in Analysis, Geometrie und Physik}. Springer\sphinxhyphen{}Verlag, Berlin Heidelberg New York, edition, 2013. ISBN 978\sphinxhyphen{}3\sphinxhyphen{}322\sphinxhyphen{}92903\sphinxhyphen{}7.
\bibitem[For17]{references:id4}
\sphinxAtStartPar
Otto Forster. \sphinxstyleemphasis{Analysis 2}. Springer, 2017.
\bibitem[Kna13]{references:id5}
\sphinxAtStartPar
Peter Knabner. \sphinxstyleemphasis{Skript zur Vorlesung "Gewöhnliche Differentialgleichungen"}. 2013.
\bibitem[Kna17]{references:id8}
\sphinxAtStartPar
Andreas Knauf. \sphinxstyleemphasis{Mathematische Physik: Klassische Mechanik}. Springer Berlin Heidelberg, 2017. \sphinxhref{https://doi.org/10.1007/978-3-662-55776-1}{doi:10.1007/978\sphinxhyphen{}3\sphinxhyphen{}662\sphinxhyphen{}55776\sphinxhyphen{}1}.
\bibitem[Kna20]{references:id7}
\sphinxAtStartPar
Andreas Knauf. \sphinxstyleemphasis{Skript zur Vorlesung "Mathematik für Physikstudierende 3"}. 2020.
\bibitem[Nol11]{references:id9}
\sphinxAtStartPar
Wolfgang Nolting. \sphinxstyleemphasis{Grundkurs Theoretische Physik 2 \sphinxhyphen{} Analytische Mechanik}. Springer Berlin Heidelberg, 2011. \sphinxhref{https://doi.org/10.1007/978-3-642-12950-6}{doi:10.1007/978\sphinxhyphen{}3\sphinxhyphen{}642\sphinxhyphen{}12950\sphinxhyphen{}6}.
\bibitem[SB18]{references:id10}
\sphinxAtStartPar
Herman Schulz\sphinxhyphen{}Baldes. \sphinxstyleemphasis{Skript zur Vorlesung "Mathematik für Physiker 3"}. 2018.
\bibitem[Ten21]{references:id12}
\sphinxAtStartPar
Daniel Tenbrinck. \sphinxstyleemphasis{Skript zur Vorlesung "Mathematik für Data Science 2"}. 2021. URL: \sphinxurl{https://fau-ammn.github.io/MathDataScience2}.
\end{sphinxthebibliography}






\renewcommand{\indexname}{Proof Index}
\begin{sphinxtheindex}
\let\bigletter\sphinxstyleindexlettergroup
\bigletter{Antiderivation}
\item\relax\sphinxstyleindexentry{Antiderivation}\sphinxstyleindexextra{vektoranalysis/diffformen}\sphinxstyleindexpageref{vektoranalysis/diffformen:\detokenize{Antiderivation}}
\indexspace
\bigletter{Grassmannalgebra "uber \$E\$.}
\item\relax\sphinxstyleindexentry{Grassmannalgebra "uber \$E\$.}\sphinxstyleindexextra{vektoranalysis/multilinear}\sphinxstyleindexpageref{vektoranalysis/multilinear:\detokenize{Grassmannalgebra "uber _E_.}}
\indexspace
\bigletter{aeussere Ableitung}
\item\relax\sphinxstyleindexentry{aeussere Ableitung}\sphinxstyleindexextra{vektoranalysis/diffformen}\sphinxstyleindexpageref{vektoranalysis/diffformen:\detokenize{aeussere Ableitung}}
\indexspace
\bigletter{aeussere\_Form}
\item\relax\sphinxstyleindexentry{aeussere\_Form}\sphinxstyleindexextra{vektoranalysis/multilinear}\sphinxstyleindexpageref{vektoranalysis/multilinear:\detokenize{aeussere_Form}}
\indexspace
\bigletter{cor:eindeutigkeit\_linear}
\item\relax\sphinxstyleindexentry{cor:eindeutigkeit\_linear}\sphinxstyleindexextra{ode/repetition}\sphinxstyleindexpageref{ode/repetition:\detokenize{cor:eindeutigkeit_linear}}
\indexspace
\bigletter{corollary\sphinxhyphen{}3}
\item\relax\sphinxstyleindexentry{corollary\sphinxhyphen{}3}\sphinxstyleindexextra{vektoranalysis/tensor}\sphinxstyleindexpageref{vektoranalysis/tensor:\detokenize{corollary-3}}
=======
\renewcommand{\indexname}{Proof Index}
\begin{sphinxtheindex}
\let\bigletter\sphinxstyleindexlettergroup
\bigletter{cor:eindeutigkeitlinear}
\item\relax\sphinxstyleindexentry{cor:eindeutigkeitlinear}\sphinxstyleindexextra{ode/repetition}\sphinxstyleindexpageref{ode/repetition:\detokenize{cor:eindeutigkeitlinear}}
>>>>>>> main
\indexspace
\bigletter{def:DGL}
\item\relax\sphinxstyleindexentry{def:DGL}\sphinxstyleindexextra{ode/repetition}\sphinxstyleindexpageref{ode/repetition:\detokenize{def:DGL}}
\indexspace
\bigletter{def:Fluss}
\item\relax\sphinxstyleindexentry{def:Fluss}\sphinxstyleindexextra{ode/fluesse}\sphinxstyleindexpageref{ode/fluesse:\detokenize{def:Fluss}}
\indexspace
\bigletter{def:LokFluss}
\item\relax\sphinxstyleindexentry{def:LokFluss}\sphinxstyleindexextra{ode/fluesse}\sphinxstyleindexpageref{ode/fluesse:\detokenize{def:LokFluss}}
\indexspace
\bigletter{def:Stabilitaet}
<<<<<<< HEAD
\item\relax\sphinxstyleindexentry{def:Stabilitaet}\sphinxstyleindexextra{ode\_stability/stabilitaetsbegriffe}\sphinxstyleindexpageref{ode_stability/stabilitaetsbegriffe:\detokenize{def:Stabilitaet}}
=======
\item\relax\sphinxstyleindexentry{def:Stabilitaet}\sphinxstyleindexextra{odestability/stabilitaetsbegriffe}\sphinxstyleindexpageref{odestability/stabilitaetsbegriffe:\detokenize{def:Stabilitaet}}
>>>>>>> main
\indexspace
\bigletter{def:anfangswertproblem}
\item\relax\sphinxstyleindexentry{def:anfangswertproblem}\sphinxstyleindexextra{ode/repetition}\sphinxstyleindexpageref{ode/repetition:\detokenize{def:anfangswertproblem}}
\indexspace
\bigletter{def:hamiltonsch}
\item\relax\sphinxstyleindexentry{def:hamiltonsch}\sphinxstyleindexextra{ode/hamilton}\sphinxstyleindexpageref{ode/hamilton:\detokenize{def:hamiltonsch}}
\indexspace
\bigletter{def:linearisierung}
<<<<<<< HEAD
\item\relax\sphinxstyleindexentry{def:linearisierung}\sphinxstyleindexextra{ode\_stability/ruhelagen}\sphinxstyleindexpageref{ode_stability/ruhelagen:\detokenize{def:linearisierung}}
\indexspace
\bigletter{def:multilinear}
\item\relax\sphinxstyleindexentry{def:multilinear}\sphinxstyleindexextra{vektoranalysis/multilinear}\sphinxstyleindexpageref{vektoranalysis/multilinear:\detokenize{def:multilinear}}
\indexspace
\bigletter{definition\sphinxhyphen{}0}
\item\relax\sphinxstyleindexentry{definition\sphinxhyphen{}0}\sphinxstyleindexextra{vektoranalysis/multilinear}\sphinxstyleindexpageref{vektoranalysis/multilinear:\detokenize{definition-0}}
\indexspace
\bigletter{definition\sphinxhyphen{}1}
\item\relax\sphinxstyleindexentry{definition\sphinxhyphen{}1}\sphinxstyleindexextra{vektoranalysis/multilinear}\sphinxstyleindexpageref{vektoranalysis/multilinear:\detokenize{definition-1}}
=======
\item\relax\sphinxstyleindexentry{def:linearisierung}\sphinxstyleindexextra{odestability/ruhelagen}\sphinxstyleindexpageref{odestability/ruhelagen:\detokenize{def:linearisierung}}
>>>>>>> main
\indexspace
\bigletter{definition\sphinxhyphen{}12}
\item\relax\sphinxstyleindexentry{definition\sphinxhyphen{}12}\sphinxstyleindexextra{ode/repetition}\sphinxstyleindexpageref{ode/repetition:\detokenize{definition-12}}
\indexspace
<<<<<<< HEAD
\bigletter{definition\sphinxhyphen{}16}
\item\relax\sphinxstyleindexentry{definition\sphinxhyphen{}16}\sphinxstyleindexextra{vektoranalysis/multilinear}\sphinxstyleindexpageref{vektoranalysis/multilinear:\detokenize{definition-16}}
\indexspace
=======
>>>>>>> main
\bigletter{definition\sphinxhyphen{}2}
\item\relax\sphinxstyleindexentry{definition\sphinxhyphen{}2}\sphinxstyleindexextra{ode/hamilton}\sphinxstyleindexpageref{ode/hamilton:\detokenize{definition-2}}
\indexspace
\bigletter{definition\sphinxhyphen{}3}
\item\relax\sphinxstyleindexentry{definition\sphinxhyphen{}3}\sphinxstyleindexextra{ode/repetition}\sphinxstyleindexpageref{ode/repetition:\detokenize{definition-3}}
\indexspace
\bigletter{definition\sphinxhyphen{}4}
\item\relax\sphinxstyleindexentry{definition\sphinxhyphen{}4}\sphinxstyleindexextra{ode/fluesse}\sphinxstyleindexpageref{ode/fluesse:\detokenize{definition-4}}
\indexspace
<<<<<<< HEAD
\bigletter{definition\sphinxhyphen{}5}
\item\relax\sphinxstyleindexentry{definition\sphinxhyphen{}5}\sphinxstyleindexextra{vektoranalysis/tensor}\sphinxstyleindexpageref{vektoranalysis/tensor:\detokenize{definition-5}}
\indexspace
\bigletter{definition\sphinxhyphen{}8}
\item\relax\sphinxstyleindexentry{definition\sphinxhyphen{}8}\sphinxstyleindexextra{ode/repetition}\sphinxstyleindexpageref{ode/repetition:\detokenize{definition-8}}
\indexspace
\bigletter{ex:10.14}
\item\relax\sphinxstyleindexentry{ex:10.14}\sphinxstyleindexextra{vektoranalysis/diffformen}\sphinxstyleindexpageref{vektoranalysis/diffformen:\detokenize{ex:10.14}}
\indexspace
=======
\bigletter{definition\sphinxhyphen{}8}
\item\relax\sphinxstyleindexentry{definition\sphinxhyphen{}8}\sphinxstyleindexextra{ode/repetition}\sphinxstyleindexpageref{ode/repetition:\detokenize{definition-8}}
\indexspace
>>>>>>> main
\bigletter{ex:bacteria}
\item\relax\sphinxstyleindexentry{ex:bacteria}\sphinxstyleindexextra{ode/dynamicSystems}\sphinxstyleindexpageref{ode/dynamicSystems:\detokenize{ex:bacteria}}
\indexspace
\bigletter{ex:freefall}
\item\relax\sphinxstyleindexentry{ex:freefall}\sphinxstyleindexextra{ode/dynamicSystems}\sphinxstyleindexpageref{ode/dynamicSystems:\detokenize{ex:freefall}}
\indexspace
\bigletter{ex:multi}
\item\relax\sphinxstyleindexentry{ex:multi}\sphinxstyleindexextra{vektoranalysis/multilinear}\sphinxstyleindexpageref{vektoranalysis/multilinear:\detokenize{ex:multi}}
\indexspace
\bigletter{ex:oscillations}
\item\relax\sphinxstyleindexentry{ex:oscillations}\sphinxstyleindexextra{ode/fluesse}\sphinxstyleindexpageref{ode/fluesse:\detokenize{ex:oscillations}}
\indexspace
\bigletter{example\sphinxhyphen{}1}
\item\relax\sphinxstyleindexentry{example\sphinxhyphen{}1}\sphinxstyleindexextra{odestability/stabilitaetsbegriffe}\sphinxstyleindexpageref{odestability/stabilitaetsbegriffe:\detokenize{example-1}}
\indexspace
\bigletter{example\sphinxhyphen{}14}
\item\relax\sphinxstyleindexentry{example\sphinxhyphen{}14}\sphinxstyleindexextra{vektoranalysis/multilinear}\sphinxstyleindexpageref{vektoranalysis/multilinear:\detokenize{example-14}}
\indexspace
\bigletter{example\sphinxhyphen{}19}
\item\relax\sphinxstyleindexentry{example\sphinxhyphen{}19}\sphinxstyleindexextra{vektoranalysis/multilinear}\sphinxstyleindexpageref{vektoranalysis/multilinear:\detokenize{example-19}}
\indexspace
\bigletter{example\sphinxhyphen{}3}
\item\relax\sphinxstyleindexentry{example\sphinxhyphen{}3}\sphinxstyleindexextra{ode/hamilton}\sphinxstyleindexpageref{ode/hamilton:\detokenize{example-3}}
\indexspace
\bigletter{example\sphinxhyphen{}4}
\item\relax\sphinxstyleindexentry{example\sphinxhyphen{}4}\sphinxstyleindexextra{ode/hamilton}\sphinxstyleindexpageref{ode/hamilton:\detokenize{example-4}}
\indexspace
\bigletter{example\sphinxhyphen{}6}
\item\relax\sphinxstyleindexentry{example\sphinxhyphen{}6}\sphinxstyleindexextra{vektoranalysis/tensor}\sphinxstyleindexpageref{vektoranalysis/tensor:\detokenize{example-6}}
\indexspace
\bigletter{example\sphinxhyphen{}7}
\item\relax\sphinxstyleindexentry{example\sphinxhyphen{}7}\sphinxstyleindexextra{vektoranalysis/tensor}\sphinxstyleindexpageref{vektoranalysis/tensor:\detokenize{example-7}}
\indexspace
\bigletter{example\sphinxhyphen{}8}
\item\relax\sphinxstyleindexentry{example\sphinxhyphen{}8}\sphinxstyleindexextra{vektoranalysis/tensor}\sphinxstyleindexpageref{vektoranalysis/tensor:\detokenize{example-8}}
\indexspace
\bigletter{example\sphinxhyphen{}8}
\item\relax\sphinxstyleindexentry{example\sphinxhyphen{}8}\sphinxstyleindexextra{odestability/ruhelagen}\sphinxstyleindexpageref{odestability/ruhelagen:\detokenize{example-8}}
\indexspace
\bigletter{example\sphinxhyphen{}9}
\item\relax\sphinxstyleindexentry{example\sphinxhyphen{}9}\sphinxstyleindexextra{odestability/ruhelagen}\sphinxstyleindexpageref{odestability/ruhelagen:\detokenize{example-9}}
\indexspace
\bigletter{geschlossen:exakt}
\item\relax\sphinxstyleindexentry{geschlossen:exakt}\sphinxstyleindexextra{vektoranalysis/diffformen}\sphinxstyleindexpageref{vektoranalysis/diffformen:\detokenize{geschlossen:exakt}}
\indexspace
\bigletter{lem:doubledual}
\item\relax\sphinxstyleindexentry{lem:doubledual}\sphinxstyleindexextra{vektoranalysis/multilinear}\sphinxstyleindexpageref{vektoranalysis/multilinear:\detokenize{lem:doubledual}}
\indexspace
\bigletter{lem:intexpglgn}
\item\relax\sphinxstyleindexentry{lem:intexpglgn}\sphinxstyleindexextra{odestability/ruhelagen}\sphinxstyleindexpageref{odestability/ruhelagen:\detokenize{lem:intexpglgn}}
\indexspace
\bigletter{lem:mpotew}
\item\relax\sphinxstyleindexentry{lem:mpotew}\sphinxstyleindexextra{ode/repetition}\sphinxstyleindexpageref{ode/repetition:\detokenize{lem:mpotew}}
\indexspace
\bigletter{lemma\sphinxhyphen{}15}
\item\relax\sphinxstyleindexentry{lemma\sphinxhyphen{}15}\sphinxstyleindexextra{vektoranalysis/multilinear}\sphinxstyleindexpageref{vektoranalysis/multilinear:\detokenize{lemma-15}}
\indexspace
\bigletter{lemma\sphinxhyphen{}17}
\item\relax\sphinxstyleindexentry{lemma\sphinxhyphen{}17}\sphinxstyleindexextra{vektoranalysis/multilinear}\sphinxstyleindexpageref{vektoranalysis/multilinear:\detokenize{lemma-17}}
\indexspace
\bigletter{lemma\sphinxhyphen{}3}
\item\relax\sphinxstyleindexentry{lemma\sphinxhyphen{}3}\sphinxstyleindexextra{vektoranalysis/multilinear}\sphinxstyleindexpageref{vektoranalysis/multilinear:\detokenize{lemma-3}}
\indexspace
\bigletter{lemma\sphinxhyphen{}4}
\item\relax\sphinxstyleindexentry{lemma\sphinxhyphen{}4}\sphinxstyleindexextra{vektoranalysis/tensor}\sphinxstyleindexpageref{vektoranalysis/tensor:\detokenize{lemma-4}}
\indexspace
\bigletter{lemma\sphinxhyphen{}8}
\item\relax\sphinxstyleindexentry{lemma\sphinxhyphen{}8}\sphinxstyleindexextra{vektoranalysis/multilinear}\sphinxstyleindexpageref{vektoranalysis/multilinear:\detokenize{lemma-8}}
\indexspace
\bigletter{lemma\sphinxhyphen{}9}
\item\relax\sphinxstyleindexentry{lemma\sphinxhyphen{}9}\sphinxstyleindexextra{vektoranalysis/multilinear}\sphinxstyleindexpageref{vektoranalysis/multilinear:\detokenize{lemma-9}}
\indexspace
\bigletter{lemma:Gronwall}
\item\relax\sphinxstyleindexentry{lemma:Gronwall}\sphinxstyleindexextra{odestability/ruhelagen}\sphinxstyleindexpageref{odestability/ruhelagen:\detokenize{lemma:Gronwall}}
\indexspace
<<<<<<< HEAD
\bigletter{pull\sphinxhyphen{}back von \$\textbackslash{}omega\$ mit \$f\$.}
\item\relax\sphinxstyleindexentry{pull\sphinxhyphen{}back von \$\textbackslash{}omega\$ mit \$f\$.}\sphinxstyleindexextra{vektoranalysis/multilinear}\sphinxstyleindexpageref{vektoranalysis/multilinear:\detokenize{pull-back von __omega_ mit _f_.}}
\indexspace
\bigletter{rem:matrixexponential\_regeln}
\item\relax\sphinxstyleindexentry{rem:matrixexponential\_regeln}\sphinxstyleindexextra{ode/repetition}\sphinxstyleindexpageref{ode/repetition:\detokenize{rem:matrixexponential_regeln}}
=======
\bigletter{rem:matrixexponentialregeln}
\item\relax\sphinxstyleindexentry{rem:matrixexponentialregeln}\sphinxstyleindexextra{ode/repetition}\sphinxstyleindexpageref{ode/repetition:\detokenize{rem:matrixexponentialregeln}}
>>>>>>> main
\indexspace
\bigletter{remark\sphinxhyphen{}0}
\item\relax\sphinxstyleindexentry{remark\sphinxhyphen{}0}\sphinxstyleindexextra{vektoranalysis/tensor}\sphinxstyleindexpageref{vektoranalysis/tensor:\detokenize{remark-0}}
\indexspace
\bigletter{remark\sphinxhyphen{}1}
<<<<<<< HEAD
\item\relax\sphinxstyleindexentry{remark\sphinxhyphen{}1}\sphinxstyleindexextra{ode/repetition}\sphinxstyleindexpageref{ode/repetition:\detokenize{remark-1}}
\indexspace
\bigletter{remark\sphinxhyphen{}10}
\item\relax\sphinxstyleindexentry{remark\sphinxhyphen{}10}\sphinxstyleindexextra{vektoranalysis/multilinear}\sphinxstyleindexpageref{vektoranalysis/multilinear:\detokenize{remark-10}}
\indexspace
\bigletter{remark\sphinxhyphen{}11}
\item\relax\sphinxstyleindexentry{remark\sphinxhyphen{}11}\sphinxstyleindexextra{vektoranalysis/multilinear}\sphinxstyleindexpageref{vektoranalysis/multilinear:\detokenize{remark-11}}
=======
\item\relax\sphinxstyleindexentry{remark\sphinxhyphen{}1}\sphinxstyleindexextra{ode/hamilton}\sphinxstyleindexpageref{ode/hamilton:\detokenize{remark-1}}
>>>>>>> main
\indexspace
\bigletter{remark\sphinxhyphen{}16}
\item\relax\sphinxstyleindexentry{remark\sphinxhyphen{}16}\sphinxstyleindexextra{ode/repetition}\sphinxstyleindexpageref{ode/repetition:\detokenize{remark-16}}
\indexspace
\bigletter{remark\sphinxhyphen{}2}
<<<<<<< HEAD
\item\relax\sphinxstyleindexentry{remark\sphinxhyphen{}2}\sphinxstyleindexextra{vektoranalysis/multilinear}\sphinxstyleindexpageref{vektoranalysis/multilinear:\detokenize{remark-2}}
\indexspace
\bigletter{remark\sphinxhyphen{}22}
\item\relax\sphinxstyleindexentry{remark\sphinxhyphen{}22}\sphinxstyleindexextra{vektoranalysis/multilinear}\sphinxstyleindexpageref{vektoranalysis/multilinear:\detokenize{remark-22}}
\indexspace
\bigletter{remark\sphinxhyphen{}4}
\item\relax\sphinxstyleindexentry{remark\sphinxhyphen{}4}\sphinxstyleindexextra{vektoranalysis/multilinear}\sphinxstyleindexpageref{vektoranalysis/multilinear:\detokenize{remark-4}}
\indexspace
\bigletter{remark\sphinxhyphen{}6}
\item\relax\sphinxstyleindexentry{remark\sphinxhyphen{}6}\sphinxstyleindexextra{vektoranalysis/multilinear}\sphinxstyleindexpageref{vektoranalysis/multilinear:\detokenize{remark-6}}
\indexspace
\bigletter{satz:hamilton\_konstant}
\item\relax\sphinxstyleindexentry{satz:hamilton\_konstant}\sphinxstyleindexextra{ode/hamilton}\sphinxstyleindexpageref{ode/hamilton:\detokenize{satz:hamilton_konstant}}
=======
\item\relax\sphinxstyleindexentry{remark\sphinxhyphen{}2}\sphinxstyleindexextra{odestability/stabilitaetsbegriffe}\sphinxstyleindexpageref{odestability/stabilitaetsbegriffe:\detokenize{remark-2}}
\indexspace
\bigletter{remark\sphinxhyphen{}4}
\item\relax\sphinxstyleindexentry{remark\sphinxhyphen{}4}\sphinxstyleindexextra{odestability/ruhelagen}\sphinxstyleindexpageref{odestability/ruhelagen:\detokenize{remark-4}}
\indexspace
\bigletter{remark\sphinxhyphen{}6}
\item\relax\sphinxstyleindexentry{remark\sphinxhyphen{}6}\sphinxstyleindexextra{odestability/ruhelagen}\sphinxstyleindexpageref{odestability/ruhelagen:\detokenize{remark-6}}
>>>>>>> main
\indexspace
\bigletter{satz:picardlindeloef}
\item\relax\sphinxstyleindexentry{satz:picardlindeloef}\sphinxstyleindexextra{ode/repetition}\sphinxstyleindexpageref{ode/repetition:\detokenize{satz:picardlindeloef}}
\indexspace
\bigletter{thm:hamconst}
\item\relax\sphinxstyleindexentry{thm:hamconst}\sphinxstyleindexextra{ode/hamilton}\sphinxstyleindexpageref{ode/hamilton:\detokenize{thm:hamconst}}
\indexspace
\bigletter{thm:piclindlokal}
\item\relax\sphinxstyleindexentry{thm:piclindlokal}\sphinxstyleindexextra{ode/repetition}\sphinxstyleindexpageref{ode/repetition:\detokenize{thm:piclindlokal}}
\indexspace
<<<<<<< HEAD
\bigletter{symplektische Form auf dem \$\textbackslash{}R\textasciicircum{}\{2n\}\$}
\item\relax\sphinxstyleindexentry{symplektische Form auf dem \$\textbackslash{}R\textasciicircum{}\{2n\}\$}\sphinxstyleindexextra{vektoranalysis/multilinear}\sphinxstyleindexpageref{vektoranalysis/multilinear:\detokenize{symplektische Form auf dem __R__2n__}}
\indexspace
\bigletter{theorem\sphinxhyphen{}2}
\item\relax\sphinxstyleindexentry{theorem\sphinxhyphen{}2}\sphinxstyleindexextra{vektoranalysis/tensor}\sphinxstyleindexpageref{vektoranalysis/tensor:\detokenize{theorem-2}}
\indexspace
\bigletter{theorem\sphinxhyphen{}20}
\item\relax\sphinxstyleindexentry{theorem\sphinxhyphen{}20}\sphinxstyleindexextra{vektoranalysis/multilinear}\sphinxstyleindexpageref{vektoranalysis/multilinear:\detokenize{theorem-20}}
\indexspace
\bigletter{theorem\sphinxhyphen{}24}
\item\relax\sphinxstyleindexentry{theorem\sphinxhyphen{}24}\sphinxstyleindexextra{vektoranalysis/multilinear}\sphinxstyleindexpageref{vektoranalysis/multilinear:\detokenize{theorem-24}}
\indexspace
\bigletter{theorem:stabilitaet\_asymptotisch\_allg}
\item\relax\sphinxstyleindexentry{theorem:stabilitaet\_asymptotisch\_allg}\sphinxstyleindexextra{ode\_stability/ruhelagen}\sphinxstyleindexpageref{ode_stability/ruhelagen:\detokenize{theorem:stabilitaet_asymptotisch_allg}}
\indexspace
\bigletter{theorem:stabilitaet\_lyapunov\_linear}
\item\relax\sphinxstyleindexentry{theorem:stabilitaet\_lyapunov\_linear}\sphinxstyleindexextra{ode\_stability/ruhelagen}\sphinxstyleindexpageref{ode_stability/ruhelagen:\detokenize{theorem:stabilitaet_lyapunov_linear}}
\indexspace
\bigletter{theorem:stabilität\_linear}
\item\relax\sphinxstyleindexentry{theorem:stabilität\_linear}\sphinxstyleindexextra{ode\_stability/ruhelagen}\sphinxstyleindexpageref{ode_stability/ruhelagen:\detokenize{theorem:stabilit_xe4t_linear}}
\indexspace
\bigletter{thm:dd}
\item\relax\sphinxstyleindexentry{thm:dd}\sphinxstyleindexextra{vektoranalysis/diffformen}\sphinxstyleindexpageref{vektoranalysis/diffformen:\detokenize{thm:dd}}
=======
\bigletter{thm:stabasymallg}
\item\relax\sphinxstyleindexentry{thm:stabasymallg}\sphinxstyleindexextra{odestability/ruhelagen}\sphinxstyleindexpageref{odestability/ruhelagen:\detokenize{thm:stabasymallg}}
\indexspace
\bigletter{thm:stablin}
\item\relax\sphinxstyleindexentry{thm:stablin}\sphinxstyleindexextra{odestability/ruhelagen}\sphinxstyleindexpageref{odestability/ruhelagen:\detokenize{thm:stablin}}
\indexspace
\bigletter{thm:stablyaplinear}
\item\relax\sphinxstyleindexentry{thm:stablyaplinear}\sphinxstyleindexextra{odestability/ruhelagen}\sphinxstyleindexpageref{odestability/ruhelagen:\detokenize{thm:stablyaplinear}}
>>>>>>> main
\end{sphinxtheindex}

\renewcommand{\indexname}{Index}
\printindex
\end{document}