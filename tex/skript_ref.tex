%% Generated by Sphinx.
\def\docclass{jupyterBook}
\documentclass[letterpaper,10pt,english]{jupyterBook}
\ifdefined\pdfpxdimen
   \let\pxdimen\pdfpxdimen\else\newdimen\pxdimen
\fi \pxdimen=.75bp\relax
%% turn off hyperref patch of \index as sphinx.xdy xindy module takes care of
%% suitable \hyperpage mark-up, working around hyperref-xindy incompatibility
\PassOptionsToPackage{hyperindex=false}{hyperref}
%% memoir class requires extra handling
\makeatletter\@ifclassloaded{memoir}
{\ifdefined\memhyperindexfalse\memhyperindexfalse\fi}{}\makeatother

\PassOptionsToPackage{warn}{textcomp}

\catcode`^^^^00a0\active\protected\def^^^^00a0{\leavevmode\nobreak\ }
\usepackage{cmap}
\usepackage{fontspec}
\defaultfontfeatures[\rmfamily,\sffamily,\ttfamily]{}
\usepackage{amsmath,amssymb,amstext}
\usepackage{polyglossia}
\setmainlanguage{english}



\setmainfont{FreeSerif}[
  Extension      = .otf,
  UprightFont    = *,
  ItalicFont     = *Italic,
  BoldFont       = *Bold,
  BoldItalicFont = *BoldItalic
]
\setsansfont{FreeSans}[
  Extension      = .otf,
  UprightFont    = *,
  ItalicFont     = *Oblique,
  BoldFont       = *Bold,
  BoldItalicFont = *BoldOblique,
]
\setmonofont{FreeMono}[
  Extension      = .otf,
  UprightFont    = *,
  ItalicFont     = *Oblique,
  BoldFont       = *Bold,
  BoldItalicFont = *BoldOblique,
]


\usepackage[Bjarne]{fncychap}
\usepackage[,numfigreset=1,mathnumfig]{sphinx}

\fvset{fontsize=\small}
\usepackage{geometry}


% Include hyperref last.
\usepackage{hyperref}
% Fix anchor placement for figures with captions.
\usepackage{hypcap}% it must be loaded after hyperref.
% Set up styles of URL: it should be placed after hyperref.
\urlstyle{same}


\usepackage{sphinxmessages}



        % Start of preamble defined in sphinx-jupyterbook-latex %
         \usepackage[Latin,Greek]{ucharclasses}
        \usepackage{unicode-math}
        % fixing title of the toc
        \addto\captionsenglish{\renewcommand{\contentsname}{Contents}}
        \hypersetup{
            pdfencoding=auto,
            psdextra
        }
        % End of preamble defined in sphinx-jupyterbook-latex %
        

\title{Mathematik für Physikstudierende C}
\date{Nov 08, 2021}

\author{J.\@{} Laubmann, T.\@{} Roith, D.\@{} Tenbrinck}



\begin{document}






\label{\detokenize{intro::doc}}


\noindent\includegraphics[width=\textwidth]{../\string_build/html/\string_images/intro\string_1\string_0.png}

\par
Das vorliegende Skript begleitet die \textbf{Vorlesung Mathematik für Physikstudierende C} und ist im Wintersemester 21/22 an der FAU Erlangen Nürnberg entstanden. Es soll den Studierenden zusätzlich zur virtuellen Vorlesung als Nachschlagewerk dienen und ist ausführlicher und genauer gehalten als die Vorlesungsnotizen.

\subsection{Referenz}

\par
Das Skript orientiert sich teilweise an dem Vorlesungsskript “Mathematik für Physikstudierende 3” \cite{Kna20} von Prof.Dr.Andreas Knauf (FAU) aus dem Sommersemester 2020 und den Folien zu “Mathematik für Physiker 3” von Prof.Dr.Hermann Schulz Baldes (FAU) \cite{SB18}. Weiterhin wird in der Vorlesung oft auch auf das Buch “Mathematische Physik: Klassiche Mechanik” \cite{Kna17} von Prof.Knauf verwiesen, was wir Ihnen als zusätzliches Nachschlagewerk empfehlen können.


\chapter{Gewöhnliche Differentialgleichungen für dynamische Systeme}
\label{\detokenize{ode/ode:gewohnliche-differentialgleichungen-fur-dynamische-systeme}}\label{\detokenize{ode/ode::doc}}
\par
In diesem ersten Kapitel der Vorlesung wollen wir weiterführende Konzepte zum Thema gewöhnlicher Differentialgleichungen einführen.
Insbesondere wollen wir uns mit gewöhnlichen Differentialgleichungen für dynamische Systeme beschäftigen.
Hierfür wiederholen wir zunächst die wichtigsten Aussagen und Begriffe, die Sie in Kaptiel 8 \cite{Ten21} kennengelernt haben.
Anschließend definieren wir zwei grundlegende mathematische Werkzeuge um dynamische Systeme zu charakterisieren, nämlich Flüsse und Phasenportraits.
Zum Schluss wollen wir diese zur Untersuchung und Lösung von Hamiltonschen Differentialgleichungen nutzen, welche eine insbesondere in der klassischen Mechanik innerhalb der Physik eine wichtige Rolle spielen.


\section{Einführung in dynamische Systeme}
\label{\detokenize{ode/dynamicSystems:einfuhrung-in-dynamische-systeme}}\label{\detokenize{ode/dynamicSystems::doc}}
\par
Dynamische Systeme spielen eine zentrale Rolle bei der Beschreibung zeitabhängiger Prozesse in vielen verschiedenen Anwendungsgebieten, wie zum Beispiel der Biologie oder der Physik.
Durch diese Art von mathematischen Modellen ist es beispielsweise möglich das Ausschwingen eines Pendels zu beschreiben oder den Bestand zweier unterschiedlicher Populationen über die Zeit in einer Räuber Beute Beziehung zu untersuchen.

\par
Maßgeblich für dynamische Systeme ist die Beobachtung, dass die beschriebenen Prozesse nicht von der Wahl des Anfangszeitpunktes abhängig sind, sondern lediglich von dem gewählten Anfangszustand.
Wir werden diese Eigenschaft später in Sektion \cref{ode/fluesse:s-fluesse}  noch genauer mathematisch charakterisieren.

\par
Je nach Anwendungsgebiet können dynamische Systeme entweder \textbf{diskret} oder \textbf{kontinuierlich} in der Zeitentwicklung sein.
Wir wollen im Folgenden zwei Beispiele zur Illustration des Unterschieds in der Zeitmodellierung diskutieren.


\subsection{Diskrete dynamische Systeme}
\label{\detokenize{ode/dynamicSystems:diskrete-dynamische-systeme}}
\par
Zur Veranschaulichung von diskreten dynamischen System wollen wir uns im Folgenden mit einem Beispiel aus der Biologie beschäftigen.
\label{ode/dynamicSystems:ex:bacteria}
\begin{example}{}{}



\par
In diesem Beispiel wollen wir annehmen, dass wir das \textbf{exponentielle Wachstum} von Bakterien durch Zellteilung als diskretes dynamisches System zu festen, äquidistanten Zeitpunkten \(t_0, t_1, \ldots \in I\) in einem offenen Zeitintervall \(I\subset\R^+_0\) untersuchen wollen.
Wir modellieren die (ungefähre) Anzahl der Bakterien zu einem Zeitpunkt \(t \in I\) als Funktion \(F \colon I \rightarrow \R_0^+\).
Da die Zeitpunkte äquidistant gewählt sind können wir eine einheitliche Wachstumsrate \(\alpha \in \R^+\) mit \(\alpha > 1\) annehmen, so dass für alle \(n \in \N\) gilt:
\begin{align*}
F(t_{n+1}) = \alpha \cdot F(t_n).
\end{align*}
\par
Wir erkennen, dass der Prozess des Bakterienwachstums nicht von der konkreten Wahl des Startzeitpunkts \(t_0 \in I\) abhängt, sondern nur von anfänglichen Anzahl der Bakterien \(F_0 \coloneqq F(t_0)\). \hyperref[\detokenize{ode/dynamicSystems:fig-bacteria}]{Fig.\@ \ref{\detokenize{ode/dynamicSystems:fig-bacteria}}} zeigt, dass eine unterschiedliche Wahl des Anfangszeitpunkt bei gleicher Wahl der Anfangspopulation keinen Effekt auf die zeitliche Dynamik hat.

\par
Dies können wir wie folgt mathematisch verifizieren. Seien \(t_m, t_n \in I\) mit \(n,m \in \N\) zwei unterschiedliche Anfangszeitpunkte für die die gleiche Anfangspopulation \(F_0 \in \N\) von Bakterien angenommen wird, d.h.,
\begin{align*}
F(t_m) = F_0 = F(t_n).
\end{align*}
\par
Betrachten wir nun für die beiden unterschiedlichen Anfangszeitpunkte das Bakterienwachstum nach \(k \in \N\) äquidistanten Zeitschritten, so ergibt sich:
\begin{align*}
F(t_{m+k}) = \alpha \cdot F(t_{m+k-1}) = \ldots = \alpha^k \cdot F(t_{m}) = \alpha^k \cdot F_0 = \alpha^k \cdot F(t_n) = F(t_{n+k}).
\end{align*}
\par
Wir erkennen also, dass unabhängig vom gewählten Anfangszeitpunkt die Bakterienpopulation nach \(k \in \N\) Zeitschritten gleich ist.
\end{example}

\begin{figure}[htbp]
\centering


\noindent\includegraphics[width=\textwidth]{../\string_build/html/\string_images/dynamicSystems\string_3\string_0.png}
\caption{Visualisierung für Beispiel \cref{ode/dynamicSystems:ex:bacteria}  Wir erkennen, dass die Dynamik der Koloniegröße nicht von der Startzeit abhängt, sondern nur vom Anfangswert. Zu beachten gilt, es ist ein diskretes System, die angezeichneten kontinuierlichen Linien dienen lediglich zur Veranschaulichung der Dynamik.}\label{\detokenize{ode/dynamicSystems:fig-bacteria}}\end{figure}

\par
Diskrete dynamische Systeme tauchen auch in anderen spannenden Anwendungen auf, wie beispielsweise in der \href{https://de.wikipedia.org/wiki/Bifurkation\_(Mathematik)\#Bifurkationsdiagramm}{Chaostheorie} und in der \href{https://de.wikipedia.org/wiki/Markow-Kette}{Stochastik}.


\subsection{Kontinuierliche dynamische Systeme}
\label{\detokenize{ode/dynamicSystems:kontinuierliche-dynamische-systeme}}
\par
Im Unterschied zu diskreten dynamischen Systemen wird die Zeit bei kontinuierlichen dynamischen Systemen nicht an abzählbar vielen Punkten modelliert, sondern als Kontinuum.
Im Folgenden beschreiben wir das physikalische Experiment des freien Falls als Spezialfall eines kontinuierlichen dynamischen Systems.
\label{ode/dynamicSystems:ex:freefall}
\begin{example}{}{}



\par
In diesem Beispiel betrachten wir ein physikalisches Modell für den freien Fall eines Steins mit Masse \(m \in \R^+\), den wir in einer Hand halten, bis wir ihn zu einem definierten Anfangszeitpunkt \(t_0 \in I\) mit \(I \subset \R^+_0\) fallen lassen.

\par
Die aktuelle Entfernung des Steins zum Boden zu einem Zeitpunkt \(t \in I\), d.h. seine gegenwärtige Höhe, ist gegeben durch eine monoton fallende Funktion \(F \colon I \rightarrow \R^+_0\).
Unsere Hand befindet sich zum Anfangszeitpunkt \(t_0\) in einer Höhe von \(F_0 > 0\).
Für jeden beliebigen Zeitpunkt \(t > t_0\) lässt sich die aktuelle Höhe des fallenden Steins mit Hilfe des Newtonschen Gravitationsgesetzes wie folgt angeben:
\begin{align*}
F(t) = \max(0, F_0 - \frac{1}{2}gt^2),
\end{align*}
\par
wobei \(g \approx 9,81 \frac{m}{s^2}\) die Erdbeschleunigungskonstante bezeichnet.

\par
Aus \hyperref[\detokenize{ode/dynamicSystems:fig-free-fall}]{Fig.\@ \ref{\detokenize{ode/dynamicSystems:fig-free-fall}}} wird klar, dass auch hier die Dynamik des freien Falls nicht von der Wahl des Anfangszeitpunkts \(t_0 \in I\) abhängt.
Anschaulich gesprochen, würde der Stein genauso fallen, wenn wir ihn noch einige Sekunden länger festhalten würden.
\end{example}

\begin{figure}[htbp]
\centering


\noindent\includegraphics[width=\textwidth]{../\string_build/html/\string_images/dynamicSystems\string_6\string_0.png}
\caption{Visualisierung für Beispiel \cref{ode/dynamicSystems:ex:freefall}  Wir erkennen, dass die Dynamik der Fallhöhe nicht von der Startzeit abhängt, sondern nur von der Starthöhe.}\label{\detokenize{ode/dynamicSystems:fig-free-fall}}\end{figure}

\par
Häufig kommen zur Beschreibung von kontinuierlichen dynamischen Systemen sogenannte \textbf{autonome gewöhnliche Differentialgleichungen} zum Einsatz, wie die in Beispiel \cref{ode/dynamicSystems:ex:freefall} implizit genutzten Bewegungsgleichungen.
Wir werden diese Art von Differentialgleichungen in Kapitel \cref{ode/fluesse:s-fluesse}  mathematisch genauer betrachten.


\section{Wiederholung: Gewöhnliche Differentialgleichungen}
\label{\detokenize{ode/repetition:wiederholung-gewohnliche-differentialgleichungen}}\label{\detokenize{ode/repetition::doc}}
\par
In diesem Abschnitt werden wir kurz die wichtigsten Definitionen und Ergebnisse zu gewöhnlichen Differentialgleichungen aus Kapitel 8 in \cite{Ten21} wiederholen und um neue Begriffe erweitern, mit denen wir die Theorie dynamischer Systeme mathematisch untersuchen können.


\subsection{Gewöhnliche Differentialgleichungen}
\label{\detokenize{ode/repetition:gewohnliche-differentialgleichungen}}
\par
Wir erinnern uns zunächst an die Definition eines gewöhnlichen Differentialgleichungssystems \(m\) ter Ordnung als Grundlage für unsere weiteren Betrachtungen.
\label{ode/repetition:def:DGL}
\begin{definition}{}{}



\par
Seien \(n,m \in \N\).
Wir betrachten im Folgenden eine offene Teilmenge \(U\subset (\R^n)^{m+1}\) und ein offenes Intervall \(I\subset\R\).
Es sei außerdem \(F:I\times U\rightarrow\R^n\) eine stetige Funktion, dann nennen wir
\begin{align}\label{equation:ode/repetition:eq:DGL}
F(x,y(x),y'(x),\ldots,y^{(m)}(x)) = 0
\end{align}
\par
ein \textbf{gewöhnliches Differentialgleichungssystem (DGL)} \(m\) ter Ordnung von \(n\) Gleichungen.
Gilt \(n=1\), das heißt die Funktion \(F\) ist skalarwertig, so sprechen wir von einer \textbf{gewöhnlichen Differentialgleichung}.

\par
Eine Funktion \(\phi\in C^m(I;\R^n)\) heißt \textbf{Lösung des Differentialgleichungssystems}, falls gilt,
\begin{align*}
F(x, \phi(x), \phi'(x), \ldots, \phi^{(m)}(x)) = 0 \quad \forall x\in I.
\end{align*}
\par
Wenn wir die DGL nach der höchsten auftauchenden Ableitung auflösen können, so dass sie die folgende Form hat
\begin{align*}
y^{(m)}(x) = F(x,y(x),y'(x),\ldots,y^{(m-1)}(x)),
\end{align*}
\par
so nennen wir die DGL \textbf{explizit}, ansonsten wird sie \textbf{implizit} genannt.
\end{definition}

\par
Folgende Bemerkung beschreibt eine alternative Notation von gewöhnlichen Differentialgleichungen 1. und 2. Ordnung, die häufig in der Literatur im Kontext dynamischer Systeme auftaucht.
\label{ode/repetition:remark-1}
\begin{emphBox}{}{}{Remark 1.1 (Zeitableitungen bei gewöhnlichen Differentialgleichungen)}



\par
Viele physikalische Phänomene können durch zeitabhängige gewöhnliche Differentialgleichungen 1. und 2. Ordnung beschrieben werden.
In diesen Fällen verwendet man häufig die Variable \(t \in \R^+_0\) als unabhängige Variable anstatt einer Variable \(x \in \R\).
Auch ändert sich häufig die Notation der Zeitableitungen der gesuchten Funktion \(y\), so dass folgende Korrespondenz für die ersten beiden Ableitungen entsteht:
\begin{enumerate}

\item {} 
\par
\(y'(x) \ \ \hat{=} \ \ \dot{y}(t)\),

\item {} 
\par
\(y''(x) \ \ \hat{=} \ \ \ddot{y}(t)\).

\end{enumerate}

\par
Damit lässt sich das gewöhnliche Differentialgleichungssystem aus \cref{ode/repetition:equation-eq-dgl} schreiben als
\begin{align}\label{equation:ode/repetition:eq:DGL_time}
F(z, y(t), \dot{y}(t), \ldots, y{(m)}(t)) = 0 \quad \forall t\in I.
\end{align}\end{emphBox}


\subsection{Autonome Differentialgleichungen}
\label{\detokenize{ode/repetition:autonome-differentialgleichungen}}
\par
Im Fall von dynamischen Systemen erhält der Definitionsbereich der Funktion \(F\) einer gewöhnlichen Differentialgleichung einen besonderen Namen, wie die folgende Bemerkung erklärt.
\label{ode/repetition:remark-2}
\begin{emphBox}{}{}{Remark 1.2 ((Erweiterter) Phasenraum)}



\par
Wird eine gewöhnliche Differentialgleichung als mathematisches Modell für ein kontinuierliches dynamisches System genutzt, so wird die offene Menge \(U\subset (\R^n)^{m+1}\) auch als \textbf{Phasenraum} bezeichnet.
Der Definitionsbereich \(I\times U\) der stetigen Funktion \(F\) wird auch als \textbf{erweiterter Phasenraum} bezeichnet.

\par
Der Phasenraum beschreibt die Menge aller möglichen Zustände des dynamischen Systems.
Jeder Punkt des Phasenraums wird hierbei eindeutig einem Zustand des Systems zugeordnet.

\par
In Kapitel \{ref\}s:fluesse werden wir spezielle Diagramme basierend auf dem Begriff des erweiterten Phasenraum betrachten (auch Phasenportraits genannt), um Lösungen von dynamischen Systemen mathematisch zu charakterisieren.
\end{emphBox}

\par
Im Fall von \textbf{kontinuierlichen dynamischen Systemen} spielt eine Familie von gewöhnlichen Differentialgleichungen eine wichtige Rolle, die wir im Folgenden definieren wollen.
Diese zeichnen sich dadurch aus, dass die Funktion \(F\) in \cref{ode/repetition:equation-eq-dgl-time} nicht explizit von der Zeit abhängt.
\label{ode/repetition:definition-3}
\begin{definition}{}{}



\par
Hängt die Funktion \(F\) in \cref{ode/repetition:def:DGL} nicht explizit von der Zeit ab, d.h., wir haben \(F:U\rightarrow\R^n\) dann heißt die Gleichung
\begin{align}\label{equation:ode/repetition:eq:autonome_DGL}
F(y(x), y'(x), \ldots, y^{(m)}(x)) = 0 \quad \forall t\in I
\end{align}
\par
\textbf{autonome DGL}.
\end{definition}

\par
Im folgenden Beispiel wollen wir unterschiedliche gewöhnliche Differentialgleichungen darauf prüfen, ob sie autonom sind.
\label{ode/repetition:example-4}
\begin{example}{}{}



\par
Wir betrachten drei verschiedene gewöhnliche Differentialgleichungen und untersuchen diese auf ihre Zeitabhängigkeit.
Der Einfachheit halber konzentrieren wir uns hierbei auf gewöhnliche Differentialgleichungen 1. Ordnung.
Sei hierzu  im Folgenden \(I \subset \R\) ein offenes Intervall.

\par
1. Die gewöhnliche Differentialgleichung
\begin{align*}
2y'(x) = y(x)\cdot x \quad \forall x \in I
\end{align*}
\par
ist \textbf{nicht autonom}, da die rechte Seite der Gleichung durch die Funktion
\begin{align*}
F(x,y(x)) = y(x) \cdot x
\end{align*}
\par
beschrieben wird und diese Funktion explizit vom Funktionsargument \(x \in I\) abhängt.



\par
2. Die gewöhnliche Differentialgleichung
\begin{align*}
2t\cdot \dot{y}(t) = y(t)\cdot t \quad \forall t \in I
\end{align*}
\par
ist hingegen \textbf{autonom}, da die Gleichung in folgende explizite Form überführt werden kann
\begin{align*}
\dot{y}(t) = \frac{1}{2} y(t) \quad \forall t \in I
\end{align*}
\par
und somit die rechte Seite der Gleichung durch die Funktion
\begin{align*}
F(t,y(t)) = \frac{1}{2}y(t)
\end{align*}
\par
beschrieben wird, welche nicht explizit vom Funktionsargument \(t \in I\) abhängt.



\par
3. Im Fall der gewöhnlichen Differentialgleichung
\begin{align*}
2y'(x) = y(x)\cdot \sin(g(x)) \quad \forall x \in I
\end{align*}
\par
können wir für beliebige Funktionen \(g \colon I \rightarrow \R\) \textbf{nicht entscheiden}, ob sie autonom ist wenn keine konkrete Form der Funktion \(g\) gegeben ist.
\end{example}


\subsection{Anfangswertprobleme}
\label{\detokenize{ode/repetition:anfangswertprobleme}}
\par
Um gewöhnliche Differentialgleichungen zu lösen, betrachtet man in der Regel sogenannte Anfangswertprobleme.
Hierbei wählt man einen ausgezeichneten Zeitpunkt \(t_0\in I\) aus dem Zeitintervall \(I\), an welchem man die Lösung explizit durch einen Anfangswert \(y_0\in U\) vorgibt.
Dieses Vorgehen wird in der folgenden Definition nochmal kurz wiederholt.
\label{ode/repetition:def:anfangswertproblem}
\begin{definition}{}{}



\par
Sei ein gewöhnliches Differentialgleichungssystem 1. Ordnung wie in \cref{ode/repetition:def:DGL} gegeben, wobei \(I \times U \subset \R_0^+ \times \R^n\) den erweiterten Phasenraum des Systems bezeichnet.
Sei außerdem \(t_0 \in I\) ein Anfangszeitpunkt und \(y_0 \in U\) der zugehörige Anfangszustand.

\par
Dann nennen wir das Gleichungssystem
\begin{align}\label{equation:ode/repetition:eq:AWP}
\dot{y}(t) &= F(t, y(t))\quad\forall t\in I, \\
y(t_0) &= y_0
\end{align}
\par
\textbf{Anfangswertproblem} des gewöhnlichen Differentialgleichungssystems.
Sofern nicht explizit angegeben werden wir im Folgenden annehmen, dass ohne Beschränkung der Allgemeinheit \(t_0=0\) gilt.
\end{definition}

\par
Die explizite Wahl des Anfangszeitpunkts und  zustands erlaubt es erst eine gewöhnliche Differentialgleichung eindeutig zu lösen.
Ohne diese zusätzlichen Informationen könnte man lediglich Funktionenscharen als Lösungsmenge angeben.
Dies wird durch das folgende Beispiel nochmal dargestellt.
\label{ode/repetition:example-6}
\begin{example}{}{}



\par
Wir betrachten eine sehr einfache gewöhnliche Differentialgleichung erster Ordnung, die sich explizit in folgender Form schreiben lässt:
\begin{align*}
y'(x) = y(x) \quad \forall x \in \R.
\end{align*}
\par
Man sieht leicht ein, dass Lösungen dieser Differentialgleichung Funktionen \(y \colon \R \rightarrow \R\) von der Form
\begin{align*}
y(x) = c\cdot e^x
\end{align*}
\par
für eine beliebige Konstante \(c \in \R\) sein müssen.
Um diese Funktionenschar weiter einzuschränken und eine eindeutige Lösung zu erhalten, müssen wir noch Anfangswertbedindungen hinzunehmen.
Hierzu reicht es eine ausgewiesene Stelle \(x_0 \in \R\) und einen Funktionswert \(y_0 = y(x_0)\) festzulegen.

\par
Wählen wir beispielsweise \(x_0 = 0\) und \(y_0 = y(0) = 2\), so erhalten wir als eindeutige Lösung der gewöhnlichen Differentialgleichung die Funktion
\begin{align*}
y(x) = 2\cdot e^x.
\end{align*}
\par
Wir sehen also, dass durch das Festlegen eines Anfangswert die unbekannte Konstante \(c \in \R\) als \(c=2\) eindeutig bestimmt wurde.
\end{example}


\subsection{Existenz und Eindeutigkeit einer Lösung}
\label{\detokenize{ode/repetition:existenz-und-eindeutigkeit-einer-losung}}
\par
Nicht jede gewöhnliche Differentialgleichung ist im Allgemeinen lösbar oder besitzt eindeutige Lösungen, wie das folgende Beispiel belegt.
\label{ode/repetition:example-7}
\begin{example}{}{}



\par
Wir wollen im folgenden zwei Beispiele von autonomen, gewöhnlichen Differentialgleichungen erster Ordnung diskutieren, für die entweder die Existenz oder die Eindeutigkeit von Lösungen nicht gegeben ist.

\par
1. Die gewöhnliche Differentialgleichung
\begin{align*}
e^{y'(x)} \equiv 0 \quad \forall x \in \R
\end{align*}
\par
besitzt keine Lösung, da die Exponentialfunktion strikt positiv ist und es somit keine Funktion \(y \colon \R \rightarrow \R\) gibt, so dass die obige Gleichung erfüllt werden kann.

\par
2. Die gewöhnliche Differentialgleichung
\begin{align*}
y'(x)(1-y'(x)) \equiv 0 \quad \forall x \in \R
\end{align*}
\par
besitzt auf Grund ihrer Symmetrieeigenschaften zwei unterschiedliche Funktionenscharen als Lösung, nämlich
\begin{align*}
y_1(x) = c \quad \text{ und } \quad y_2(x) = x + c \quad \forall x \in \R,
\end{align*}
\par
wobei \(c \in \R\) eine beliebige Konstante darstellt.
\end{example}

\par
Die wichtigste Eigenschaft für die Existenz und Eindeutigkeit von Lösungen gewöhnlicher Differentialgleichungen ist die \textbf{(lokale) Lipschitzstetigkeit} der rechten Seite \(F \colon I \times U \rightarrow U\).
Diese wollen wir der Vollständigkeit halber im Folgenden definieren.
\label{ode/repetition:definition-8}
\begin{definition}{}{}



\par
Sei \(F \colon G \to \R^n\) eine Funktion mit dem erweiterten Phasenraum \(G \, \coloneqq \, I \times U \subset \R\times\R^n\).
Man sagt, dass \(F\) in \(G\) einer \textbf{globalen Lipschitz Bedingung} genügt (bezüglich der Variablen \(y \in U\)) mit der Lipschitz Konstanten \(L\geq0\), wenn gilt
\begin{align*}
\Vert F(t,y) - F(t,\widetilde{y}) \Vert \leq L \Vert y-\widetilde{y}\Vert\quad\text{ für alle }(t,y), (t,\widetilde{y})\in G\,.
\end{align*}
\par
Man sagt, \(F\) genüge in \(G\) einer \textbf{lokalen Lipschitz Bedingung}, falls jeder Punkt \((a,b)\in G\) im erweiterten Phasenraum eine Umgebung \(V\) besitzt, sodass \(F\) in \(G\cap V\) einer Lipschitzbedingung mit einer gewissen (von \(V\) abhängigen) Konstanten \(L\in\R_0^+\) genügt.
\end{definition}

\par
Für die \textbf{(lokale) Existenz von Lösungen} haben wir in Kapitel 8.4 \cite{Ten21} den Satz von Picard Lindelöf formuliert, den wir im Folgenden wiederholen werden.
\label{ode/repetition:satz:picardlindeloef_lokal}
\begin{theorem}{}{}



\par
Sei \(F\colon G\to\R^n\) eine stetige Funktion mit erweitertem Phasenraum \(G \coloneqq I \times U \subset \R\times\R^n\), die lokal Lipschitz stetig auf \(G\) bezüglich der \(y\) Variablen ist.
Dann existiert zu jedem Anfangswert \((t_0,y_0) \in G\) ein \(\varepsilon>0\), sowie genau eine Lösung
\begin{align*}
\phi \colon \left[t_0-\varepsilon, t_0+\varepsilon\right] \to \R^n
\end{align*}
\par
der gewöhnlichen Differentialgleichung
\begin{align*}
\dot{y}(t) \ = \ F(t,y(t))
\end{align*}
\par
unter der Anfangsbedingung \(\phi(t_0)=y_0\).
\end{theorem}

\begin{proof}
 Siehe Kapitel 12, Satz 4 Kapitel 8.4 \cite{For17}
\end{proof}

\par
Bisher haben wir nur die Existenz und Eindeutigkeit von Lösungen gewöhnlicher Differentialgleichungen in lokalen Intervallen betrachtet.
Unter den strengeren Voraussetzungen einer rechten Seite \(F\) der gewöhnlichen Differentialgleichung, die einer globalen Lipschitzbedingung genügt, lässt sich jedoch eine \textbf{globale Existenzaussage} formulieren, die besonders für konkrete Anwendungen sehr praktisch ist.
\label{ode/repetition:satz:picardlindeloef}
\begin{theorem}{}{}



\par
Sei \(F\colon G\to\R^n\) eine stetige Funktion mit erweitertem Phasenraum \(G \, \coloneqq \, I \times U \subset \R\times\R^n\), die eine globale Lipschitzbedingung auf \(G\) bezüglich der \(y\) Variablen erfüllt.
Dann existiert zu jedem Anfangswert \((t_0,y_0) \in G\) eine globale Lösung
\begin{align*}
\phi \colon I \to \R^n
\end{align*}
\par
der gewöhnlichen Differentialgleichung
\begin{align*}
\dot{y}(t) \ = \ F(t,y(t))
\end{align*}
\par
unter der Anfangsbedingung \(\phi(t_0)=y_0\).
Es existieren außerdem keine weiteren (lokalen) Lösungen.
\end{theorem}

\begin{proof}
 Siehe Kapitel 2.3 \cite{Kna13}
\end{proof}
\label{ode/repetition:cor:eindeutigkeit_linear}
\begin{emphBox}{}{}{Corollary 1.1}



\par
Das Anfangswertproblem jedes \textbf{linearen} gewöhnlichen Differentialgleichungssystems 1. Ordnung hat eine eindeutige globale Lösung.
\end{emphBox}

\begin{proof}
 Siehe Theorem 2.25, Kapitel 2.3 \cite{Kna13}
\end{proof}


\subsection{Lösungen von linearen Differentialgleichungssystemen}
\label{\detokenize{ode/repetition:losungen-von-linearen-differentialgleichungssystemen}}\label{\detokenize{ode/repetition:s-lineare-dglsysteme}}
\par
Analog zu Kapitel 8 in \cite{Ten21} wollen wir uns mit Lösungen für \textbf{homogene lineare Differentialgleichungen} beschäftigen, jedoch dieses Mal nicht im skalaren Fall \(n=1\), sondern für ein Anfangswertproblem von der Form
\begin{align}\label{equation:ode/repetition:eq:lin_hom_dglsystem}
\dot{x}(t) &= A x(t), \quad \forall t \in I \subset \R^+_0, \\
x(t_0) &= x_0 \in U \subset \R^n.
\end{align}
\par
Wir bemerken hierbei, dass im Gegensatz zum skalaren Fall hier die Koeffizientenmatrix \(A \in \C^{n\times n}\) nicht von der Zeit abhängt, wir also ein autonomes Differentialgleichungssystem betrachten.

\par
Bevor wir Lösungen von \cref{ode/repetition:equation-eq-lin-hom-dglsystem} angeben, wollen wir ein hilfreiches Funktionalkalkül einführen, dass die Notation im Fall von Differentialgleichungssystemen erleichtert.
\label{ode/repetition:definition-12}
\begin{definition}{}{}



\par
Sei \(n \in \N\) und \(A \in \C^{n \times n}\) eine beliebige quadratische Matrix.
Das \textbf{Matrixexponential} \(e^A\) von \(A\), ist diejenige \(n\times n\) Matrix, welche durch die folgende Potenzreihe definiert ist:
\begin{align*}
e^A \equiv \exp(A) \ \coloneqq \ \sum_{k=0}^\infty \frac{A^k}{k!} = I_n + A + \frac{A^2}{2} + \frac{A^3}{6} + \ldots.
\end{align*}
\par
Analog zur gewöhnlichen Exponentialfunktion konvergiert die Reihe für alle \(A \in \C^{n \times n}\), woraus die Wohldefiniertheit der Definition folgt.
Für den Spezialfall \(n=1\) entspricht das Matrixexponential der gewöhnlichen Exponentialfunktion.
\end{definition}
\label{ode/repetition:rem:matrixexponential_regeln}
\begin{emphBox}{}{}{Remark 1.3 (Rechenregeln für das Matrixexponential)}



\par
Für das Matrixexponential gelten die gleichen Rechenregeln wie für die gewöhnliche Exponentialfunktion, wie zum Beispiel:
\begin{itemize}
\item {} 
\par
\(e^{tA}e^{sA} = e^{(t+s)A}, \quad\) für \(s,t \in \R\)

\item {} 
\par
\(\frac{d}{dt} e^{tA} = Ae^{tA}, \quad\) für \(t \in \R\)

\item {} 
\par
\( e^{D} = \operatorname{diag}(e^{a_1}, \ldots, e^{a_n})\) ist Diagonalmatrix für eine Diagonalmatrix \(D = \operatorname{diag}(a_1, \ldots, a_n)\).

\end{itemize}
\end{emphBox}

\par
Folgendes Lemma stellt einen interessanten Zusammenhang des Matrixexponentials zur Spektraltheorie her.
\label{ode/repetition:lem:mpotew}
\begin{lemma}{}{}



\par
Sei \(A \in \C^{n\times n}\) eine beliebige quadratische Matrix und sei \(\lambda \in \C\) ein Eigenwert von \(A\) zum
Eigenvektor \(v \in \C^n\).
Dann ist der Vektor \(v\) auch Eigenvektor des Matrixexponentials \(e^A\) zum zugehörigen Eigenwert \(e^\lambda\).
\end{lemma}

\begin{proof}
 In der Hausaufgabe zu zeigen.
\end{proof}

\par
Mit Hilfe des Matrixexponentials lässt sich die Lösung des homogenen linearen Differentialgleichungssystems \cref{ode/repetition:equation-eq-lin-hom-dglsystem} kompakt angeben, wie uns folgendes Lemma zeigt.
\label{ode/repetition:lemma-15}
\begin{lemma}{}{}



\par
Sei \(n\in \N\), \(I \subset \R^+_0\) und \(A \in \C^{n\times n}\) eine beliebige quadratische Matrix.
Das Anfangswertproblem \cref{ode/repetition:equation-eq-lin-hom-dglsystem} hat die eindeutige Lösung
\begin{align*}
x(t) = e^{A(t-t_0)}x_0, \quad \forall t \in I.
\end{align*}\end{lemma}

\begin{proof}
 Wir zeigen zunächst, dass die Lösung \(x(t)\) die Anfangswertbedingung erfüllt:
\begin{align*}
x(t_0) = e^{A(t_0-t_0)}x_0 = e^0x_0 = I_n x_0.
\end{align*}
\par
Um zu zeigen, dass \(x(t)\) das lineare homogene Differentialgleichungssystem \cref{ode/repetition:equation-eq-lin-hom-dglsystem} löst, berechnen wir die entsprechende Zeitableitung als
\begin{align*}
\dot{x}(t) = \frac{d}{dt}(e^{A(t-t_0)}x_0) = A \cdot e^{A(t-t_0)}x_0 = A x(t), \quad \forall t \in I.
\end{align*}
\par
Vergleichen wir die linke und rechte Seite dieser Gleichung so erkennen wir, dass \(x(t)\) in der Tat eine Lösung des Differentialgleichungssystems ist.

\par
Nach \cref{ode/repetition:cor:eindeutigkeit_linear} ist die Lösung eindeutig, da es sich um ein lineares Differentialgleichungssystem 1. Ordnung handelt.
\end{proof}

\par
Im Allgemeinen kann man bei linearen Differentialgleichungssystemen nicht davon ausgehen, dass diese in der einfachsten Form wie in \cref{ode/repetition:equation-eq-lin-hom-dglsystem} vorliegen.
Außerdem ist die konkrete Berechnung des Matrixexponentials zur Bestimmung einer Lösungsfunktion \(x(t)\) in der Regel ungeeignet.
Hierzu wollen wir die abschließende Bemerkung machen.
\label{ode/repetition:remark-16}
\begin{emphBox}{}{}{Remark 1.4}



\par
1. Zur Berechnung einer konkreten Lösung \(x(t)\) des linearen homogenen Differentialgleichungssystems \cref{ode/repetition:equation-eq-lin-hom-dglsystem} bietet es sich an, die \textbf{Jordansche Normalform} \(J = SAS^{-1}\) von \(A\) aus Kapitel 2.7 in \cite{Ten21} auszunutzen, da für diese das Matrixexponential wie folgt berechnet werden kann:
\begin{align*}
e^{tA} =  \sum_{k=0}^\infty \frac{(t A)^k}{k!} = \sum_{k=0}^\infty \frac{(tS^{-1}JS)^k}{k!} = S^{-1} \sum_{k=0}^\infty \frac{(tJ)^k}{k!} S = S^{-1} e^{tJ}S = S^{-1} e^{t(D+N)}S = S^{-1} e^{tD} e^{tN} S
\end{align*}
\par
für eine Transformationsmatrix \(S \in \C^{n \times n}\), eine Diagonalmatrix \(D \in \C^{n \times n}\) mit den Eigenwerten von \(A\) und einer nilpotenten Matrix \(N \in \C^{n \times n}\), für die die Reihendarstellung des zugehörigen Matrixexponentials nach endlich vielen Summanden (entsprechend dem Nilpotenzindex von \(N\)) abbricht.

\par
2. Ist das vorliegende lineare Differentialgleichungssystem \textbf{inhomogen}, das heißt für eine stetige Störfunktion \(b \colon I \rightarrow \R^n\) von der Form
\begin{align}\label{equation:ode/repetition:eq:lin_inhom_dglsystem}
\dot{x}(t) &= A x(t) + b(t), \quad \forall t \in I \subset \R^+_0, \\
x(t_0) &= x_0 \in U \subset \R^n,
\end{align}
\par
so lässt sich über die Variation der Konstanten aus Kapitel 8.2 in \cite{Ten21} eine eindeutige Lösung des Anfangswertproblems \cref{ode/repetition:equation-eq-lin-inhom-dglsystem} angeben als
\begin{align*}
x(t) = e^{tA}x_0 + \int_0^t e^{(t-s)A}b(s) \, \mathrm{d}s.
\end{align*}
\par
3. Im Falle eines homogenen, linearen Differentialgleichungssystems, das \textbf{nicht autonom} ist, das heißt die Koeffizientenmatrix \(A = A(t)\) ist zeitabhängig, können wir nicht mehr die Spektraltheorie zur konkreten Berechnung von Lösungen nutzen.
Formal lassen sich dennoch Lösungen als sogenanntes \textbf{zeitgeordnetes Produkt} angeben, was jedoch den Rahmen dieser Vorlesung sprengen würde.
\end{emphBox}


\section{Phasenflüsse und Phasenportraits}
\label{\detokenize{ode/fluesse:phasenflusse-und-phasenportraits}}\label{\detokenize{ode/fluesse:s-fluesse}}\label{\detokenize{ode/fluesse::doc}}
\par
In diesem Abschnitt führen wir die grundlegende mathematischen Konzepte zur Analyse von kontinuierlichen dynamischen Systemen ein. Insbesondere diskutieren wir Flüsse als Lösungen von autonomen gewöhnlichen Differentialgleichungen und definieren sogenannte Phasenportraits, die es uns erlauben dynamische Systeme geometrisch zu interpretieren.


\subsection{Phasenflüsse}
\label{\detokenize{ode/fluesse:phasenflusse}}
\par
Wir beginnen zunächst damit eine Klasse von Funktionen einzuführen, welche die Beschreibung zeitabhängiger Systeme vereinfacht.
Die folgende Definition ist zunächst sehr allgemein für beliebige dynamische Systreme gehalten und wird später im Kontext von konkreten Anwendungsbeispielen spezieller diskutiert.
\label{ode/fluesse:def:Fluss}
\begin{definition}{}{}



\par
Sei \(U \subset \R^n\) eine offene Teilmenge und \(I=\R^+_0\), dann heißt eine Abbildung \(\Phi:I\times U\rightarrow U\) \textbf{(Phasen )Fluss}, falls gilt,
\begin{enumerate}

\item {} 
\par
\(\Phi(0, x) = x\) für alle \(x\in U\),

\item {} 
\par
\(\Phi(t, \Phi(s,x)) = \Phi(s + t, \Phi(0, x)) = \Phi(s + t, x)\) für alle \(x\in U\) und alle \(s,t\in I\).

\end{enumerate}

\par
Das Tripel \((I, U, \Phi)\) heißt \textbf{dynamisches System}.

\par
Zur Vereinfachung der Notation schreibt man häufig auch das erste Argument des Flusses als Index wie folgt
\begin{align*}
\Phi_t(x) \coloneqq \Phi(t, x).
\end{align*}\end{definition}

\par
Für die Analyse von dynamischen Systemen beschreibt der Fluss die Bewegung im Phasenraum in Abhängigkeit zur Zeit.
Im Folgenden wollen wir speziell die \textbf{Lösungen einer autonomen DGL}
\begin{align*}
\dot{x} = F(x).
\end{align*}
\par
für \(F\in C^1(U;\R^n)\) als Fluss interpretieren.
Hierbei soll das zweite Argument des Flusses jeweils den Anfangswert \(x_0\in U\) angeben und \(\Phi(x_0) = \Phi(\cdot, x_0)\) dann eine Lösung der DGL sein, d.h.,
\begin{align*}
\frac{\d}{\d t} \Phi(x_0) = F(\Phi(x_0))
\end{align*}
\par
So werden durch den Phasenfluss die Lösungen des dynamischen Systems in Abhängigkeit vom Anfangszustand angegeben.
Im folgenden Beispiel betrachten wir den \textbf{Fluss eines Vektorfeldes}, das die rechte Seite eines gewöhnlichen Differentialgleichungssystems beschreibt.
\label{ode/fluesse:example-1}
\begin{example}{}{}



\par
Sei \(I\subset \R_0^+\) ein offenes Zeitintervall.
Wir interessieren uns für Lösungen des autonomen gewöhnlichen Differentialgleichungssystems
\begin{align*}
\dot{\vec{x}}(t) = F(\vec{x}) \quad \forall t\in I,
\end{align*}
\par
dessen rechte Seite durch das Vektorfeld \(F \colon \R^2 \rightarrow \R^2\) mit \(F(x,y) \, \coloneqq \, (y, -x)\) gegeben ist.
Abbildung \textbackslash{}xxx illustriert das Vektorfeld in \(\R^2\).

\par
Wir wollen den Fluss des Vektorfeldes \(F\) angeben, der die Bewegung entlang der Lösungskurven der durch das Vektorfeld gegebenen gewöhnlichen Differentialgleichung beschreibt.
Dieser ist gegeben durch
\begin{align*}
\Phi(t,(x,y)) = (\cos(t)x + \sin(t)y, -\sin(t)x + \cos(t)y).
\end{align*}
\par
Das die Funktion \(\Phi \colon I \times \R^2 \rightarrow \R^2\) ein Fluss ist, lässt sich leicht verifizieren durch Nachrechnen der beiden Eigenschaften eines Flusses aus Definition \textbackslash{}ref.

\par
1. Es gilt \(\Phi(0, (x,y)) = (x,y)\) für beliebige Paare \((x,y) \in \R^2\), da
\begin{align*}
\Phi(0, (x,y)) = (1\cdot x + 0\cdot y, - 0 \cdot x + 1 \cdot y) = (x,y).
\end{align*}
\par
2. Es gilt \(\Phi(t, \Phi(s,(x,y)) = \Phi(s + t, (x,y))\) für beliebige Paare \((x,y) \in \R^2\) und Zeitpunkte \(s,t \in I\), da wegen der Additionstheoreme von Sinus und Cosinus gilt
\begin{align*}
\Phi(t, \Phi(s,(x,y))) &= \Phi(t, (\cos(s)x + \sin(s)y, -\sin(s)x + \cos(s)y)) \\
&= [\cos(t)(\cos(s)x + \sin(s)y) + \sin(t)(-\sin(s)x + \cos(s)y), \\
& \ \ -\sin(t)(\cos(s)x + \sin(s)y) + \cos(t)(-\sin(s)x + \cos(s)y)]\\
&= \ [ (\cos(t)\cos(s) - \sin(t)\sin(s))x + (\cos(t)\sin(s) + \sin(t)\cos(s))y, \\
& \quad (-\sin(t)\cos(s) - \cos(t)\sin(s))x + (\cos(t)\cos(s) - \sin(t)\sin(s))y ] \\
&= (\cos(s+t)x + \sin(s+t)y, -\sin(s+t)x + \cos(s+t)y).
\end{align*}
\par
Nun verfizieren wir noch, dass der Fluss tatsächlich Lösungen des gewöhnlichen Differentialgleichungssystems realisiert.
Es gilt
\begin{align*}
\dot{\Phi}(t, (x,y)) &= \frac{d}{dt}(\cos(t)x + \sin(t)y, -\sin(t)x + \cos(t)y) \\
&= (-\sin(t)x + \cos(t)y, -\cos(t)x - \sin(t)y) = F(\Phi(t,(x,y)), \quad \forall t \in I, (x,y) \in U.
\end{align*}
\par
Offensichtlich ist der Fluss \(\Phi \colon I \times \R^2 \rightarrow \R^2\) Lösung des gewöhnlichen Differentialgleichungssystems.
\end{example}


\subsection{Lokale Flüsse}
\label{\detokenize{ode/fluesse:lokale-flusse}}
\par
Nach dem Satz von Picard Lindelöf \cref{ode/repetition:satz:picardlindeloef_lokal} wissen wir, dass für jeden Anfangswert \(x_0\in U\) ein \(\epsilon(x_0)>0\) existiert, so dass es lokal eine eindeutige Lösung \(\phi: [-\epsilon(x_0), \epsilon(x_0)]\) gibt, falls die rechte Seite \(F\) lokal Lipschitzstetig bezüglich der \(y\) Variablen ist.
In diesem Fall müssen wir das Zeitintervall \(I(x_0)=[-\epsilon(x_0), \epsilon(x_0)]\) wählen und können also nicht wie in \cref{ode/fluesse:def:Fluss} auf ganz \(I = \R^+_0\) als Zeitintervall arbeiten.
Stattdessen können wir nur Tupel der Form \((t, x_0) \in I(x_0) \times \{x_0\}\) betrachten, wobei \(x_0\in U\) fixiert ist und \(t\) aus dem lokalen Existenzintervall \(I(x_0)\) gewählt werden kann.

\par
Diese Einschränkung führt uns auf den Begriff des \textbf{lokalen Phasenflusses}.
\label{ode/fluesse:def:LokFluss}
\begin{definition}{}{}



\par
Sei \(U \subset \R^n\) eine offene Teilmenge und der erweiterte Phasenraum \(G\subset \R^+_0\times U\) sei gegeben als
\begin{align*}
G = \bigcup_{x_0\in U} I(x_0) \times \{x_0\},
\end{align*}
\par
wobei \(0\in I(x_0)\) für jedes \(x_0\in U\) gelte.

\par
Dann heißt eine Abbildung \(\Phi: G\rightarrow U\) \textbf{lokaler (Phasen )Fluss}, falls
\begin{enumerate}

\item {} 
\par
\(\Phi(0,x) = x\) für alle \(x\in U\),

\item {} 
\par
\(\Phi(t, \Phi(s, x)) = \Phi(s+t, x)\) für alle \(x\in U\) und alle \(s,t\) mit \(s, s+t\in I(x)\) und \(t\in I(\Phi(s,x))\).

\end{enumerate}
\end{definition}

\par
Im nächsten Lemma werden wir sehen, dass die Lösung eines autonomen gewöhnlichen Differentialgleichungssystems tatsächlich als solch ein lokaler Fluss interpretiert werden kann.
In diesem Fall spircht man auch vom \textbf{Fluss einer Differentialgleichung}.
\label{ode/fluesse:lemma-3}
\begin{lemma}{}{}



\par
Sei \(U\subset\R^n\) eine offene Teilmenge und es sei \(F \colon U \rightarrow \R^n\) eine lokal Lipschitzstetige Abbildung.
Dann existieren Intervalle \(I(x_0)\), so dass es für den erweiterten Phasenraum
\begin{align*}
G = \bigcup_{x_0\in U} I(x_0)\times\{x_0\}
\end{align*}
\par
eine Funktion \(\Phi \colon G\rightarrow \R^n\) gibt, mit folgenden Eigenschaften
\begin{enumerate}

\item {} 
\par
\(\frac{\d}{\d t} \Phi(t, x_0) = F(\Phi(t, x_0))\) für alle \((t,x_0)\in G\),

\item {} 
\par
\(\Phi\) ist ein lokaler Fluss auf \(G\).

\end{enumerate}
\end{lemma}

\begin{proof}
 Da die rechte Seite \(F\) des autonomen gewöhnlichen Differentialgleichungssystems nach Voraussetzung lokal Lipschitzstetig ist, existiert nach dem Satz von Picard Lindelöf \cref{ode/repetition:satz:picardlindeloef_lokal} für jedes \(x_0\in U\) ein \(\epsilon(x_0)>0\), so dass eine Lösung des Differentialgleichungssystems \(\Phi_{x_0} \colon [-\epsilon(x_0),\epsilon(x_0)] \rightarrow U\) mit dem Anfangswert \(x_0\) existiert, d.h.,
\begin{align*}
\dot{\Phi}_{x_0}(t) &= F(\Phi_{x_0}(t)) \quad \forall t \in [-\epsilon(x_0),\epsilon(x_0)],\\
\Phi_{x_0}(0) &= x_0.
\end{align*}
\par
Daher können wir den erweiterten Phasenraum als
\begin{align*}
G = \bigcup_{x_0\in U} [-\epsilon(x_0),\epsilon(x_0)] \times\{x_0\}
\end{align*}
\par
wählen und die Abbildung \(\Phi\) als Einschränkung auf die Funktionen \(\Phi_{x_0}\) so definieren, dass
\begin{align*}
\frac{\d}{\d t} \Phi(t, x_0) &= \dot{\Phi}_{x_0}(t) = F(\Phi_{x_0}(t)) = F(\Phi(t, x_0))\\
\Phi(0, x_0) &= \Phi_{x_0}(0) = x_0
\end{align*}
\par
für alle \((t, x_0)\in G\).
Damit haben wir sowohl die erste Aussage des Lemmas als auch die erste Flusseigenschaft aus \cref{ode/fluesse:def:Fluss} gezeigt.

\par
Die zweite Flusseigenschaft ist eine direkte Folgerung aus der Eindeutigkeit der Lösung des gewöhnlichen Differentialgleichungssystems.
Wir führen den Beweis trotzdem im Folgenden explizit aus.
Es sei \(x_0\in U, s\in [-\epsilon(x_0), \epsilon(x_0)]\) und zusätzlich \(t\) so gewählt, dass \(s+t \in [-\epsilon(x_0), \epsilon(x_0)]\) und \(t\in [-\epsilon(\Phi(s,x_0)), \epsilon(\Phi(s,x_0))]\).
Per Definition löst die Funktion
\begin{align*}
\phi_1(\tau) \ \coloneqq \ \Phi(s + \tau, x_0)
\end{align*}
\par
sowie auch die Funktion
\begin{align*}
\phi_2(\tau) \ \coloneqq \ \Phi(\tau, \Phi(s,x_0))
\end{align*}
\par
das gewöhnliche Differentialgleichungssystem auf dem Intervall \([t, \epsilon(x_0)]\), da \(\Phi\) eine Lösung ist.
Weiterhin wissen wir auf Grund der ersten Flusseigenschaft, dass
\begin{align*}
\phi_1(0) = \Phi(s, x_0) = \Phi(0, \Phi(s, x_0)) = \phi_2(0).
\end{align*}
\par
Somit stimmen also beide Funktionen an einem Punkt überein und sind somit schon auf dem gesamten Intervall \([t, \epsilon(x_0)]\) gleich, was eine direkte Folgerung aus dem Eindeutigkeitssatz 8.20 aus \cite{Ten21} ist.
Wir haben also insgesamt
\begin{align*}
\Phi(s + \tau, x_0) = \phi_1(\tau) = \phi_2(\tau) = \Phi(\tau, \Phi(s,x_0))
\end{align*}
\par
für jedes \(\tau\in [t, \epsilon(x_0)]\), was die zweite Flusseigenschaft aus \cref{ode/fluesse:def:Fluss} zeigt.
\end{proof}


\subsection{Phasenporträts}
\label{\detokenize{ode/fluesse:phasenportrats}}
\par
Die teilweise abstrakten Konzepte und Eigenschaften von Phasenflüssen aus den vorangegangenen Abschnitten werden wir im Folgenden mit einfachen geometrischen Anschauungen illustrieren.
Dafür benötigen wir zunächst die folgenden Definitionen.
\label{ode/fluesse:definition-4}
\begin{definition}{}{}



\par
Es sei \(\Phi:G\rightarrow U\) ein Phasenfluss eines gewöhnlichen Differentialgleichungssystems für den erweiterten Phasenraum \(G = I \times U\subset \R^+_0\times \R^n\).
Dann können wir folgende Begriffe für den Fluss einführen:
\begin{itemize}
\item {} 
\par
Für jedes \(x_0\in U\) heißt die Funktion \(t\mapsto \Phi(t, x_0)\) \textbf{Bahnkurve} durch \(x_0\).

\item {} 
\par
Die Menge \(\mathcal{O}(x_0) := \{\Phi(t, x_0): (t, x_0)\in G\}\) heißt \textbf{Orbit} oder \textbf{Trajektorie} durch \(x_0\).

\item {} 
\par
Ein Punkt \(x_0 \in U\) heißt \textbf{Ruhelage}, falls \(\mathcal{O}(x_0) = \{x_0\}\).

\item {} 
\par
Ein Anfangswert \(x_0\in U\) heißt \textbf{periodisch} mit Periode \(T>0\), falls \(\Phi(T, x_0) = x_0\).

\end{itemize}

\par
Wir nennen die Zerlegung des erweiterten Phasenraums \(G\) in Orbits ein \textbf{Phasenporträt} des dynamischen Systems \((I,U, \Phi)\).
\end{definition}

\par
Phasenporträts erlauben es uns das charakteristische Verhalten kontinuierlicher dynamischer Systeme zu visualisieren und graphisch zu analysieren.
An ihnen lassen sich beispielsweise die Existenz und Stabilität von Fixpunkten und periodischen Orbits direkt erkennen.
Da ein Phasenporträt den gesamten Phasenraum zerlegt werden typischerweise nur einige charakteristische Orbits gezeichnet um die Übersichtlichkeit zu gewährleisten.
Aus dem gleichen Grund beschränkt man sich in der Regel außerdem auf ein  und zweidimensionale Phasenräume.

\par
Ein klassisches Beispiel aus der Mechanik ist besonders gut geeignet, um die eben eingeführten Konzepte näher zu diskutieren   die gedämpfte Schwingungsgleichung.
\label{ode/fluesse:ex:oscillations}
\begin{example}{}{}



\par
Die \textbf{gedämpfte Schwingungsgleichung} ist gegeben durch
\begin{align}\label{equation:ode/fluesse:eq:schwingungsgleichung}
m\ddot{x}(t) + r\dot{x}(t) + kx(t)=0
\end{align}
\par
und beschreibt beispielsweise die horizontale (eindimensionale) Auslenkung eines Federpendels, das durch Reibungsverluste Schwingungsenergie über die Zeit verliert.

\par
Hierbei bezeichnet
\begin{itemize}
\item {} 
\par
\(x(t)\) die horizontale Auslenkung des Federpendels zum Zeitpunkt \(t\),

\item {} 
\par
\(m\) die Masse des Objekts,

\item {} 
\par
\(r\) die Dämpfungskonstante,

\item {} 
\par
\(k\) die Federkonstante.

\end{itemize}

\par
Durch Einführung der Variablen \(p(t):= m\dot{x}(t)\) als Impuls erhalten wir das folgende gewöhnliche Differentialgleichungssystem
\begin{align*}
\dot{p}(t) &= - \frac{r}{m}p(t) -kx(t), \\
\dot{x}(t) &= \frac{1}{m}p(t).
\end{align*}
\par
Dies lässt sich in kompakter Form schreiben als:
\begin{align*}
\begin{pmatrix} \dot{p} \\ \dot{x} \end{pmatrix}(t) = \begin{pmatrix} -\frac{r}{m} & -k \\ \frac{1}{m} & 0\end{pmatrix} \begin{pmatrix}p \\ x\end{pmatrix}(t)
\end{align*}
\par
Betrachten wir speziell den ungedämpften Fall für \(r=0\), d.h. ohne Reibungsverluste, so geht die Gleichung in die \textbf{Bewegungsgleichung für einen harmonischen Oszillator} über.
In diesem Fall erhalten wir zum Anfangswert \((p,x) \in U \subset \R^2 \) die Lösung
\begin{align*}
\Phi(t, (p,x)) = 
\begin{pmatrix}
p \cos(\omega t) - m x \sin(\omega t) \\
\frac{p}{\omega m}\sin(\omega t) + x\cos(\omega t)
\end{pmatrix},
\end{align*}
\par
wobei \(\omega=\sqrt{\frac{k}{m}}\) die Eigenfrequenz des Systems ist.
\end{example}

\par
Wir wollen im Folgenden die Phasenporträts der gedämpften Schwingungsgleichung und des harmonischen Oszillators illustrieren.

\par
In beiden Abbildungen wird die horizontale Auslenkung \(x(t)\) des Federpendels auf der x Achse und der Impuls \(p(t) = m\dot{x}(t)\) auf der y Achse aufgetragen.

\par
Das in \hyperref[\detokenize{ode/fluesse:fig-harmonic-oscillator}]{Fig.\@ \ref{\detokenize{ode/fluesse:fig-harmonic-oscillator}}} dargestellte Phasenportrait illustriert anschaulich, dass im Fall des harmonischen Oszillators ohne Dämpfung die Orbits elliptisch sind und somit jeder Startwert \(x_0 \in U\) periodisch ist.

\begin{figure}[htbp]
\centering


\noindent\includegraphics[width=\textwidth]{../\string_build/html/\string_images/fluesse\string_3\string_0.png}
\caption{Visualisierung des Phasenporträts und einiger Orbits für den Phasenfluss des harmonischen Oszillators aus \cref{ode/fluesse:ex:oscillations}  Das Phasenporträt zeigt das charakteristische Verhalten von Lösungen der gedämpften Schwingungsgleichung für reibungsfreie Prozesse, d.h., für eine Dämpfungskonstante \(r = 0\).}\label{\detokenize{ode/fluesse:fig-harmonic-oscillator}}\end{figure}

\par
Betrachten wir nun für eine positive Dämpfungskonstante \(r > 0\) den Fall der allgemeinen gedämpften Schwingungsgleichung, so sieht man am dargestellen Phasenportrait in \hyperref[\detokenize{ode/fluesse:fig-damped-oscillator}]{Fig.\@ \ref{\detokenize{ode/fluesse:fig-damped-oscillator}}}, dass die Trajektorien in den Ursprung konvergieren, der als Orbit in Ruhelage einen Fixpunkt des dynamischen Systems darstellt.
Dies macht auch physikalisch Sinn, da jedes Federpendel auf Grund der Reibung nach endlicher Zeit zum Stillstand kommt.

\begin{figure}[htbp]
\centering


\noindent\includegraphics[width=\textwidth]{../\string_build/html/\string_images/fluesse\string_6\string_0.png}
\caption{Visualisierung des Phasenporträts und einiger Orbits für den Phasenfluss der gedämpften Schwingungsgleichung aus \cref{ode/fluesse:ex:oscillations} für eine relativ groß gewählte Dämpfungskonstante \(r > 0\).}\label{\detokenize{ode/fluesse:fig-damped-oscillator}}\end{figure}


\section{Hamiltonsche Differentialgleichungen}
\label{\detokenize{ode/hamilton:hamiltonsche-differentialgleichungen}}\label{\detokenize{ode/hamilton::doc}}
\par
Ein wichtiges Prinzip für viele physikalischen Anwendungen und dynamische Systeme sind \emph{Erhaltungssätze} und die dazugehörigen \emph{Erhaltungsgrößen}.
Aus der klassichen Mechanik kennen wir beispielsweise die \emph{Energieerhaltung} oder die \emph{Impulserhaltung}.
In \cref{ode/fluesse:s-fluesse}  haben wir Bewegungsgleichungen als System von gewöhnlichen Differentialgleichungen hergeleitet und gelöst, deshalb wollen wir nun die nötige Theorie entwickeln, die es uns erlaubt Erhaltungsgrößen direkt aus der Formulierung des Differentialgleichungssystems abzulesen.

\par
Hamiltonsche Differentialgleichungen haben in der Physik eine besondere Rolle, insbesondere in der klassischen Mechanik bei Abwesenheit von Reibung.
Typischerweise tauchen diese bei der Untersuchung von Bewegungen im Phasenraum auf, d.h., bei der Betrachtung von Paaren aus Orts  und Impulswerten.
Ihre Lösungen liefern uns Trajektorien im Phasenraum für die die Gesamtenergie des Systems erhalten bleibt.
Dies macht sie für uns besonders interessant.

\par
Bevor wir die hamiltonschen Differentialgleichungen und ihre Eigenschaften näher diskutieren führen wir zunächst ein wann wir ein Vektorfeld auf dem Phasenraum Hamiltonsch nennen und was eine Hamilton Funktion dieses Vektorfelds ist.
\label{ode/hamilton:def:hamiltonsch}
\begin{definition}{}{}



\par
Sei \(n \in N\) die \textbf{Anzahl der Freiheitsgrade} des betrachteten dynamischen Systems und sei \(U\subset \R^n \times \R^n\) der zugehörige Phasenraum.
Wir nennen ein Vektorfeld \(X \colon U \rightarrow \R^{2n}\) mit \(X \in C^1(P;\R^{2n})\) \textbf{Hamiltonsch}, falls eine reellwertige Funktion \(H \colon U \rightarrow \R\) sowie eine Matrix \(J \, \coloneqq \, \begin{pmatrix}0 & -\mathbf{1}\\ \mathbf{1} & 0 \end{pmatrix} \in \R^{2n \times 2n}\) existiert, so dass sich das Vektorfeld darstellen lässt als
\begin{align}\label{equation:ode/hamilton:eq:hamilton_Gleichung}
X(p,q) = J \, \nabla H (p,q) \quad \forall (p,q) \in U.
\end{align}
\par
In diesem Fall nennen wir die Funktion \(H\) eine \textbf{Hamilton Funktion} des Vektorfelds \(X\).
\end{definition}

\par
Folgende Bemerkungen zur Hamilton Funktion wollen wir festhalten.
\label{ode/hamilton:remark-1}
\begin{emphBox}{}{}{Remark 1.5}


\begin{enumerate}

\item {} 
\par
Die Hamilton Funktion lässt sich auch als Legendre Transformation der Lagrange Funktion des Systems herleiten, was weitere interessante Zusammenhänge in der Physik erklärt.
In dieser Vorlesung verzichten wir auf diesen Zugang zur Hamilton Funktion und verweisen die interessierten Leser*innen auf Kapitel 2 \cite{Nol11}.

\item {} 
\par
Im Folgenden werden wir annehmen, dass die Hamilton Funktion \(H\) nicht explizit von der Zeitvariable \(t \in I\) abhängt, was jedoch im Allgemeinen sein kann.

\end{enumerate}
\end{emphBox}

\par
Basierend auf der Hamilton Funktion aus \cref{ode/hamilton:def:hamiltonsch} können wir nun die Hamiltonschen Differentialgleichungen definieren.
\label{ode/hamilton:definition-2}
\begin{definition}{}{}



\par
Sei \(x(t) = (p(t),q(t)) \in U\) eine Bahnkurve des Phasenraums \(U \subset \R^{2n}\).
Wird das hamiltonsche Vektorfeld auf der linken Seite von \cref{ode/hamilton:equation-eq-hamilton-gleichung} als
\begin{align*}
X = \dot{x}(t) = \begin{pmatrix} \dot{p} \\ \dot{q} \end{pmatrix} (t)
\end{align*}
\par
gewählt, so lässt sich die Gleichung für \(J \, \coloneqq \, \begin{pmatrix}0 & -\mathbf{1}\\ \mathbf{1} & 0 \end{pmatrix} \in \R^{2n \times 2n}\) schreiben als
\begin{align}\label{equation:ode/hamilton:eq:hamilton_DGL}
\dot{x}(t) = J \nabla H(x(t)).
\end{align}
\par
In dieser Form wird die entstehende Differentialgleichung in \cref{ode/hamilton:equation-eq-hamilton-dgl} \textbf{Hamiltonsche Differentialgleichung} genannt.

\par
Äquivalent lässt sich dieses System von gewöhnlichen Differentialgleichungen auch explizit für die \(2n\) unbekannten Orts  und Impulsfunktionen \(q_i, p_i\) für \(1 \leq i \leq n\) schreiben als
\begin{align*}
\dot{q_i}(t) = \frac{\partial H}{\partial p_i}(t), \quad \dot{p_i}(t) = -\frac{\partial H}{\partial q_i}(t), \quad i=1,\ldots,n.
\end{align*}\end{definition}

\par
Für den einfachen Fall einer zeitunabhängigen Hamilton Funktion \(H\) lässt sich beobachten, dass die Lösungskurven der Hamiltonschen Differentialgleichungen sich nicht schneiden und durch jeden Punkt des Phasenraums eine Lösungskurve verläuft.

\par
Die Hamilton Funktion \(H\) als Funktion des Phasenraumes kann als die Energie eines Systems von Teilchen aufgefasst werden.
Wir wollen uns die Rolle der Hamilton Funktion \(H\) an Hand eines physikalischen Beispiels klar machen.
\label{ode/hamilton:example-3}
\begin{example}{}{}



\par
Im folgenden Beispiel wollen wir die Bewegung eines Teilchens mit Masse \(m>0\) in einem Kraftfeld \(F \colon \R^3 \rightarrow \R^3\)  untersuchen, welches nur vom Ort \(q \in \R^3\) abhängt.
Nach dem 2. Newtonschen Gesetz erhalten wir die Bewegungsgleichung
\begin{align}\label{equation:ode/hamilton:eq:newton}
m\ddot{q}(t) = F(q(t)).
\end{align}
\par
Die gewöhnliche Differentialgleichung 2. Ordnung in \cref{ode/hamilton:equation-eq-newton} lässt sich durch die Definition des Impulses des Teilchens \(p(t) \, \coloneqq \, m \dot{q(t)}\) in ein gewöhnliches Differentialgleichungssystem 1. Ordnung überführen:
\begin{align*}
\dot{p}(t) = F(q(t)), \quad \dot{q}(t) = \frac{1}{m}p(t).
\end{align*}
\par
Wir nehmen zur Vereinfachung nun an, dass das gegebene Kraftfeld \(F\) \emph{konservativ} sei, d.h., wir können annehmen, dass \(F = - \nabla V\) gilt für ein Potential \(V \colon \R^3 \rightarrow \R\) (z.B. die Erdanziehungskraft).
Dann können wir das physikalische Modell als kontinuierliches dynamisches System interpretieren mit dem erweiterten Phasenraum \(I \times U \subset \R^+_0 \times \R^6\).
Betrachten wir nun einen Punkt \(x = \begin{pmatrix} p \\ q\end{pmatrix} \in U\) im Phasenraum, so lässt sich das autonome gewöhnliche Differentialgleichungssystem kompakt schreiben als
\begin{align}\label{equation:ode/hamilton:eq:newton_DGL}
\dot{x}(t) = \begin{pmatrix} \dot{p} \\ \dot{q} \end{pmatrix}(t) = \begin{pmatrix} -\nabla V(q) \\ \frac{p}{m} \end{pmatrix}(t)
\end{align}
\par
Wählen wir nun die \textbf{Hamilton Funktion} aus \cref{ode/hamilton:def:hamiltonsch} \begin{align*}
H(p,q) \, \coloneqq \, \frac{||p||^2}{2m} + V(q),
\end{align*}
\par
so erkennen wir, dass diese sich aus \emph{kinetischer} und \emph{potentieller Energie} zusammensetzt.
Durch diese Hamilton Funktion \(H\) lässt sich \cref{ode/hamilton:equation-eq-newton} als \textbf{Hamiltonsche Differentialgleichung} schreiben mit
\begin{align*}
\dot{x}(t) = \begin{pmatrix}\dot{p} \\ \dot{q} \end{pmatrix}(t) = \begin{pmatrix} -\nabla V(q) \\ \frac{p}{m} \end{pmatrix}(t) = \begin{pmatrix}0 & -\mathbf{1}\\ \mathbf{1} & 0 \end{pmatrix} \begin{pmatrix} \frac{p}{m} \\ \nabla V(q) \end{pmatrix}(t) = J \nabla H(p(t),q(t)).
\end{align*}\end{example}

\par
Ergänzend wollen wir noch folgendes Beispiel einer Hamilton Funktion nennen.
\label{ode/hamilton:example-4}
\begin{example}{}{}



\par
Im Fall des eindimensionalen harmonischen Oszillators mit Masse \(m > 0\) aus \cref{ode/fluesse:ex:oscillations} lässt sich ebenfalls eine Hamilton Funktion des dynamischen Systems angeben.
Sei \((x,p) \in U\) als Punkt des Phasenraums \(U \subset \R^2\) der Ort und Impuls eines Pendels.
Dann lässt sich die zugehörige Hamilton Funktion \(H \colon U \rightarrow \R\) angeben als:
\begin{align*}
H(x,p) = \frac{p^2}{2m} + \frac{m}{2} \omega^2 x^2.
\end{align*}
\par
Hierbei bezeichnet \(\omega = \sqrt{\frac{k}{m}}\) die Eigenfrequenz des Systems und \(k > 0\) die Federkonstante.
\end{example}

\par
Bisher haben wir noch nicht den Grund diskutiert, warum die Hamilton Funktion eine besondere Rolle im Kontext dynamischer Systeme spielt.
Das wollen wir nun im folgenden Satz nachholen.
\label{ode/hamilton:satz:hamilton_konstant}
\begin{theorem}{}{}



\par
Sei \(n\in \N, U \subseteq \mathbb{R}^{2n}\) ein (offener) Phasenraum und \(J= \begin{pmatrix} 0 & - \mathbf{1} \\ \mathbf{1} & 0 \end{pmatrix} \in \mathbb{R}^{2n \times 2n}\).
Ist die Hamilton Funktion \(H \in C^2(U; \mathbb{R})\), dann ist sie entlang der Lösungskurven der Hamiltonschen Differentialgleichung
\begin{align*}
\dot x = J \nabla H(x)
\end{align*}
\par
konstant.
\end{theorem}

\begin{proof}
 In der Hausaufgabe zu zeigen.
\end{proof}

\par
\cref{ode/hamilton:satz:hamilton_konstant} sagt uns also, dass die Orbits des kontinuierlichen Systems innerhalb der Niveaumengen der Hamilton Funktion verlaufen.
Dies erlaubt es uns dynamische Systeme auf diese häufig auch \emph{Energieschalen} genannten Niveaumengen \(H^{-1}(E)\) für \(E \in \R\) zu restringieren.
Diese Energieschalen bilden Untermannigfaltigkeiten des Phasenraums \(U\).

\par
Für den einfachen Fall eines Freiheitsgrades, d.h., für \(n = 1\), lassen sich für eine gegebene Hamilton Funktion \(H\) die Orbits des dynamischen Systems bestimmen.
Für einen Punkt \(x \in U\) im Phasenraum \(U \subset \R^2\) unterscheiden wir zwei Fälle:
\begin{enumerate}

\item {} 
\par
Ist \(\nabla H(x) = 0\), so ist der Orbit wegem \cref{ode/hamilton:equation-eq-hamilton-dgl} von der Form \(O(x) = {x}\).

\item {} 
\par
Ist \(\nabla H(x) \neq 0\), so ist der Orbit \(O(x)\) gegeben durch die zusammenhängende Menge

\end{enumerate}
\begin{align*}
O(x) = \{y \in U | H(y) = H(x), \nabla H(y) \neq 0\}
\end{align*}
\par
Die Orientierung des Orbits erhält man durch die Richtung, die orthogonal zum Gradienten \(\nabla H\) steht, d.h., durch Drehung des Gradienten im Uhrzeigersinn um \(\frac{\pi}{2}\).
Die Matrix \(J\) entspricht eben einer solchen Drehung.
\label{ode/hamilton:remark-6}
\begin{emphBox}{}{}{Remark 1.6}



\par
Eine Formulierung der Bewegungsgleichungen eines dynamischen Systems als Hamiltonsche Differentialgleichungen hat den Vorteil, dass sie unter den sogenannten \emph{kanonischen Transformationen} in manchen Fällen in eine einfachere, lösbare Form gebracht werden können.
\end{emphBox}


\chapter{Stabilitätsanalyse für dynamische Systeme}
\label{\detokenize{ode_stability/stabilitaetsanalyse:stabilitatsanalyse-fur-dynamische-systeme}}\label{\detokenize{ode_stability/stabilitaetsanalyse::doc}}
\par
In diesem Abschnitt beschäftigen wir uns mit der Stabilitätstheorie für kontinuierliche dynamische Systeme.
Hierbei interessieren wir uns für die Frage, wie sich \emph{kleine Störungen} von bestimmten Zuständen des Systems auf die Lösungen der zu Grunde liegenden gewöhnlichen Differentialgleichungen auswirken.
Der untersuchte Zustand kann beispielsweise ein periodischer Orbit oder eine Ruhelage des dynamischen Systems sein.
Letztere sind oftmals von besonderes Interesse, da man in vielen technischen und physikalischen Anwendungen daran interessiert ist das System in eine oder nahe einer Gleichgewichtslage zu bringen.

\par
Im Folgenden werden wir verschiedene Stabilitätsbegriffe für dynamische Systeme einführen und speziell Kriterien für die Stabilität von Ruhelagen diskutieren.


\section{Stabilitätsbegriffe}
\label{\detokenize{ode_stability/stabilitaetsbegriffe:stabilitatsbegriffe}}\label{\detokenize{ode_stability/stabilitaetsbegriffe::doc}}
\par
Im Folgenden wollen wir grundlegende Begriffe der Stabilitätsanalyse von Ruhelagen einführen und diskutieren.
Wie in \cref{ode/fluesse:s-fluesse}  definiert, nennen wir einen Punkt \(x\in U\) im Phasenraum \(U\) \textbf{Ruhelage}, falls für den zugehörigen Phasenfluss \(\Phi \colon I \times U \rightarrow U\) des dynamischen Systems gilt: \(\Phi(t,x) = x, \forall t \in I\), d.h., wenn für alle \(t \in I\) der Zustand \(x \in U\) ein \textbf{Fixpunkt des Flusses} ist.

\par
Für autonome Differentialgleichungssysteme mit
\begin{align*}
\dot{x}(t) = F(x)
\end{align*}
\par
ist \(x \in U\) auch eine Ruhelage, falls \(F(x) = 0\) gilt, d.h., falls \(x\) eine Nullstelle von \(F\) ist.
Das ist einfach zu verstehen, da die Zeitableitung auf der linken Seite für eine Ruhelage Null ist und somit die Funktion \(F\), die nur vom Ort abhängt, sich nicht ändern kann.

\par
Anschaulich versteht man unter der Stabilitätsanalyse von Ruhelagen die mathematische Untersuchung, ob benachbarte Lösungen von einer Ruhelage wegstreben oder nicht.
Dies ist insbesondere in technischen Anwendungen wichtig, da man dort häufig danach strebt ein dynamisches System in eine Gleichgewichtslage zu bringen.
Da dies nur bis zu einer gewissen Genauigkeit möglich ist, muss man also mit kleinen Störungen rechnen.

\par
Ist eine Ruhelage stabil, dann bleiben benachbarte Lösungen auch für zukünftige Zeitpunkte \(t \in I\) nahe der Ruhelage.
Ist sie jedoch nicht stabil, so muss das im Allgemeinen nicht gelten und die Lösungen können dann mit der Zeit von der Ruhelage divergieren.
Diese Anschauung wollen wir in der folgenden Definition mathematisch formalisieren.
Hierbei werden wir den Stabilitätsbegriff für allgemeine Lösungen einführen und später Ruhelagen als ein Spezialfall dieser Lösungen interpretieren.
\label{ode_stability/stabilitaetsbegriffe:def:Stabilitaet}
\begin{definition}{}{}



\par
Sei \(\Phi \colon I \times U \rightarrow U\) der Phasenfluss zu dem Vektorfeld \(F\in C^1(U;\R^n)\) auf \(U\), dass durch die rechte Seite des zugehörigen Differentialgleichungssystems gegeben ist.

\par
1. Eine Lösung \(t \in [0,\infty) \mapsto \Phi_t(x)\) heißt \textbf{(Lyapunov )stabil}, wenn zu jedem \(\epsilon > 0\) ein \(\delta>0\) existiert mit:
\begin{align*}
\|x-y\|<\delta \ \Rightarrow \ \sup_{t\geq0}\|\Phi_t(x)-\Phi_t(y)\|<\epsilon.
\end{align*}
\par
2. Eine Lösung \( t \in [0,\infty) \mapsto \Phi_t(x)\) heißt \textbf{asymptotisch stabil}, wenn ein \(\delta > 0\) existiert mit:
\begin{align*}
\|x-y\|<\delta \ \Rightarrow \ \lim_{t\to\infty}\|\Phi_t(x)-\Phi_t(y)\|=0.
\end{align*}
\par
3. Eine Lösung heißt \textbf{instabil}, wenn sie nicht (Lyapunov )stabil ist.
\end{definition}

\par
Es ist klar, dass der Begriff der asymptotischen Stabilität \emph{stärker} als der Begriff der Lyapunov Stabilität von Lösungen ist, da jede asymptotisch stabile Lösung auch schon Lyapunov stabil ist.
Die Umkehrung gilt jedoch im Allgemeinen nicht.
Dies wird durch das folgende Beispiel nochmal illustriert.
\label{ode_stability/stabilitaetsbegriffe:example-1}
\begin{example}{}{}



\par
Der Phasenfluss für den harmonischen Oszillator ist, wie wir in \cref{ode/fluesse:ex:oscillations} gesehen haben, gegeben durch
\begin{align*}
\Phi(t, (p,x)) = \begin{pmatrix}
p \cos(\omega t) - m x \sin(\omega t)\\
\frac{p}{\omega m}\sin(\omega t) + x\cos(\omega t)
\end{pmatrix}
\end{align*}
\par
Wir suchen nun einen Fixpunkt \((p_r,x_r) \in U\) des Flusses der unabhängig ist vom Zeitpunkt \(t\).
Man sieht leicht ein, dass eine \textbf{Ruhelage} sich bei \((p_r,x_r) = (0,0)^T \in U\) befindet, da \(\Phi(t,(0,0)) = (0,0)^T\) ist für alle \(t \in I\).
Die gefundene Ruhelage ist \textbf{Lyapunov stabil}, denn wie wir im Phasenporträt in  gesehen haben, ist jeder Orbit um die Ruhelage \((0,0)\) periodisch. Damit kann das dynamische System insgesamt nicht wegstreben von der Ruhelage.

\par
Mathematisch lässt sich diese Eigenschaft wie folgt zeigen.
Für ein beliebiges \(\epsilon > 0\) sei \((p,y) \in U\) ein Punkt im Phasenraum mit periodischen Orbit \(O(p,y)\) um die Ruhelage \((p_r,x_r) = (0,0)^T \in U\), so dass dessen maximaler Abstand zur Ruhelage kleiner als \(\epsilon\) ist, d.h.
\begin{align*}
\sup_{t \geq 0} ||\Phi_t(p_r,x_r) - \Phi_t(p,y)|| < \epsilon
\end{align*}
\par
Auf Grund der ersten Eigenschaft des Phasenflusses \(\Phi_0(p,y) = (p,y)\) gilt dann aber schon
\begin{align*}
||(p_r, x_r) - (p,y)|| = ||\Phi_0(p_r, x_r) - \Phi_0(p,y)|| < \epsilon.
\end{align*}
\par
Wählen wir nun \(\delta \coloneqq \epsilon\), so haben wir gezeigt, dass die Ruhelage \((p_r, x_r) = (0,0)^T\) Lyapunov stabil ist.
Sie ist jedoch auf Grund der Periodizität der Orbits um die Ruhelage \textbf{nicht asymptotisch stabil}, da für beliebige Punkte \((p,y) \in U\) mit \(||(p_r,x_r) - (p,y)|| < \delta\) für ein \(\delta > 0\) gilt
\begin{align*}
\lim_{t\to\infty}\|\Phi_t(p_r, x_r)-\Phi_t(p,y)\| \neq 0.
\end{align*}\end{example}

\par
Im allgemeinen Fall der gedämpften Schwingungsgleichung in \cref{ode/fluesse:ex:oscillations} hängt die Stabilität der Ruhelage im Ursprung intuitiverweise von der Reibungskonstanten ab, wie folgende Bemerkung festhält.
\label{ode_stability/stabilitaetsbegriffe:remark-2}
\begin{emphBox}{}{}{Remark 2.1 (Stabilität bei der gedämpften Schwingungsgleichung)}



\par
Für den Fall der gedämpften Schwingungsgleichung in \cref{ode/fluesse:equation-eq-schwingungsgleichung} lässt sich folgendes Stabilitätsverhalten der Ruhelage im Ursprung in Abhängigkeit der Reibungskonstanten \(r \in \R\) beobachten:
\begin{enumerate}

\item {} 
\par
Die Ruhelage ist \textbf{asymptotisch stabil} für den Fall mit positiver Reibung \(r>0\).

\item {} 
\par
Die Ruhelage ist \textbf{Lyapunov stabil} für den reibungsfreien Fall \(r=0\).

\item {} 
\par
Die Ruhelage ist \textbf{instabil} für den Fall einer negativen Reibung \(r < 0\), d.h. für einen externen Antrieb.

\end{enumerate}
\end{emphBox}


\section{Stabilität von Ruhelagen}
\label{\detokenize{ode_stability/ruhelagen:stabilitat-von-ruhelagen}}\label{\detokenize{ode_stability/ruhelagen::doc}}
\par
Zunächst wollen wir die Stabilität von dynamischen System im einfachen Fall von Ruhelagen für allgemeine \textbf{lineare} Differentialgleichungssysteme untersuchen.
Diese Familie von gewöhnlichen Differentialgleichungssystemen haben wir schon in Kapitel 8 in \cite{Ten21} kennen gelernt.

\par
Das folgende Theorem beschreibt die Existenz und Eindeutigkeit einer Ruhelage eines dynamischen System, das durch ein lineares Differentialgleichungssystem charakterisiert wird und gibt Bedingungen für die Stabilität der Ruhelage.
\label{ode_stability/ruhelagen:theorem-0}
\begin{theorem}{}{}



\par
Sei \(A\in \C^{n\times n}\) eine Matrix mit den Eigenwerten \(\lambda_1,\dots, \lambda_n\in \C\).
Dann beschreibt der zugehörige Phasenfluss \(\Phi\) zum homogenen linearen Differentialgleichungssystem
\begin{align*}
\dot{x}(t) = Ax(t)
\end{align*}
\par
eine Ruhelage in \(\vec{0} \in \C^n\).
Diese ist sogar eindeutig, falls \(\lambda_i\neq 0, i=1,\ldots,n\) gilt.

\par
Für
\begin{align*}
\gamma \coloneqq \max_{i=1,\dots,n} \mathcal{Re}(\lambda_i)
\end{align*}
\par
kann die Stabilität der Ruhelage wie folgt charakterisiert werden:
\begin{enumerate}

\item {} 
\par
Falls \(\gamma <0\) gilt, ist die Ruhelage \(\vec{0}\) \emph{asymptotisch stabil}

\item {} 
\par
Falls \(\gamma >0\) gilt, ist die Ruhelage \(\vec{0}\) \emph{instabil}.

\end{enumerate}
\end{theorem}

\begin{proof}
 Wir wissen, dass für einen beliebigen Startpunkt \(x_0 \in U\) im Phasenraum der Phasenfluss \(\Phi \colon I \times U \rightarrow U\) eine Lösung des Differentialgleichungssystems realisiert.
Für homogene, lineare Differentialgleichungssysteme haben wir bereits in \cref{ode/repetition:s-lineare-dglsysteme}  Lösungen mittels des \emph{Matrixexponentials} hergeleitet.

\par
Sei \(J = S^{-1}AS\) die Jordansche Normalform von \(A\) mit Transformationsmatrizen \(S^{-1},S \in \C^{n \times n}\), so erhalten wir die Abschätzung
\begin{align*}
\|\Phi_t(x_0)\| &= \|e^{tA}x_0\| = \|S^{-1}e^{tJ}Sx_0\| = \|S^{-1}e^{tD}e^{tN}Sx_0\| \\
&\leq \|S^{-1}\| \cdot \|e^{tD}\| \cdot \|e^{tN}\| \cdot \|S\| \cdot \|x_0\| \leq C_1 \cdot \|e^{tD}\| \cdot \|e^{t N}\|,
\end{align*}
\par
für eine Konstante \(C_1 > 0\), die unabhängig von \(t\) ist.
Hierbei haben wir ausgenutzt, dass sich die Jordannormalform \(J\) von \(A\) als Summe einer Diagonalmatrix \(D\) mit den Eigenwerten \(\lambda_i \in \C\), \(i=1,\ldots,n\) von \(A\) und einer nilpotenten Matrix \(N\) schreiben lässt als \(J = D + N\).
Diese Matrizen kommutieren, d.h., \(D \cdot N = N \cdot D\).

\par
Wir sehen nun ein, dass \(e^{tN}\) wegen der Nilpotenz von \(N\) eine endliche Reihe bildet der Form
\begin{align*}
e^{tN} = \sum_{k=0}^m \frac{(tN)^k}{k!} = \sum_{k=0}^m t^k\frac{N^k}{k!},
\end{align*}
\par
welches ein Polynom vom Grad \(m\) darstellt, wobei \(m \in \N\) der Nilpotenzindex der Matrix \(N\) ist.

\par
Sei nun \(\epsilon > 0\) beliebig klein gewählt.
Dann lässt sich die Norm des Polynoms mit einer genügend großen Konstanten \(C_2 > 0\), die von \(\epsilon\) jedoch nicht von \(t\) abhängt, durch eine gewöhnliche Exponentialfunktion abschätzen mit
\begin{align*}
 \|e^{tN}\| = \| \sum_{k=0}^m t^k\frac{N^k}{k!} \| \leq \sum_{k=0}^m t^k \frac{\|N^k\|}{k!} \leq C_2  e^{t \epsilon}.
\end{align*}
\par
Wählen wir nun \(\gamma \coloneqq \max_{i=1,\dots,n} \mathcal{Re}(\lambda_i)\), so folgt direkt, dass gilt
\begin{align*}
||e^{tD}|| \leq C_3 e^{t\gamma}.
\end{align*}
\par
Insgesamt erhalten wir also für die Norm des Phasenflusses
\begin{align*}
\|\Phi_t(x_0)\| \leq C_1 \cdot \|e^{tN}\| \cdot \|e^{tD}\| \leq C_1 \cdot C_2 e^{t \epsilon} \cdot C_3 e^{t\gamma} = C e^{t \epsilon} e^{t\gamma}.
\end{align*}
\par
Da \(\epsilon > 0\) beliebig klein ist, können wir \(|\epsilon| < |\gamma|\) wählen.
Damit hängt das Verhalten der Norm des Flusses nur noch vom Vorzeichen von \(\gamma\) ab.
Wir unterscheiden daher zwei Fälle:

\par
1. Wenn \(\gamma >0\) ist, so existiert zum Eigenwert \(\gamma\) von \(A\) ein zugehöriger Eigenvektor \(v\in U\), so dass die Eigenwertgleichung \(A v = \gamma v\) gilt.
Nach Lemma \cref{ode/repetition:lem:mpotew} ist dann \(e^{t\gamma}\) ein Eigenwert des Matrixexponentials \(e^{tA}\) mit zugehörigem Eigenvektor \(v\).
Insgesamt erhalten wir also
\begin{align*}
||\Phi_t(\alpha v)|| = ||e^{tA}\alpha v|| = ||\lambda e^{t\gamma} \alpha v|| \to \infty, \quad \text{ für } \ t \to \infty, \quad  \forall \alpha>0.\end{align*}
\par
Also enthält jede beliebig kleine Umgebung der Ruhelage \(0\) Punkte, für die die entsprechenden Lösungen divergieren.
In diesem Fall ist die Ruhelage also \textbf{instabil}.

\par
2. Falls \(\gamma <0\) gilt, so können wir abschätzen, dass
\begin{align*}
0\leq \|\Phi_t(x_0)-0\|\leq C e^{t\epsilon} e^{t \gamma} \to 0 \quad \text{ für } \ t \to \infty.
\end{align*}
\par
Dies liefert uns also \textbf{asymptotische Stabilität} der Ruhelage \(0\).
\end{proof}

\par
Wir haben also gesehen, dass im Fall eines homogenen, linearen Differentialgleichungssystems die \(0\) immer eine Ruhelage des zugehörigen dynamischen Systems darstellt, deren Stabilität einzig vom Vorzeichen des größten Eigenwerts abhängt.


\subsection{Linearisierung um Ruhelage}
\label{\detokenize{ode_stability/ruhelagen:linearisierung-um-ruhelage}}
\par
In diesem Abschnitt wollen wir unsere Erkentnisse zur Stabilitätsanalysie vom Fall eines linearen Differentialgleichungssystems auf den allgemeinen Fall übertragen, da man es in den meisten Anwendungen leider nur selten mit linearen Differentialgleichungen zu tun hat.
Darüber hinaus ist es erstrebenswert Stabilitätsaussagen zu Differentialgleichungen zu machen, deren Lösungen man nicht explizit analytisch herleiten kann.
Daher betrachten wir im Folgenden das Anfangswertproblem eines \textbf{allgemeinen Differentialgleichungssystem erster Ordnung} auf dem Phasenraum \(U\in \R^n\), das nicht notwendigerweise linear sein muss und für ein Vektorfeld \(F\in C^1(U;\R^n)\) wie folgt formuliert ist
\begin{align}\label{equation:ode_stability/ruhelagen:eq:awpallg}
\dot{x}(t) &= F(x(t)), \quad \forall t \in I \subset \R^+_0\\
x(0) &= x_0.
\end{align}
\par
Wir nehmen an, dass \(x_F \in U\) eine Ruhelage des dynamischen Systems ist, so dass dementsprechend \(F(x_F) = 0\) gilt.
Durch einfache Translation der Koordinaten des Systems um \(x_F \in U\), können wir ohne Beschränkung der Allgemeinheit annehmen, dass die Ruhelage sich im Nullpunkt befindet.

\par
Im Folgenden definieren wir zwei wichtige Werkzeuge zur Untersuchung der Stabilität von Ruhelagen für allgemeine Differentialgleichungssysteme.
\label{ode_stability/ruhelagen:def:linearisierung}
\begin{definition}{}{}



\par
Sei \(F\in C^1(U;\R^n)\) ein Vektorfeld auf dem Phasenraum \(U \subset \R^n\) und \(0\) eine Ruhelage des dynamischen Systems, dass durch das allgemeine Differentialgleichungssystem in \eqref{equation:ode_stability/ruhelagen:eq:awpallg} charakterisiert wird.
Sei nun \((DF)(x)\) die Jacobi Matrix der Funktion \(F\) im Punkt \(x \in U\) (vgl. Kapitel 6.2 in \cite{Ten21}).
Dann bezeichnen wir mit \(A := (DF)(0)\) die \textbf{Linearisierung} von \(F\) in der Ruhelage \(0 \in U\).
Außerdem bezeichnen wir die Funktion \(R \in C^1(U; \R^n)\) mit
\begin{align*}
R(x) \ \coloneqq \ F(x) - Ax
\end{align*}
\par
als die \textbf{Abweichung} (auch \textbf{Residuum} genannt) des Vektorfeldes \(F\) von seiner Linearisierung \(A\) in der Ruhelage.
\end{definition}

\par
Mit diesen Hilfswerkzeugen werden wir im Folgenden zeigen, dass die Lösung des Differentialgleichungssystem in führender Ordnung durch die Linearisierung \(A\) von \(F\) kontrolliert werden, solange wir uns nah genug zur Ruhelage befinden. Dies wird durch das folgende Lemma ausgedrückt.
\label{ode_stability/ruhelagen:lem:intexpglgn}
\begin{lemma}{}{}



\par
Wir betrachten das Anfangswertproblem aus \eqref{equation:ode_stability/ruhelagen:eq:awpallg} auf dem Phasenraum \(U \subset \R^n\) für ein Vektorfeld \(F\in C^1(U;\R^n)\).
Außerdem sei \(A \coloneqq (DF)(0)\) die Linearisierung des Vektorfelds in der Ruhelage \(0\) des dynamischen Systems und \(R(x) \coloneqq F(x) - Ax\) die Abweichung von \(F\) von seiner Linearisierung \(A\) im Nullpunkt.

\par
Dann lassen sich Lösungen des Differentialgleichungssystems mittels der Linearisierung \(A\) und der Abweichung \(R\) explizit angeben als
\begin{align*}
x(t) = e^{At}x_0 + \int_0^t e^{A(t-s)} R(x(s))\, \mathrm{d}s, \quad \forall t \in I.
\end{align*}\end{lemma}

\begin{proof}
 Wir setzen zunächst die unbekannte Lösung \(x(t)\) des Anfangswertproblems \eqref{equation:ode_stability/ruhelagen:eq:awpallg} in der allgemeinen Form
\begin{align*}
x(t) = e^{At}c(t),\quad \text{mit }c(0) = x_0
\end{align*}
\par
an, und suchen eine Bestimmungsgleichung für die unbekannte Funktion \(c(t)\) mittels \textbf{Variation der Konstanten} (vgl. Kapitel 8.2 in \cite{Ten21}).

\par
Mittels der Rechenregeln für das Matrixexponentials in \cref{ode/repetition:rem:matrixexponential_regeln} können wir die Ableitung der Funktion \(x\) mittels Produktregel angeben als
\begin{align*}
\dot{x}(s) = A e^{As}c(s)+ e^{As}\dot{c}(s) = Ax(s) + e^{As}\dot{c}(s).
\end{align*}
\par
Aus der Definition des Residuums in \cref{ode_stability/ruhelagen:def:linearisierung} folgt aber auch
\begin{align*}
\dot{x}(s) = F(x(s)) = Ax(s) + R(x(s)).
\end{align*}
\par
Vergleichen wir die beiden Gleichungen, so sieht man ein, dass
\begin{align*}
e^{As}\dot{c}(s) = R(x(s))
\end{align*}
\par
gelten muss.
Äquivalent können wir auch folgern, dass \(\dot{c}(s) = e^{-As}R(x(s))\) gilt.

\par
Nach dem Hauptsatz der Differential  und Integralrechnung (vgl. Theorem 5.3 in \cite{Ten21}) gilt dann für die unbekannte Funktion \(c\) der folgende Zusammenhang
\begin{align*}
c(t) = c(0) + \int_0^t \dot{c}(s)\, \mathrm{d}s = x_0+ \int_0^t e^{-As}R(x(s)) \, \mathrm{d}s.
\end{align*}
\par
Setzen wir dies in die erste Gleichung unserer Ansatzfunktion ein und nutzen die Rechenregeln des Matrixexponnentials aus \cref{ode/repetition:rem:matrixexponential_regeln}  so erhalten wir schließlich die Aussage des Lemmas
\begin{align*}
x(t) = e^{At}x_0+ \int_0^t e^{A(t-s)}R(x(s)) \, \mathrm{d}s.
\end{align*}\end{proof}

\par
Auf den ersten Blick nützt uns die Identität in \cref{ode_stability/ruhelagen:lem:intexpglgn} nicht viel, denn auch auf der rechten Seite taucht \(x(s)\), also die unbekannte Lösung des Anfangswertproblems \eqref{equation:ode_stability/ruhelagen:eq:awpallg} auf.
Es stellt sich jedoch heraus, dass wir die \textbf{Gronwall Ungleichung} auf diese Integralgleichung anwenden können.
Diese wichtige Abschätzung in der Theorie von Differentialgleichungen ähnelt Münchhausens Methode, sich an den eigenen Haaren aus dem Sumpf zu ziehen.
\label{ode_stability/ruhelagen:lemma:Gronwall}
\begin{lemma}{}{}



\par
Für zwei stetige Funktionen \(f,g\in C([t_0,t_1]; \R^+)\) gelte für eine Konstante \(a \geq 0\) die Ungleichung
\begin{align*}
f(t) \leq a + \int_{t_0}^t f(s)g(s)\, \mathrm{d}s \quad \forall t\in [t_0,t_1].
\end{align*}
\par
Dann lässt sich der Wert der Funktion \(f\) durch die Funktion \(g\) wie folgt abschätzen
\begin{align*}
f(t) \leq a \exp{ \left(\int_{t_0}^t g(s)\, \mathrm{d}s \right)} \quad \forall t\in [t_0,t_1].
\end{align*}\end{lemma}

\begin{proof}
 Wir definieren zunächst eine Hilfsfunktion
\begin{align*}
h(t) \ \coloneqq \ a + \int_{t_0}^t f(s)g(s)\, \mathrm{d}s
\end{align*}
\par
und bemerken, dass \(0 \leq f(t) \leq h(t)\) nach Voraussetzung gilt für alle \(t \in [t_0, t_1]\).
Nun führen wir eine einfache Fallunterscheidung durch:

\par
1. Ist \(h(t)=0\), so folgt mit der Abschätzung \(f(t) \leq h(t)\) schon, dass \(f(t) = 0\) gelten muss, so dass die Behauptung des Lemmas trivialerweise erfüllt ist.

\par
2. Sei also im Folgenden \(h(t) > 0\).
Aus dem Haupsatz der Integral  und Differentialrechnung wissen wir, dass \(h'(t) = f(t)g(t)\) gilt.
Wegen \(f(t) \leq h(t)\) für alle \(t \in [t_0, t_1]\) folgt sofort, dass
\begin{align*}
f(t)g(t) \leq h(t)g(t) \quad \forall t \in [t_0,t_1].
\end{align*}
\par
Kombinieren wir diese Abschätzung mit der Identität der Ableitung \(h'(t)\), so erhalten wir durch Umstellen
\begin{align*}
\frac{h'(t)}{h(t)} \leq g(t) \quad \forall t \in [t_0, t_1].
\end{align*}
\par
Da wir \(h(t) > 0\) angenommen haben erhalten wir durch Integration beider Seiten die Abschätzung
\begin{align*}
\int_{t_0}^t \frac{h'(s)}{h(s)} \, \mathrm{d}s \leq \int_{t_0}^t g(s) \, \mathrm{d}s\end{align*}
\par
für alle \(t \in [t_0, t_1]\).
Für die linke Seite können wir das Integral explizit angeben als
\begin{align*}
\int_{t_0}^t \frac{h'(s)}{h(s)} \, \mathrm{d}s = \ln(h(t)) - \ln(h(t_0)) = \ln(h(t)) - \ln(a) = \ln\left(\frac{h(t)}{a}\right).
\end{align*}
\par
Es gilt also nun
\begin{align*}
\ln \left(\frac{h(t)}{a}\right) \leq \int_{t_0}^t g(s)\, \mathrm{d}s.
\end{align*}
\par
Durch Anwenden der Exponentialfunktion auf beiden Seiten erhalten wir schließlich die Behauptung des Lemmas
\begin{align*}
 h(t)\leq a \exp{\left( \int_{t_0}^t g(s)\, ds \right)} \quad \forall t \in [t_0,t_1].
\end{align*}\end{proof}

\par
Wir wollen folgende Bemerkungen zur Gronwall Ungleichung festhalten.
\label{ode_stability/ruhelagen:remark-4}
\begin{emphBox}{}{}{Remark 2.2}



\par
1. Die in \cref{ode_stability/ruhelagen:lemma:Gronwall} beschriebene Gronwall Ungleichung ist eigentlich ein Spezialfall für eine konstante Funktion \(a(t) \equiv a \geq 0\).
Die ursprünglich bewiesene Aussage gilt auch für allgemeinere Funktionen.

\par
2. Man kann sich die Abschätzung in der Gronwall Ungleichung leicht merken wenn man Gleichheit der beiden Seiten annimmt.
Die Integralgleichung
\begin{align*}
f(t) = a + \int_{t_0}^t f(s)g(s)\, \mathrm{d}s \quad t\in [t_0,t_1]
\end{align*}
\par
entspricht nämlich dem \textbf{linearen Anfangswertproblem}
\begin{align*}
\dot{f}(t) &= f(t)\cdot g(t) \quad \forall t \in [t_0, t_1], \\
f(t_0) &= a,
\end{align*}
\par
welches für alle \(t \in [t_0, t_1]\) die folgende explizite Lösung besitzt
\begin{align*}
f(t) = a \exp{\left( \int_{t_0}^t g(s)\, \mathrm{d}s \right)}.
\end{align*}\end{emphBox}

\par
Wir werden die Resultate der beiden Lemmata in den folgenden Abschnitten anwenden, um die Stabilität von Ruhelagen eines allgemeinen dynamischen Systems durch eine Linearisierung zu untersuchen.


\subsection{Asymptotische Stabilität von Ruhelagen}
\label{\detokenize{ode_stability/ruhelagen:asymptotische-stabilitat-von-ruhelagen}}\label{ode_stability/ruhelagen:theorem-5}
\begin{theorem}{}{}



\par
Eine Gleichgewichtslage \(x_s\in  U\subset \R^n\) der DGL
\begin{align*}
\dot{x} = F(x), \quad f\in C^1(U,\R^n)
\end{align*}
\par
ist asymptotisch stabil, wenn für die Eigenwerte \(\lambda\) von \(Df(x_s)\) gilt: \(Re(\lambda)<0\).
\end{theorem}

\begin{proof}
   Wieder können wir durch eine Verschiebung \(x_s = 0\) erreichen. Da \(U\subseteq\R^n \) offen ist, gehört eine Kugelumgebung vom positiven Radius \(\vec{r}\) zu \(U\).
\begin{itemize}
\item {} 
\par
Für die Eigenwerte \(\lambda_i\) von \(A:=Df(x_s)\) gelte die Abschätzung \(Re(\lambda_i)< -\Lambda,\ \Lambda>0\) geeignet. Dann gibt es ein \(c>0\), sodass gilt

\end{itemize}
\begin{align*}
\|e^{At}\| \leq c\cdot e^{-\Lambda t}\quad (t\geq 0).
\end{align*}
\par
Das ersieht man aus der Tatsache, dass die Einträge von \(e^{At}\) Summen von \(\lambda_i\) Quasipolynomen \(e^{\lambda_i t}p_i(t)\) sind und dass wegen \(Re(\lambda_i) + \Lambda <0\)
\begin{align*}
\lim_{t\to\infty} \exp((\lambda_i + \Lambda)t)p_i(t) = 0
\end{align*}
\par
ist.
\begin{itemize}
\item {} 
\par
Nun existiert ein Radius \(r\in (0,\vec{r})\) mit

\end{itemize}
\begin{align*}
\|R(x)\|\leq \frac{\Lambda}{2c} \|x\|, \quad \text{falls } \|x\|\leq r,
\end{align*}
\par
denn \(\lim_{x\to 0} \frac{\|Re(x)\|}{\|x\|} = \lim_{x\to}\frac{\|f(x)- Df(0)\cdot x\|}{\|x\|} = 0\).
\begin{itemize}
\item {} 
\par
Wenn wir nun zeigen können, dass aus

\end{itemize}
\begin{align}\label{equation:ode_stability/ruhelagen:eq:residuumabsch}
\|x(0)\|\leq \epsilon <\frac{r}{c}
\end{align}
\par
folgt, dass für \(t\geq 0\) gilt:
\begin{align*}
\|x(t)\|\leq c\epsilon e^{-\Lambda t/2},
\end{align*}
\par
dann haben wir den Satz bewiesen, denn für die rechte Seite gilt: \(c\epsilon e^{-\Lambda t/2} \leq c\epsilon<r<\tilde{r}\), und für \(t\to \infty\) strebt sie gegen Null.

\par
Nun gilt nach Lemma \cref{ode_stability/ruhelagen:lem:intexpglgn}  \(x(t) = e^{At}x_0 + \int_0^t e^{A(t-s)} R(x(s))\, ds\) woraus mit \eqref{equation:ode_stability/ruhelagen:eq:residuumabsch} die Ungleichung
\begin{align*}
\|x(t)\|\leq ce^{-\Lambda t}\|x_0\| + \int_0^tce^{-\Lambda (t-s)}\frac{\Lambda}{2c}\|x(s)\|\, ds
\end{align*}
\par
folgt, soweit \(\|x\|\leq r\).

\par
Setzt man \(F(t):=e^{\Lambda t}\|x(t)\|\), dann gilt
\begin{align*}
F(t)\leq \underbrace{c\|x_0\|}_{=:a} + \int_0^t \underbrace{\frac{\Lambda}{2}}_{=:g(s)} F(s)\, ds,
\end{align*}
\par
also nach dem Gronwall Lemma \cref{ode_stability/ruhelagen:lemma:Gronwall} \begin{align*}
F(t) \leq c \|x_0\| + \exp{\left( \frac{1}{2} \int_0^t \Lambda \, ds \right) }
\leq c \epsilon e^{\frac{\Lambda}{2} t} \leq r e^{\frac{\Lambda}{2} t}
\end{align*}
\par
oder \(\|x(t)\|\leq re^{-\frac{\Lambda}{2}t}\). Die Lösungskurve bleibt also für alle positiven Zeiten in der Vollkugel \(\{x\in \R^n|\|x\|\leq r\}\) und konvergiert gegen Null.
\end{proof}
\label{ode_stability/ruhelagen:remark-6}
\begin{emphBox}{}{}{Remark 2.3}



\par
Der Beweis liefert zusätzlich die Aussage, dass alle \(x\in U\) mit \(\|x\|<\frac{r}{c}\) zu gegen die Gleichgewichtslage konvergierenden Orbits gehören, also in deren Einzugsbereich, dem so genannten \emph{Bassin}, liegen.
\end{emphBox}


\subsection{Lyapunov Stabilität von Ruhelagen}
\label{\detokenize{ode_stability/ruhelagen:lyapunov-stabilitat-von-ruhelagen}}
\par
Während ein hinreichendes Kriterium für das Vorliegen \emph{asymptotischer Stabilität} die strikte Ungleichung \(Re(\lambda_i)<0\) für die Eigenwerte \(\lambda_i\) der Jacobi Matrix war, ist die Situation bezüglich der Lyapunov Stabilität komplizierter.
\label{ode_stability/ruhelagen:theorem-7}
\begin{theorem}{}{}



\par
Besitzen die Eigenwerte \(\lambda_i\) von \(A\in \R^{n\times n}\) Realteil \(Re(\lambda_i) \leq 0\), und ist im Fall \(Re(\lambda_i)=0\) die geometrische Vielfachheit gleich der algebraischen, dann ist \(0\in \R^n\) Lyapunov stabile Gleichgewichtslage der DGL \(\dot{x} = Ax\).
\end{theorem}

\begin{proof}
 Aus der \textbf{Jordan Normalform} \(A=VJV^{-1}\) mit Jordan Matrix \(J\) der Form
\begin{align*}
J=
\begin{pmatrix}
J_{r_1}(\lambda_1)& & & 0\\
 & J_{r_2}(\lambda_2) & & \\
 & & \ddots & \\
 0 & & J_{r_k}(\lambda_k)
\end{pmatrix}, 
\quad J_r(\lambda_r):=
\begin{pmatrix}
\lambda_r & 1 & & 0\\ 
& \ddots & \ddots & \\
& & \ddots & 1\\
0 & & & \lambda_r
\end{pmatrix} 
\in \C^{r\times r}
\end{align*}
\par
folgt mit
\begin{align*}
\exp{(Jt)} = \begin{pmatrix}
\exp{(J_{r_1}(\lambda_1)t)} & & 0\\
 & \ddots & \\
 0& & \exp{(J_{r_k}(\lambda_k)t)}
 \end{pmatrix}
\end{align*}
\par
die Liapunov Stabilität der Gleichgewichtslage aus
\begin{align*}
\|\exp{(At)}\|= \|V\exp{(Jt)}V^{-1}\|\leq \|V\|\|V^{-1}\|\|\exp{(Jt)}\|,
\end{align*}
\par
wenn für alle Jordanblöcke \(J_{r_i}(\lambda_i)\) von \(J\) der Ursprung \(0\in \C^{r_i}\) Liapunov stabile Gleichgewichtslage der DGL \(\dot{y} = J_{r_i}(\lambda_i)y\) ist.

\par
Aus \(Re(\lambda_i)<0\) folgt sogar asymptotische Stabilität.

\par
Da für einen komplexen Eigenwert \(\lambda\) von \(A\) mit \(Re(\lambda)=0\) die geometrische Vielfachheit nach Vorraussetzung gleich der algebraischen ist, sind die ihm zugeordneten Jordanblöcke alle eindimensional: \(J_{r_i}(\lambda_i) = \lambda\). Damit ist in diesem Fall
\begin{align*}
\|exp{(J_{r_i}(\lambda_i)t)}\|= |\cos{(Im(\lambda)t)} + i \sin{(Im(\lambda)t)}|=1.
\end{align*}\end{proof}
\label{ode_stability/ruhelagen:remark-8}
\begin{emphBox}{}{}{Remark 2.4}



\par
Dass man die Gleichheit von geometrischer und algebraischer Vielfachheit fordern muss, sieht man schon am Beispiel von
\begin{align*}
A = \begin{pmatrix} 0&1\\0&0\end{pmatrix} \quad \text{mit} \quad e^{At} = \begin{pmatrix} 0&t\\0&0\end{pmatrix}.
\end{align*}\end{emphBox}

\par
Leider kann man nicht wie im Fall der asymptotischen Stabilität vom linearen auf den nicht linearen Fall folgern.
\label{ode_stability/ruhelagen:example-9}
\begin{example}{}{}



\par
DGL \(\dot{x} = \alpha x+ \beta x^3\) mit Parametern \(\alpha, \beta\in \R\). \textbackslash{}
Die Null ist Gleichgewichtslage der DGL und ihre Linearisierung \(\dot{y} = \alpha y\).


\begin{center}
\centering
\begin{tabularx}{\linewidth}[{\linewidth}]{|c|c|c|}\hline

\par

& 
\par
linearisierte Gleichung
& 
\par
nicht lineare Gleichung
\\
\hline
\par
\(\alpha<0\)
&
\par
asymptotisch stabil
&
\par
asymptotisch stabil
\\
\hline
\par
\(\alpha>0\)
&
\par
instabil
&
\par
instabil
\\
\hline
\par
\(\alpha=0\)
&
\par
Liapunov stabil
&
\par
asymptotisch stabil für \(\beta<0\)
\\
\hline
\par

&
\par

&
\par
stabil für \(\beta =0 \)
\\
\hline
\par

&
\par

&
\par
instabil \(\beta > 0\)
\\
\hline
\end{tabularx}
\end{center}
\end{example}
\label{ode_stability/ruhelagen:remark-10}
\begin{emphBox}{}{}{Remark 2.5}



\par
Anschaulich gesprochen kann asymptotische Stabilität nur vorliegen, wenn der Fluss in der Nähe der Gleichgewichtslage das Phasenraumvolumen verkleinert. Daher kann man in physikalischen Situationen ohne Reibungseffekte höchstens Liapunov Stabilität erwarten. das hat z.B. zur Folge, dass die Frage der \textbf{Stabilität des Sonnensystems} sehr subtil ist.
\end{emphBox}


\chapter{Bibliography}
\label{\detokenize{references:bibliography}}\label{\detokenize{references::doc}}
\par


\begin{sphinxthebibliography}{For17}
\bibitem[For17]{references:id4}
\par
Otto Forster. \emph{Analysis 2}. Springer, 2017.
\bibitem[Kna13]{references:id5}
\par
Peter Knabner. \emph{Skript zur Vorlesung "Gewöhnliche Differentialgleichungen"}. 2013.
\bibitem[Kna17]{references:id8}
\par
Andreas Knauf. \emph{Mathematische Physik: Klassische Mechanik}. Springer Berlin Heidelberg, 2017. \href{https://doi.org/10.1007/978-3-662-55776-1}{doi:10.1007/978 3 662 55776 1}.
\bibitem[Kna20]{references:id7}
\par
Andreas Knauf. \emph{Skript zur Vorlesung "Mathematik für Physikstudierende 3"}. 2020.
\bibitem[Nol11]{references:id9}
\par
Wolfgang Nolting. \emph{Grundkurs Theoretische Physik 2   Analytische Mechanik}. Springer Berlin Heidelberg, 2011. \href{https://doi.org/10.1007/978-3-642-12950-6}{doi:10.1007/978 3 642 12950 6}.
\bibitem[SB18]{references:id10}
\par
Herman Schulz Baldes. \emph{Skript zur Vorlesung "Mathematik für Physiker 3"}. 2018.
\bibitem[Ten21]{references:id12}
\par
Daniel Tenbrinck. \emph{Skript zur Vorlesung "Mathematik für Data Science 2"}. 2021. URL: \url{https://fau-ammn.github.io/MathDataScience2}.
\end{sphinxthebibliography}






\renewcommand{\indexname}{Proof Index}


\renewcommand{\indexname}{Index}

\end{document}